\documentclass[12pt]{article}
\usepackage{lscape}
\usepackage{amssymb}
\usepackage[algo2e,ruled,linesnumbered]{algorithm2e} % for algorithms
\hoffset -25truemm
\usepackage{latexsym}
\oddsidemargin=30truemm             %%
\evensidemargin=25truemm            %% inner margin 30mm, outer margin 25mm
\textwidth=155truemm                %%
\voffset -25truemm
\topmargin=25truemm                 %% top margin of 25mm
\headheight=0truemm                 %% no head
\headsep=0truemm                    %% no head
\textheight=220truemm               
\renewcommand{\thefootnote}{}
\newtheorem{definition}{Definition}[section]
\newtheorem{lemma}[definition]{Lemma}
\newtheorem{theorem}[definition]{Theorem}
\newtheorem{corollary}[definition]{Corollary}
\newtheorem{remark}[definition]{Remark}
\newtheorem{problem}[definition]{Problem}
\newenvironment{proof}{\normalsize {\sc Proof}:}{{\hfill $\Box$ \\}}

\def\X{{\bf X}}
\def\O{{\bf O}}

\def\SL{{\rm SL}}
\def\GL{{\rm GL}}
\def\GU{{\rm GU}}
\def\GSp{{\rm GSp}}
\def\U{{\rm U}}
\def\PSL{{\rm PSL}}
\def\PSp{{\rm PSp}}
\def\PSU{{\rm PSU}}
\def\GF{{\rm GF}}
\def\Sp{{\rm Sp}}
\def\SU{{\rm SU}}
\def\SX{{\rm SX}}
\def\PX{{\rm PX}}
\def\GX{{\rm GX}}
\def\PSX{{\rm PSX}}
\def\PGL{{\rm PGL}}
\def\q{\quad}
\def\centreline{\centerline}

\begin{document}

\title{Constructive recognition of classical groups in odd characteristic} 
\author{C.R.\ Leedham-Green and E.A. O'Brien}
\date{}
\maketitle

\begin{abstract}
Let $G = \langle X \rangle \leq \GL(d, F)$ be 
one of $\SL (d, F)$, $\Sp (d, F)$, or $\SU (d, F)$
where $F$ is a finite field of odd characteristic. 
We present algorithms to construct
standard generators for $G$ which allow us
to write an element of $G$ as a straight-line
program in $X$.  The algorithms run in Las Vegas polynomial-time, 
subject to the existence of a discrete log oracle for $F$. 
\end{abstract}

\footnote{This work was supported in part by the Marsden Fund of
New Zealand via grant UOA 412. We thank John Bray and Robert
Wilson for helpful discussions on its content.
2000 {\it Mathematics Subject Classification}.
Primary 20C20, 20C40.}

\section{Introduction}
\label{intro}

The major goal of the ``matrix recognition project"
is the development of efficient
algorithms for the investigation of 
subgroups of $\GL(d, F)$ where $F$ 
is a finite field. 
We refer to the recent survey \cite{OBrien05}
for background related to this work.
A particular aim is to 
identify the composition factors
of $G \leq \GL(d, F)$. If a problem
can be solved for the composition factors,
then it can be frequently be solved for $G$.

One may intuitively think of a {\it straight-line program} (SLP)
for $g \in G = \langle X \rangle$ as an efficiently stored group word
on $X$ that evaluates to $g$.  For a formal definition, we
refer the reader to Section \ref{SLP}.
A critical property of an SLP is 
that its length is proportional to the number of 
multiplications and exponentiations used in
constructing the corresponding group element. 
Babai \& Szemer{\'e}di \cite{BabaiSzemeredi84}
prove that every element of 
$G$ has an SLP in $X$ of length at most $O(\log^2|G|)$.

Informally, a {\it constructive recognition algorithm}
constructs an explicit isomorphism
between a group $G$ and a ``standard" (or natural)
representation of $G$, and exploits this isomorphism
to write an arbitrary element of $G$ as
an SLP in its defining generators. 
For a more formal definition, see \cite[p. 192]{Seress03}.

We consider subgroups of $\GL(d, q)$  which
preserve a certain non-degenerate bilinear or sesquilinear
form on $V$, the vector space of $d$-dimensional
row vectors over $\GF (q)$ on which $\GL(d, q)$  acts
naturally.  Let $\Sp (d, q)$, for even $d$, denote the
subgroup of $\GL(d,q)$ consisting of all linear 
transformations of $V$ which preserve 
a non-degenerate alternating bilinear form.
Let $\U(d, q)$  denote the subgroup of 
$\GL(d,q^2)$ consisting of all linear 
transformations of the underlying vector space of 
degree $d$ over $\GF(q^2)$ which preserve 
a non-degenerate hermitian form.
%Let $\SL(d, q)$ denote
%the group of matrices of determinant 1 which
%preserve no non-degenerate bilinear, quadratic, 
%or sesquilinear form.
We use the notation $\GX(d,q)$ 
to denote $\GL(d,q)$, or $\Sp(d,q)$, or $\U(d,q)$
and $\SX(d, q)$ denotes the corresponding subgroup of 
matrices of determinant 1. 

%Let $\SX(d,q)$ denote $\SL(d,q)$, or $\Sp(d,q)$ 
We present and analyse two algorithms that
take as input a generating subset $X$ of $\SX(d,q)$ for $q$ odd, and
return as output {\it standard generators} of this group as 
SLPs in $X$. Usually, these  
generators are defined with respect to a 
basis different to that for which $X$ was defined, 
and a change-of-basis 
matrix is also returned to relate these bases.

Similar algorithms are under development for the orthogonal groups
in odd characteristic.  Further, characteristic 2 can also be 
addressed in the same style, but the resulting algorithms are more complex.
We shall consider these cases in later papers.

Our principal result is the following.
\begin{theorem} \label{main}
Let $G = \langle X \rangle \leq \GL(d, q)$ denote 
$\SL(d,q)$, or $\Sp(d,q)$ for even $d$, or $\SU(d,q)$
where, in all cases, $q$ is odd. 
There are Las Vegas algorithms which,
given the input $X$, 
construct a new standard generating set $S$ 
for $G$ having the property that 
an SLP of length $O(d^2 \log q)$
can be found from $S$ for any $g \in G$. 
Assuming the existence of a discrete log oracle 
for $\GF (q)$, the algorithm to construct
$S$ runs in $O(d (\xi + d^3 \log q + \chi))$ field operations, 
where $\xi$ is the cost of constructing an independent 
(nearly) uniformly distributed random element of $G$,
and $\chi$ is the cost of a call to a discrete log oracle for $\GF(q)$.
\end{theorem}

We prove this theorem by exhibiting algorithms
with the stated complexity. If we assume that
a random element can be constructed in 
$O(d^3)$ field operations, then, for fixed $q$,
an upper bound to the complexity of constructing 
the standard generators is $O(d^4)$. 

Brooksbank's algorithms \cite{Brooksbank03} for the natural 
representation of $\Sp(d, q)$, $\SU(d, q)$, and $\Omega^\epsilon(d, q)$ 
have complexity $O(d^5)$ for fixed $q$. More precisely, 
the complexity of his algorithm to construct standard
generators for the classical group is 
$$O(d^3 \log q (d + \log d \log^3 q) + \xi(d + \log \log q) + d^5 \log^2 q
   + \chi(\log q)).$$
%$O(d^2 \log d (\mu + \chi \log q + d\log^4 q))$,
%where $\epsilon$ is the cost of constructing  a independent 
%(nearly) uniformly distributed random element,
%and $\chi$ is the cost of a call to a discrete log oracle for $\GF(q)$.
The algorithm of Celler \& Leedham-Green \cite{CellerLeedhamGreen98}
for $\SL(d, q)$ has complexity $O(d^4 \cdot q)$.

Once we have constructed these standard generators for $G$,
a generalised echelonisation algorithm can 
now be used to write a given element of $G$ 
as an SLP in terms of these generators.
We do not consider this task here, but refer 
the interested reader to the algorithm
of \cite[Section 5]{Brooksbank03},
which performs this task in $O(d^3\log q + \log^2 q)$ field operations. 

The two algorithms presented here reflect a tension 
between two competing tasks: the speed of construction
of the standard generators, and minimising the 
length of the resulting SLPs for the standard 
generators in $X$.
The first is designed for optimal efficiency; 
the second to produce short SLPs.
We consider this topic in more detail in Section \ref{SLP}.

We establish some notation. 
Let $g\in G \leq \GL(d, q)$, 
let $\bar{G}$ denote $G / G \cap Z$
where $Z$ denotes the centre of $\GL(d, q)$,
and let $\bar{g}$ denote the image of $g$ in $\bar{G}$.
The {\it projective centraliser} of $g \in G$
is the preimage in $G$ of $C_{\bar{G}} (\bar{g})$.
Further $g \in G$ is a {\it projective involution} 
if $g^2$ is scalar, but $g$ is not.

A central component of both algorithms
is the use of involution centralisers.
In Section \ref{cent} we summarise
the structure of involution centralisers
for elements of classical groups in odd characteristic.
In Section \ref{standard} we define
standard generators for the classical groups.
In Sections \ref{Alg1} and \ref{Alg2} the two 
algorithms are described. They
rely on finding involutions  whose eigenspaces  have
approximately the same  dimension in the case of the first algorithm,
and exactly the same dimension  in the second. The cost 
of constructing such involutions 
is analysed in Sections \ref{Involution} and Section \ref{Equal}.
We frequently compute high powers of elements of linear groups; 
an algorithm for doing this
efficiently is described in Section \ref{Exp}. 
In Section \ref{Pow}, we discuss how to  
construct both the derived group of a group 
containing a classical group and also the 
the factors of a direct product of two classical groups. 
The centraliser of an involution  is constructed using an 
algorithm of Bray \cite{Bray}; this is considered
in Section \ref{Bray}. The base cases of the
algorithms (when $d \leq 4$) are discussed in Section \ref{base}. 
The complexity of the algorithms and the 
length of the resulting SLPs for the standard generators
are discussed in Section \ref{Analysis} and \ref{SLP}.  
Finally we report on our implementation of the algorithm, 
publicly available in {\sc Magma} \cite{Magma}.

\section{Centralisers of involutions in classical groups}\label{cent} 
We briefly review the structure of involution centralisers 
in (projective) classical groups 
defined over fields of odd characteristic.
A detailed account can be found in \cite[4.5.1]{GLS3}.
\begin{enumerate}
\item 
If $u$ is an involution in $\SL(d,q)$, with eigenspaces $E_+$ and
$E_-$, then the centraliser of $u$ in $\SL(d,q)$ is
$(\GL(E_+)\times\GL(E_-))\cap\SL(d,q)$. The centraliser of the
image of $u$ in $\PSL(d,q)$ is the image of the centraliser of $u$ in
$\SL(d,q)$ if $E_+$ and $E_-$ have different dimensions. If
$E_+$ and $E_-$ have the same dimension, then in $\PSL(d,q)$ these
eigenspaces may be interchanged by the centraliser of the image of
$u$, which is now the image of $(\GL(d/2,q)\wr C_2)\cap\SL(d,q)$ in
$\PSL(d,q)$.

\item 
If $u$ is an involution in $\Sp(2n,q)$, with eigenspaces $E_+$ and
$E_-$, these spaces are  mutually orthogonal, and the form restricted
to either is non-singular. Thus the centraliser of $u$ is 
$\Sp(E_+) \times \Sp(E_-)$. The centraliser of the image of $u$ in
$\PSp(2n,q)$ is the image of $\Sp(E_+)\times\Sp(E_-)$, except when the
eigenspaces have the same dimension, when the centraliser again
permutes the eigenspaces. An element of the projective
centraliser permuting the eigenspaces sends $(v,w)$ to
$(w\theta,-v\theta)$, where  $\theta$ is an isometry that normalises 
these spaces, so the image of $\Sp(E_+)\times\Sp(E_-)$ has index 2
in the projective centraliser.

\item 
If $u$ is an involution in $\SU(d,q)$, the situation is similar.
Again the eigenspaces of $u$ are mutually orthogonal, and the form
restricted to the eigenspaces is non-degenerate. The centraliser of
$u$ in $\SU(d,q)$  is $(\U(E_+)\times\U(E_-))\cap\SU(d,q)$. The
centraliser of the image of $u$ in $\PSU(d,q)$ is the image of the
centraliser of $u$ in $\SU(d,q)$  except where the eigenspaces of
$u$ have the same dimension, when the centraliser is the image of
$(\U(d/2,q)\wr C_2)\cap\SU(d,q)$ in $\PSU(d,q)$.
\end{enumerate}
\section{Standard generators for classical groups}
\label{standard}
We now describe {\it standard generators} for
the perfect classical groups $\SL(d,q)$, $\Sp(d,q)$ and $\SU(d,q)$
where $q$ is odd in all cases.

We use the notation $\SX(d,q)$ to denote any  one of these 
groups, and $\PX(d,q)$ to denote the corresponding central quotient.

Let $V$ be the natural module for a perfect classical group $G$ of the
above kind.  The standard generators for $G$ are defined with respect to 
a hyperbolic basis for $V$, which will be defined in terms of the
given basis by a change-of-basis matrix. 
We define a {\it hyperbolic} basis for $V$ as follows. 
\begin{enumerate}
\item 
If $G=\SL(d,q)$ then any ordered basis, say $(e_1, \ldots, e_d)$, is hyperbolic. 

\item 
If $G=\Sp(2n,q)$ then a hyperbolic basis for $V$ is an
ordered basis of the form $(e_1,f_1,\ldots,e_n,f_n)$, where, if the
image of a pair of vectors $(v,w)$ under the form is written as $v.w$,
then $e_i.e_j=f_i.f_j=0$ for all $i,j$ (including the case $i=j$), and
$e_i.f_j=0$ for $i\ne j$, and $e_i.f_i=-f_i.e_i=1$ for all $i$. 

\item 
If $G=\SU(2n,q)$, then a hyperbolic
basis is exactly as
for $\Sp(2n,q)$ except that, the form being hermitian, the
condition  $e_i.f_i=-f_i.e_i=1$ for all $i$ is replaced by the
condition $e_i.f_i=f_i.e_i=1$ for all $i$. 

\item 
If $G=\SU(2n+1,q)$, then a hyperbolic basis for $V$ is 
an ordered basis of the form 
$(e_1,f_1,\ldots,e_n,f_n,w)$, where the above equations hold, and in
addition $e_i.w=f_i.w=0$ for all $i$, and $w.w=1$. 
\end{enumerate}
For uniformity of exposition, we sometimes
label the ordered basis for 
$\SL(2n, q)$ as $(e_1,f_1,\ldots,e_n,f_n)$
and that for $\SL(2n + 1, q)$ 
as $(e_1,f_1,\ldots,e_n,f_n, w)$.

The standard generators for $\SX (d, q)$ are with respect
to a hyperbolic basis for $V$; these  are defined
in Table \ref{standard-table}, subject to the following conventions.
\begin{enumerate}
\item 
For the unitary groups, $\omega$ is a primitive element for $\GF(q^2)$
and $\alpha = \omega^{(q + 1)/2}$. 
In all other cases $\omega$ is a primitive element for $\GF(q)$.

%\item 
%For the unitary groups,  $\alpha = \omega^{(q + 1)/2}$. 
%For $\SU(2n+1, q^2)$, 
%$\beta$ and $\gamma$ satisfy the
%equation $\gamma^{q+1}+\beta+\beta^q=0$,
%and $\beta \not\in \GF(q)$.
%Both $u$ and $v$ are trivial for $\SU(3, q)$.

\item In all but one case, we describe $v$
as a permutation matrix acting on the 
hyperbolic basis for $V$.

\item For $\SU(2n+1, q)$, the matrices $x$ and $y$
normalise the subspace $U$ having ordered basis $B = (e_1,w,f_1)$ and
centralise $\langle e_2, f_2, \ldots, e_{n}, f_{n}\rangle$.
We list their action on $U$ with respect to basis $B$.

\item 
All other generators 
normalise either a 2-dimensional or 4-dimensional subspace $U$ 
having ordered basis $B$ which is $(e_1, f_1)$ or $(e_1, f_1, e_2, f_2)$
and centralise the space spanned by the remaining basis vectors.
We list the action of a generator on $U$ with respect to basis $B$.

\item 
To facilitate uniform exposition, we introduce trivial generators. 
Observe that $vx$ is a $2n$-cycle for $\SL(2n, q)$. 
If the dimension required to define a generator is too large, 
then the generator is assumed to be trivial.
\end{enumerate}

It is of course a triviality to  {\it write down} the standard generators
for $G=\SX(d, q)$.  However we must construct these elements
as SLPs in the given generators of $G$.

Once a hyperbolic basis has been chosen for $V$, the Weyl group of $G$
can be defined as a section of $G$, namely as the group of monomial 
matrices in $G$ modulo diagonal
matrices, thus defining a subgroup of the symmetric group $S_d$. For
$G=\SL(d,q)$, this group is $S_d$. For $\Sp(2n,q)$ the Weyl group 
is the subgroup of $S_{2n}$ that preserves the system of imprimitivity 
with blocks
$\{e_i,f_i\}$ for $1\le i\le n$, and is thus $C_2\wr S_n$. For
each of $\SU(2n,q)$ and $\SU(2n+1,q)$, the Weyl group is also $C_2\wr S_n$. 

\begin{lemma}
Let $G = \SU(d, q)$ for $d \geq 2$.  
Then $G = \langle s,t, \delta, u,v, x, y\rangle$. 
\end{lemma}
\begin{proof}
%With respect to the basis $(e_1, w, f_1)$, 
%$$x = \left(\matrix{1&\gamma&\beta\cr 0&1 & -\gamma^q \cr 0 & 0 & 1\cr}\right) \;\;
%y = \left(\matrix{\omega &0&0\cr 0&\omega^{q-1} \cr 0 & 0 & \omega^{-q}\cr}\right)$$
%where $\gamma^{q+1} + \beta + \beta^q = 0$.
If $d = 2n + 1$, then a direct computation shows that 
$$x^y = 
\left(\matrix{
%1 & \omega^{q - 2} \gamma & \omega^{-(q + 1)} \beta \cr
%0 & 1                     & -\omega^{-2q + 1} \gamma^q \cr
1 & \omega^{q - 2} & -\omega^{-(q + 1)} /2 \cr
0 & 1                     & \omega^{-2q + 1} \cr
0 & 0                     & 1 
\cr}\right).
$$
Thus $S = \langle x^{y^k} : 1 \leq k \leq |y| \rangle$
is a non-abelian group of order $q^3$, having
derived group and centre of order $q$. 
%Its derived group and centre coincide with the set of matrices
%having  the $1,2$ and $2,3$ entries equal to $0$. 
A similar calculation for $d = 2n$ shows that 
$\langle x^{y^k} : 1 \leq k \leq |y| \rangle$
is a group of order $q^2$.
These correspond to the subgroups $X_S^1$  of \cite{Carter};
the result now follows from \cite[Proposition 13.6.5]{Carter}. 
\end{proof}

In other cases, the standard generators of $\SX(d, q)$ have the property that 
it is easy to construct from them any root group, and 
consequently they generate $\SX(d, q)$.  The root groups 
are defined with respect to a maximal split torus, the group of 
diagonal matrices in $\SX(d, q)$; for a detailed description see \cite{Carter}.

We now summarise other important properties of our generating sets.  
\begin{theorem}
Let $G = \langle s,t, \delta, u,v, x\rangle$. 
\begin{enumerate}
\item 
Let $G = \SL(d, q)$ for $d \geq 2$. Then $W := \langle s, v, x \rangle$
defines the Weyl group of $G$ modulo its diagonal matrices.
The root groups of $\SL(d, q)$ are obtained by conjugating 
$\langle t^{\delta^k} : 1 \leq k \leq |y| \rangle$ 
by elements of $W$.
If $d = 2n$ then $Y_0:=\{s,t,\delta,u,v\}$
generates $\SL(2,q)\wr C_{n}$.

\item 
Let $G$ be $\Sp(2n, q)$ for $n \geq 2$.
Then $W := \langle s, u, v \rangle$
defines the Weyl group of $G$ modulo its diagonal matrices.
The long root groups of $G$ are obtained
by conjugating $\langle t^{\delta^k} : 1 \leq k \leq |y| \rangle$ 
by elements of $W$, the short root groups of $G$ are obtained
by conjugating $\langle x^{\delta^k} : 1 \leq k \leq |y| \rangle$ by elements of $W$, 
and $Y_0:=\{s,t,\delta,u,v\}$ generates $\SL(2,q)\wr S_n$.

\item 
Let $G$ be $\SU(d, q)$ for $d \geq 3$.
Then $W := \langle s, u, v \rangle$
defines the Weyl group of $G$ modulo its diagonal matrices.
If $d = 2n$ then 
%the long root groups of $G$ are obtained
%by conjugating $\langle t^{y^k} : 1 \leq k \leq |y| \rangle$ 
%by elements of $W$, the short root groups of $G$ are obtained
%by conjugating $\langle x^{y^k} : 1 \leq k \leq |y| \rangle$ 
%by elements of $W$, and 
$Y_0:=\{s,t,\delta,u,v\}$ generates $\SU(2, q) \wr S_n$.
\end{enumerate}
\end{theorem}

\begin{landscape}
\begin{table} \label{standard-table}\tiny 
\begin{center}
\begin{tabular}{|r||c|c|c|c|c|c|c|} 
\hline 
Group & $s$ & $t$ & $\delta$ & $u$ & $v$ & $x$ & $y$ \\ \hline 

$\SL(2n, q)$ 
& 
$\begin{array}{cc} \left(\matrix{0&1\cr-1&0\cr}\right) \end{array} \;$
& 

$\begin{array}{cc} \left(\matrix{1&1\cr0&1\cr}\right) \end{array} \;$

& 
$\left(\matrix{\omega&0\cr0&\omega^{-1}\cr}\right)$
& 
$I_2$
& 

$ (e_1, e_2, \ldots, e_{n})(f_1,f_2,\ldots, f_n) $

& 

$\left(\matrix{0&1&0&0\cr0&0&1&0\cr0&0&0&1\cr-1&0&0&0\cr}\right)$

& 

$I_4$

\\ \hline 

$\SL(2n + 1, q)$ & 
$\begin{array}{cc} \left(\matrix{0&1\cr-1&0\cr}\right) \end{array} \;$
& 

$\begin{array}{cc} \left(\matrix{1&1\cr0&1\cr}\right) \end{array} \;$

& 
$\left(\matrix{\omega&0\cr0&\omega^{-1}\cr}\right)$
& 
$I_2$
& 

$\left(\matrix{ 0 & 1 \cr -I_{2n} & 0 \cr }\right)$
& 

$I_{4}$

& 

$I_4$

\\ \hline 

$\Sp(2n, q)$ & 
$\left(\matrix{0&1\cr-1&0\cr}\right)$
& 

$ \begin{array}{cc} \left(\matrix{1&1\cr0&1\cr}\right) \end{array} $

& 
$\left(\matrix{\omega&0\cr0&\omega^{-1}\cr}\right)$
& 
$\left(\matrix{0&0&1&0\cr0&0&0&1\cr1&0&0&0\cr0&1&0&0\cr}\right)$
& 

$ (e_1, e_2, \ldots, e_{n})(f_1,f_2,\ldots, f_n) $

& 

$\left(\matrix{1&0&0&0\cr0&1&1&0\cr0&0&1&0\cr1&0&0&1\cr}\right)$

& 

$I_4$

\\ \hline 

$\SU(2n, q)$ & 

$\left(\matrix{0&\alpha\cr \alpha^{-q}&0\cr} \right)$

& 

$\left(\matrix{1&\alpha\cr0&1\cr}\right)$

& 

%$\left(\matrix{\omega&0&0&0 \cr 0&\omega^{-q} & 0 & 0\cr 
%                0 & 0 & w^{-1} & 0 \cr 0 & 0 & 0 & w^q\cr}\right)$

$\left(\matrix{\omega^{q + 1}&0\cr0&\omega^{-(q+1)}\cr}\right)$

& 
$\left(\matrix{0&0&1&0\cr0&0&0&1\cr1&0&0&0\cr0&1&0&0\cr}\right)$

& 
$ (e_1, e_2, \ldots, e_{n})(f_1,f_2,\ldots, f_n) $
& 

$\left(\matrix{1&0&1&0\cr0&1&0&0\cr0&0&1&0\cr0&-1&0&1\cr}\right)$

& 

$\left(\matrix{\omega&0&0&0 \cr 0&\omega^{-q} & 0 & 0\cr 
                0 & 0 & \omega^{-1} & 0 \cr 0 & 0 & 0 & \omega^q\cr}\right)$

\\ \hline 

$\SU(2n + 1, q)$ & 

$\left(\matrix{0&\alpha\cr \alpha^{-q}&0\cr} \right)$

& 

$\left(\matrix{1&\alpha\cr0&1\cr}\right)$

& 

$\left(\matrix{\omega^{q + 1}&0\cr0&\omega^{-(q+1)}\cr}\right)$

& 

$\left(\matrix{0&0&1&0\cr0&0&0&1\cr1&0&0&0\cr0&1&0&0\cr}\right)$

& 
$ (e_1, e_2, \ldots, e_{n})(f_1,f_2,\ldots, f_n) $

& 

%$\left(\matrix{1&\beta&\gamma\cr 0&1&0\cr 0&-\gamma^q&1\cr}\right)$
% $\left(\matrix{1&\gamma&\beta\cr 0&1 & -\gamma^q \cr 0 & 0 & 1\cr}\right)$
$\left(\matrix{1& 1 & -1/2 \cr 0&1 & -1 \cr 0 & 0 & 1\cr}\right)$

& 

%$\left(\matrix{\omega&0 & 0 \cr0&\omega^{-q} & 0 \cr 0 & 0 & \omega^{q - 1}\cr}\right)$
$\left(\matrix{\omega &0&0\cr 0&\omega^{q-1} \cr 0 & 0 & \omega^{-q}\cr}\right)$

\\ \hline 

\end{tabular}
\end{center}
\caption{Standard generators for classical groups}
\label{tab1}
\end{table}
\end{landscape}


\section{Algorithm {\tt One}}
\label{Alg1}

Algorithm {\tt One} takes as input a generating set $X$ for
$G=\SX(d,q)$, and returns standard generators for $G$ as SLPs in $X$.
The generators are in standard form 
when referred to a basis constructed  by the algorithm. The change-of-basis 
matrix that expresses this basis in terms of the standard basis for the natural
module is also returned.

The algorithm employs a ``divide-and-conquer" strategy. 
\begin{definition}
A {\it strong involution} in $\SX(d,q)$ for $d > 4$ is an involution whose
$-1$-eigenspace has dimensions in the range $(d/3,2d/3]$. 
\end{definition}

\begin{algorithm2e}[H] 
\caption{\tt OneEven$(X,{\it type})$}
\label{alg1:even}
\tcc{
$X$ is a generating set for
the perfect classical group $G$
in odd characteristic, of type SL or Sp or SU, in even dimension.
Return the standard generating set $Y_0$ for 
$\SL(2, q) \wr C_{d/2}$ if {\it type} is SL, 
otherwise for $\SL(2, q) \wr S_{d/2}$ as subgroup of  $ G$, 
the SLPs for the elements of $Y_0$ and the change-of-basis matrix.
}
\Begin{
 $d$ := the rank of the matrices in $X$; 

if $d  = 2$ then return {\tt BaseCase} (X, {\it type});

$q$ := the size of the field over which these matrices are defined;   

if {\it type} = SU then $q := q^{1/2}$;  

Find by random search $g \in G:=\langle X\rangle$ of 
even order such that $g$ powers to 
a strong involution $h$;

Let $2k$ be the dimension of the $-1$-eigenspace of $h$;

Find generators for the centraliser  $C$ of $h$ in $G$;

In the derived subgroup $C'$ of $C$ find generating sets $X_1$ and $X_2$ 
for the images in $\SX(2k, q)$ and $\SX(d - 2k, q)$ of 
the direct factors of $C'$;

$(s_{1},t_1,\delta_1,u_1,v_1)$ := {\tt OneEven}$(X_1,{\it type})$;

$(s_2,t_2,\delta_2,u_2,v_2)$ := {\tt OneEven}$(X_2,{\it type})$;

Let $(e_1, f_1, \ldots, e_{k}, f_{k}, e_{k+1}, f_{k+1}, \ldots, e_d, f_d)$
be the concatenation of the hyperbolic bases obtained in lines 10 and 11;

% $k := (\delta_1^{(q-1)/2})^{v_1^{-1}}\delta_2^{(q-1)/2}$;
$a := (s_1^2)^{v_1^{-1}}(s_2^2)$;

Find generators for the centraliser $D$ of $a$ in $G$;

In the derived subgroup $D'$ of $D$ find a generating set $X_3$ for 
the image in $\SX(4, q)$ of the direct factor
that acts faithfully on $\langle e_k,f_k,e_{k+1},f_{k+1}\rangle$;

In $\langle X_3\rangle$ find the permutation matrix 
$b=(e_k,e_{k+1})(f_k,f_{k+1})$;

$v := v_1 b v_2$;

return $(s_1,t_1,\delta_1,u_1,v)$ and the change-of-basis matrix;
}
\end{algorithm2e}

For $\Sp(d,q)$ and $\SU(d,q)$ the
eigenspaces of an involution $u$ are mutually orthogonal, and the form
restricted to either eigenspace is non-degenerate. Thus, if these
spaces have dimensions $e$ and $d-e$, then the derived subgroup of the
centraliser of $u$ in $\SX(d,q)$ is $\SX(e,q)\times\SX(d-e,q)$. Note
that the dimension of the $-1$-eigenspace of an involution in
$\SX(d,q)$ is always even.

Algorithm {\tt OneEven} addresses the case of even $d$.


If the type is SL, then the centraliser of $h$ is 
$\GL(E_+)\times\GL(E_-)\cap\SL(d,q)$ where $E_+$ and $E_-$ are the
eigenspaces  of $h$. If the type is Sp, it is
$\Sp(E_+)\times\Sp(E_-)$, and if the type is SU, it is
$(U(E_+)\times U(E_-))\cap\SU(d,q)$. In these last two
cases the restriction of the form to each of the eigenspaces is
non-singular, and each eigenspace is orthogonal to the other. Thus
the concatenation of a hyperbolic basis of one eigenspace with a
hyperbolic basis for the other eigenspace is a hyperbolic basis for
the whole space. Hence we need to {\it construct} a hyperbolic 
basis only for the base cases of the algorithm.

We make the following observations on Algorithm {\tt OneEven}. 
\begin{enumerate}
\item 
As presented the algorithm has been simplified. 
In lines 10 and 11 we have ignored
the change-of-basis matrices that are also returned; the change-of-basis 
returned at line 18 is defined by the concatenation of these bases.

\item 
The SLPs that express the standard generators 
in terms of $X$ are also returned.

\item 
Generators for the involution centralisers in lines 
8 and 14 are constructed using the algorithm of Bray \cite{Bray},
see Section \ref{Bray}. 
We need only a subgroup of the centraliser that 
contains its derived subgroup. 

\item 
The generators for the direct summands
constructed in lines 9 and 15 are constructed by forming suitable powers of
the generators of the centraliser. This step is discussed in
Section \ref{Pow}.

\item 
The algorithms for the {\tt BaseCase} call in line 3 are 
discussed in Section \ref{base}.
In summary {\tt BaseCase}($X$, {\it type}) returns
the standard generating set $Y$ and the associated SLPs
for the classical group $\langle X \rangle$ of the specified type. 
%{\tt BaseCase}($X$, {\it type, false}) returns $Y_0$.

\item 
The search for an element that powers to a suitable involution is
discussed in Section \ref{Involution}.

\item 
The recursive calls in lines 10 and 11 are
in smaller dimension. Not only are the groups of
smaller Lie rank, but the matrices have degree at most $2d/3$.
Hence these calls only affect the time or space complexity of the
algorithm up to a constant multiple; however they contribute to the length of
the SLPs produced.

\item 
It is easy to check that $a$ in line 13 is an involution: 
its $-1$-eigenspace is $\langle e_k,f_k,e_{k+1},f_{k+1}\rangle$ and its
$+1$-eigenspace is
$\langle e_1,f_1,\ldots, e_{k - 1}, f_{k-1}, e_{k+2},f_{k+2}, \ldots, e_d, f_d \rangle$.
We discuss how to find the ``glue" element $b$ in Section \ref{glue-element}.
\end{enumerate}
 
Algorithm {\tt OneOdd}, which considers 
the case of odd degree $d$, is similar
to Algorithm {\tt OneEven}.
Our commentary on the even degree case also applies. 

\begin{algorithm2e}[H]
\caption{\tt OneOdd$(X,{\it type})$}
\label{alg1:odd}
\tcc{
$X$ is a generating set for
the perfect classical group $G$
in odd characteristic and odd dimension, of type SL or SU.
Return the standard generating set for $G$,
the SLPs for elements of this generating set, 
and the change-of-basis matrix.
}

\Begin{
$d$ := the rank of the matrices in $X$;

if $d = 3$ then return {\tt BaseCase} (X, {\it type});

$q$ := the size of the field over which these matrices are defined;  

if {\it type} = SU then $q := q^{1/2}$;  

Find by random search $g \in G:=\langle X\rangle$ of even order
such that $g$ powers to a strong involution $h$;

Let $2k$ be the dimension of the $-1$-eigenspace of $h$;

Find generators for the centraliser $C$ of $h$ in $G$;

In the derived subgroup $C'$ of $C$ find generating 
sets $X_1$ and $X_2$ for the 
images in $\SX(2k, q)$ and $\SX(d - 2k, q)$ of 
the direct factors 
of $C'$, where $X_1$ centralises the $-1$-eigenspace of $h$;

$(s_1,t_1,\delta_1,u_1,v_1)$ := {\tt OneEven}$(X_1,{\it type})$;

$(s_2,t_2, \delta_2,u_2,v_2, x, y)$ := {\tt OneOdd}$(X_2,{\it type})$;

Let $(e_1, f_1, \ldots, e_k, f_k, e_{k+1}, f_{k+1}, \ldots, e_d, f_d)$
be the concatenation of the hyperbolic bases obtained in lines 10 and 11;

%$k := (\delta_1^{(q-1)/2})^{v_1^{-1}}\delta_2^{(q-1)/2}$;
$a := (s_1^2)^{v_1^{-1}}(s_2^2)$;

Find generators for the centraliser $D$ of $a$ in $G$;

In the derived subgroup $D'$ of $D$ find a generating set $X_3$ for 
the image in $\SX(4, q)$ of the direct factor
that acts faithfully on $\langle e_k,f_k,e_{k+1},f_{k+1}\rangle$;

In $\langle X_3 \rangle$ find the permutation matrix 
$b=(e_k,e_{k+1})(f_k,f_{k+1})$;

$v := v_1 b v_2$;

return $(s_1,t_1,\delta_1,u_1,v,x,y)$ and the change-of-basis matrix; 
}
\end{algorithm2e}

We summarise the main algorithm as Algorithm {\tt OneMain}. 

\begin{algorithm2e}[ht]
\caption{\tt OneMain$(X,{\it type})$}
\label{alg1:main}
\tcc{ $X$ is a generating set for the perfect classical group $G$
in odd characteristic, of type SL or Sp or SU.
Return the standard generators for $G$, the SLPs
for these generators, and change-of-basis matrix.}

\Begin{
 
$d$ := the rank of the matrices in $X$;

%if $d \leq 3$ then return {\tt BaseCase}(X, {\it type});

  \eIf{$d$ is odd}
   {
       $(s, t, \delta, u, v, x, y)$, $B$ := {\tt OneOdd}(X,{\it type});

       %return $(s,t, \delta,u,v, x)$ and the change-of-basis matrix;
       }
       {
       $(s,t,\delta, u, v)$, $B$ := {\tt OneEven}$(X,{\it type})$;

       Construct additional elements $x$ and $y$; 
       }


return $(s, t,\delta,u,v,x,y)$ and the change-of-basis matrix $B$;
}
\end{algorithm2e}

The correctness and complexity of this algorithm, 
and the lengths of the resulting SLPs for the 
standard generators, are discussed in the rest
of this paper.

\section{Algorithm {\tt Two}} 
\label{Alg2}

We present a variant of the algorithms in Section \ref{Alg1} based on  
one recursive call rather than two. Again we denote 
the groups $\SL(d,q)$, $\Sp(d,q)$
and $\SU(d,q)$ by $\SX(d,q)$, and the corresponding projective group
by $\PX(d,q)$.

The key idea is as follows. Suppose that $d$ is a multiple of 4.  
We find an involution $h \in \SX (d, q)$, as in line 6 of {\tt OneEven},
but insist that it should have both eigenspaces of dimension $d/2$. 

Let $\bar{h}$ be the image of $h$ in $\PX(d,q)$.
The centraliser of $\bar{h}$ in $\PX(d,q)$
acts on the pair of eigenspaces $E_+$ and $E_-$ of $h$, 
interchanging them. We construct the
projective centraliser of $h$ by applying the algorithm 
of \cite{Bray} to $\bar{h}$ and $\PX(d, q)$.

If we now find recursively the subset $Y_0$ of
standard generators for $\SX(E_+)$ with respect the basis $\cal B$,
then $Y_0^g$ is a set of standard generators for $\SX(E_-)$ with
respect to the basis ${\cal B}^g$. We now use these 
to construct standard generators for 
$\SX(d,q)$ exactly as in Algorithm {\tt One}.

If $d$ is an odd multiple of 2, we find an involution with one
eigenspace of dimension exactly 2. The centraliser of this
involution allows us to construct $\SX(2,q)$ and $\SX(d-2,q)$. The $d-2$
factor is now processed as above, since $d-2$ is a multiple of 4, and
the 2 and $d-2$ factors are combined as in the first algorithm. Thus
the algorithm deals with $\SX(d,q)$, for even  values of $d$, in a way
that is similar in outline to the familiar method of powering, that
computes $a^n$, by recursion on $n$, as $(a^2)^{n/2}$ for even $n$ and
as $a(a^{n-1})$ for odd $n$.

Algorithms {\tt TwoTimesFour} and {\tt TwoTwiceOdd} 
describe the case of even $d$. 
Algorithm {\tt TwoTimesFour} calls no new procedures except in line 5,
where we construct an involution with eigenspaces of equal dimension.
This construction is discussed in Section \ref{Equal}.
Algorithm {\tt TwoEven}, which summarises the even degree case,
returns the generating set $Y_0$ defined in Section \ref{standard}. 
We complete the construction of $Y$ exactly as in Section \ref{Alg1}.

If $d$ is odd, then we find an involution whose $-1$-eigenspace has
dimension 3, thus splitting $d$ as $(d-3)+3$. Since $d-3$ is even, we
apply the odd case {\it precisely once}.

The resulting {\tt TwoOdd} is the same as {\tt OneOdd},
except that it calls {\tt TwoEven} rather than {\tt OneEven};
similarly {\tt TwoMain} calls {\tt TwoOdd} and {\tt TwoEven}.

The primary advantage of the second algorithm 
lies in its one recursive call. 
As we show in Section \ref{SLP}, 
this reduces the lengths of the 
SLPs for the standard generators.

\begin{algorithm2e}[H]
\caption{\tt TwoTimesFour$(X,{\it type})$}
\label{alg2:even-b}
\tcc{$X$ is a generating set for
the perfect classical group $G$
in odd characteristic, of type SL or Sp or SU, in dimension a multiple of 4.
Return the standard generating set $Y_0$ for a copy 
of $\SL(2, q) \wr C_{d/2}$ if {\it type} is SL, otherwise 
$\SL(2, q) \wr S_{d/2}$ as subgroup of $G$, the 
SLPs for the elements of $Y_0$ and the change-of-basis matrix.
}

\Begin{

$d$ := the rank of the matrices in $X$;

%if $d = 4$ then return {\tt BaseCase} (X, {\it type, false});

$q$ := the size of the field over which these matrices are defined;  

if {\it type} = SU then $q := q^{1/2}$;  

Find by random search $g \in G:=\langle X\rangle$ of even order such that 
$g$ powers to an involution $h$ with eigenspaces of dimension $2k = d/2$;

Find generators for the projective centraliser $C$ of $h$ in $G$
and identify an element $g$ of $C$ that interchanges the two eigenspaces;

In the derived subgroup $C'$ of $C$ find a generating 
set $X_1$ for the image in $\SX(2k, q)$ one of the direct factors of $C'$;

$(s_1,t_1,\delta_1,u_1,v_1)$ := {\tt TwoEven}$(X_1,{\it type})$;

$X_2 := X_1^g$;

Conjugate all elements of $(s_1,t_1,\delta_1,u_1,v_1)$ by $g$ to 
obtain solution $(s_{2},t_2,\delta_2,u_2,v_2)$ for $X_2$;

Let $(e_1, f_1, \ldots, e_k, f_k, e_{k+1}, f_{k+1}, \ldots, e_d, f_d)$
be the concatenation of the hyperbolic basis obtained in line 9 and
its image under $g$;

% $k := (\delta_1^{(q-1)/2})^{v_1^{-1}}\delta_2^{(q-1)/2}$;
$a := (s_1^2)^{v_1^{-1}}(s_2^2)$;

Find generators for the centraliser $D$ of $a$ in $G$;

In the derived subgroup $D'$ of $D$ find a generating set $X_3$ 
for the image in $\SX(4, q)$ of the direct factor
that acts faithfully on $\langle e_k,f_k,e_{k+1},f_{k+1}\rangle$;

In $\langle X_3\rangle$ find the permutation matrix $b=(e_k,e_{k+1})(f_k,f_{k+1})$;

$v := v_1 b v_2$;

return $(s_1,t_1,\delta_1,u_1,v)$ and the change-of-basis matrix;
%return $(s_1,t_1,\delta_1,u_1,v)$, the change-of-basis matrix, and $X_3$.
}
\end{algorithm2e}

\begin{algorithm2e}[H]
\caption{\tt TwoTwiceOdd$(X,{\it type})$}
\label{alg2:even-a}
\tcc{ $X$ is a generating set for
the perfect classical group $G$
in odd characteristic, of type SL or Sp or SU, 
in dimension $d = 2(k + 1)$ for even $k \geq 0$.
Return the standard generating set $Y_0$ for a copy 
of $\SL(2, q) \wr C_{d/2}$ if {\it type} is SL, or $\SL(2, q) \wr S_{d/2}$ as subgroup
of $G$, the SLPs for the elements of $Y_0$ and the change-of-basis matrix.
}

\Begin{

$d$ := the rank of the matrices in $X$; 

if $d = 2$ then return {\tt BaseCase} (X, {\it type, false});

$q$ := the size of the field over which these matrices are defined;  

if {\it type} = SU then $q := q^{1/2}$;  

Find by random search $g \in G:=\langle X\rangle$ of even order
such that $g$ powers to an involution $h$ with eigenspaces of dimension 
2 and $d-2$.

Find generators for the centraliser $C$ of $h$ in $G$;

In the derived subgroup $C'$ of $C$ find generating sets 
$X_1$ and $X_2$ for the images in $\SX(d - 2, q)$ and $\SX(2, q)$ 
of the direct factors 
of $C'$ where $X_1$ centralises the eigenspace of dimension 2;

$(s_{1},t_1, \delta_1,u_1,v_1)$ := {\tt TwoTimesFour}$(X_1,{\it type})$;

$(s_{2},t_2, \delta_2,u_2,v_2)$ := {\tt TwoTwiceOdd}$(X_2,{\it type})$;

Let $(e_1, f_1, \ldots, e_k, f_k, e_{k + 1}, f_{k+1})$
be the concatenation of the hyperbolic bases obtained in lines  9 and 10;

%$k := (\delta_1^{(q-1)/2})^{v_1^{-1}}\delta_2^{(q-1)/2}$;
$a := (s_1^2)^{v_1^{-1}}(s_2^2)$;

Find generators for the centraliser $D$ of $a$ in $G$;

In the derived subgroup $D'$ of $D$ find a generating set $X_3$ 
for the image in $\SX(4, q)$ of the direct factor
that acts faithfully on $\langle e_{k},f_k, e_{k+1},f_{k+1}\rangle$;

In $\langle X_3\rangle$ find the permutation matrix $b=(e_k,e_{k+1})(f_k,f_{k+1})$;

$v := v_1 b v_2$;

return $(s_1,t_1,\delta_1,u_1,v)$ and the change-of-basis matrix.

}
\end{algorithm2e}

\begin{algorithm2e}[H]
\caption{\tt TwoEven$(X,{\it type})$}
\label{alg2-main:even}
\tcc{ $X$ is a generating set for
the perfect classical group $G$
in odd characteristic, of type SL or Sp or SU, in even dimension.
Return standard generating set $Y_0$ for a copy 
of $\SL(2, q) \wr S_{d/2} \leq G$, the 
SLPs for the elements of $Y_0$, and the change-of-basis matrix.
}

\Begin{
$d$ := the rank of the matrices in $X$;

\eIf {$d\bmod4=2$} 
  {
 
   return {\tt TwoTwiceOdd}$(X,{\it type})$;
  }{
    return {\tt TwoTimesFour}$(X,{\it type})$;
  }
}
\end{algorithm2e}


\section{Finding strong involutions}
\label{Involution}

In the first step of our main algorithms, as outlined in 
Sections \ref{Alg1} and \ref{Alg2},
by random search, we obtain an element of
even order that has as a power a strong involution. 
We now establish a lower bound
to the proportion of elements of $\SX(d,q)$ that power up to give
a strong involution. 
%We denote the natural module for $\SX(d,q)$ by $V$.

A matrix is {\it separable} if its characteristic polynomial
has no repeated factors.
Fulman, Neumann \& Praeger \cite{FNP} 
prove the following. 
\begin{theorem}\label{Lemma5.1}  
The probability that an 
element of $\GL(d,q)$ or $\U(d,q)$ is separable is 
at least $1-2/q$.  The probability that an 
element of $\Sp(d,q)$ is separable is 
at least $1-3/q + O(1/q^2)$.
\end{theorem}

We use these results to assist in our analysis of the 
proportion of strong involutions in $\SX(d, q)$.

\subsection{The special linear case}\label{sl}
We commence our analysis with $\SL(d, q)$.
We first estimate the probability that a random element of
$\GL(d,q)$ has a power that is an involution having an eigenspace
of dimension within a given range and then derive 
similar results for $\SL(d, q)$.

\begin{lemma}\label{monic} 
The number of irreducible monic polynomials of degree
$e>1$ with coefficients in $\GF(q)$ is $k$ where
$(q^e-1)/e>k\ge q^e(1-q^{-1})/e$.
\end{lemma}
\begin{proof}
Let $k$ denote the number of such polynomials.
We use the inclusion-exclusion principle to count the
number of elements of $\GF(q^e)$ that do not lie in any  maximal
subfield containing $\GF(q)$, and divide this number by $e$, since
every irreducible monic polynomial of degree $e$ over $\GF(q)$ corresponds
to exactly $e$ such elements.  Thus 
$$k = {q^e-\sum_iq^{e/p_i}+\sum_{i<j}q^{e/p_ip_j}-\cdots\over e}$$ 
where $p_1<p_2<\cdots$ are the distinct prime divisors of $e$. 
The inequality $(q^e-1)/e >k$ is obvious.  If $e$ is a prime,
then $k=(q^e-q)/e\ge q^e(1-1/q)/e$, with equality if $e=2$.
Now suppose that $e\ge4$, and let $\ell$ denote the largest prime
dividing $e$.  Then from the above formula 
\begin{eqnarray*}
ek & \ge & q^e-q^{e/\ell}-q^{(e/\ell)-1}-\ldots -1 \\
   & = &  q^e - (q^{1+e/\ell}-1)/(q-1)\\
   & \ge & q^e - q^{1+e/2}+1 \\
   & > & q^e - q^{e-1} \\
\end{eqnarray*}
as $1+e/2\le e-1$.
\end{proof}

\begin{lemma}\label{Lemma5.3} Let $e>d/2$ for $d \geq 4$. 
The proportion of elements of
$\GL(d,q)$ whose characteristic polynomial has an irreducible factor
of degree $e$ is $(1/e)(1+O(1/q))$. 
More precisely, there are
universal  constants $c_1$  and $c_2$ such that the proportion is always
between $(1/e)(1-c_2/q)$ and $(1/e)(1-c_1/q)$.
\end{lemma}
\begin{proof}
Let the characteristic polynomial of $g\in\GL(d,q)$ have an
irreducible factor $h(x)$ of degree $e$. Then $\{w\in V:w.h(g)=0\}$ is
a subspace of $V$ of dimension $e$. It follows that the number of
elements of $\GL(d,q)$ of the required type  is $k_1k_2k_3k_4k_5$
where $k_1$ is the number of subspaces of $V$ of dimension $e$, 
$k_2$ is the number of irreducible monic polynomials of degree $e$
over $\GF(q)$, $k_3$ is the number of elements of $\GL(e,q)$ that
have a given irreducible characteristic polynomial, $k_4$ is the
order of $\GL(d-e,q)$, and $k_5$ is the number of complements in
$V$ to a subspace of dimension $e$.
In more detail, 
\begin{eqnarray*}
k_1 & = & (q^d-1)(q^d-q)\cdots(q^d-q^{e-1})\over(q^e-1)(q^e-q) \cdots (q^e-q^{e-1}) \\
k_3 & = & (q^e-1)(q^e-q)\cdots(q^e-q^{e-1})\over (q^e-1) \\
k_4 & = & (q^{d-e}-1)(q^{d-e}-q) \cdots (q^{d-e}-q^{d-e-1}) \\
k_5 & = & q^{e(d-e)}.
\end{eqnarray*}
The formula for $k_3$ arises by taking
the index in $\GL(e,q)$ of the centraliser of an irreducible element,
this centraliser being cyclic of order $q^e-1$. 
The formula for $k_2$ is given in Lemma \ref{monic}. 
Hence $k_1 k_2 k_3 k_4 k_5 = \vert\GL(d,q)\vert\times k_2/(q^e-1)$. 
The result follows.  Note that $c_1$ and $c_2$ may be taken to be positive.
\end{proof}

\begin{lemma}\label{Lemma5.4} Let $e\in(d/3,d/2]$ for $d \geq 4$.
Let $S_1$ denote the number of elements of $G:=\GL(d,q)$ whose characteristic polynomial
has two distinct irreducible factors of degree $e$; let $S_2$ denote the number
of elements of $G$ whose characteristic polynomial has a repeated factor of
degree $e$; and let $S_3$ denote the number of elements of $G$ whose characteristic
polynomial has exactly one irreducible factor of degree $e$. 
Then $S_1={1\over2}\vert\GL(d,q)\vert e^{-2}(1+O(q^{-1}))$, and 
$S_2=\vert\GL(d,q)\vert O(q^{-1})$, 
and $S_3=\vert\GL(d,q)\vert(e^{-1}-e^{-2})(1+O(q^{-1}))$,
where the constants implied by the $O$ notation are absolute constants,
independent of $d$.
\end{lemma}
\begin{proof}  In the proof of Lemma \ref{Lemma5.3} the fact 
that $e>d/2$ was only used
to ensure that the characteristic polynomial of $g\in \GL(d,q)$ has 
only one irreducible factor of degree $e$.  
If the proof of Lemma \ref{Lemma5.3} is repeated to estimate $S_3$, the
factor $k_4$ must be replaced by the number of elements of $\GL(d-e,q)$ whose
characteristic polynomials do not have an irreducible factor of degree $e$. A
direct application of this lemma then shows that this number is the order
of $\GL(d-e,q)$ multiplied by a factor of the form $(1-e^{-1})(1+O(1/q))$.
Thus $S_3=\vert\GL(d,q)\vert(e^{-1}-e^{-2})(1+O(q^{-1}))$. 

The proportion of non-separable elements of $\GL(d,q)$ is
$O(q^{-1})$ by Theorem \ref{Lemma5.1}, so $S_2=\vert\GL(d,q)\vert O(q^{-1})$,
and we may ignore the condition that the irreducible factors in question are
distinct in our estimate of $S_1$.
 If the proof of
Lemma \ref{Lemma5.3} is repeated again to estimate $S_1$, the factor $k_4$ may
be replaced by the number of elements of $\GL(d-e,q)$ whose characteristic
polynomials have an irreducible factor of degree $e$, and the total
must be divided by 2, since the group elements in question normalise
two subspaces of dimension $e$.  
Thus $S_1={1\over2}\vert\GL(d,q)\vert e^{-2}(1+O(q^{-1}))$.  
\end{proof}

We will now show that the results of these lemmas hold if 
$\GL(d,q)$ is replaced by $\SL(d,q)$. First we need a preliminary
result. The norm map $N:\GF(q^e)\to\GF(q)$ defines a homomorphism from
$\GF(q^e)^\times$ onto $\GF(q)^\times$.  
Let $I_a$ denote the intersection of the pre-image of an arbitrary
$a \in \GF(q)^\times$ with the set of elements of $\GF(q^e)$
that lie in no proper subfield.  
Since in general the norm map will
not map a proper subfield of $\GF(q^e)$ onto $\GF(q)$, 
the size of $I_a$ varies with $a$.  
\begin{lemma}\label{norm}
$\vert I_a\vert=c_a{q^e-1 \over q-1}$, where $c_a=1+O(1/q)$.
\end{lemma}
\begin{proof}
The set of elements of $\GF(q^e)$ mapped to $a\in \GF (q)^\times$ 
under the norm map $x \mapsto x^{1+q+q^2 +\ldots+q^{e-1}}$ 
has size $1+q + \ldots + q^{e - 1} = (q^e-1)/(q - 1)$.
To obtain $I_a$, we must remove those elements 
which lie in a proper subfield containing $\GF(q)$. 
Following the proof of Lemma \ref{monic}, the number of elements 
of $\GF(q^e)^\times$ which
are in a proper subfield is less than $q^{1+e/2} - 1 < q^{e-2}$ for $e \geq 6$.
If $e$ is prime, the number of elements of $\GF(q)$ of norm $a$ is 
either 0 or $\gcd (q - 1, e)$.  If $e = 4$, the number of elements of 
$\GF(q^2)$ of norm $a$ is either 0 or $2(q + 1)$.
In all cases the result holds. 
\end{proof}

\begin{lemma}\label{Lemma5.5} The results of 
Lemmas $\ref{Lemma5.3}$ and $\ref{Lemma5.4}$ hold 
if $\GL(d,q)$ is replaced by $\SL(d,q)$. 
\end{lemma}
\begin{proof}
We first prove that the proportion quoted in 
Lemma \ref{Lemma5.3} is also true for 
$\SL(d,q)$.  In this case $k_4$ must be replaced by
the number of elements of $\GL(d-e,q)$ of a specified determinant.
But the number of such elements is exactly the number of elements of
$\GL(d-e,q)$ divided by $q-1$; so the result follows.  Similarly with
Lemma \ref{Lemma5.4}, if $e\ne d/2$ the proportions in the two cases are exactly
equal.  The point is that in these cases we consider the number of
elements in $\GL(d-e,q)$ or in $\GL(d-2e,q)$, and we need to replace
these numbers by the number of such elements having a given determinant, thus
reducing the number by a factor of $q-1$.
This also applies to $S_2$.  

However, this argument breaks down when
$e=d/2$, as in this case we cannot adjust the determinant of the
element being constructed by requiring an element of $\GL(d-2e,q)$
to have a given determinant.  
Indeed, in this case the proportions are 
not exactly equal for $\GL(d,q)$ and $\SL(d,q)$.

Consider the norm map $N:\GF(q^e)\to\GF(q)$.  
Let $I_a$ be the intersection of the pre-image of an arbitrary
$a \in \GF(q)^\times$ with the set of elements of $\GF(q^e)$
that lie in no proper subfield.  
Lemma \ref{norm} implies that 
$\vert I_a\vert=c_a (q^e-1) / (q-1)$, where $c_a=1+O(1/q)$.
Now $S_1$ is approximated, in the case of $\SL(d,q)$, by 
$${1\over2}\sum_a\vert I_a\vert\vert 
I_{a^{-1}}\vert\left({\vert\GL(e,q)\vert\over e-1} \right)^2$$
where the error in the approximation is due to the fact that we have 
ignored the condition that the irreducible factors of the characteristic 
polynomial should be
distinct.  The analogous estimate for $S_1$ in the case of $\GL(d,q)$ replaces
$\sum_a\vert I_a\vert \vert I_{a^{-1}}\vert$ by $\sum_{a,b}\vert I_a\vert\vert I_b\vert$,
which is approximately $q-1$ times as big, the error arising from the fact that
the cardinality of $I_a$ is not quite constant.  We have seen that this error 
corresponds to a factor of the form $1+O(1/q)$, as does ignoring the condition
that the irreducible factors in the characteristic polynomial should be 
different,
and omitting the contribution of $S_2$.  
Note that the assumption $d\ge 4$ enforces
the condition $e\ge 2$.
\end{proof}

We now obtain a lower bound for the proportion of 
$g\in\SL(d,q)$ such that $g$ has even order $2n$, and $g^n$
has an eigenspace with dimension in a given range. 
To perform this calculation, we consider the cyclic groups $C_{q^e-1}$ of order
$q^e-1$. If $n$ is an integer, we write $v_2(n)$ for the $2$-adic
value of $n$.

\begin{lemma}\label{Lemma5.6} If $v_2(m)=v_2(n)$ then $v_2(q^m-1) = v_2(q^n-1)$.
\end{lemma}
\begin{proof} 
 It suffices to consider the case where $m=kn$, and $k$ is odd.
Then $(q^m-1)/(q^n-1)$ is the sum of $k$ powers of $q^n$, and so is odd.
\end{proof}

\begin{lemma}\label{Lemma5.7} If $u<v$ then $v_2(q^{2^u}-1)<v_2(q^{2^v}-1)$, 
and if $u>0$ then $v_2(q^{2^u}-1) =v_2(q^{2^{u+1}}-1)-1$.
\end{lemma}
\begin{proof} 
Observe that $(q^{2^{u+1}}-1)/(q^{2^u}-1)=q^{2^u}+1$ which is even. Now
$v_2(q^{2^u}-1)>1$ if $u>0$. It then follows that $v_2(q^{2^u}+1)=1$.
\end{proof}

\begin{theorem}\label{Theorem5.1}  
For some absolute constant $c$, the proportion
of $g \in \SL(d,q)$ of even order, such that a power of $g$
is an involution with its $-1$-eigenspace of dimension in the range
$(d/3,2d/3]$, is at least $c/d$.
\end{theorem}
\begin{proof} 
Let $2^k$ be the unique power of $2$ in the range
$(d/3,2d/3]$. By Lemma \ref{Lemma5.5} it suffices to prove that 
if $g\in\SL(d,q)$
has an  irreducible factor of degree $2^k$ then the probability that
$g$ has the required property is bounded away from 0.

Let $\{W_i:i\in I\}$ be the set of composition factors of $V$
under the action of $\langle g\rangle$. Let $n_i$ be the order of the
image of $g$ in $\GL(W_i)$, and set $w_i=v_2(n_i)$, and
$w=\max_i(w_i)$, and $d_i=\dim(W_i)$. 
If $w>0$, then $g$ has even order $2n$  say,
and in  this case the $-1$-eigenspace of $z :=  g^n$ has
dimension $\sum d_i$, where the sum is over those values of $i$ for
which $w_i=w$. 

Suppose now that the characteristic polynomial of $g$
has exactly one irreducible factor of degree $2^k$. By renumbering if
necessary we may assume that $d_1=2^k$. Set $x=v_2(q^{2^k}-1)$.  The
probability that $w_1=x$ is slightly greater than $1/2$. This is
because the action of $g$ on $W_1$ embeds $g$ at random in
$\GF(q^{2^k})$, which is a cyclic group of order an odd multiple of
$2^x$. The distribution of possible values of $g$ is uniform among
those elements that do not lie in a proper subfield of $\GF(q^{2^k})$.
But non-zero elements of such subfields do not have order a multiple
of $2^x$. If $w_1=x$ then necessarily $w_i<w_1$ for all $i>1$,
and $w_1=w$. It follows that $g$ will then have even order, and that
$z$ will be an involution whose $-1$-eigenspace will have
dimension exactly $2^k$. 
There is a slight problem with the elements of $\SL(d,q)$
whose characteristic polynomials have two irreducible factors of
degree $2^k$, as such elements may power up to an involution
whose  $-1$-eigenspace has dimension $2^{k+1}$, but the estimate of $S_1$
in Lemma \ref{Lemma5.4} shows that this problem does 
not affect the correctness of the theorem.
\end{proof}

\begin{corollary}\label{Corollary5.1} Such an element $g$ in $\SL(d,q)$ 
can be found with at most $O(d(\xi + d^3\log q))$ field operations,
where $\xi$ is the cost of constructing a random element.
\end{corollary}
\begin{proof}
Theorem \ref{Theorem5.1} implies that a search of length $O(d)$ 
will find such an element $g$. 
%Theorem \ref{Lemma5.1} implies that we can
%afford to discard elements whose characteristic   
%polynomials have repeated roots. 
In $O(d^3)$ field operations 
the characteristic polynomial $f(t)$ of $g$ can be
computed (see \cite[Section 7.2]{HoltEickOBrien05}); 
in $O(d^2 \log q)$ field operations it can be factorised as 
$f(t)=\prod_{i=1}^kf_i(t)$, where the $f_i(t)$ are irreducible 
(see \cite[Theorem 14.14]{vzg}).

Following the notation of the proof of Theorem \ref{Theorem5.1},  
we may take $W_i$ to be the
kernel of $f_i(g)$. It remains to calculate $w_i$. Let $m_i$ be the
odd part of $q^{d_i}-1$, where $d_i$ is the degree of $f_i$. Now
compute $s :=(f_i(t))+t^{m_i}$ in $\GF(q)[t]/(f_i(t))$, and iterate 
$s := s^2$  until $s$ is the identity. The number of iterations
determines $w_i$, and it is now easy to determine whether or not $g$
satisfies the required conditions. All of the above steps may be carried
out in at most $O(d^3 \log q)$ field operations. 
\end{proof}


\subsection{The symplectic and unitary groups}
We first consider the symplectic groups.
If $h(x)\in\GF(q)[x]$ is a
monic polynomial with non-zero constant term, let
$\tilde{h}(x)\in\GF(q)[x]$ be the monic polynomial  whose zeros are the
inverses of the zeros of $h(x)$. Hence the multiplicity of a zero of
$h(x)$ is the multiplicity of its inverse in $\tilde{h}(x)$, and 
$h(x)\tilde{h}(x)$ is a symmetric  polynomial.  We start with this
analogue of Lemma \ref{Lemma5.3}.

\begin{lemma}\label{Lemma5.8} Let $m>n/2$ where $n \geq 2$. 
The proportion of elements of
$\Sp(2n,q)$ whose characteristic polynomial has a factor $h(x)$ where
$h(x)$ is irreducible of degree $m$ and $h(x)\ne \tilde{h}(x)$ is
$(1/2m)(1+O(1/q))$, where the constants implied by the $O$ notation 
are absolute constants, independent of $n$.
\end{lemma}
\begin{proof}
Let $g\in\Sp(2n,q)$ act on the natural module $V$, and let
$h(x)$ be an irreducible factor of degree $m$
of the characteristic polynomial
$f(x)$ of $g$. 
Let $V_0$ be the kernel of $h(g)$. Since    
$h(x)\ne \tilde{h}(x)$ it follows that $V_0$ is totally isotropic. Also
$\tilde{h}(x)$ is a factor of $f(x)$, and if  $V_1$ is the kernel of
$\tilde{h}(x)$ then $V_1$ is totally isotropic. Since $h(x)$ and
$\tilde{h}(x)$ divide $f(x)$ with multiplicity 1, $V_0$ and $V_1$ are
uniquely determined, and the form restricted to $V_0\oplus V_1$ is
non-singular. Now let $e_1,\ldots,e_m$ be a basis for $V_0$. A basis
$f_1,\ldots,f_m$ for $V_1$ is then determined by the conditions
$B(e_i,f_j)=0$ for $i \ne j$, and $B(e_i,f_i)=1$ for all $i$, where
$B(-,-)$ is the symplectic form that is preserved. The matrix for $g$
restricted to $V_0$ now determines the matrix of $g$ restricted to
$V_1$, since $g$ preserves the form. 

Thus the number of possibilities
for $g$ is the product $k_1k_2k_3k_4k_5/2$, where $k_1$ is
the number of choices for $V_0$, and $k_2$ is the number of choices
for $V_1$ given $V_0$, and $k_3$ is the number of irreducible monic 
polynomials $h(x)$ of degree $m$ over $\GF(q)$ such that $h(x)\ne
\tilde{h}(x)$, and $k_4$ is the number of elements of $\GL(m,q)$ with a
given irreducible characteristic polynomial, and $k_5$ is the order of
$\Sp(2n-2m,q)$.   The factor $1/2$ in the above expression arises from
the fact that every such element $g$ is counted twice, because of the
symmetry between $h(x)$ and $\tilde h(x)$. 
In more detail 
\begin{eqnarray*}
k_1 & = & {(q^{2n}-1)(q^{2n-1}-q)(q^{2n-2}-q^2)\cdots(q^{2n-m+1}-q^{m-1})\over
(q^m-1)(q^m-q)(q^m-q^2)\cdots(q^m-q^{m-1})} \\                   
k_2 & = & q^{(2n-m)+(2n-m-1)+(2n-m-2)+\cdots+(2n-2m+1)} \\
k_3 & \sim  & q^m/m \\
k_4 & = & {(q^m-1)(q^m-q)(q^m-q^2)\cdots(q^m-q^{m-1})\over q^m-1}\\
k_5 & = & q^{(n-m)^2} \prod_{i = 1}^{n - m}(q^{2i}-1). 
\end{eqnarray*}

These results are obtained as follows. For $k_1$, we count the number of
sequences of linearly  independent  elements $(e_1,e_2,\ldots)$  such
that each is orthogonal to its predecessors, and divide by the order
of $\GL(m,q)$. For $k_2$, we observe that there is a 1-1
correspondence between the set of candidate subspaces for $V_1$ and
the set of sequences $(f_1,f_2,\ldots,f_m)$ of elements of $V$ such that
each $f_j$ satisfies $m$ linearly independent conditions
$B(e_i,f_j)=0$ for $i \ne j$, and 
$B(e_j,f_j)=1$, and $B(f_k,f_j)=0$ for $k < j$.
We observe that $k_3$ is the number of orbits of the Galois group of
$\GF(q^m)$ over $\GF(q)$ acting on those $a \in \GF(q^m)$ that 
do not lie
in a proper subfield containing $\GF(q)$, and have the property that 
the orbit of $a$ does not contain $a^{-1}$.  This last condition is 
equivalent to the statement that $h(x)\ne \tilde{h}(x)$.   
Note that $h(x)=\tilde{h}(x)$ if and only if
$m$ is even, and $a^{-1}=a^{q^{m/2}}$.
A precise formula for $k_3$
would be rather complex, so we obtain instead the following
estimate. If we ignore this last condition, then
Lemma \ref{monic} estimates $k_3$.
Now it is clear that if $a\in\GF(q^m)$ satisfies the
above equation then the norm of $a$ is 1. In other words, the
constant term of $h(x)$ is $1$. But this is exactly the problem solved in
the proof of Lemma \ref{Lemma5.5}, so we find that, for some
absolute constants $c_1$ and $c_2$, $k_3$ lies between $(1-c_2 /q)q^m/m$ and
$(1-c_1 /q)q^m/m$.  The product of the $k_i$ is
$k_3\vert\Sp(2n,q)\vert/(q^m-1)$ and the result follows. 
\end{proof}

\begin{lemma}\label{Lemma5.9} Let $m\in(n/3,n/2]$. 
Let $S_1$ denote the number of elements of $G:=\Sp(2n,q)$ whose characteristic 
polynomial
has four distinct irreducible factors of degree $m$, of the 
form $h(x)$, $\tilde h(x)$,
$k(x)$, and $\tilde k(x)$;
 let $S_2$ denote the number
of elements of $G$ whose characteristic polynomial has two distinct 
repeated factors of degree $m$, of the form $h(x)$ and $\tilde h(x)$; and 
let $S_3$ denote the number 
of elements of $G$ whose characteristic
polynomial has exactly two distinct irreducible factors of degree $m$,
of the form $h(x)$ and $\tilde h(x)$. 
Then $S_1={1\over8}\vert\Sp(2n,q)\vert m^{-2}(1+O(q^{-1}))$, and 
$S_2=\vert\Sp(2n,q)\vert O(q^{-1})$, 
and $S_3={1\over2}\vert\Sp(2n,q)\vert(m^{-1}-{1\over2}m^{-2})(1+O(q^{-1}))$,
where the constants implied by the $O$ notation are absolute constants,
independent of $n$.
\end{lemma}
\begin{proof} 
The proof is similar to that of Lemma \ref{Lemma5.4}.  Care has to
be taken with counting the number of times that elements of the analogue
of $S_1$ are counted.  The characteristic polynomial of such an element
$g$ now has four distinct irreducible factors $h(x)$, $\tilde h(x)$, $k(x)$,
and $\tilde k(x)$ of degree $m$.  This leads to such elements being counted 
eight times.
\end{proof}

\noindent
We now obtain the analogue of Theorem \ref{Theorem5.1}.
\begin{theorem}\label{Theorem5.2}  
For some absolute constant $c>0$, the proportion
of $g \in \Sp(2n,q)$ of even order, such that a power of $g$
is an involution with its $-1$-eigenspace of dimension in the range
$(2n/3,4n/3]$, is at least $c/n$.
\end{theorem}
\begin{proof} 
Given Lemmas \ref{Lemma5.8} and \ref{Lemma5.9}, the proof is 
essentially the same as that of Theorem \ref{Theorem5.1}.   We
adopt the notation of that proof.  One
must consider the contribution of $W_i$ to the eigenspaces of $z$
when the characteristic polynomial of $g$ restricted to $W_i$ is
an irreducible polynomial $h(x)$ such that $h(x)=\tilde h(x)$.  But
in this case if $\alpha$ is a zero of $h(x)$ then so is $\alpha^{-1}$,
and $\alpha^{q^m}=\alpha^{-1}$, where $W_i$ has dimension $2m$, and the 
order of $g$
divides $q^m+1$.  It is easy to see that if $i$ is even then $v_2(q^i+1)=1$,
and if $i$ is odd then $v_2(q^i+1)=v_2(q+1)$.  Thus $W_i$ will
contribute nothing to the dimension of the $-1$-eigenspace of $z$.
The result follows.
\end{proof}

We finally turn to the unitary groups. 
\begin{theorem}\label{Theorem5.3}  For some absolute constant $c>0$, 
the proportion of $g\in \SU(d,q)$ that have even order, 
such that a power of $g$
is an involution with its $-1$-eigenspace of dimension in the range
$(d/3,2d/3]$, is at least $c/d $.
\end{theorem}
\begin{proof} 
The analysis in this case is almost
exactly the same as for the symplectic groups.  The only difference
comes from the analysis of the restriction of $g$ to $W_i$ where
now we require $h(x)$ to be the image of $\tilde h(x)$ under the Frobenius
map $a\mapsto a^q$.  This now requires $W_i$ to have odd dimension
$2t+1$, say, and then the order of $g$ will divide $q^{2t+1} +1$.
\end{proof}

In summary, Theorems \ref{Theorem5.1}, \ref{Theorem5.2} and \ref{Theorem5.3} 
provide an estimate of the
complexity of finding a strong involution of the type required 
as $O(d(\xi + d^3 \log q))$ field operations. 

In Algorithm {\tt TwoMain}, we search in $\SX(d, q)$
for an element that powers to an involution
with eigenspaces of dimension 2 and $d - 2$ if $d$ 
is twice odd, and $3$ and dimension $d - 3$ if $d$ is odd.
Lemma \ref{Lemma5.3} and its analogues for the classical
groups show that we can find such elements  in $O(d)$ trials. 

\section{Involutions with eigenspaces of equal dimension}\label{Equal}
Our next objective is to describe and analyse an algorithm 
to construct an involution in
$\SX(d,q)$ with eigenspaces of equal dimension. This necessarily
presupposes that $d$ is a multiple of $4$. 
We use such an element in Algorithm {\tt TwoEven}. 

We describe a recursive procedure to construct an 
involution in $G=\SL(d, q)$ whose $-1$-eigenspace has a specified 
even dimension $e$. 

\begin{enumerate}
\item 
Search randomly for $g \in G$ of even order
that powers to an involution $h_1$
satisfying the conditions of Theorem \ref{Theorem5.1}.

\item Let $r$ and $s$ denote the ranks of the 
$-1$- and $+1$-eigenspaces of $h_1$.

\item If $r = e$ then $h_1$ is the desired involution.
 
\item 
Consider the case where $s \leq e < r$.
Construct the centraliser in $G$ of  $h_1$ and 
obtain generators for the special linear group
$S_-$ on the $-1$-space, where $S_-$ acts as the identity on the
$+1$-eigenspace of $h_1$. 
By recursion on $d$, an involution 
can be found in $S_-$ whose $-1$-eigenspace    
has dimension $e$. 

\item 
Consider the case where $e \leq \min(r, s)$.
If $r \leq s$ then 
construct the centraliser in $G$ of  $h_1$; 
using the method of Section \ref{Pow}, obtain 
generators for the special linear group
$S_-$ on this $-1$-eigenspace,
where $S_-$ acts as the identity on the $+1$-eigenspace.
By recursion on $d$, an involution 
can be found in $S_-$ whose $-1$-eigenspace    
has dimension $e$. 
Similarly, if $s < r$ then construct $S_+$,
and search in $S_+$ for an involution
whose $-1$-eigenspace has dimension $e$. 

\item 
Consider the cases where $s \geq e > r$ or $e \geq \max(r, s)$.
Construct the centraliser in $G$ of $h_1$, and obtain generators
for the special linear group $S_+$ on the $+1$-eigenspace of $h_1$,
where $S_+$ acts as the identity on the $-1$-eigenspace.
Now an involution $h_2$ is found recursively in
$S_+$ whose $-1$-eigenspace has dimension $e-r$. 
Then $h_1h_2$ is an involution of the required type. 

\end{enumerate}
The recursion is founded trivially with the case $d=4$.

\begin{theorem}\label{Corollary5.2}  
Using this algorithm, an involution in $\SL(d,q)$
can be constructed with $O(d(\xi + d^3 \log q))$ field operations that has its
$-1$-eigenspace of any even dimension in $[0,d]$.
\end{theorem}
\begin{proof}
Corollary \ref{Corollary5.1} implies that $h_1$ can be constructed with at most 
$O(d(\xi + d^3 \log q))$ field operations. We shall see in 
Sections \ref{Pow} and \ref{Bray} 
that generators for $S_-$ and $S_+$ can be constructed
in at most $O(d \xi + d^3 \log q)$ field operations. Thus the above algorithm requires
$O(d(\xi + d^3 \log q))$ field operations, plus the number of field operations
required in the recursive call. Since the dimension of the matrices
in a recursive call is at most $2d/3$, the total
complexity is as stated.
\end{proof}

\noindent
Similar results can be obtained for the other classical groups.

\section{Exponentiation}
\label{Exp}

A frequent step in our algorithms is computing the 
power $g^n$ for some $g\in \GL(d,q)$ and integer $n$.
Sometimes we raise $g$ to a high power in order to construct an
involution, and we may write down
this involution without performing the calculation. However, if, for example, 
we want to construct elements of one direct factor of a direct product 
of two groups by exponentiation, then we must explicitly
compute the required power. 

The value of $n$ may be as large as $O(q^d)$. We
could construct $g^n$ with $O(\log(n))$ multiplications using the familiar
black-box squaring technique. 
Instead, we describe the following faster algorithm to perform this task.
\begin{enumerate}
\item 
Construct the Frobenius normal form of $g$ and record
the change-of-basis matrix. 

\item 
 From the Frobenius normal form, 
we read off the minimal polynomial
$h(x)$ of $g$, and factorise $h(x)$ 
as a product of irreducible polynomials.

\item 
This form determines a multiplicative upper bound
to the order of $g$. 
If $\{f_i(x):i\in I\}$ is the set of distinct
irreducible factors of $h(x)$, and if $d_i$ is the degree of
$f_i(x)$, then the order of the semi-simple part of $g$ divides
$\prod_iq^{d_i}-1$, and the order of the idempotent part of $g$ can be
read off directly. The product of these two factors  gives the
required upper bound $m$. 

\item If $n>m$ we replace $n$ by $n\bmod m$. 
By repeated squaring we calculate $x^n\bmod h(x)$ 
as a polynomial of degree $d$. 

\item This polynomial is evaluated in $g$ to give $g^n$. 

\item Conjugate $g^n$ by the inverse of the change-of-basis 
matrix to return to the original basis.
\end{enumerate}

We now consider the complexity of this algorithm.
\begin{lemma}\label{compute-power}
Let $g\in\GL(d,q)$ and let $0\le n<q^d$. Then
$g^n$ can be computed using the above algorithm
with $O(d^3 + d^2 \log d \log \log d \log q)$ field operations.
\end{lemma}
\begin{proof}
The Frobenius normal form of $g$ can be computed with
$O(d^3)$ field operations \cite{Storjohann98}
and provides the minimal polynomial.
The minimal polynomial can be factored in 
$O(d^2 \log q)$ field operations \cite[Theorem 14.14]{vzg}.
Calculating $x^n \bmod h(x)$ requires $O(\log(n))$
multiplications in $\GF(q)[x]/(h(x))$, 
at most $O(d^2 \log d \log \log d \log q)$ 
field operations \cite{vzg}. Evaluating the resultant polynomial in $g$ requires
$O(d)$ matrix multiplications;  but multiplying by   
$g$ only costs $O(d^2)$ field operations, since $g$ is sparse 
in Frobenius normal form. Finally, conjugating $g$ by the inverse of 
the change-of-basis matrix costs a further $O(d^3)$ field operations.
\end{proof}

One should consider the cost of
dividing $m$ by $n$, even though this does not 
contribute to the number of field operations. 
However, for our applications, the exponent $n$ 
is always less than $q^d$, so
reducing $m$ modulo $n$ is unnecessary.

There is no need to prefer one normal form for $g$ to another,
provided that the normal form can be computed in at most $O(d^3)$ field
operations, the form is sparse, and the minimum polynomial
and multiplicative upper bound for the order of $g$
can be determined readily from the normal form.

This algorithm is similar to 
that of \cite{CLG97} to determine 
the order of an element of $\GL(d, q)$.

\section{Derived subgroups and direct products} 
\label{Pow}
In this section, we solve two problems which have
closely related solutions. 
\begin{enumerate}
\item 
Given a generating set for $H$ where $\SX(e,q)\le H\le \GX(e,q)$,
construct a generating set for the derived subgroup of $H$.
\item 
Given a generating set of $\SX(e,q)\times\SX(d-e,q)$,
construct a generating set for one of the direct factors.
\end{enumerate}
Our solutions and their analysis rely on the work
of Niemeyer \& Praeger \cite{NP,JAMS}.
We assume throughout that the characteristic is odd.

\subsection{Constructing the derived group}
The one-sided Monte Carlo recognition algorithm of \cite{NP} 
takes as input a subset $Y$ of $\GX(d,q)$ and 
endeavours to prove that $G:=\langle Y\rangle$ contains $\SX(d,q)$, given that
$G$ is an irreducible subgroup of $\GX(d,q)$ that does not
preserve any bilinear or quadratic form not preserved by $\GX(d,q)$.

We say that $P$ is a set of {\it test primes} if, whenever $S \subseteq G$
and $S$ has an element of order a multiple of $p$ for all $p \in P$, 
then $\langle S \rangle$ either contains $\SX(d,q)$, or is reducible, 
or preserves a form not preserved by $\SX(d,q)$.  
Now $S$ is a set of {\it test elements} for $G$ if 
for every $p \in P$, there is $g \in S$ of order a multiple of $p$.
To find a suitable set $S$ of test elements, the expected 
number of random elements to be examined is at most 
$O(\log\log d)$; see \cite[Proposition 7.5]{NP}.

A {\it primitive prime divisor} of $q^e-1$ is a prime divisor of
$q^e-1$ that does not divide $q^i-1$ for any positive integer $i<e$.  
If $r$ is a primitive prime divisor of $q^e-1$ then $r\equiv 1\bmod e$, 
and so $r\ge e+1$.  Further, $r$ is a {\it large} primitive prime divisor 
of $q^e-1$ if $r$ is a primitive prime divisor of $q^e-1$, and either 
$r>e+1$ or $r^2$ divides $q^e-1$.  Finally, $r$ is defined to be a {\it basic}
primitive prime divisor of $q^e-1$ if $r$ is a primitive prime divisor 
of $p^{ae}-1$ where $p$ is prime and $q=p^a$.  
A {\it ppd}$(d,q;e)$ element of $G$ is one whose order is 
a multiple of a primitive prime divisor of $q^e-1$. 

Omitting the orthogonal groups, Theorem 5.7 of \cite{NP} proves the following.
\begin{theorem}
Let $\SX(d,q)\le G\le\GX(d,q)$.
The proportion ppd$(G,e)$ of ppd$(d,q;e)$ elements
of $G$ satisfies $1/(e+1)\le {\rm ppd}(G,e)\le 1/e$, 
except for odd $e$ in the symplectic case 
and even $e$ in the unitary case when it is 0.
\end{theorem}

Consider first the general case where $G$ is a {\it generic subgroup} 
of $\GL(d, q)$ \cite[Definition 3.2]{NP}.
Now, in all but one exceptional case, $P$ has the property that 
each prime in $P$ does not divide $q - 1$
when $G = \SL(d, q)$ and does not divide $q^2 - 1$ when $G = \SU(d, q)$.
Hence the elements of $S$ {\it remain} test elements when raised to the 
power $n$, 
where $n=q-1$ in the case of $\GL(d,q)$, and is $q+1$ in 
the case of $\U(d,q)$.
(Recall that $\Sp(d, q)$ is perfect.)
Thus the $n$-th powers of the test elements also generate either 
$\SX(d,q)$, or a reducible group, or a group that preserves some 
form not preserved by $\GX(d,q)$.  

In more detail, the set $S$ of test elements for $G$ contains two 
elements, $g_1$ and $g_2$: 
one is a basic and the other a large 
ppd$(d,q;e_i)$ elements where $e_1\ne e_2$,
and $e_i>d/2$ for $i=1,2$.  Hence our putative generating set for $G'$ 
contains two elements $h_1$ and $h_2$, powers of $g_1$ and $g_2$, that are
ppd$(d,q;e_i)$ elements. Thus they act irreducibly on subspaces $W_1$ 
and $W_2$ of $V$ of dimensions $e_1$ and $e_2$ respectively.  
It is now easy to deduce that the probability that 
$\langle h_1, h_2 \rangle$ acts irreducibly on the underlying space
$V$ is at least $\prod_{i=1}^{\infty} (1 - 1/2^i) > 0.28$.
Hence in general the subspaces will span $V$; if not, 
then $h_1$ may be replaced by a suitable $G$-conjugate of $h_1$.
Thus we may assume that $H$ acts irreducibly on $V$.  

It remains to decide whether $H$ preserves some form not preserved 
by $\SX(d,q)$.  
To prove that $g\in \SL(d,q)$ does not preserve a non-degenerate symplectic or
symmetric bilinear form, it suffices to prove that the order of $g$ is a multiple
of a primitive prime divisor of $q^e-1$ for some odd $e>d/2$.  Since $d>2$ such an
$e$ exists, and as $e$ is odd such a primitive prime divisor exists.  
Theorem 5.7 of \cite{NP} proves that 
the proportion of such elements is at least $1/(e+1)$; and so the
proportion quantifying over all odd $e>d/2$ is at least $1/6$. 
%since if $f(d)=\sum
%1/(e+1)$, the sum being taken over all $e$ in the range $(d/2,d]$, then $f(d)\ge 1/6$.
%In fact $f(6)=1/6$, and $\lim_{d\to \infty}f(d)={1\over2}\log_e2$.  (COMPUTER CHECK).
Thus we expect to find such an element in a constant number of trials.

To prove that $g\in\SL(d,q)$ does not preserve a hermitian form, where $q=q_0^2$, it
suffices to prove that the order of $g$ is a multiple of a primitive prime divisor of
$q_0^e-1$ for some odd $e>d/2$.  Note that a necessary condition for $g\in\SL(d,q)$ to
be of order a multiple of a primitive prime divisor of $q_0^e-1$ or of $q^e-1$ is that
$g$ should have an irreducible factor of degree $e$ in its characteristic polynomial,
since $e$ is odd, and thus define a conjugacy class of $e$ elements of $\GF(q^e)^\times$.
We have seen in the proof of Lemma \ref{Lemma5.3} that if $g$ is 
uniformly distributed in $\SL(d,q)$
then the conjugacy classes of $\GF(q^e)^\times$ that arise are uniformly distributed
amongst those classes not contained in $\GF(q^k)^\times$ for any proper subfield
$\GF(q^k)$ of $\GF(q^e)$.  As we are concerned with elements of order a multiple of a
primitive prime divisor of $q_0^e-1$, the conjugacy class of $\GF(q^e)^\times$ in
question cannot fall into any proper subfield of $\GF(q^e)$ containing $\GF(q)$.  
Since the proportion of elements of $\GF(q^e)^\times$ whose orders are a 
multiple of a given primitive
prime divisor $\ell$ of $q_0^e-1$ is $1-1/\ell$, and the same holds for primitive prime
divisors of $q^e-1$, the ratio of the number of elements of $\SL(d,q)$ whose orders are
a multiple of a primitive prime divisor of $q_0^e-1$ to those whose orders are a multiple
of a primitive prime divisor of $q^e-1$ is greater than $2/3$.  
%This limit would be
%approached if 3 were the only primitive prime divisor of $q_0^e-1$, but $q^e-1$ had a 
%large primitive prime divisor.  
It follows again that examining a constant
number of elements of $\SL(d,q)$ will find an appropriate ppd-element.

%NOTE: if  $r$ is ppd $(q, e)$ then $q^e = 1 \bmod r$ 
%So q has order e mod r 
%so e is an element of cyclic group of r.
%$r\equiv 1\bmod e$, 
%Need to consider last paragraph in light of this
%Can we get down to 3?.
%Thus we can ensure that the 
%putative generators of $\SX(d,q)$ generate an irreducible subgroup 
%of $\GL(d,q)$ that preserves no non-degenerate bilinear form. 

Consider now the exceptional generic case.
In \cite[Case 1, p.\ 159]{NP}, the authors discuss how
to distinguish subgroups of $Z\times \PGL(2,7)$ from 
$G = \SL(3,q)$ when $q=3.2^s-1$ for some $s\ge2$.  This they do by observing that
$\SL(3,q)$ contains many elements of order a multiple of 8, and 
$Z\times\PGL(2,7)$ contains none.  
In this case $G$ has a reasonable proportion of 
elements $g$ of order a multiple of 16, and, since $q - 1$
is an odd multiple of 2, $g^{q - 1}$ has order a multiple of 8.
This fails when $s = 2$ and $q = 11$.
But $1/40$ of the elements of $\GL(3, 11)$ have order a 
multiple of 8 and determinant 1, and so we search for such 
elements and do not power them.

If $G$ is not generic, then 
the set of test elements is more elaborate.
For example, in \cite[Table 8, p.\ 248]{JAMS},
the set of test elements for $\SU(3,3)$ and $\SU(3,5)$ include one
of even order, and elements of the type required cannot be obtained from 
the corresponding general unitary groups by raising to the power 
4 (in the first case) or 6 (in the second).  
%These appear to be the only problem cases in this limited context.
These cases can be treated individually; alternatively, since $d \leq 6$, we
can use the black-box algorithm of \cite[Theorem 2.4.8]{Seress03}
to construct the derived group in time $O(\log^2 q)$.

Niemeyer \& Praeger \cite{NP} exclude the case $d=2$.  In this case, we can find by 
random search an element that powers up to an element $g_1$ of determinant 1
and order $q+1$, take a conjugate $g_2$ of this element that does not
commute with $g_1$.  With high probability, these elements will generate $\SL(2,q)$, 
and can be found by considering at most $O(\log \log q)$ random elements.
If $q=5$, we take an additional element of order 5 to exclude the 
possibility that this pair of elements generates $2.A_4$.

We now consider the overall cost of the algorithm to construct the derived group.
\begin {theorem}
Let $G$ be a generic subgroup of  $\GX(d,q)$. 
A generating set for the derived group of $G$ can be constructed 
in Las Vegas time $O(\xi \log \log d + d^3 \log^2 d \log q)$ field operations.
\end{theorem}
\begin{proof}
The algorithm of \cite{NP} 
has running time approximately $O(d^3\log^2 d\log q)$
field operations. For a more precise statement of its complexity, see \cite{Bath}. 

We need up to four exponentiations, the exponent being $q-1$ or 
$q+1$; its cost is estimated in Theorem \ref{compute-power}.  

At a cost of $O(d^3)$ field operations 
\cite[Section 7.2]{HoltEickOBrien05}, we compute a constant 
number of characteristic polynomials to exclude the possibility that 
we have constructed a group that preserves a form. 

We must also construct the spaces $W_1$ and $W_2$
fixed by two elements, $h_1$ and $h_2$. 
Let $h_1$ have characteristic polynomial $f(x)$.  Then 
$f(x)=u(x)w(x)$, where $u$ is irreducible of degree $e$ for some 
$e>d/2$, and $W_1$ is the image of $w(g)$.  Let $w(x) = 
a_0+a_1x+\cdots+x^{d-e}$, and compute $y := v.w(g)$ in $O(d^3)$ field 
operations.  Then $y\in W_1$,
and the probability that $y$ is zero is $1:q^{d-e}$.  If $y$ is 
non-zero, we ``spin" $y$ under $h_1$
to obtain a basis for $W_1$.
The total Las Vegas time for computing $W_1$ is $O(d^3)$ \cite{HoltRees94}.  
The same applies to computing $W_2$. To determine
whether or not $W_1+W_2=V$ costs $O(d^3)$ field operations.  
\end{proof}

\subsection{Decomposing a direct product}
Assume we have a generating set for $G=\SX(e,q)\times\SX(d-e,q)$, 
and wish to construct generating sets for the direct factors.
We assume that we have performed a base change
on $G$ and so can readily read off the projection of an element
of $G$ onto each factor.

We use essentially the same algorithm as that for constructing the derived 
group of $\GX(d,q)$. 
Namely, we construct an element of $\SX(e,q)$ by taking a random 
element $(g_1,g_2)$ 
of $G$, and raise this element to the power $n$, where 
now $n$ is the order of $g_2$.  In general, a test element has an order that is a 
multiple of some prime, and we need to assess the probability that the 
order of $g_2$ will not be a multiple of this prime.

It follows from Theorem 5.7 of \cite{NP} 
that the probability of the relevant ppd-property of $g_2$ being 
destroyed by powering is less than $2/d$.  This remains the case if 
we raise $(g_1,g_2)$ to the power given by the {\it pseudo-order} of $g_2$,
thus avoiding problems with integer factorisation. (See Section \ref{Bray}
for definition.)

In the non-generic case, the value of $d$ is at most 6,
and so we can use the following result of \cite{BPS} in these cases.
\begin{theorem}\label{bps}
Let $C$ be a finite simple classical group,
with natural module of dimension $d$.
For a prime $p$, the proportion of $p$-regular 
elements of $C$ is greater than $1/{2d}$.
\end{theorem}

\section{Constructing an involution centraliser}
\label{Bray}

The centraliser of an involution in a black-box group having an order
oracle can be constructed using an algorithm of Bray \cite{Bray}. 
Elements of the centraliser are constructed using the following result.
\begin{theorem}
\label{thm:bray}
If $u$ is an involution in a group $G$, and $g$ is an arbitrary element of $G$,
then $[u,g]$ either has odd order $2k+1$, in which case
$g[u,g]^k$ commutes with $u$, or has even order $2k$, in which case
both $[u,g]^k$ and $[u,g^{-1}]^k$ commute with $u$.
\end{theorem}
That these elements centralise $u$ follows from elementary
properties of dihedral groups. 

Bray \cite{Bray} also proves that if $g$ is uniformly
distributed among the elements of $G$ for which $[u,g]$
has odd order, then $g[u,g]^k$ is uniformly distributed among the
elements of the centraliser of $u$. If $[u,g]$ has even even, 
then the elements returned are involutions; but if just
one of these is selected, then it is independently and uniformly 
distributed within that class of involutions.

Let $u \in \SL(d, q)$ and let $E_+$ and $E_-$ denote the 
eigenspaces of $u$.
We apply the Bray algorithm in the following contexts. 
\begin{enumerate}
\item 
Construct a generating set for a subgroup of the centraliser 
of $u$ that contains $\SL(E_+)\times \SL(E_-)$. 
\item 
The eigenspaces, $E_+$ and $E_-$, have the same dimension. 
Construct the projective centraliser of $u$.
As we observed in Section \ref{cent}, its preimage in 
$G$ contains an element which interchanges the eigenspaces.
\end{enumerate}
The other contexts are similar, but with $\SL(d,q)$ 
replaced by the other classical groups. 

Parker \& Wilson \cite{PW05} prove the following:
\begin{theorem}\label{clasthm}
There is an absolute constant $c$ such that if $G$ is a finite
quasisimple classical group, with natural module
of dimension $d$ over a field of odd order,
and $u$ is an involution in $G$, then $[u,g]$ has odd order
for at least a proportion $c/d$ of the elements $g$ of $G$.
\end{theorem}

Hence, by a random search of length at most $O(d)$, we construct
random elements of the centraliser of the involution. 
Liebeck \& Shalev \cite{lish} prove that if $H_0 \leq H \leq {\rm Aut}(H_0)$,
where $H_0$ is a finite simple group, then the probability that
two random elements of $G$ generate a group containing 
$H_0$ tends to 1 as $|H_0|$ tends to infinity. A similar
result clearly holds for a direct product of two simple groups.

In its black-box application, this algorithm 
assumes the existence of an order oracle.
We do not require such an oracle for a linear group.
If a multiplicative upper-bound $B$ for the
order of $g \in G$ is available,
then we can learn in polynomial time
the {\it exact} power of $2$ (or of any specified prime)
which divides $|g|$.
By repeated division by 2, we write $B = 2^m b$ where
$b$ is odd. Now we compute $h = g^{b}$, and determine
its order which divides $2^m$ by repeated squaring.
If $g \in \GL(d, q)$, then a multiplicative upper 
bound of magnitude $O(q^d)$ can be obtained for $|g|$
using the algorithms of 
\cite{CLG97} and \cite{Storjohann98}
in at most $O(d^3 \log q)$
field operations.  We call this upper bound the
{\it pseudo-order} of $g$.
Further, as discussed in \cite{Ryba-paper},
the construction of the centraliser
of an involution requires only
knowledge of the pseudo-order.

In summary, \cite[Theorem 7]{Ryba-paper} implies the following.
\begin{theorem}
The Bray algorithm to construct the centraliser
of an involution in $\SX(d, q)$ has complexity
$O(d(\xi + d^3 \log q))$ field operations.
\end{theorem}

This algorithm can be readily adapted (using projective rather
than linear pseudo-orders) to compute the preimage in 
$\SX(d, q)$ of the centraliser of an involution in $\PX(d, q)$.

Once we construct a subgroup of  the centraliser containing
its derived group, we can apply 
the algorithms of Section \ref{Pow} to obtain its projection 
onto each factor and then obtain the derived group of each projection.
Now we can use the algorithm of \cite{NP} to deduce that 
the projection contains a perfect classical group in its 
natural representation.  If the factors have the
same dimension, there is a small possibility that the given elements
generate a group that contains a diagonal embedding of $\SX(d/2,q)$
in $\SX(d/2,q)\times\SX(d/2,q)$ but does not contain the full direct product.
This case is easily detected.  We can also readily detect when an element
of the centraliser interchanges the eigenspaces.

We summarise the preceding discussion. 
\begin{theorem}
Let $h$ be an involution in $\langle X \rangle = G$, where
$\SX(d, q) \leq G \GX(d, q)$. Assume that the 
$-1$-eigenspace of $h$ has dimension $e$ in the range $(d/3, 2d/3]$.
Generating sets for the images in 
$\SX(e, q)$ and $\SX(d - e, q)$ that centralise
the eigenspaces can be found in $O(d(\xi + d^3 \log q))$ field operations.
If $e = d/2$ so that $d \equiv 0 \bmod 4$, then
we can similarly find an element in $\SX(d, q) \wr C_2$ which
interchanges the two copies of $\SX(d, q)$.
\end{theorem}

\section{The base cases}
\label{base}
We now consider the base cases for 
Algorithms {\tt One} and {\tt Two}.
Recall that, for $d = 2n$, 
$Y_0 = \{ s, t, \delta, u, v \}$
generates $\SX(2, q) \wr C_n$  or $\SX(2, q) \wr S_n$.
This observation influences the organisation of our 
algorithms and particularly impacts on our handling of the base cases. 
As the first and major part of each
algorithm, we construct $Y_0$.   
Then, as a final step, we construct the 
additional elements $x, y$.
Clearly the elements of $Y_0$ could be constructed
by constructively recognising $\SX(4, q)$; however, both $u$ and $v$
can be obtained by constructively recognising $\SX(2, q) \wr C_2$, a 
computation practically easier than that for $\SX(4, q)$.
Hence we designate the following as 
base cases: $\SX(2, q)$, $\SX(2, q) \wr C_2$, $\SX(3, q)$ and $\SX(4, q)$. 
The last two arise {\it at most once} during
an application of Algorithm {\tt One} or {\tt Two}.

The construction of a hyperbolic basis for a vector space with a given
symplectic or hermitian form, can be carried out
in $O(d^3)$ field operations \cite[Chapter 2]{Grove02}.
This calculation is performed for base cases only.

In the remainder of this section, 
we outline the specialised algorithms
for the base cases. 
We first summarise their cost.

\begin{theorem}\label{ryba-alg}
Subject to the availability of a discrete log oracle 
for $\GF(q)$, {\rm SLPs} for standard generators and other 
elements of $\SX(d, q)$ for $d \leq 4$ can be 
constructed in $O(\xi \log \log q + \log q)$ field operations.
\end{theorem}

\subsection{$\SL(2,q)$}
The base case encountered most frequently is $\SL(2,q)$
in its natural representation.
An algorithm to construct an element of $\SL(2,q)$ as an SLP  
in an arbitrary generating set is described in \cite{Conderetal05}. 
This algorithm requires $O(\log q)$ field operations, and the 
availability of discrete logarithms in $\GF(q)$.
Observe that $\SU(2, q)$ is isomorphic to $\SL(2, q)$
and can be written over $\GF(q)$ using the algorithm 
of \cite{GLO} in $O(\log q)$ operations.

%For $\SL(3,q)$ we use the algorithm of \cite{sl3q}
%to perform the same task. 
%It assumes the existence of an oracle
%to recognise constructively $\SL(2, q)$
%and its complexity is that of the oracle.

\subsection{$\SL(2, q) \wr C_2$}\label{glue-element}
In executing Algorithms {\tt OneEven} or {\tt OneOdd}, or
{\tt TwoTimesFour} or {\tt TwiceTwiceOdd}, 
each pair of recursive calls generates
an instance of the following problem.

\begin{problem} \label{glue}
Let $V$ be the natural module of  $G=\SX(4,q)$, and let
$(e_1,f_1,e_2,f_2)$ be a hyperbolic basis for $V$. Given a generating
set for $X$, and the involution $u$, where $u$ maps $e_1$ to $-e_1$
and $f_1$ to $-f_1$, and centralises the other basis
elements, construct the involution  $b$  that permutes the basis
elements, interchanging $e_1$ with $e_2$, and $f_1$ with $f_2$.
\end{problem}

Consider the procedure {\tt OneEven}.
Observe that in l.\ 15 we construct $\SX(4, q)$.
Now $b$ is the permutation matrix used in l.\ 16 to 
``glue" $v_1$ and $v_2$ together to form $v$, the long cycle. 
We could use the algorithm of Section \ref{ryba-base}  to find
$b$ directly in $\SX(4, q)$.
Instead, for reasons of practical efficiency, we 
use the following algorithm to 
find $b$ inside the projective centraliser of $u \in \SX(4, q)$. 

\begin{enumerate}
\item 
Construct the projective
centraliser $H$ of $u$ in $\SX(4,q)$, using the Bray algorithm. 

\item 
Since $\SL(2,q)\wr C_2 \leq H \leq \GL(2, q) \wr C_2$,  
we find $h\in\SL(2,q)\wr C_2$ that
interchanges the spaces $\langle e_1,f_1\rangle$ and $\langle
e_2,f_2\rangle$. 

\item Now $bh$ lies in $\SL(2,q)\times\SL(2,q)$.  Using the
algorithms described in Section \ref{Pow}, 
we construct the two direct factors, solve in each
direct factor for the projection of $jh$ and so 
construct $bh$ as an SLP.  
We can now solve for $b$.
\end{enumerate}
This algorithm requires $O(\log q)$ field operations.
Observe that we can conjugate, using $h$, the solution from one 
copy of $\SL(2, q)$ to the other, thus 
requiring just one constructive recognition of $\SL(2, q)$.

%\subsection{The involution-centraliser algorithm} 
\subsection{$\SX(3, q)$ and $\SX(4, q)$}
\label{ryba-base}
We use the involution-centraliser algorithm of \cite{Ryba-paper}
to construct standard generators for $\SX(3,q)$, and the additional
elements $x, y \in \SX(4, q)$.  

We briefly summarise this algorithm.  
Assume $G = \langle X \rangle$ is a black-box group
with order oracle. We are given 
$g \in G$ to  be expressed as an SLP in $X$.
In our description, if we ``find" an element $g$ of $G$, then we 
obtain its SLP in $X$.
First find by random search $h\in G$ such that
$gh$ has even order $2\ell$, and $z:=(gh)^\ell$ is a non-central
involution. Now  find, by random search and powering, an involution
$x\in G$ such that $xz$ has even order $2m$, and $y:=(xz)^m$ is a
non-central involution. Note that an SLP is known for $x$, but, at this
stage, not for either of $y$ nor $z$. 
Observe that $x$, $y$ and $z$ are non-central involutions. 
We construct their centralisers using the Bray algorithm.
We assume that we can  solve the explicit membership problem 
in these centralisers.
In particular, we find $y$ as an element of the centraliser in $G$ of $x$, 
and $z$ as an
element of the centraliser in $G$ of $y$, and $gh$ as an element of the
centraliser in $G$ of $z$. Now that we know an SLP for $gh$ and 
$h$, we can write down an SLP for $g$.

In summary, this algorithm reduces the constructive
membership test for $G$ to three constructive membership  tests  in involution
centralisers in $G$.  But  this is an imperfect recursion, since  the 
algorithm may not be applicable to these centralisers. 
We do not rely on the recursion; instead we construct
explicitly the desired elements of the centralisers, 
since their derived groups are (direct products of) $\SL(2,q)$
and we can use the algorithm of \cite{Conderetal05}.
In this context, the complexity of the involution-centraliser
algorithm is that stated in Theorem \ref{ryba-alg}.

As presented, this is a black-box algorithm  requiring an order oracle.
If $G$ is a linear group, the algorithm does not require 
an order oracle, exploiting instead the multiplicative 
bound for the order of an element which can
be obtained in polynomial time as described in Section \ref{Bray}.

Since the practical performance of
this algorithm is rather slow for large fields,
we organised Algorithm {\tt One} and {\tt Two}
to ensure that they each need {\it at most one application}. 
If the dimension $d$ of the input group is odd, then we invoke 
this algorithm 
once to construct standard generators for $\SX(3, q)$. 
If $d$ is even, then as a final step, we construct the additional generators 
$x$ and $y$ using this algorithm. 
Let $h \in G=\SX(d, q)$ be the involution whose $-1$-eigenspace
is $\langle e_1, f_1, e_2, f_2 \rangle$. 
Observe that $h$ 
% = (s \cdot s^v)^{(p - 1)/2}$, so it 
can be readily constructed from $Y_0$, and that both $x$ and $y$
are elements of $C_G(h)$. 
%These employ the involution-centraliser algorithm to find both,
%and Theorem \ref{ryba-alg} again applies.

\section{Complexity of the algorithms} 
\label{Analysis}
We now analyse the principal
algorithms, and in the next section estimate the length of the SLPs
that express the canonical generators as words in  the given
generators. The time analysis is based on counting the number of
field operations, and the number of calls to 
the discrete logarithm oracles. Use of discrete
logarithms in a given field requires first the setting up of certain
tables, and these tables are consulted for each application. The
time spent in the discrete logarithm algorithm, and the space that it
requires, are  not proportional to the number of applications in a
given field.

Babai \cite{Babai91} presented a Monte Carlo algorithm to
construct in polynomial time nearly uniformly distributed 
random elements of a finite group.  An alternative is the 
{\it product replacement algorithm} of Celler
{\it et al.\ }\cite{Celleretal95}.
That this is also polynomial time was
established by Pak \cite{Pak00}.
For a discussion of both algorithms, we refer
the reader to \cite[pp.\ 26-30]{Seress03}.

We now complete our analysis of the main algorithms.
\begin{theorem}\label{Theorem1}  
The number of field operations carried out 
in Algorithm {\tt OneEven} is 
$O(d (\xi + d^3 \log q)$. 
\end{theorem}
\begin{proof} 
The proportion of elements of $G$ with the required property in line 6
is at least $k/d$ for some absolute constant $k$, as proved in Section
\ref{Involution}.

The number of field operations required in lines 8 and 14 is 
$O(d(\xi + d^3 \log q))$,
as proved in Section \ref{Bray}.

The recursive calls in  lines 10 and 11 are to cases of dimension at
most $2d/3$, and hence they increase by only a constant factor 
the number of field operations. 

The number of field operations required in lines 9 and 13 is at most 
$O(d^3\log q)$, as proved in Section \ref{Pow}. 

The result follows.
\end{proof}

We estimate the number of calls to the $\SL(2, q)$ constructive
recognition algorithm and the associated discrete logarithm oracle.
\begin{theorem}\label{Theorem2} 
If $d > 2$, then Algorithms {\tt OneEven} and {\tt TwoTimesFour}
generate at most $2d-3$ and $3 \log d$ calls  to 
the discrete logarithm oracle for $\GF(q)$ respectively.
\end{theorem}
\begin{proof}
Each call to the constructive recognition oracle for $\SL(2,q)$ generates 
three calls to the discrete logarithm oracle for $\GF(q)$ \cite{Conderetal05}.
Each solution to Problem \ref{glue} requires 3 calls to the oracle. 

Let $f(d)$ be the number of calls generated by 
applying {\tt OneEven} to $\SX(d, q)$. 
Then $f(2) = f(4) = 3$ and 
$f(d) = f(e) + f(d - e) + 3$ for $d > 4$ and some $e \in (d/3, 2d/3]$.
It follows that $f(d) \leq 2d - 3$ for $d > 2$.

Let $g(d)$ be the number of calls generated by 
applying {\tt TwoTimesFour} to $\SX(d, q)$, where $d = 4n$.
Again $g(2) = g(4) = 3$ and $g(4n) = g(2n) + 3$ for $n > 2$.
Hence $g(d) \leq 3 \log d$.
\end{proof}

Similar results hold for the other algorithms.
If we use the involution-centraliser algorithm \cite{Ryba-paper} 
to construct either standard generators for $\SX(3, q)$, 
or additional generators $x, y \in \SX(4, q)$, 
then the number of calls to the oracle in each case is 9.

\section{Straight-line programs}\label{SLP}
We now consider the length of the SLPs for the 
standard generators for $\SX(d, q)$ constructed by our algorithms.

In its simplest form, 
an SLP on a subset $X$ of a group $G$ is a
string, each of whose entries is either a pointer to an element of $X$, 
or a pointer to a previous
entry of the string, or an ordered pair of pointers to not necessarily 
distinct previous entries.
Every entry of the string defines an element of $G$.  An 
entry that points to an element of
$X$ defines that element.  An entry that points to a previous entry defines 
the inverse of the element defined by that entry.  An entry that points to 
two previous entries defines the
product, in that order, of the elements defined by those entries.

Such a simple SLP defines an element 
of $G$, namely the element defined by the last entry, and it 
can be obtained by computing in turn
the elements for successive entries.
The SLP is primarily used by replacing the elements 
$X$ of $G$ by the elements $Y$ of some group
$H$, where $X$ and $Y$ are in one-to-one correspondence, and then evaluating 
the element of $H$ that the SLP then defines.

We now identify other desirable features of SLPs. 
\begin{enumerate}
\item 
We need to replace the second type of node, that defines the 
inverse of a previously defined element, by a type of node with two fields, 
one pointing to a previous entry, and one containing 
a possibly negative integer.  The element defined is then the 
element defined by the entry to which
the former field points, raised to the power defined by 
the latter field.  This reflects the fact that we
raise group elements to very large powers, and have 
an efficient algorithm described in Section \ref{Exp} for performing this.

\item 
An SLP may define a number of elements of $G$,
and not just one element, so a sequence of nodes may be specified as 
giving rise to elements
of $G$.  Thus we wish to return a single SLP that defines 
all of the standard generators of $\SX(d,q)$, rather than an SLP
for each generator.
This avoids duplication when two or more of the standard generators 
rely on common calculations.

%\item 
%In order to preserve space the structure of an SLP needs to 
%be enhanced to ensure that, when the
%SLP is evaluated in some other group, the element defined by a 
%node is only calculated when it
%will be needed later, and is discarded when it it is no longer needed.  
%Discarding the element of $H$ defined by a node when it is no longer 
%required in an evaluation ensures that the space complexity of evaluating 
%an SLP is at worst proportional to the space complexity of the space 
%required to construct the corresponding element (or elements) of $G$ 
%in the first place, given a bound to the space required to store 
%an element of $H$.

\item 
A critical concern is how the number of trials in a random search for 
a group element affects the length of an SLP that defines that element.  
Any discussion of this requires consideration of the algorithm
used to generate random elements.  We make two reasonable assumptions:
\begin{enumerate}
\item[(a)] the associated random process is a stochastic process taking place 
in a graph whose vertices are defined by a {\it seed};
\item[(b)] a random number generator now determines which 
edge adjoining the current vertex in the graph will be followed in 
the stochastic process.  
\end{enumerate}
By default, the length of the SLP will then increase by a constant amount 
for {\it every trial, successful or unsuccessful}.  
Should its length reflect only those trials which are successful? 
One additional assumption which allows us to explore this question is the following:
{\it When embarking on a search that is expected to require $d$ trials, 
we record the value of the seed, and repeatedly carry
out a random search, using our random process, but returning, after 
every $\ell (d)$ steps, for some function $\ell$ of $d$, to the 
stored value of the seed, until we succeed.}
We hypothesise that values for $\ell(d)$ range from $\log d$ to $d$ 
and analyse the lengths of the SLPs for these values below.
\end{enumerate}

\begin{theorem} 
If the {\rm SLPs} constructed satisfy properties $1-3$ above,
then their lengths are the following.
%\begin{table}[h]
%%\caption{Length of SLP}
%\label{SLP-len}
\begin{center}
\begin{tabular}
{|c|r|r|} \hline
$\ell (d)$  &   {\tt OneMain}  & {\tt TwoMain} \rule{0cm}{2.5ex}\\
\hline
%1  &  $d + \log d$ &   $\log^2 d$    \rule{0cm}{2.5ex}\\ \hline
$\log d$  &  $O(d)$ &   $O(\log^3 d)$    \rule{0cm}{2.5ex}\\ \hline
$d$  &  $O(d \log d)$ &   $O(d \log d)$    \rule{0cm}{2.5ex}\\ \hline
\end{tabular}
\end{center}
%\end{table}

\end{theorem}
\begin{proof}
We wish to find functions $f(d)$ and $g(d)$ such that the lengths of
the SLPs returned by Algorithms {\tt One} and {\tt Two} are bounded
above by these functions respectively.

It suffices for $f$ to satisfy
$f(d)\ge f(e)+f(d-e) + c\ell(d)$ whenever $d\ge5$ for some constant
$c$, if $f(d)$ is large enough for small values of $d$. Since of
necessity $f(d)>f(e)+f(d-e)$ it follows that $f(d)$ is at least linear in $d$.

Consider, for example, the case where $\ell(d)=d$.
Suppose that we take a constant $k$ such that
$k>c/\log(3/2)$, taking all logarithms to base 2. Suppose now that
$f(n) < kn\log(n)$ for  all $n<d$ for some $d>4$, and let
$e\in(d/3,2d/3]$. Then 
\begin{eqnarray*}
f(e)+f(d-e) +cd & < & ke\log(e) + k(d-e)\log(d-e) + cd  \\
                 & < & kd\log(d) -kd\log(3/2)+cd \\
                 & < & kd\log(d),
\end{eqnarray*}
as required. 

Algorithm {\tt Two} recurses either from the case $d=4n$ to the case
$d=2n$ in one step, or from the case $d=4n+2$ to the case $d=4n$ and
then to the case $d=2n$. It is easy to see that the effect on the
length of the SLP in the latter situation is dominated by the second
step. If $d$ is initially odd, then the contribution of the reduction to
the even case, which is carried out once, may also be ignored here.
The main contribution to the length of the SLP in passing from $d=4n$
to $d=2n$ arises from constructing an involution whose eigenspaces
have dimension $2n$. This involution is constructed recursively,
where the length of the recursion is $O(\log d)$. Thus the
contribution to the length of the SLP in constructing this involution
is $O(\log(d)\ell(d))$. 
Hence $g(4n)\le g(2n)+c\log(n)\ell(n)$ and $g(4n+2)\le g(2n)+c\log(n)\ell(n)$.
If $\ell(d)=O(\log d)$, then   the inequality 
$g(n)\le g(\lceil n/2\rceil)+c\log^2(n)$
is satisfied by $g(n)=k\log^3(n)$ for large enough $k$. 

Similar calculations can be carried for the other two cases,
yielding the stated results. 
\end{proof}

\section{An implementation}
Our implementation of these algorithms is publicly available in {\sc Magma}.
It uses:
\begin{itemize}
\item 
the product replacement algorithm \cite{Celleretal95}
to generate random elements; 
\item our implementations of Bray's algorithm \cite{Bray}
and the involution-centraliser algorithm \cite{Ryba-paper}.
\item our implementations of the algorithm of \cite{Conderetal05}
and \cite{sl3q}.
\end{itemize}


The computations reported in Table \ref{table1} were carried out
using {\sc Magma} V2.13 on a Pentium IV 2.8 GHz processor.
The input to the algorithm is $\SX (d, q)$.
In the column entitled ``Time", we list the CPU time in seconds
taken to construct the standard generators.

\begin{table}[h]
\caption{Performance of implementation for a sample of groups}
\label{table1}
\begin{center}
\begin{tabular}
{|c|r|} \hline
Input  &   Time  \rule{0cm}{2.5ex}\\
\hline
%$SL_3(11)$ & $2.1$      \rule{0cm}{2.5ex}\\ \hline
$\SL(6, 5^8)$  & 2.2 \rule{0cm}{2.5ex}\\ \hline
$\SL(40, 5^8)$  & 22.5 \rule{0cm}{2.5ex}\\ \hline
$\SL(80, 5^8)$  & 130.8 \rule{0cm}{2.5ex}\\ \hline
$\Sp(10,5^{10})$ & 19.5 \rule{0cm}{2.5ex}\\ \hline
$\Sp(40,5^{10})$ & 280.4 \rule{0cm}{2.5ex}\\ \hline
%$\SU(6,5^{5})$ & 22.6 \rule{0cm}{2.5ex}\\ \hline
$\SU(8,3^{16})$ & 22.6 \rule{0cm}{2.5ex}\\ \hline
$\SU(20, 5^{12})$ & 47.6 \rule{0cm}{2.5ex}\\ \hline
$\SU(70, 5^2)$ & 191.3 \rule{0cm}{2.5ex}\\ \hline
\end{tabular}
\end{center}
\end{table}

\begin{thebibliography}{10}

\bibitem{Babai91}
L\'aszl\'o Babai,  
Local expansion of vertex-transitive graphs and
  random generation in finite groups,  {\it Theory of Computing}, (Los
  Angeles, 1991), pp.\ 164--174. Association for Computing Machinery, 
New York, 1991.

\bibitem{BabaiSzemeredi84}
L\'aszl\'o Babai and Endre Szemer\'edi.
\newblock On the complexity of matrix group problems, {I}.
\newblock In {\em Proc.\ $25$th IEEE Sympos.\ Foundations Comp.\ Sci.}, pages
  229--240, 1984.

\bibitem{BabaiBeals99}
L\'aszl\'o Babai and Robert Beals,
\newblock A polynomial-time theory of black-box groups. {I}.
\newblock In {\em Groups St. Andrews 1997 in Bath, I}, volume 260 of {\em
  London Math. Soc. Lecture Note Ser.}, pages 30--64, Cambridge, 1999.
  Cambridge Univ. Press.

\bibitem{BPS} L. Babai, P.\ P{\'a}lfy and J.\ Saxl, On the number
of $p$-regular elements in simple groups, preprint.

\bibitem{Bray} J.N. Bray, An improved method of finding
the centralizer of an involution, {\it Arch. Math. (Basel)}
{\bf 74} (2000), 241--245.

\bibitem{Magma}
Wieb Bosma, John Cannon, and Catherine Playoust,
\newblock The {\sc Magma} algebra system I: The user language,
\newblock {\em J.\ Symbolic Comput.}, {\bf 24}, 235--265, 1997.

\bibitem{Brooksbank03}
P.A. Brooksbank,
\newblock Constructive recognition of classical groups
in their natural representation.
\newblock {\em J. Symbolic Comput.} {\bf 35} (2003), 195--239.


\bibitem{Carter}
Roger Carter.
\newblock Simple groups of Lie Type.
\newblock Wiley-Interscience, 1989.

\bibitem{Celleretal95}
Frank Celler, Charles R.\ Leedham-Green, Scott H.\ Murray, Alice C.\
  Niemeyer and E.A.\ O'Brien, Generating random elements of a 
finite group, {\it Comm.\ Algebra}, {\bf 23} (1995), 4931--4948.

\bibitem{CLG97}
Frank Celler and C.R.\ Leedham-Green,
\newblock Calculating the order of an invertible matrix,
\newblock In {\em {Groups and Computation {II}}}, volume~28 of {\em Amer.\
  Math.\ Soc.\ DIMACS Series}, pages 55--60. (DIMACS, 1995), 1997.


\bibitem{CellerLeedhamGreen98}
F.~Celler and C.R. Leedham-Green.
\newblock A constructive recognition algorithm for the special linear group.
\newblock In {\em The atlas of finite groups: ten years on (Birmingham, 1995)},
  volume 249 of {\em London Math. Soc. Lecture Note Ser.}, pages 11--26,
  Cambridge, 1998. Cambridge Univ. Press.

%\bibitem{CMT}
%Arjeh M.\ Cohen, Scott H.\ Murray, and D.E.\ Taylor.
%\newblock Computing in groups of Lie type.
%\newblock {\em Math. Comp.\ }{\bf73}, 1477-1498, 2003.

%\bibitem{ConderLeedhamGreen01}
%Marston Conder and Charles~R. Leedham-Green.
%\newblock Fast recognition of classical groups over large fields.
%\newblock In {\em Groups and Computation, III (Columbus, OH, 1999)}, volume~8
%  of {\em Ohio State Univ. Math. Res. Inst. Publ.}, pages 113--121, Berlin,
%  2001. de Gruyter.

\bibitem{Conderetal05}
M.D.E. Conder, C.R. Leedham-Green, and E.A. O'Brien.
\newblock Constructive recognition of PSL$(2, q)$.
\newblock {\em Trans.\ Amer.\ Math.\ Soc.} {\bf 358}, 1203--1221, 2006.

\bibitem{FNP}
Jason Fulman, Peter M. Neumann, and Cheryl E. Praeger.
A Generating Function Approach to the Enumeration
of Matrices in Classical Groups over Finite Fields.
Mem. Amer. Math. Soc. {\bf 176}, no.\ 830, 2005.

\bibitem{GLO} S.P.\ Glasby, C.R.\ Leedham-Green, and E.A. O'Brien.
Writing projective representations over subfields.
{\it J. Algebra}, 295, 51-61, 2006.

\bibitem{GLS3}
Daniel Gorenstein, Richard Lyons, and  Ronald Solomon.
The classification of the finite simple groups. Number 3. Part I,
American Mathematical Society, Providence, RI, 1998. 

\bibitem{Grove02}
Larry C. Grove. Classical Groups and Geometric Algebra.
AMS Graduate Studies in Math.\ {\bf 39}.

%\bibitem{GuralnickLubeck01}
%R.M. Guralnick\ and\ F. L\"ubeck.
% On $p$-singular elements in Chevalley groups in characteristic $p$.
%\newblock In {\em Groups and Computation, III (Columbus, OH, 1999)}, volume~8
%  of {\em Ohio State Univ. Math. Res. Inst. Publ.}, pages 113--121,
%Berlin, 2001. de Gruyter.

\bibitem{HoltRees94} D.F.\,Holt and S.\,Rees. Testing modules for
irreducibility. {\it J. Austral. Math. Soc. Ser. A} {\bf 57} (1994), 1--16.

\bibitem{HoltEickOBrien05}
Derek~F.\ Holt, Bettina Eick, and Eamonn~A.\ O'Brien.
\newblock {\em Handbook of computational group theory}.
\newblock Chapman and Hall/CRC, London, 2005.

\bibitem{Ryba-paper} P.E. Holmes, S.A. Linton, E.A. O'Brien, A.J.E. Ryba and
R.A. Wilson, Constructive membership in black-box groups, preprint.

\bibitem{KantorSeress01}
William~M. Kantor and {\'A}kos Seress.
\newblock Black box classical groups.
\newblock {\em Mem. Amer. Math. Soc.}, {\bf 149}, 2001.
                             
%\bibitem{LG01}
%Charles R.\ Leedham-Green,
%The computational matrix group project, in
%{\it Groups and Computation}, III (Columbus, OH, 1999), 229--247, Ohio
%State Univ. Math. Res. Inst. Publ., {\bf 8}, de Gruyter, Berlin, 2001.
                                                                                
%\bibitem{LiebeckOBrien05}
%Martin~W. Liebeck and E.A.\ O'Brien.
%\newblock Finding the characteristic of a group of Lie type.
%\newblock Preprint, 2005.

\bibitem{lish} M.W. Liebeck and A. Shalev. The probability of generating
a finite simple group, {\it Geom. Ded.} {\bf 56} (1995), 103--113.

\bibitem{sl3q}
F.\ L{\"u}beck, K.\ Magaard, and E.A. O'Brien. 
Constructive recognition of $\SL_3(q)$.
Preprint 2005.

\bibitem{Huppert67}
B.\ Huppert.
\newblock {\em {E}ndliche {G}ruppen {I}}, volume 134 of {\em Grundlehren Math.\
  Wiss.}
\newblock Springer-Verlag, Berlin, Heidelberg, New York, 1967.

\bibitem{NeumannPraeger92}
Peter~M.\ Neumann and Cheryl~E.\ Praeger.
\newblock A recognition algorithm for special linear groups.
\newblock {\em Proc.\ London Math.\ Soc.\ $(3)$}, 65:555--603, 1992.

\bibitem{NP} A.C. Niemeyer and C.E. Praeger.
A recognition algorithm for classical groups over finite fields,
{\it Proc. London Math. Soc.} {\bf 77} (1998), 117--169.

\bibitem{NP2}
Alice~C. Niemeyer and Cheryl~E. Praeger.
\newblock Implementing a recognition algorithm for classical groups.
\newblock In {\em Groups and Computation, II (New Brunswick, NJ, 1995)},
  volume~28 of {\em DIMACS Ser. Discrete Math. Theoret. Comput. Sci.}, pages
  273--296, Providence, RI, 1997. Amer. Math. Soc.

\bibitem{JAMS}
Alice C.Niemeyer and Cheryl E.Praeger.
A recognition algorithm for non-generic classical groups over finite fields. 
J. Austral. Math. Soc. Ser. A {\bf 67} (1999), no. 2, 223--253.

\bibitem{OBrien05}
E.A. O'Brien. Towards effective algorithms for linear groups.
{\it Finite Geometries, Groups and Computation},
(Colorado), pp. 163-190. De Gruyter, Berlin, 2006.

\bibitem{Pak00}
Igor Pak. The product replacement algorithm is polynomial.
In {\it 41st Annual Symposium on Foundations of Computer Science
(Redondo Beach, CA, 2000)}, 476--485,
IEEE Comput. Soc. Press, Los Alamitos, CA, 2000.

\bibitem{Bath}
Cheryl E.\ Praeger.
 Primitive prime divisor elements in finite classical groups.
In {\it Groups St. Andrews 1997 in Bath, II}, 605--623,
Cambridge Univ. Press, Cambridge, 1999.

\bibitem{PW05}
C.W. Parker and R.A. Wilson.
Recognising simplicity in black-box groups. 
Preprint 2005.

\bibitem{Seress03}
{\'A}kos Seress.
\newblock {\em Permutation group algorithms}, volume 152 of {\em Cambridge
  Tracts in Mathematics}.
\newblock Cambridge University Press, Cambridge, 2003.

\bibitem{Storjohann98}
Arne Storjohann.
An $O(n\sp 3)$ algorithm for the Frobenius normal form. In
{\em Proceedings of the 1998 International Symposium on Symbolic
and Algebraic Computation} (Rostock), 101--104, ACM, New York, 1998.

\bibitem{vzg}
Joachim von zur Gathen and J\"urgen Gerhard,
{\it Modern Computer Algebra}, Cambridge University Press, 2002.
\end{thebibliography}

\begin{tabbing}
\=\hspace{70mm}\=\kill
\>School of Mathematical Sciences \>Department of Mathematics    \\
\>Queen Mary, University of London \>Private Bag 92019, Auckland \\
\>London E1 4NS, United Kingdom   \>University of Auckland     \\
\>United Kingdom                  \> New Zealand     \\
\> C.R.Leedham-Green@qmul.ac.uk   \> obrien@math.auckland.ac.nz
\end{tabbing}

\vspace*{2mm}
\noindent 
Last revised \today

\end{document}
