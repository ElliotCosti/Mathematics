\documentclass[12pt]{article}
\usepackage{amssymb}
\usepackage[algo2e,ruled,linesnumbered]{algorithm2e} % for algorithms
\hoffset -25truemm
\usepackage{latexsym}
\oddsidemargin=30truemm             %%
\evensidemargin=25truemm            %% inner margin 30mm, outer margin 25mm
\textwidth=155truemm                %%
\voffset -25truemm
\topmargin=25truemm                 %% top margin of 25mm
\headheight=0truemm                 %% no head
\headsep=0truemm                    %% no head
\textheight=220truemm               
\renewcommand{\thefootnote}{}
\newtheorem{definition}{Definition}[section]
\newtheorem{lemma}[definition]{Lemma}
\newtheorem{theorem}[definition]{Theorem}
\newtheorem{corollary}[definition]{Corollary}
\newtheorem{remark}[definition]{Remark}
\newtheorem{problem}[definition]{Problem}
\newenvironment{proof}{\normalsize {\sc Proof}:}{{\hfill $\Box$ \\}}

\def\SL{{\rm SL}}
\def\GL{{\rm GL}}
\def\U{{\rm U}}
\def\PSL{{\rm PSL}}
\def\PSp{{\rm PSp}}
\def\Stab{{\rm Stab}}
\def\PSU{{\rm PSU}}
\def\GF{{\rm GF}}
\def\Sp{{\rm Sp}}
\def\SU{{\rm SU}}
\def\SX{{\rm SX}}
\def\PX{{\rm PX}}
\def\GX{{\rm GX}}
\def\PSX{{\rm PSX}}
\def\PGL{{\rm PGL}}
\def\q{\quad}
\def\centreline{\centerline}

\begin{document}

\title{An Algorithm to Find an Element of $\SL(d, q)$ as a Word in its Generators} 
\author{Elliot Costi}
\date{April 2006}
\maketitle

\section{}
\label{}

$\SL(d, q)$ is the set of all $d \times d$ matrices over a finite field with $q = p^e$ elements. Elements of a finite field; can be written in two ways. Firstly as powers of a primite element $\omega$ (except 0) and secondly as a vector space in $omega$ over the prime field with $p$ elements as the following table shows for $F_9$:

$0 -> 0$

$\omega -> \omega$

$\omega^2 -> 1 + \omega$ 

$\omega^3 -> 1 + 2\omega$

$\omega^4 -> 2$

$\omega^5 -> 2\omega$

$\omega^6 -> 2 + 2\omega$

$\omega^7 -> 2 + \omega$

$\omega^8 -> 1$
\\

$\SL(V)$ is the set of all linear transformations from the vector space $V$ to itself. If $V$ is ${F_q}^d$, then the natural representation of $\SL(V)$ is $\SL(d, q)$. Algorithms to find any element $A$ of $\SL(d, q)$ as a word in its generators is long established. I produced a similar algorithm that worked in the following way. You take as generators of $\SL(d, q)$ the following matrices:
\\

$t = $\left(\matrix
{1 & 1 & 0 & 0 & \cdots & 0 \cr 
0 & 1 & 0 & 0 & \cdots & 0 \cr
0 & 0 & 1 & 0 & \cdots & 0 \cr
\vdots & \vdots & \ddots & \ddots & \ddots & \vdots \cr
0 & 0 & 0 & \cdots & 1 & 0  \cr
0 & 0 & 0 & \cdots & 0 & 1}\right)$$

a transvection
\\

$u = $\left(\matrix
{0 & 1 & 0 & 0 & \cdots & 0 \cr 
-1 & 0 & 0 & 0 & \cdots & 0 \cr
0 & 0 & 1 & 0 & \cdots & 0 \cr
\vdots & \vdots & \ddots & \ddots & \ddots & \vdots \cr
0 & 0 & 0 & \cdots & 1 & 0  \cr
0 & 0 & 0 & \cdots & 0 & 1}\right)$$

a 2-cycle
\\

$v = $\left(\matrix
{0 & -1 & 0 & 0 & \cdots & 0 \cr 
0 & 0 & -1 & 0 & \cdots & 0 \cr
0 & 0 & 0 & -1 & \cdots & 0 \cr
\vdots & \vdots & \ddots & \ddots & \ddots & \vdots \cr
0 & 0 & 0 & \cdots & 0 & -1  \cr
1 & 0 & 0 & \cdots & 0 & 0}\right)$$

an n-cycle
\\

$\delta = $\left(\matrix
{$\omega$ & 0 & 0 & 0 & \cdots & 0 \cr 
0 & $\omega^{-1}$ & 0 & 0 & \cdots & 0 \cr
0 & 0 & 1 & 0 & \cdots & 0 \cr
\vdots & \vdots & \ddots & \ddots & \ddots & \vdots \cr
0 & 0 & 0 & \cdots & 1 & 0  \cr
0 & 0 & 0 & \cdots & 0 & 1}\right)$$

an element to extend the field to $p^e$ as opposed to just $p$.
\\

The first step is to continuously multiply $A$ by $\delta$ to get 1 in the $(1, 1)$ entry. The generators $u$ and $v$ generate the permutation group $S_d$. With these you can manipulate the matrix $A$ in question to move row $i$ to row 1 and column $j$ to column 1 and then use various combinations of conjugates of $t$ and $\delta$ to continuously add multiples of the first row/column to every other entry in first row/column until they are all zero. Working through the matrix $A$ in this way, you will eventually be left with the identity matrix. You will then have $x_1 \dots x_m A x_{m+1} \dots x_n = I$, where the $x_i$ are elements of the generating set. Then you can rearrange the equation to get $A$ in terms of the generators.
\\

However, as has already been said, algorithms to solve this problem have already been found. The idea is to now find a similar algorithm, that will probably also utilize linear algebra to solve this problem, when you are no longer working in the natural representation. So you are still working on $\SL({F_q}^d)$ but the matrices that you have that represent these transformations are of dimension $n$, where $n > d$.

I have just started to work on this problem and will be attempting to solve it using an idea put forward by my supervisor Charles Leedham-Green. He has given me the general outline of how it should work and it will be left for me to fill in the details. It is this that I will be outlining today.

The first step in order to solve this problem is to look at a specific example. I have taken $n =$ \left(\matrix {d \cr 2 \cr}\right)$$

and the representation in question to be the exterior square.

What is the exterior square of a module? You choose a basis \{$v_i$\} for $V$, you form the tensor square $V \otimes V$ which is generated by the basis \{$v_i \otimes v_j$\} and then you quotient out the symmetric elements. That is to say, $v \wedge v$ = 0 for all $v \in V$, where $\wedge$ is the symbol you use to denote the product in the exterior square (obviously different from $\otimes$ as $v \wedge v$ = 0.

Now consider the subgroup $H \leq \SL(d, q)$. $H = $\left(\matrix
{det^{-1} & 0 & 0 & 0 & 0 & 0 \cr 
* &  &  &  &  &  \cr
* &  &  &  &  &  \cr
* &  &   $\GL(d-1, q)$   &  \cr
* &  &  &  &  & }\right)$$

This fixes a 1 dimensional space and is isomorphic to $C_{q^{(d-1)}} \rtimes GL(d-1, q)$. Now we map $H$ from the natural representation to $SL(n, q)$ by a map $\phi$. Now, $\phi(H)$ acts reducibly on the underlying vector space ${F_q}^n$ since it has a normal $p$-subgroup (a theorem from representation theory). The normal $p$-subgroup in question is $C_{q^{(d-1)}}$. So there is a non-trivial submodule $U$ of ${F_q}^n$. Now, $H$ is maximal and normalises $U$ and so $H = N(U)$. By normaliser we mean $\{g \in \SL(n, q)|gU = U \}$. Now let $W = U^g$. We want to find out the first row of the matrix $g \in SL(n, q)$.

Consider ${g_2}^\alpha, {g_3}^\alpha, \ldots, {g_n}^\alpha \in \SL(d, q)$ and say that these elements are the preimage of \{$I + \alpha\delta_{1i}\} \in \SL(n, q)$, where $\alpha$ is a primitive element of $F_q$. We want to find $\alpha_2, \alpha_3, \dots, \alpha_d$ such that $W^{{g_2}^{\alpha_2} \dots {g_d}^{\alpha_d}} = U$. We then have that $g{g_2}^{\alpha_2} \dots {g_d}^{\alpha_d} \in H$ and hence we now have the whole problem reduced by a dimension. This process is then repeated on the next dimension down.

\end{document}

