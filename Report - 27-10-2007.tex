\documentclass[12pt]{article}
\usepackage{amssymb}
\hoffset -25truemm
\usepackage{latexsym}
\oddsidemargin=27truemm             %%
\evensidemargin=25truemm            %% inner margin 30mm, outer margin 25mm
\textwidth=157truemm                %%
\voffset -25truemm
\topmargin=22truemm                 %% top margin of 25mm
\headheight=0truemm                 %% no head
\headsep=0truemm                    %% no head
\textheight=240truemm               
\renewcommand{\thefootnote}{}
\newtheorem{definition}{Definition}[section]
\newtheorem{lemma}[definition]{Lemma}
\newtheorem{theorem}[definition]{Theorem}
\newtheorem{corollary}[definition]{Corollary}
\newtheorem{remark}[definition]{Remark}
\newtheorem{problem}[definition]{Problem}
\newenvironment{proof}{\normalsize {\sc Proof}:}{{\hfill $\Box$ \\}}

\def\SL{{\rm SL}}
\def\GL{{\rm GL}}
\def\U{{\rm U}}
\def\PSL{{\rm PSL}}
\def\PSp{{\rm PSp}}
\def\Stab{{\rm Stab}}
\def\PSU{{\rm PSU}}
\def\GF{{\rm GF}}
\def\Sp{{\rm Sp}}
\def\SU{{\rm SU}}
\def\SX{{\rm SX}}
\def\PX{{\rm PX}}
\def\GX{{\rm GX}}
\def\PSX{{\rm PSX}}
\def\PGL{{\rm PGL}}
\def\q{\quad}
\def\centreline{\centerline}

\begin{document}

\title{Quaterly Report}
\author{Elliot Costi}
\date{October 2007}
\maketitle

\newpage

\section{Reading}

\begin{enumerate}
\item Introduction to Lie Algebras and Representation Theory by James Humphreys - Chapter 1 of 7.
\end{enumerate}

\section{Research}

\begin{enumerate}
\item Writing an algorithm to write a word in the generators for $\Omega^-(d, F)$, char $F \ne 2$ in a non-natural representation
\item Writing code to test and debug the above algorithm
\item Writing code to test and debug the natural representation code for $\Omega^-(d, F)$.
\item Fixing bugs in the non-natural representation code for $\SU(d, F)$, $d$ both odd and even (two different algorithms for each).
\item Fixing bugs in the non-natural representation code for $\Omega^+(d, F)$.
\item Fixing bugs in the natural representation code for $\SU(d, F)$, $d$ both odd.
\end{enumerate}

\section{Seminars}

\begin{enumerate}
\item I gave two seminars on material based on James Humphreys' Reflection Groups and Coxeter Groups book.
\end{enumerate}


\subsection{Things To Be Done}

\begin{enumerate}
\item Calculate the complexity of the algorithms for which this has not yet been done.

\item Construct algorithms to write an element of $\PSX(d, q)$ as an element of its generators by considering how the generators act on the projective points.

\item Get these algorithms to work for characteristic 2.

\item Write a paper on the above material.

\item Give talk at the Pure Maths Seminar in January.

\item Write thesis.

\end{enumerate}

\end{document}

