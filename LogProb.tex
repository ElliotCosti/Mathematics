\documentclass[12pt]{article}
\usepackage{amssymb}
\hoffset -25truemm
\usepackage{latexsym}
\usepackage{float}
\usepackage[algo2e,ruled,linesnumbered]{algorithm2e}
\oddsidemargin=27truemm             %%
\evensidemargin=25truemm            %% inner margin 30mm, outer margin 25mm
\textwidth=157truemm                %%
\voffset -25truemm
\topmargin=22truemm                 %% top margin of 25mm
\headheight=0truemm                 %% no head
\headsep=0truemm                    %% no head
\textheight=240truemm               
\renewcommand{\thefootnote}{}
\newtheorem{definition}{Definition}[section]
\newtheorem{lemma}[definition]{Lemma}
\newtheorem{theorem}[definition]{Theorem}
\newtheorem{corollary}[definition]{Corollary}
\newtheorem{remark}[definition]{Remark}
\newtheorem{problem}[definition]{Problem}
\newenvironment{proof}{\normalsize {\sc Proof}:}{{\hfill $\Box$ \\}}

\usepackage{setspace} 
\doublespacing

\def\SL{{\rm SL}}
\def\GL{{\rm GL}}
\def\U{{\rm U}}
\def\PSL{{\rm PSL}}
\def\PSp{{\rm PSp}}
\def\Stab{{\rm Stab}}
\def\PSU{{\rm PSU}}
\def\GF{{\rm GF}}
\def\Sp{{\rm Sp}}
\def\SU{{\rm SU}}
\def\SO{{\rm SO}}
\def\SX{{\rm SX}}
\def\PX{{\rm PX}}
\def\GX{{\rm GX}}
\def\PSX{{\rm PSX}}
\def\PGL{{\rm PGL}}
\def\q{\quad}
\def\centreline{\centerline}

\begin{document}

$${\rm Omega Minus char 2}$$

We are considering the natural representation only. We wish to kill an arbitrary element of $\Omega^-(d, q)$, where $q$ is even. In order to do this, we wish to embed $\Omega^+(d-2, q)$ into it. That is to say write the generators of $\Omega^+(d-2, q)$ in terms of those of $\Omega^-(d, q)$. There is only one generator of $\Omega^+(d-2, q)$ that we need to consider and that is $t$:

$$\left(\matrix{1&0&0&1\cr0&1&0&0\cr0&1&1&0\cr0&0&0&1}\right).$$

The following is the proof in the thesis of how to create this element. Can you think of any way of finding out $z$ without using logarithms? One of the problems I have here is that I don't have $\gamma + \gamma^q$ as a power of $\omega$ (notation explained below).

\begin{lemma}
Let $B(h) = (h^{v^2})^{-1} ({(h^\delta)^v}) h^{v^2}$ and let $\omega^z$ be (4, 1) entry of $B(h)$, where $v$ and $\delta$ are generators for $\Omega^-(2d, q)$ as they appear in the above table, $h$ is some other generator and $\omega$ is the primitive element of the ground field. Then for even characteristic:
\begin{itemize}
\item $t \in \Omega^+(2d-2, q)$ is formed by $({t^v} B(t))^{\delta^z} \in \Omega^-(2d, q)$;
\item $r \in \Omega^+(2d-2, q)$ is formed by $({r^v} B(r))^{\delta^z} \in \Omega^-(2d, q)$;
\item $t^\prime \in \Omega^+(2d-2, q)$ is formed by $(({t^v} B(t))^{\delta^z})^s \in \Omega^-(2d, q)$;
\item $r^\prime \in \Omega^+(2d-2, q)$ is formed by $(({r^v} B(r))^{\delta^z})^s \in \Omega^-(2d, q)$;
\end{itemize}

\end{lemma}

\begin{proof}
Firstly, consider $({t^v})^{\delta}$. This gives a matrix of the following form:

$$\left(\matrix{1 & \omega^{-2} & 0 & 0 & 0 & \ldots & 0 & * & * \cr
0 & 1 & 0 & 0 & 0 & \ldots & 0 & 0 & 0 \cr
0 & 0 & 1 & 0 & 0 & \ldots & 0 & 0 & 0 \cr
0 & 0 & 0 & 1 & 0 & \ldots & 0 & 0 & 0 \cr
0 & 0 & 0 & 0 & 1 & \ldots & 0 & 0 & 0 \cr
\vdots & \vdots & \vdots & \vdots & \vdots & \ddots & \vdots & \vdots & \vdots \cr
0 & 0 & 0 & 0 & 0 & \ldots & 1 & 0 & 0 \cr
0 & * & 0 & 0 & 0 & \ldots & 0 & 1 & 0 \cr
0 & * & 0 & 0 & 0 & \ldots & 0 & 0 & 1 \cr
}\right),$$
where the asterisks represent arbitrary elements of GF($q$).

Now we conjugate by $t^{v^2}$, which only affects the top left 4 $\times$ 4 block. A simple calculation shows that this gives the following matrix:

$$B(t) = (t^{v^2})^{-1} ({t^v})^{\delta} t^{v^2} = \left(\matrix{1 & \omega^{-2} & 0 & x & 0 & \ldots & 0 & * & * \cr
0 & 1 & 0 & 0 & 0 & \ldots & 0 & 0 & 0 \cr
0 & x & 1 & 0 & 0 & \ldots & 0 & 0 & 0 \cr
0 & 0 & 0 & 1 & 0 & \ldots & 0 & 0 & 0 \cr
0 & 0 & 0 & 0 & 1 & \ldots & 0 & 0 & 0 \cr
\vdots & \vdots & \vdots & \vdots & \vdots & \ddots & \vdots & \vdots & \vdots \cr
0 & 0 & 0 & 0 & 0 & \ldots & 1 & 0 & 0 \cr
0 & * & 0 & 0 & 0 & \ldots & 0 & 1 & 0 \cr
0 & * & 0 & 0 & 0 & \ldots & 0 & 0 & 1 \cr
}\right),$$
where $x$ is an element of GF($q$) and the asterisks represent the same arbitrary elements of GF($q$) as in the first step.

We now need to work out how to set the (1, 2) entry to 0. By direct calculation, we can see that conjugating the above matrix by $\delta^{-1}$ gives:

$$\left(\matrix{1 & 1 & 0 & y & 0 & \ldots & 0 & 1 & 0 \cr
0 & 1 & 0 & 0 & 0 & \ldots & 0 & 0 & 0 \cr
0 & y & 1 & 0 & 0 & \ldots & 0 & 0 & 0 \cr
0 & 0 & 0 & 1 & 0 & \ldots & 0 & 0 & 0 \cr
0 & 0 & 0 & 0 & 1 & \ldots & 0 & 0 & 0 \cr
\vdots & \vdots & \vdots & \vdots & \vdots & \ddots & \vdots & \vdots & \vdots \cr
0 & 0 & 0 & 0 & 0 & \ldots & 1 & 0 & 0 \cr
0 & 0 & 0 & 0 & 0 & \ldots & 0 & 1 & 0 \cr
0 & \gamma + \gamma^q & 0 & 0 & 0 & \ldots & 0 & 0 & 1 \cr
}\right),$$
where $y \in$ GF($q$) and $\gamma$ is the primitive element of GF($q^2$). As the asterisked entries were not changed by conjugating by $t^{v^2}$, the portion of the matrix outside the $4 \times 4$ block will look like $t^v$, since ${t^v}^{\delta \delta^{-1}} = t^v$.

Pre-multiplying by $t^v$ then gives:

$$\left(\matrix{1 & 0 & 0 & y & 0 & \ldots & 0 & 0 & 0 \cr
0 & 1 & 0 & 0 & 0 & \ldots & 0 & 0 & 0 \cr
0 & y & 1 & 0 & 0 & \ldots & 0 & 0 & 0 \cr
0 & 0 & 0 & 1 & 0 & \ldots & 0 & 0 & 0 \cr
0 & 0 & 0 & 0 & 1 & \ldots & 0 & 0 & 0 \cr
\vdots & \vdots & \vdots & \vdots & \vdots & \ddots & \vdots & \vdots & \vdots \cr
0 & 0 & 0 & 0 & 0 & \ldots & 1 & 0 & 0 \cr
0 & 0 & 0 & 0 & 0 & \ldots & 0 & 1 & 0 \cr
0 & 0 & 0 & 0 & 0 & \ldots & 0 & 0 & 1 \cr
}\right).$$

Hence conjugating the above element by $\delta^z$ gives the required matrix and so we're done.

The other three equations can be shown to hold by a similar method.
\end{proof}



\end{document}

