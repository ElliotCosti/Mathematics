\documentclass[12pt]{article}
\usepackage{amssymb}
\usepackage[algo2e,ruled,linesnumbered]{algorithm2e} % for algorithms
\hoffset -25truemm
\usepackage{latexsym}
\oddsidemargin=30truemm             %%
\evensidemargin=25truemm            %% inner margin 30mm, outer margin 25mm
\textwidth=155truemm                %%
\voffset -25truemm
\topmargin=25truemm                 %% top margin of 25mm
\headheight=0truemm                 %% no head
\headsep=0truemm                    %% no head
\textheight=220truemm               
\renewcommand{\thefootnote}{}
\newtheorem{definition}{Definition}[section]
\newtheorem{lemma}[definition]{Lemma}
\newtheorem{theorem}[definition]{Theorem}
\newtheorem{corollary}[definition]{Corollary}
\newtheorem{remark}[definition]{Remark}
\newtheorem{problem}[definition]{Problem}
\newenvironment{proof}{\normalsize {\sc Proof}:}{{\hfill $\Box$ \\}}

\def\X{{\bf X}}
\def\O{{\bf O}}

\def\SL{{\rm SL}}
\def\GL{{\rm GL}}
\def\U{{\rm U}}
\def\PSL{{\rm PSL}}
\def\PSp{{\rm PSp}}
\def\PSU{{\rm PSU}}
\def\GF{{\rm GF}}
\def\Sp{{\rm Sp}}
\def\SU{{\rm SU}}
\def\SX{{\rm SX}}
\def\PX{{\rm PX}}
\def\GX{{\rm GX}}
\def\PSX{{\rm PSX}}
\def\PGL{{\rm PGL}}
\def\q{\quad}
\def\centreline{\centerline}

\begin{document}

\title{Constructive recognition of classical groups in odd characteristic} 
\author{C.R.\ Leedham-Green and E.A. O'Brien}
\date{}
\maketitle

\begin{abstract}
Let $G = \langle X \rangle \leq \GL(d, F)$ be 
one of $\SL (d, F)$, $\Sp (d, F)$, or $\SU (d, F)$
where $F$ is a finite field of odd characteristic. 
We present recognition algorithms to construct
standard generators for $G$ which allow us
to write an element of $G$ as a straight-line
program in $X$.
The algorithms are Las Vegas polynomial-time, 
subject to the existence of a discrete log oracle for $F$. 
\end{abstract}

\footnote{This work was supported in part by the Marsden Fund of
New Zealand via grant UOA 412.
2000 {\it Mathematics Subject Classification}.
Primary 20C20, 20C40.}

\section{Introduction}
\label{intro}

A major goal of the ``matrix recognition project"
is the development of efficient
algorithms for the investigation of 
subgroups of $\GL(d, F)$ where $F$ 
is a finite field. 
We refer to the recent survey \cite{OBrien05}
for background related to this work.
A particular aim is to 
identify the composition factors
of $G \leq \GL(d, F)$. If a problem
can be solved for the composition factors,
then it can be frequently be solved for $G$.

One may intuitively think of a {\it straight-line program} (SLP)
for $g \in G = \langle X \rangle$ as an efficiently stored group word
on $X$ that evaluates to $g$.  For a formal definition, we
refer the reader to \cite[p.\ 10]{Seress03} or  
Section \ref{SLP}.
A critical property of an SLP is 
that its length is proportional to the number of 
multiplications and exponentiations used in
constructing the corresponding group element. 
Babai \& Szemer{\'e}di \cite{BabaiSzemeredi84}
prove that every element of 
$G$ has an SLP in $X$ of length at most $O(\log^2|G|)$.

Informally, a {\it constructive recognition algorithm}
constructs an explicit isomorphism
between a group $G$ and a ``standard" (or natural)
representation of $G$, and exploits this isomorphism
to write an arbitrary element of $G$ as
an SLP in its defining generators. 
%That the constructive membership problem has a solution was established

In this paper we present constructive
recognition algorithms for certain of the classical groups.
Let $\SX(d,q)$ denote $\SL(d,q)$, or $\Sp(d,q)$ (for 
even $d$), or $\SU(d,q)$, and let $\GX(d,q)$ denote
$\GL(d,q)$, or $\Sp(d,q)$, or $\U(d,q)$. 
More precisely, we present and analyse two algorithms that
take as input a generating subset $X$ of $\SX(d,q)$ for $q$ odd, and
return as output {\it standard generators} of this group as 
SLPs in $X$. Usually, these  
generators are defined with respect to a 
basis different to that for which $X$ was defined, 
and a change-of-basis 
matrix is also returned to relate these bases.
We can readily write a given element of a classical group 
as an SLP in terms of these generators, using row-and-column operations.

Similar algorithms are under development for the orthogonal groups
in odd characteristic.  Further, characteristic 2 can also be 
addressed in the same style, but the resulting algorithms are more complex.
We shall consider these cases in later papers.

Our principal result is the following.
\begin{theorem} \label{main}
Let $G = \langle X \rangle \leq \GL(d, q)$ denote 
$\SL(d,q)$, or $\Sp(d,q)$ for even $d$, or $\SU(d,q)$
where, in all cases, $q$ is odd. 
There are Las Vegas algorithms which,
given the input $X$, 
construct a new standard generating set $S$ 
for $G$ having the property that 
an SLP of length $O(d^2 \log q)$
can be found from $S$ for any $g \in G$. 
Assuming the existence of a discrete log oracle 
for $\GF (q)$, 
the algorithm to construct
$S$ runs in $O(d (\xi + d^3 \log q + \chi))$ field operations, 
where $\xi$ is the cost of constructing an independent 
(nearly) uniformly distributed random element of $G$,
and $\chi$ is the cost of a call to a discrete log oracle for $\GF(q)$.
\end{theorem}

We prove this theorem by exhibiting algorithms
with the stated complexity. If we assume that
a random element can be constructed in 
$O(d^3)$ field operations, then $O(d^4)$ is 
an upper bound to the complexity for fixed $q$. 

Brooksbank's algorithms \cite{Brooksbank03} 
for the natural 
representation of $\Sp(d, q)$, $\SU(d, q)$, and $\Omega^\epsilon(d, q)$ 
have complexity $O(d^5)$ for fixed $q$. More precisely, 
the complexity of his algorithm is 
$$O(d^3 \log q (d + \log d \log^3 q) + \xi(d + \log \log q) + d^5 \log^2 q
   + \chi(\log q)).$$
%$O(d^2 \log d (\mu + \chi \log q + d\log^4 q))$,
%where $\epsilon$ is the cost of constructing  a independent 
%(nearly) uniformly distributed random element,
%and $\chi$ is the cost of a call to a discrete log oracle for $\GF(q)$.
The algorithm of Celler \& Leedham-Green \cite{CellerLeedhamGreen98}
for $\SL(d, q)$ has complexity $O(d^4 \cdot q)$.

The two algorithms presented here reflect a tension 
between two competing tasks: the speed of construction
of the standard generators, and minimising the 
length of the resulting SLPs for the standard 
generators in $X$.
The first is designed for optimal efficiency; 
the second to produce short SLPs.
We consider this topic in more detail in Section \ref{SLP}.

We establish some notation. 
Let $g\in G \leq \GL(d, q)$, 
let $\bar{G}$ denote $G / G \cap Z$
where $Z$ denotes the centre of $\GL(d, q)$,
and let $\bar{g}$ denote the image of $g$ in $\bar{G}$.
The {\it projective centraliser} of $g \in G$
is the preimage in $G$ of $C_{\bar{G}} (\bar{g})$.
Further $g \in G$ is a {\it projective involution} 
if $g^2$ is scalar, but $g$ is not.

A central component of both algorithms
is the use of involution centralisers.
In Section \ref{cent} we summarise
the structure of involution centralisers
for elements of classical groups in odd characteristic.
In Section \ref{standard} we define
standard generators for the classical groups.

In Sections \ref{Alg1} and \ref{Alg2} the two 
algorithms are described. They
rely on finding involutions  whose eigenspaces  have
approximately the same  dimension in the case of the first algorithm,
and exactly the same dimension  in the second. The probability
of obtaining such involutions by random search
is analysed in Sections \ref{Involution} and Section \ref{Equal}.
The centraliser of an involution  is constructed using an 
algorithm of Bray \cite{Bray}; this is considered
in Section \ref{Bray}. The base cases of the
algorithms (when $d \leq 4$) are discussed in Section \ref{base}. 
We frequently compute high powers of elements of linear groups; 
an algorithm for doing this
efficiently is described in Section \ref{Exp}. The use of powering to
construct the direct factors from the direct product of two classical
groups is discussed in Section \ref{Pow}. 
The complexity of the algorithms and the 
length of the resulting SLPs for the standard generators
are discussed in Section \ref{Analysis} and \ref{SLP}.  
%In Section \ref{grey} we consider the modifications
%necessary to apply the algorithms to other representations 
%of the groups in defining characteristic. 
Finally we report on our implementation of the algorithm, 
publicly available in {\sc Magma} \cite{Magma}.

\section{Centralisers of involutions in classical groups}\label{cent} 
We briefly review the structure of involution centralisers 
in (projective) classical groups 
defined over fields of odd characteristic.
A detailed account can be found in \cite{GLS3}.
\begin{enumerate}
\item 
If $u$ is an involution in $\SL(d,q)$, with eigenspaces $E_+$ and
$E_-$, then the centraliser of $u$ in $\SL(d,q)$ is
$(\GL(E_+)\times\GL(E_-))\cap\SL(d,q)$. The centraliser of the
image of $u$ in $\PSL(d,q)$ is the image of the centraliser of $u$ in
$\SL(d,q)$ if $E_+$ and $E_-$ have different dimensions. If
$E_+$ and $E_-$ have the same dimension, then in $\PSL(d,q)$ these
eigenspaces may be interchanged by the centraliser of the image of
$u$, which is now the image of $(\GL(d/2,q)\wr C_2)\cap\SL(d,q)$ in
$\PSL(d,q)$.

\item 
If $u$ is an involution in $\Sp(2n,q)$, with eigenspaces $E_+$ and
$E_-$, these spaces are  mutually orthogonal, and the form restricted
to either is non-singular. Thus the centraliser of $u$ is 
$\Sp(E_+) \times \Sp(E_-)$. The centraliser of the image of $u$ in
$\PSp(2n,q)$ is the image of $\Sp(E_+)\times\Sp(E_-)$, except when the
eigenspaces have the same dimension, when the centraliser again
permutes the eigenspaces. An element of the projective
centraliser permuting the eigenspaces sends $(v,w)$ to
$(w\theta,-v\theta)$, where  $\theta$ is an isometry that permutes
these spaces, so the image of $\Sp(E_+)\times\Sp(E_-)$ has index 2
in the projective centraliser.

\item 
If $u$ is an involution in $\SU(d,q)$, the situation is similar.
Again the eigenspaces of $u$ are mutually orthogonal, and the form
restricted to the eigenspaces is non-degenerate. The centraliser of
$u$ in $\SU(d,q)$  is $(\U(E_+)\times\U(E_-))\cap\SU(d,q)$. The
centraliser of the image of $u$ in $\PSU(d,q)$ is the image of the
centraliser of $u$ in $\SU(d,q)$  except where the eigenspaces of
$u$ have the same dimension, when the centraliser is the image of
$(\U(d/2,q)\wr C_2)\cap\SU(d,q)$ in $\PSU(d,q)$.
\end{enumerate}

\section{Standard generators for classical groups}
\label{standard}

We now describe {\it standard generators} for
the perfect classical groups $\SL(d,q)$, $\Sp(d,q)$ and $\SU(d,q)$
where $q$ is odd in all cases.

We use the notation $\SX(d,q)$ to denote any  one of these 
groups, and $\PX(d,q)$ to denote the corresponding central quotient.

Let $V$ be the natural module for a perfect classical group $G$ of the
above kind.  We define a {\it hyperbolic} basis for $V$ as
follows. If $G=\SL(d,q)$ then any ordered basis is hyperbolic. If
$G=\Sp(d,q)$ then $d$ is even, say $d=2n$, and $G$ preserves a
non-degenerate symplectic form. A hyperbolic basis for $V$ is then an
ordered basis of the form $\{e_1,f_1,\ldots,e_n,f_n\}$, where, if the
image of a pair of vectors $(v,w)$ under the form is written as $v.w$,
then $e_i.e_j=f_i.f_j=0$ for all $i,j$ (including the case $i=j$), and
$e_i.f_j=0$ for $i\ne j$, and $e_i.f_i=-f_i.e_i=1$ for all $i$. If
$G=\SU(d,q)$, and $d=2n$ is even, then the definition is exactly as
for the case of $\Sp(d,q)$ except that, the form being hermitian, the
condition  $e_i.f_i=-f_i.e_i=1$ for all $i$ is replaced by the
condition $e_i.f_i=f_i.e_i=1$ for all $i$. If $G=\SU(d,q)$, where
$d=2n+1$, a hyperbolic basis is of the form
$(e_1,f_1,\ldots,e_n,f_n,v)$, where the above equations hold, and in
addition $e_i.v=f_i.v=0$ for all $i$, and $v.v=1$. 

That a hyperbolic basis exists for $V$ is easily established;
it can be constructed from an arbitrary basis in
$O(d^3)$ field operations. For details, see 
for example, \cite[Chapter 2]{Grove02}.

The standard generators introduced here are defined in terms of
a hyperbolic basis for $V$, which will be defined in terms of the
given basis by a change-of-basis matrix. 

It is of course a triviality to  {\it write down} the standard generators
(once they have been defined).  However we must construct these elements
as SLPs in the given generators.

Once a hyperbolic basis has been chosen for $V$, the Weyl group of $G$
can be defined as a section of $G$, namely as the group of monomial 
matrices in $G$ modulo diagonal
matrices, thus defining a subgroup of the symmetric group $S_d$. For
$G=\SL(d,q)$, this group is $S_d$. For $\Sp(2n,q)$ the Weyl group 
is the subgroup of $S_{2n}$ that preserves the system of imprimitivity with blocks
$\{e_i,f_i\}$ for $1\le i\le n$, and is thus $C_2\wr S_n$. For
each of $\SU(2n,q)$ and $\SU(2n+1,q)$, the Weyl group 
is also $C_2\wr S_n$. 

%Our standard generators for $G$ are in two intersecting sets. The first is
%a generating set for a minimal case, and the second consists
%of signed permutation matrices that generate the Weyl group of $G$, modulo 
%diagonal matrices. The
%permutation matrices are signed, since otherwise matrices of determinant
%$-1$ would be needed in some cases. 

%The minimal cases are as follows.
%If $G=\SL(d,q)$ then the minimal case is $\SL(2,q)$. If $G=\Sp(d,q)$
%the minimal case is $\Sp(4,q)$. If $G=\SU(2n,q)$ the minimal case
%is $\SU(4,q)$, and if $G=\SU(2n+1,q)$ the minimal case is
%$\SU(3,q)$.
%Note that $\Sp(2,q)=\SU(2,q)=\SL(2,q)$.

In detail, the standard generating set $Y$ for $G$ with respect
to a hyperbolic basis for $V$ is as follows:
\begin{enumerate}
\item 
If $G=\SL(d,q)$ then $Y=\{s,\delta,u,v\}$ is defined as follows.
All but $v$ lie in the copy of $\SL(2,q)$ that normalises $\langle
e_1,e_2\rangle$ and centralises $\langle e_3,\ldots,e_d\rangle$, and
these act on  $\langle e_1,e_2\rangle$ with respect to this ordered
basis as follows: 
$$s=\left(\matrix{1&1\cr0&1\cr}\right)\quad
%t_2=\left(\matrix{1&0\cr1&1\cr}\right)$$            
\delta = \left(\matrix{\omega&0\cr0&\omega^{-1}\cr}\right)\quad
u=\left(\matrix{0&1\cr-1&0\cr}\right)$$  
where $\omega$ is a primitive element for $\GF(q)$.
Finally $v$ is defined by
$e_1\mapsto e_d\mapsto -e_{d-1}\mapsto -e_{d-2}\mapsto
-e_{d-3}\cdots \mapsto -e_1$, [NOT TRUE: Signs wrong -- correct description is below]
Finally 
$$v= \left(\matrix{ 0 & 0 &  \ldots & 0 & 1 \cr 
                  -1 & 0 &  \ldots & 0 & 0 \cr 
                   0 & -1 &  \ldots & 0 & 0 \cr 
              \ldots  & \ldots    & \ldots & \ldots   & \ldots \cr
                   0 &  0  & \ldots & -1 & 0 \cr 
}
\right)\quad$$
the signs chosen to ensure that $v$ has determinant 1. 
Clearly $u$ and $v$ generate the Weyl
group, modulo the group of diagonal matrices.
Note that $v=u$ if $d=2$.

\item 
If $G=\Sp(d,q)$, where $d=2n$ and $n>1$, then
$Y=\{s,t,\delta,u,v\}$ where $s$ and $ \delta$ 
are as defined for $\SL(d,q)$; and $t$ is the element of $G$ 
that centralises $\langle e_i,f_i:i>2\rangle$, 
normalises the space $\langle e_1,f_1,e_2,f_2\rangle$,
and acts on the space with matrix referred to this hyperbolic basis given by
$$t=\left(\matrix{1&0&0&0\cr0&1&1&0\cr0&0&1&0\cr1&0&0&1\cr}\right);$$ 
and $u$ and $v$ are permutation matrices defined by
$u=(e_1,e_2)(f_1,f_2)$ and 
$v=(e_1,e_2,\ldots,e_n)(f_1,f_2,\ldots,f_n)$.
Note that $v=u$ if $n=2$.

\item 
If $G=\SU(d,q)$, where $d=2n$ and $n>1$, then
$Y=\{s,t,x,\delta, u,v\}$, where $u$ and $v$ 
are as defined for $\Sp(d,q)$;  and $\delta$, $x$, and $s$ centralise
all but the first two basis vectors, normalise the space spanned
by the first two basis vectors, and act on this space, with respect to
the ordered basis $(e_1,f_1)$ as 
$$\delta = \left(\matrix{\omega^{q + 1}&0\cr0&\omega^{-(q+1)}\cr}\right)\quad
s= \left(\matrix{1&\alpha\cr0&1\cr}\right) \quad
x= \left(\matrix{0&\alpha\cr \alpha^{-q}&0\cr}\right)$$
where $\omega$ is a primitive element of $\GF(q^2)$,
and $\alpha=\omega^{(q+1)/2}$; and $t$
centralises all but the first four basis vectors, normalises the space
spanned by the first four basis elements, and acts on this space,
with respect to the ordered basis $(e_1,f_1,e_2,f_2)$ as
$$t=\left(\matrix{1&0&\omega&0\cr0&1&0&0\cr0&0&1&0\cr0&-\omega^q&0&1\cr}\right).$$

\item
If $G=\SU(d,q)$, where $d=2n+1$ and $n>0$, then
$Y=\{s,t,x,y,\delta,u,v\}$, where the generators except for $x$ and $y$
are as for the even case, but with $t$ omitted if $d=3$. Now $x$ and $y$
centralise all but the first two and the last basis vectors, normalise the
space that these three vectors span, and act on this space with respect to
the ordered basis $(e_1,f_1,v)$ with matrices of the form
$$\left(\matrix{1&\beta&\gamma\cr
0&1&0\cr
0&-\gamma^q&1\cr}\right).$$
In each case the equation $\gamma^{q+1}+\beta+\beta^q=0$ is satisfied.
The two values of $\beta$ (one for $x$ and one for $y$)
are chosen so that they span $\GF(q^2)$ over $\GF(q)$.
\end{enumerate}

In all cases, these generators have the property that it is easy to construct from them
any element of any root group, and consequently these generators generate
the group in question.  The root groups are defined with respect to a maximal
split torus, which we take to be the group of diagonal matrices in the group in question
(with the additional restriction, in the case of $\SU(2n+1,q)$, that the final diagonal
entry is 1).  These root groups can then be constructed as follows.

\begin{enumerate}
\item If $G=\SL(d,q)$ then the root groups are of the form
$\{s^{\delta^ig}:0\le i\le q-2\}$,
where $g\in\langle u,v\rangle$.

\item If $G=\Sp(2n,q)$ then the root groups corresponding to
long roots are again of the form $\{s^{\delta^ig}:0\le i\le q-2\}$,
where $g\in\langle u,v\rangle$, and root groups corresponding to
short roots are of the form $\{t^{\delta^ig}:0\le i\le q-2\}$,
where $g\in\langle u,v\rangle$.

\item If $G=\SU(2n,q)$ then the root groups are defined by the same formulae
as in the case $G=\Sp(2n,q)$.

\item If $G=\SU(2n+1,q)$ then the root groups are as in the previous case, together
with a family of two-parameter groups, namely the set of elements that
normalise the space spanned by $\{e_i,f_i,v\}$, centralise the other basis elements,
and act on the above 3-space, with respect to the ordered basis $(e_i,v,f_i)$, as the
set of matrices  of the form
$$\left(\matrix{1&\beta&\gamma\cr
0&1&-\beta^q\cr
0&0&1\cr}\right),$$
where the equation $\beta^{q+1}+\gamma+\gamma^q=0$ is satisfied.
This is a non-abelian group of order $q^3$. Its derived group and
centre coincide, and these form the set of matrices with $\beta=0$.  If $i=1$ then
conjugating by $\delta$ multiplies all three matrix entries above the diagonal
by a primitive element of $\GF(q)$, so these root groups can be written as
$\{(x^{\delta^i}y^{\delta^j}[x,y]^{\delta^k})^g:0\le i,j,k\le q-2\}$ for
$g\in\langle u,v\rangle$.
\end{enumerate}

Define $Y_0:=\{s,\delta,u,v\}$. 
If $G$ is $\Sp(2n, q)$ where $n > 1$, 
or $\SU (2n, q)$ or $\SU(2n + 1, q)$ for $n > 0$,
then $Y_0$ generates $\SL(2,q)\wr S_n$.
For these groups, the first and major step in our algorithm constructs $Y_0$.
As a final step, we construct the additional 
element or elements to obtain $Y$. 

If $G = \SL(2n, q)$, the first step also constructs 
$\SL(2,q)\wr S_{n}$; in a final step we obtain the $2n$-cycle. 

\section{Algorithm {\tt One}}
\label{Alg1}
Algorithm {\tt One} takes as input a generating set $X$ for
$G=\SX(d,q)$, and returns standard generators for $G$ as SLPs in $X$.
The generators are in standard form 
when referred to a basis constructed  by the algorithm. The change-of-basis 
matrix that expresses this basis in terms of the standard basis for the natural
module is also returned.

The algorithm employs a ``divide-and-conquer" strategy. Define a
{\it strong involution} in $\SX(d,q)$ to be an involution whose
eigenspaces have dimensions in the range $(d/3,2d/3]$ if $d>5$, and 
in the range $[2,3]$ if $d=5$. For $\Sp(d,q)$ and $\SU(d,q)$ the
eigenspaces of an involution $u$ are mutually orthogonal, and the form
restricted to either eigenspace is non-degenerate. Thus, if these
spaces have dimensions $e$ and $d-e$, then the derived subgroup of the
centraliser of $u$ in $\SX(d,q)$ is $\SX(e,q)\times\SX(d-e,q)$. Note
that the dimension of the $-1$-eigenspace of an involution in
$\SX(d,q)$ is always even.

Algorithm {\tt OneEven} addresses the case of even $d$.

\begin{algorithm2e}[H] 
\caption{\tt OneEven$(X,{\it type})$}
\label{alg1:even}
\tcc{
$X$ is a generating set for
the perfect classical group $G$
in odd characteristic, of type SL or Sp or SU, in even dimension.
Return standard generating set $Y_0$ for a copy 
of $\SL(2, q) \wr S_{d/2} \leq G$, the 
SLPs for the elements of $Y_0$, the change-of-basis matrix,
and generators for centraliser of involution $k$ defined in line 13.
}
\Begin{
 $d$ := the rank of the matrices in $X$; 

if $d \leq 4$ then return {\tt BaseCase} (X, {\it type, false});

$q$ := the size of the field over which these matrices are defined;   

if {\it type} = SU then $q := q^{1/2}$;  

Find by random search $g \in G:=\langle X\rangle$ of 
even order such that $g$ powers to 
a strong involution $h$;

Let $n$ be the dimension of the $+1$-eigenspace of $h$;

Find generators for the centraliser  $C$ of $h$ in $G$;

In the derived subgroup $C'$ of $C$ find generating sets 
$X_1$ and $X_2$ for the direct factors of $C'$;

$(s_{1},\delta_1,u_1,v_1)$ := {\tt OneEven}$(X_1,{\it type})$;

$(s_2,\delta_2,u_2,v_2)$ := {\tt OneEven}$(X_2,{\it type})$;

Let $(e_1, f_1, \ldots, e_n, f_n, e_{n+1}, f_{n+1}, \ldots, e_d, f_d)$
be the concatenation of the hyperbolic bases constructed in lines 10 and 11;

$k := (\delta_1^{(q-1)/2})^{v_1^{-1}}\delta_2^{(q-1)/2}$;

Find generators for the centraliser $D$ of $k$ in $G$;

In the derived subgroup $D'$ of $D$ find a generating set $X_3$ for 
the direct factor
that acts faithfully on $\langle e_n,e_{n+1},f_n,f_{n+1}\rangle$;

In $\langle X_3\rangle$ find the permutation matrix 
$j=(e_n,e_{n+1})(f_n,f_{n+1})$;

$v := v_1 j v_2$;

return $(s_1,\delta_1,u_1,v)$, the change-of-basis matrix, and $X_3$.
}
\end{algorithm2e}

If the type is SL, then the centraliser of $h$ is 
$\GL(E_+)\times\GL(E_-)\cap\SL(d,q)$ where $E_+$ and $E_-$ are the
eigenspaces  of $h$. If the type is Sp, it is
$\Sp(E_+)\times\Sp(E_-)$, and if the type is SU, it is
$(U(E_+)\times U(E_-))\cap\SU(d,q)$. In these last two
cases the restriction of the form to each of the eigenspaces is
non-singular, and each eigenspace is orthogonal to the other. Thus
the concatenation of a hyperbolic basis of one eigenspace with a
hyperbolic basis for the other eigenspace is a hyperbolic basis for
the whole space. 

As presented the algorithm has been simplified. 
In lines 11 and 12 we have ignored
the change-of-basis matrices that are also returned; the change-of-basis 
returned at line 18 is the concatenation of these bases.

We make the following additional observations on Algorithm {\tt OneEven}. 

\begin{enumerate}
\item 
The SLPs that express the standard generators 
in terms of $X$ are also returned.

\item 
Generators for the involution centraliser in line 
8 are constructed using the algorithm of Bray \cite{Bray},
see Section \ref{Bray}. Of course, $g$ is an element of
this centraliser. We need only a subgroup of the centraliser that 
contains its derived subgroup. 

\item 
The generators for the direct summands
constructed in line 9 are constructed by forming suitable powers of
the generators of the centraliser. This step is discussed in
Section \ref{Pow}.

\item 
The algorithms for the {\tt BaseCase} calls in lines 3 and 16 are 
discussed in Section \ref{base}.

\item 
The search for an element that powers to a suitable involution is
discussed in Section \ref{Involution}.

\item 
The recursive calls in lines 10 and 11 are
in smaller dimension. Not only are the groups of
smaller Lie rank, but the matrices have degree at most $2d/3$.
Hence these calls only affect the time or space complexity of the
algorithm up to a constant multiple; however they contribute to the length of
the SLPs produced.

\item 
Note that $k$ in line 12 is an involution: 
its $-1$-eigenspace is $\langle e_n,f_n,e_{n+1},f_{n+1}\rangle$ and its
$+1$-eigenspace is
$\langle e_1,f_1,\ldots, e_{n - 1}, f_{n-1}, e_{n+2},f_{n+2}, \ldots, e_d, f_d \rangle$.
\end{enumerate}
 
Algorithm {\tt OneOdd}, which considers 
the case of odd degree $d$, is similar
to Algorithm {\tt OneEven}.
Our commentary on the even degree case also applies. 

\begin{algorithm2e}[H]
\caption{\tt OneOdd$(X,{\it type})$}
\label{alg1:odd}
\tcc{
$X$ is a generating set for
the perfect classical group $G$
in odd characteristic and degree, of type SL or SU.
If $G = \SL(d, q)$, then 
return standard generating set $Y$ for $G$;
if $G = \SU(d, q)$ then 
return generating set $Y_0$ for $\SL(2, q) \wr S_{(d - 1)/2}$. 
Also return the SLPs for elements of this generating set, 
the change-of-basis matrix,
and generators for centraliser of involution $k$ defined in line 13.
}

\Begin{
$d$ := the rank of the matrices in $X$;

if $d = 3$ then return {\tt BaseCase} (X, {\it type, false});

 $q$ := the size of the field over which these matrices are defined;  

 if {\it type} = SU then $q := q^{1/2}$;  

Find by random search $g \in G:=\langle X\rangle$ of even order
 such that $g$ powers to a strong involution $h$;

Let $n$ be the dimension of the $+1$-eigenspace of $h$;

Find generators for the centraliser $C$ of $h$ in $G$;

In the derived subgroup $C'$ of $C$ find generating 
sets $X_1$ and $X_2$ for the direct factors 
of $C'$, where $X_1$ centralises the $-1$-eigenspace of $h$;

 $(s_1,\delta_1,u_1,v_1)$ := {\tt OneOdd}$(X_1,{\it type})$;

 $(s_2,\delta_2,u_2,v_2)$ := {\tt OneEven}$(X_2,{\it type})$;

Let $(e_1, f_1, \ldots, e_n, f_n, e_{n+1}, f_{n+1}, \ldots, e_d, f_d)$
be the concatenation of the hyperbolic bases constructed in lines 10 and 11;

 $k := (\delta_1^{(q-1)/2})^{v_1^{-1}}\delta_2^{(q-1)/2}$;

Find generators for the centraliser $D$ of $k$ in $G$;

In the derived subgroup $D'$ of $D$ find a generating set 
$X_3$ for the direct factor
that acts faithfully on $\langle e_n,e_{n+1},f_n,f_{n+1}\rangle$;

In $\langle X_3 \rangle$ find the permutation matrix 
$j=(e_n,e_{n+1})(f_n,f_{n+1})$;

$v := v_1 j v_2$;

return $(s_1,\delta_1,u_1,v)$, the change-of-basis matrix and $X_3$.
}
\end{algorithm2e}

We summarise the main algorithm as Algorithm {\tt OneMain}. 

If $G = \SL(2n, q)$ and $n > 2$, we construct an additional element
$a$ which is used to construct a $2n$-cycle. 
It is an element of the centraliser of the involution $k$ 
computed in each of {\tt OneEven} and {\tt OneOdd}. It acts 
on the subspace spanned by the basis vectors $e_n, f_n, e_{n+1}, f_{n +1}$ as follows:
$$\left(\matrix{0&1&0&0\cr0&0&1&0\cr0&0&0&1\cr-1&0&0&0\cr}\right)$$ 
and centralises the remaining $2n - 4$ basis vectors. 
The product $av$ is a $2n$-cycle; we perform a change-of-basis that 
permutes the basis and changes sign to produce the desired one.

\begin{algorithm2e}[H]
\caption{\tt OneMain$(X,{\it type})$}
\label{alg1:main}
\tcc{ $X$ is a generating set for the perfect classical group $G$
in odd characteristic, of type SL or Sp or SU.
Return standard generators $Y$ for $G$, the SLPs
for these generators, and change-of-basis matrix.}

\Begin{
 
$d$ := the rank of the matrices in $X$;

if $d \leq 4$ then return {\tt BaseCase}(X, {\it type, true});

  \eIf{$d$ is odd}
   {
       $(s, \delta, u, v)$, $X_3$ := {\tt OneOdd}(X,{\it type});

       \If{type={\rm SL}}
       {
          return $(s,\delta,u,v)$ and the change-of-basis matrix;
       }
   }{
       $(s,\delta, u, v)$, $X_3$ := {\tt OneEven}$(X,{\it type})$;
       }


\eIf {type = {\rm SL}} {
In $\langle X_3 \rangle$ construct additional element $a$ defined above;

   $v := (av)^{-1}$; 

   change basis to obtain desired $d$-cycle $v$;

   return $(s,\delta,u,v)$ and the change-of-basis matrix.
}
{
In $\langle X_3 \rangle$ construct additional element $t$ defined above; 

return $(s, t,\delta,u,v)$ and the change-of-basis matrix.
}
}
\end{algorithm2e}

The correctness and complexity of this algorithm, 
and the lengths of the resulting SLPs for the 
standard generators, are discussed in the rest
of this paper.

\section{Algorithm {\tt Two}} 
\label{Alg2}

We present a variant of the algorithms in Section \ref{Alg1} based on  
one recursive call rather than two. Again we denote 
the groups $\SL(d,q)$, $\Sp(d,q)$
and $\SU(d,q)$ by $\SX(d,q)$, and the corresponding projective group
by $\PX(d,q)$.

The key idea is as follows. Suppose that $d$ is a multiple of 4.  
We find an involution $h \in \SX (d, q)$, as in line 7 of {\tt OneEven},
but insist that it should have both eigenspaces of dimension $d/2$. 

Let $\bar{h}$ be the image of $h$ in $\PX(d,q)$.
The centraliser of $\bar{h}$ in $\PX(d,q)$
acts on the pair of eigenspaces $E_+$ and $E_-$ of $h$, 
interchanging them. In practice, we construct the
projective centraliser of $h$ by applying the algorithm 
of \cite{Bray} to $\bar{h}$ and $\PX(d, q)$, but with the
additional requirement that we find $\bar{g} \in \PX(d, q)$ 
that interchanges the two eigenspaces. 

If we now find recursively a set $\cal S$ of
standard generators for $\SX(E_+)$ with respect the basis $\cal B$,
then ${\cal S}^g$ is a set of standard generators for $\SX(E_-)$ with
respect to the basis ${\cal B}^g$. We now use these 
to construct standard generators for 
$\SX(d,q)$ exactly as in Algorithm {\tt One}.

If $d$ is an odd multiple of 2, we find an involution with one
eigenspace of dimension exactly 2. The centraliser of this
involution gives us $\SX(2,q)$ and $\SX(d-2,q)$. The $d-2$
factor is now processed as above, since $d-2$ is a multiple of 4, and
the 2 and $d-2$ factors are combined as in the first algorithm. Thus
the algorithm deals with $\SX(d,q)$, for even  values of $d$, in a way
that  is similar in outline to the familiar method of powering, that
computes $a^n$, by recursion on $n$, as $(a^2)^{n/2}$ for even $n$ and
as $a(a^{n-1})$ for odd $n$.

Algorithms {\tt TwoTimesFour} and {\tt TwoTwiceOdd} 
describe the case of even $d$. 

\begin{algorithm2e}
\caption{\tt TwoTimesFour$(X,{\it type})$}
\label{alg2:even-b}
\tcc{ $X$ is a generating set for
the perfect classical group $G$
in odd characteristic, of type SL or Sp or SU, in dimension a multiple of 4.
Return standard generating set $Y_0$ for a copy 
of $\SL(2, q) \wr S_{d/2} \leq G$, the 
SLPs for the elements of $Y_0$, the change-of-basis matrix,
and generators for centraliser of involution $k$ defined in line 14.
}

\Begin{

$d$ := the rank of the matrices in $X$;

if $d \leq 4$ then return {\tt BaseCase} (X, {\it type, false});

$q$ := the size of the field over which these matrices are defined;  

if {\it type} = SU then $q := q^{1/2}$;  

 Find by random search $g \in G:=\langle X\rangle$ of even order such that 
$g$ powers to an involution $h$ with eigenspaces of dimension $n = d/2$;

Let $n$ be the dimension of the $+1$-eigenspace of $h$;

 Find generators for the projective centraliser $C$ of $h$ in $G$
and identify an element $g$ of $C$ that interchanges the two eigenspaces;

In the derived subgroup $C'$ of $C$ find a generating 
set $X_1$ for one of the direct factors of $C'$;

$(s_1,\delta_1,u_1,v_1)$ := {\tt TwoEven}$(X_1,{\it type})$;

Let $X_2 = X_1^g$;

Conjugate all elements of $(s_1,\delta_1,u_1,v_1)$ by $g$ to 
obtain  solution $(s_{2},\delta_2,u_2,v_2)$ for $X_2$;

Let $(e_1, f_1, \ldots, e_n, f_n, e_{n+1}, f_{n+1}, \ldots, e_d, f_d)$
be the concatenation of the hyperbolic bases constructed in lines 10 and 12;

$k := (\delta_1^{(q-1)/2})^{v_1^{-1}}\delta_2^{(q-1)/2}$;

Find generators for the centraliser $D$ of $k$ in $G$;

In the derived subgroup $D'$ of $D$ find a generating set $X_3$ for the direct factor
that acts faithfully on $\langle e_n,e_{n+1},f_n,f_{n+1}\rangle$;

In $\langle X_3\rangle$ find the permutation matrix $j=(e_n,e_{n+1})(f_n,f_{n+1})$;

$v := v_1 j v_2$;

return $(s_1,\delta_1,u_1,v)$, the change-of-basis matrix, and $X_3$.
}
\end{algorithm2e}

\begin{algorithm2e}
\caption{\tt TwoTwiceOdd$(X,{\it type})$}
\label{alg2:even-a}
\tcc{ $X$ is a generating set for
the perfect classical group $G$
in odd characteristic, of type SL or Sp or SU, in twice odd dimension.
Return standard generating set $Y_0$ for a copy 
of $\SL(2, q) \wr S_{d/2} \leq G$, the 
SLPs for the elements of $Y_0$, the change-of-basis matrix,
and generators for centraliser of involution $k$ defined in line 13.
}

\Begin{

$d$ := the rank of the matrices in $X$;

if $d \leq 4$ then return {\tt BaseCase} (X, {\it type, false});

$q$ := the size of the field over which these matrices are defined;  

if {\it type} = SU then $q := q^{1/2}$;  

 Find by random search $g \in G:=\langle X\rangle$ of even order
 such that $g$ powers to an involution $h$ with eigenspaces of dimension 
2 and $d-2$.

Let $n$ be the dimension of the $+1$-eigenspace of $h$;

Find generators for the centraliser $C$ of $h$ in $G$;

 In the derived subgroup $C'$ of $C$ find generating sets 
$X_1$ and $X_2$ for the direct factors 
of $C'$ where $X_2$ centralises the eigenspace of dimension 2;

$(s_{1},\delta_1,u_1,v_1)$ := {\tt TwoTwiceOdd}$(X_1,{\it type})$;

$(s_{2},\delta_2,u_2,v_2)$ := {\tt TwoTimesFour}$(X_2,{\it type})$;

Let $(e_1, f_1, \ldots, e_n, f_n, e_{n+1}, f_{n+1}, \ldots, e_d, f_d)$
be the concatenation of the hyperbolic bases constructed in lines  and 10;

$k := (\delta_1^{(q-1)/2})^{v_1^{-1}}\delta_2^{(q-1)/2}$;

Find generators for the centraliser $D$ of $k$ in $G$;

In the derived subgroup $D'$ of $D$ find a generating set $X_3$ for the direct factor
that acts faithfully on $\langle e_n,e_{n+1},f_n,f_{n+1}\rangle$;

In $\langle X_3\rangle$ find the permutation matrix $j=(e_n,e_{n+1})(f_n,f_{n+1})$;

$v := v_1 j v_2$;

return $(s_1,\delta_1,u_1,v)$, the change-of-basis matrix, and $X_3$.

}
\end{algorithm2e}


Algorithm {\tt TwoTimesFour} calls no new procedures except in line 6,
where we construct an involution with eigenspaces of equal dimension.
This construction is discussed in Section \ref{Involution}.

Algorithm {\tt TwoEven}, which summarises the even degree case,
returns the generating set $Y_0$ defined in Section \ref{standard}. 
We complete the construction of $Y$ exactly as in Section \ref{Alg1}.

\begin{algorithm2e}
\caption{\tt TwoEven$(X,{\it type})$}
\label{alg2-main:even}
\tcc{ $X$ is a generating set for
the perfect classical group $G$
in odd characteristic, of type SL or Sp or SU, in even dimension.
Return standard generating set $Y_0$ for a copy 
of $\SL(2, q) \wr S_{d/2} \leq G$, the 
SLPs for the elements of $Y_0$, the change-of-basis matrix,
and generators for centraliser of involution. 
}

\Begin{
$d$ := the rank of the matrices in $X$;

\eIf {$d\bmod4=2$} 
  {
 
   return {\tt TwoTwiceOdd}$(X,{\it type})$;
  }{
    return {\tt TwoTimesFour}$(X,{\it type})$;
  }
}
\end{algorithm2e}

If $d$ is odd, then we find an involution whose $-1$-eigenspace has
dimension 3, thus splitting $d$ as $(d-3)+3$. Since $d-3$ is even, we
apply the odd case precisely once.

The resulting {\tt TwoOdd} is the same as {\tt OneOdd},
except that it calls {\tt TwoEven} rather than {\tt OneEven};
similarly {\tt TwoMain} calls {\tt TwoOdd} and {\tt TwoEven}.

The primary advantage of the second algorithm 
lies in its one recursive call. 
This significantly reduces the lengths of the 
SLPs for the standard generators.

\section{Finding strong involutions}
\label{Involution}

In the first step of our main algorithms, as outlined in 
Sections \ref{Alg1} and \ref{Alg2},
by random search, we obtain an element of
even order that has as a power a strong involution. 
We must establish a lower bound
to the proportion of elements of $\SX(d,q)$ that power up to give
a strong involution. 
%We denote the natural module for $\SX(d,q)$ by $V$.

A matrix is {\it separable} if its characteristic polynomial
has no repeated factors.
Fulman, Neumann \& Praeger \cite{FNP} 
provide detailed estimates for the proportions
of separable matrices in $\GX(d, q)$.
%, which are valid for all finite fields.
In particular they establish the following. 
\begin{theorem}\label{Lemma5.1}  
The probability that an 
element of $\GL(d,q)$ or $\U(d,q)$ is separable is 
at least $1-2/q$.
The probability that an 
element of $\Sp(d,q)$ is separable is 
at least $1-3/q + O(1/q^2)$.
\end{theorem}

We use these results to assist in our analysis of the 
proportion of strong involutions in $\SX(d, q)$.

\subsection{The special linear case}\label{sl}
We commence our analysis with $\SL(d, q)$.
We first estimate the probability that a random element of
$\GL(d,q)$ has a power that is an involution having an eigenspace
of dimension within a given range and then derive 
similar results for $\SL(d, q)$.

%Since we are willing to estimate within an error that is
%$O(1/q)$, we may assume, by Theorem \ref{Lemma5.1}, that the characteristic 
%polynomial of $g \in \GL(d, q)$ has no repeated factors. Thus the natural module 
%$V$ splits up as the direct sum of the 
%irreducible $\langle g\rangle$-submodules of $V$.

\begin{lemma}\label{monic} 
The number of irreducible monic polynomials of degree
$e>1$ with coefficients in $\GF(q)$ is $k$ where
$(q^e-1)/e>k\ge q^e(1-q^{-1})/e$.
\end{lemma}
\begin{proof}
Let $k$ denote the number of such polynomials.
We use the inclusion-exclusion principle to count the
number of elements of $\GF(q^e)$ that do not lie in any  maximal
subfield containing $\GF(q)$, and divide this number by $e$, since
every irreducible monic polynomial of degree $e$ over $\GF(q)$ corresponds
to exactly $e$ such elements.  Thus 
$$k = {q^e-\sum_iq^{e/p_i}+\sum_{i<j}q^{e/p_ip_j}-\cdots\over e}$$ 
where $p_1<p_2<\cdots$ are the distinct prime divisors of $e$. 
The inequality $(q^e-1)/e >k$ is obvious.  Also, if $e$ is a prime
then $k=(q^e-q)/e\ge q^e(1-1/q)/e$, with equality if $e=2$.
Now suppose that $e\ge4$, and let $\ell$ denote the largest prime
dividing $e$.  Then from the above formula 
\begin{eqnarray*}
ek & \ge & q^e-q^{e/\ell}-q^{(e/\ell)-1}-\ldots -1 \\
   & = &  q^e - (q^{1+e/\ell}-1)/(q-1)\\
   & \ge & q^e - q^{1+e/2}+1 \\
   & > & q^e - q^{e-1} \\
\end{eqnarray*}
as $1+e/2\le e-1$.
\end{proof}

\begin{lemma}\label{Lemma5.3} Let $e>d/2$ for $d \geq 4$. 
The proportion of elements of
$\GL(d,q)$ whose characteristic polynomial has an irreducible factor
of degree $e$ is $(1/e)(1+O(1/q))$. 
More precisely, there are
universal  constants $c_1$  and $c_2$ such that the proportion is always
between $(1/e)(1-c_2/q)$ and $(1/e)(1-c_1/q)$.
\end{lemma}
\begin{proof}
Let the characteristic polynomial of $g\in\GL(d,q)$ have an
irreducible factor $h(x)$ of degree $e$. Then $\{w\in V:w.h(g)=0\}$ is
a subspace of $V$ of dimension $e$. It follows that the number of
elements of $\GL(d,q)$ of the required type  is $k_1k_2k_3k_4k_5$
where $k_1$ is the number of subspaces of $V$ of dimension $e$, 
$k_2$ is the number of irreducible monic polynomials of degree $e$
over $\GF(q)$, $k_3$ is the number of elements of $\GL(e,q)$ that
have a given irreducible characteristic polynomial, $k_4$ is the
order of $\GL(d-e,q)$, and $k_5$ is the number of complements in
$V$ to a subspace of dimension $e$.
In more detail, 
\begin{eqnarray*}
k_1 & = & (q^d-1)(q^d-q)\cdots(q^d-q^{e-1})\over(q^e-1)(q^e-q) \cdots (q^e-q^{e-1}) \\
k_3 & = & (q^e-1)(q^e-q)\cdots(q^e-q^{e-1})\over (q^e-1) \\
k_4 & = & (q^{d-e}-1)(q^{d-e}-q) \cdots (q^{d-e}-q^{d-e-1}) \\
k_5 & = & q^{e(d-e)}.
\end{eqnarray*}
The formula for $k_3$ arises by taking
the index in $\GL(e,q)$ of the centraliser of an irreducible element,
this centraliser being cyclic of order $q^e-1$. 
The formula for $k_2$ is given in Lemma \ref{monic}. 
Hence $k_1 k_2 k_3 k_4 k_5 = \vert\GL(d,q)\vert\times k_2/(q^e-1)$. 
The result follows.  Note that $c_1$ and $c_2$ may be taken to be positive.
\end{proof}

\begin{lemma}\label{Lemma5.4} Let $e\in(d/3,d/2]$ for $d \geq 4$.
Let $S_1$ denote the number of elements of $G:=\GL(d,q)$ whose characteristic polynomial
has two distinct irreducible factors of degree $e$; let $S_2$ denote the number
of elements of $G$ whose characteristic polynomial has a repeated factor of
degree $e$; and let $S_3$ denote the number of elements of $G$ whose characteristic
polynomial has exactly one irreducible factor of degree $e$. 
Then $S_1={1\over2}\vert\GL(d,q)\vert e^{-2}(1+O(q^{-1}))$, and 
$S_2=\vert\GL(d,q)\vert O(q^{-1})$, 
and $S_3=\vert\GL(d,q)\vert(e^{-1}-e^{-2})(1+O(q^{-1}))$,
where the constants implied by the $O$ notation are absolute constants,
independent of $d$.
\end{lemma}
\begin{proof}  In the proof of Lemma \ref{Lemma5.3} the fact 
that $e>d/2$ was only used
to ensure that the characteristic polynomial of an element $g\in G=\GL(d,q)$ has 
only one irreducible factor of degree $e$.  In the case of the present lemma,
 If the proof of 
Lemma \ref{Lemma5.3} is repeated to estimate $S_3$ the
factor $k_4$ must be replaced by the number of elements of $\GL(d-e,q)$ whose
characteristic polynomials do not have an irreducible factor of degree $e$. A
direct application of this lemma then shows that this number is the order
of $\GL(d-e,q)$ multiplied by a factor of the form $(1-e^{-1})(1+O(1/q))$.
Thus $S_3=\vert\GL(d,q)\vert(e^{-1}-e^{-2})(1+O(q^{-1}))$. 

The proportion of inseparable elements of $\GL(d,q)$ is
$O(q^{-1})$ by Theorem\ref{Lemma5.1}, so $S_2=\vert\GL(d,q)\vert O(q^{-1})$,
and we may ignore the condition that the irreducible factors in question are
distinct in our estimate of $S_1$.
 If then the proof of
Lemma \ref{Lemma5.3} is repeated again to estimate $S_1$ the factor $k_4$ may
be replaced by the number of elements of $\GL(d-e,q)$ whose characteristic
polynomials do have an irreducible factor of degree $e$, and the total
must be divided by 2, since the group elements in question normalise
two subspaces of dimension $e$.  
Thus $S_1={1\over2}\vert\GL(d,q)\vert e^{-2}(1+O(q^{-1}))$.  
\end{proof}

\begin{lemma}\label{Lemma5.5} The results of 
Lemmas $\ref{Lemma5.3}$ and $\ref{Lemma5.4}$ hold if $\GL(d,q)$
is replaced by $\SL(d,q)$. 
\end{lemma}
\begin{proof}
We first prove that in Lemma \ref{Lemma5.3} the proportions are exactly the same in
the case of $\SL(d,q)$.  For in this case $k_4$ must be replaced by
the number of elements of $\GL(d-e,q)$ of a specified determinant.
But the number of such elements is exactly the number of elements of
$\GL(d-e,q)$ divided by $q-1$; so the result follows.  Similarly with
Lemma \ref{Lemma5.4}, if $e\ne d/2$ the proportions in the two cases are exactly
equal.  The point is that in these cases we consider the number of
elements in $\GL(d-e,q)$ or in $\GL(d-2e,q)$, and we need to replace
these numbers by the number of such elements of a given determinant, thus
reducing the number by a factor of $q-1$.
This also applies to $S_2$.  However, this argument breaks down when
$e=d/2$, as in this case we cannot adjust the determinant of the
element being constructed by requiring an element of $\GL(d-2e,q)$
to have a given determinant.  Indeed, in this case it seems that
the proportions are not exactly equal in the cases of $\GL(d,q)$
and $\SL(d,q)$.
To deal with the case $d=e/2$ we need first to consider the norm
map $N:GF(q^e)\to\GF(q)$.  This defines a homomorphism from
$\GF(q^e)^\times$ onto $\GF(q)^\times$.  We are concerned with
the size of the intersection $I_a$ of the pre-image of an arbitrary
element $a$ of $\GF(q)^\times$ with the set of elements of $\GF(q^e)$
that lie in no proper subfield.  Since in general the norm map will
not map a proper subfield of $\GF(q^e)$ onto $\GF(q)$ the size of $I_a$
will vary with $a$.  But clearly $\vert I_a\vert=c_a{q^e-1
\over q-1}$, where $c_a=1+O(1/q)$.
Now $S_1$ is approximated, in the case of $\SL(d,q)$, by 
$${1\over2}\sum_a\vert I_a\vert\vert I_{a^{-1}}\vert\left({\vert\GL(e,q)\vert\over e-1}
\right)^2$$
where the error in the approximation is due to the fact that we have ignored the 
condition that the irreducible factors of the characteristic polynomial should be
distinct.  The analogous estimate for $S_1$ in the case of $\GL(d,q)$ replaces
$\sum_a\vert I_a\vert \vert I_{a^{-1}}\vert$ by $\sum_{a,b}\vert I_a\vert\vert I_b\vert$,
which is approximately $q-1$ times as big, the error arising from the fact that
the cardinality of $I_a$ is not quite constant.  We have seen that this error 
corresponds to a factor of the form $1+O(1/q)$, as does ignoring the condition
that the irreducible factors in the characteristic polynomial should be different,
and omitting the contribution of $S_2$.  Note that the assumption $d\ge 3$ enforces
the condition $e\ge 2$.
\end{proof}

We now obtain a lower bound for the proportion of 
$g\in\SL(d,q)$ such that $g$ has even order $2n$, and $g^n$
has an eigenspace with dimension in a given range. 
%We assume for the rest of this section that $q$ is odd. 
To perform this calculation, we consider the cyclic groups $C_{q^e-1}$ of order
$q^e-1$. If $n$ is an integer, we write $v_2(n)$ for the $2$-adic
value of $n$.

\begin{lemma}\label{Lemma5.6} If $v_2(m)=v_2(n)$ then $v_2(q^m-1) = v_2(q^n-1)$.
\end{lemma}
\begin{proof} 
 It suffices to consider the case where $m=kn$, and $k$ is odd.
Then $(q^m-1)/(q^n-1)$ is the sum of $k$ powers of $q^n$, and so is odd.
\end{proof}

\begin{lemma}\label{Lemma5.7} If $u<v$ then $v_2(q^{2^u}-1)<v_2(q^{2^v}-1)$, 
and if $u>0$ then $v_2(q^{2^u}-1) =v_2(q^{2^{u+1}}-1)-1$.
\end{lemma}
\begin{proof} 
Observe that $(q^{2^{u+1}}-1)/(q^{2^u}-1)=q^{2^u}+1$ which is even. Now
$v_2(q^{2^u}-1)>1$ if $u>0$. It then follows that $v_2(q^{2^u}+1)=1$.
\end{proof}

\begin{theorem}\label{Theorem5.1}  
For some absolute constant $c$, the proportion
of $g \in \SL(d,q)$ of even order, such that a power of $g$
is an involution with eigenspaces of dimensions in the range
$(d/3,2d/3]$, is at least $c/d$.
\end{theorem}
\begin{proof} 
Let $2^k$ be the unique power of $2$ in the range
$(d/3,2d/3]$. By Lemma \ref{Lemma5.5} it suffices to prove that 
if $g\in\SL(d,q)$
has an  irreducible factor of degree $2^k$ then the probability that
$g$ has the required property is bounded away from 0.

Let $\{W_i:i\in I\}$ be the set of composition factors of $V$
under the action of $\langle g\rangle$. Let $n_i$ be the order of the
image of $g$ in $\GL(W_i)$, and set $w_i=v_2(n_i)$, and
$w=\max_i(w_i)$, and $d_i=\dim(W_i)$. 
If $w>0$, then $g$ has even order $2n$  say,
and in  this case the $-1$-eigenspace of $z :=  g^n$ has
dimension $\sum d_i$, where the sum is over those values of $i$ for
which $w_i=w$. 

Suppose now that the characteristic polynomial of $g$
has exactly one irreducible factor of degree $2^k$. By renumbering if
necessary we may assume that $d_1=2^k$. Set $x=v_2(q^{2^k}-1)$.  The
probability that $w_1=x$ is slightly greater than $1/2$. This is
because the action of $g$ on $W_1$ embeds $g$ at random in
$\GF(q^{2^k})$, which is a cyclic group of order an odd multiple of
$2^x$. The distribution of possible values of $g$ is uniform among
those elements that do not lie in a proper subfield of $\GF(q^{2^k})$.
But non-zero elements of such subfields do not have order a multiple
of $2^x$. If $w_1=x$ then necessarily $w_i<w_1$ for all $i>1$,
and $w_1=w$. It follows that $g$ will then have even order, and that
$z$ will be an involution whose $-1$-eigenspace will have
dimension exactly $2^k$. There is a slight problem with the elements of $\SL(d,q)$
whose characteristic polynomials have two irreducible factors of
degree $2^k$, as such elements may power up to an involution
whose  $-1$-eigenspace has dimension $2^{k+1}$, but the estimate of $S_1$
in Lemma \ref{Lemma5.4} shows that this problem does not affect the truth of the theorem.
\end{proof}

\begin{corollary}\label{Corollary5.1} Such an element $g$ in $\SL(d,q)$ 
can be found with at most $O(d(\xi + d^3\log q))$ field operations,
where $\xi$ is the cost of constructing a random element.
\end{corollary}
\begin{proof}
Theorem \ref{Theorem5.1} implies that a search of length $O(d)$ 
will find such an element $g$. 
%Theorem \ref{Lemma5.1} implies that we can
%afford to discard elements whose characteristic   
%polynomials have repeated roots. 
In $O(d^3)$ field operations 
the characteristic polynomial $f(t)$ of $g$ can be
computed (see \cite[Section 7.2]{HoltEickOBrien05}); 
in $O(d^2 \log q)$ field operations it can be factorised as 
$f(t)=\prod_{i=1}^kf_i(t)$, where the $f_i(t)$ are irreducible 
(see \cite[Theorem 14.14]{vzg}).

Following the notation of the proof of Theorem \ref{Theorem5.1},  
we may take $W_i$ to be the
kernel of $f_i(g)$. It remains to calculate $w_i$. Let $m_i$ be the
odd part of $q^{d_i}-1$, where $d_i$ is the degree of $f_i$. Now
compute $s :=(f_i(t))+t^{m_i}$ in $\GF(q)[t]/(f_i(t))$, and iterate 
$s := s^2$  until $s$ is the identity. The number of iterations
determines $w_i$, and it is now easy to determine whether or not $g$
satisfies the required conditions. All of the above steps may be carried
out in at most $O(d^3 \log q)$ field operations. 
\end{proof}


\subsection{The symplectic and unitary groups}
We first consider the symplectic groups.
If $h(x)\in\GF(q)[x]$ is a
monic polynomial with non-zero constant term, let
$\tilde{h}(x)\in\GF(q)[x]$ be the monic polynomial  whose zeros are the
inverses of the zeros of $h(x)$. Hence the multiplicity of a zero of
$h(x)$ is the multiplicity of its inverse in $\tilde{h}(x)$ so that
$h(x)\tilde{h}(x)$ is a symmetric  polynomial.  We start with this
analogue of Lemma \ref{Lemma5.3}.

\begin{lemma}\label{Lemma5.8} Let $m>n/2$ where $n \geq 2$. 
The proportion of elements of
$\Sp(2n,q)$ whose characteristic polynomial has a factor $h(x)$ where
$h(x)$ is irreducible of degree $m$ and $h(x)\ne \tilde{h}(x)$ is
$(1/2m)(1+O(1/q))$, where the constants implied by the $O$ notation 
are absolute constants, independent of $n$.
\end{lemma}
\begin{proof}
Let $g\in\Sp(2n,q)$ act on the natural module $V$, and let
$h(x)$ be an irreducible factor of degree $m$
of the characteristic polynomial
$f(x)$ of $g$. 
Let $V_0$ be the kernel of $h(g)$. Since    
$h(x)\ne \tilde{h}(x)$ it follows that $V_0$ is totally isotropic. Also
$\tilde{h}(x)$ is a factor of $f(x)$, and if  $V_1$ is the kernel of
$\tilde{h}(x)$ then $V_1$ is totally isotropic. Since $h(x)$ and
$\tilde{h}(x)$ divide $f(x)$ with multiplicity 1, $V_0$ and $V_1$ are
uniquely determined, and the form restricted to $V_0\oplus V_1$ is
non-singular. Now let $e_1,\ldots,e_m$ be a basis for $V_0$. A basis
$f_1,\ldots,f_m$ for $V_1$ is then determined by the conditions
$B(e_i,f_j)=0$ for $i \ne j$, and $B(e_i,f_i)=1$ for all $i$, where
$B(-,-)$ is the symplectic form that is preserved. The matrix for $g$
restricted to $V_0$ now determines the matrix of $g$ restricted to
$V_1$, since $g$ preserves the form. 

Thus the number of possibilities
for $g$ is the product $k_1k_2k_3k_4k_5/2$, where $k_1$ is
the number of choices for $V_0$, and $k_2$ is the number of choices
for $V_1$ given $V_0$, and $k_3$ is the number of irreducible monic 
polynomials $h(x)$ of degree $m$ over $\GF(q)$ such that $h(x)\ne
\tilde{h}(x)$, and $k_4$ is the number of elements of $\GL(m,q)$ with a
given irreducible characteristic polynomial, and $k_5$ is the order of
$\Sp(2n-2m,q)$.   The factor $1/2$ in the above expression arises from
the fact that every such element $g$ is counted twice, because of the
symmetry between $h(x)$ and $\tilde h(x)$. 
In more detail 
\begin{eqnarray*}
k_1 & = & {(q^{2n}-1)(q^{2n-1}-q)(q^{2n-2}-q^2)\cdots(q^{2n-m+1}-q^{m-1})\over
(q^m-1)(q^m-q)(q^m-q^2)\cdots(q^m-q^{m-1})} \\                   
k_2 & = & q^{(2n-m)+(2n-m-1)+(2n-m-2)+\cdots+(2n-2m+1)} \\
k_3 & \sim  & q^m/m \\
k_4 & = & {(q^m-1)(q^m-q)(q^m-q^2)\cdots(q^m-q^{m-1})\over q^m-1}\\
k_5 & = & q^{(n-m)^2} \prod_{i = 1}^{n - m}(q^{2i}-1). 
\end{eqnarray*}

These results are obtained as follows. For $k_1$, we count the number of
sequences of linearly  independent  elements $(e_1,e_2,\ldots)$  such
that each is orthogonal to its predecessors, and divide by the order
of $\GL(m,q)$. For $k_2$, we observe that there is a 1-1
correspondence between the set of candidate subspaces for $V_1$ and
the set of sequences $(f_1,f_2,\ldots,f_m)$ of elements of $V$ such that
each $f_j$ satisfies $m$ linearly independent conditions
$B(e_i,f_j)=0$ for $i \ne j$, and $B(e_j,f_j)=1$, and $B(f_k,f_j)=0$ for $k < j$.
We observe that $k_3$ is the number of orbits of the Galois group of
$\GF(q^m)$ over $\GF(q)$ acting on those $a \in \GF(q^m)$ that 
do not lie
in a proper subfield containing $\GF(q)$, and have the property that 
the orbit of $a$ does not contain $a^{-1}$.  This last condition is equivalent to the
statement that $h(x)\ne \tilde{h}(x)$.   Note that $h(x)=\tildeh(x)$ if and only if
$m$ is even, and $a^{-1}=a^{q^{m/2}}$.
A precise formula for $k_3$
would be rather complex, so we obtain instead the following
estimate. If we ignore this last condition, then
Lemma \ref{monic} estimates $k_3$.
Now it is clear that if $a\in\GF(q^m)$ satisfies the
above equation then the norm of $a$ is 1. In other words, the
constant term of $h(x)$ is $1$. But this is exactly the problem tackled in
the proof of Lemma \ref{Lemma5.5}, so we find that, for some
absolute constants $c_1$ and $c_2$, $k_3$ lies between $(1-c_2 /q)q^m/m$ and
$(1-c_1 /q)q^m/m$. 

But the product of the $k_i$ is
$k_3\vert\Sp(2n,q)\vert/(q^m-1)$ and the result follows. 
\end{proof}

\begin{lemma}\label{Lemma5.9} Let $m\in(n/3,n/2]$. 
Let $S_1$ denote the number of elements of $G:=\Sp(2n,q)$ whose characteristic polynomial
has four distinct irreducible factors of degree $m$, of the form $h(x)$, $\tilde h(x)$,
$k(x)$, and $\tilde k(x)$;
 let $S_2$ denote the number
of elements of $G$ whose characteristic polynomial has two distinct repeated factors of
degree $m$, of the form $h(x)$ and $\tilde h(x)$; and let $S_3$ denote the number 
of elements of $G$ whose characteristic
polynomial has exactly two distinct irreducible factors of degree $m$,
of the form $h(x)$ and $\tilde h(x)$. 
Then $S_1={1\over8}\vert\Sp(2n,q)\vert m^{-2}(1+O(q^{-1}))$, and 
$S_2=\vert\Sp(2n,q)\vert O(q^{-1})$, 
and $S_3={1\over2}\vert\Sp(2n,q)\vert(m^{-1}-{1\over2}m^{-2})(1+O(q^{-1}))$,
where the constants implied by the $O$ notation are absolute constants,
independent of $n$.
\end{lemma}
\begin{proof} 
The proof is similar to that of Lemma \ref{Lemma5.4}.  Care has to
be taken with counting the number of times that elements of the analogue
of $S_1$ are counted.  The characteristic polynomial of such an element
$g$ now has four distinct irreducible factors $h(x)$, $\tilde h(x)$, $k(x)$,
and $\tilde k(x)$ of degree $m$.  This leads to such elements being counted eight times.
\end{proof}


\noindent
We now obtain the analogue of Theorem \ref{Theorem5.1}.
\begin{theorem}\label{Theorem5.2}  
For some absolute constant $c>0$, the proportion
of elements $g \in \Sp(2n,q)$ of even order, such that a power of $g$
is an involution with eigenspaces of dimensions in the range
$(2n/3,4n/3]$, is at least $c/n$.
\end{theorem}
\begin{proof} 
Given Lemmas \ref{Lemma5.8} and \ref{Lemma5.9}, the proof is 
essentially the same as that of Theorem \ref{Theorem5.1}.   We
adopt the notation of that proof.
One
must consider the contribution of $W_i$ to the eigen-spaces of $z$
when the characteristic polynomial of $g$ restricted to $W_i$ is
an irreducible polynomial $h(x)$ such that $h(x)=\tilde h(x)$.  But
in this case if $\alpha$ is a zero of $h(x)$ then so is $\alpha^{-1}$,
and $\alpha^{q^m}=\alpha^{-1}$, where $W_i$ has dimension $2m$, and the order of $g$
divides $q^m+1$.  It is easy to see that if $i$ is even then $v_2(q^i+1)=1$,
and if $i$ is odd then $v_2(q^i+1)=v_2(q+1)$.  Thus $W_i$ will
contribute nothing to the dimension of the $-1$-eigen-space of $z$.
The result follows.
\end{proof}

We finally turn to the unitary groups. 
\begin{theorem}\label{Theorem5.3}  For some absolute constant $c>0$, 
the proportion of elements $g\in \SU(d,q)$ that have even order, such that a power of $g$
is an involution with eigenspaces of dimensions in the range
$(d/3,2d/3]$, is at least $c/d $.
\end{theorem}
\begin{proof} 
The analysis in this case is almost
exactly the same as for the symplectic groups.  The only difference
comes from the analysis of the restriction of $g$ to $W_i$ where
now we require $h(x)$ to be the image of $\tilde h(x)$ under the Frobenius
map $a\mapsto a^q$.  This now requires $W_i$ to have odd dimension
$2t+1$, say, and then the order of $g$ will divide $q^{2t+1} +1$.
\end{proof}

In summary, Theorems \ref{Theorem5.1}, \ref{Theorem5.2} and \ref{Theorem5.3} 
provide an estimate of the
complexity of finding a strong involution of the type required 
as $O(d(\xi + d^3 \log q))$ field operations. 
%While, from a practical point of view, we regard this
%as conservative estimate, from a theoretical point of view it is
%adequate: there are other components of the main
%algorithms with complexity bounded by $O(d^4 \log q)$ field operations. 

\section{Involution with eigenspaces of equal dimension}\label{Equal}
Our next objective is to describe and analyse an algorithm 
to construct an involution in
$\SX(d,q)$ with eigenspaces of equal dimension. This necessarily
presupposes that $d$ is a multiple of $4$. 
We use such an element in Algorithm {\tt TwoEven}. 

We describe a recursive procedure to construct an 
involution in $\SL(d, q)$ whose $-1$-eigenspace has a specified 
even dimension $e$. 

\begin{enumerate}
\item 
Search randomly for an element $g$ of even order
that powers to an involution $h_1$
satisfying the conditions of Theorem \ref{Theorem5.1}.

\item Let $r$ and $s$ denote the ranks of the 
$-1$- and $+1$-eigenspaces of $h_1$.

\item If $r = e$ then $h_1$ is the desired involution.
 
\item 
Consider the case where $s \leq e < r$.
Construct the centraliser of  $h_1$, and by powering, obtain 
generators for the special linear group
$S_-$ on the $-1$-space, where $S_-$ acts as the identity on the
$+1$-eigenspace of $h_1$. 
By recursion on $d$, an involution 
can be found in $S_-$ whose $-1$-eigenspace    
has dimension $e$. 

\item 
Consider the case where $e \leq \min(r, s)$.
If $r \leq s$ then 
construct the centraliser of  $h_1$, and by powering, obtain 
generators for the special linear group
$S_-$ on this $-1$-eigenspace,
where $S_-$ acts as the identity on the $+1$-eigenspace.
By recursion on $d$, an involution 
can be found in $S_-$ whose $-1$-eigenspace    
has dimension $e$. 
Similarly, if $s < r$ then construct $S_+$,
and search in $S_+$ for an involution
whose $-1$-eigenspace has dimension $e$. 

\item 
Consider the case where $s \geq e > r$.
Construct the centraliser of $h_1$, and obtain generators
for the special linear group $S_+$ on the $+1$-eigenspace of $h_1$,
where $S_+$ acts as the identity on the $-1$-eigenspace.
Now an involution $h_2$ is found recursively in
$S_+$ whose $-1$-eigenspace has dimension $e-r$. 
Then $h_1h_2$ is an involution of the required type. 

\item Finally consider the case where $e \geq \max(r, s)$.
This is identical to the last case.

\end{enumerate}
The recursion is founded trivially with the case $d=4$.

\begin{theorem}\label{Corollary5.2}  
Using this algorithm, an involution in $\SL(d,q)$
can be constructed with $O(d(\xi + d^3 \log q))$ field operations that has its
$-1$-eigenspace of any even dimension in $[0,d]$.
\end{theorem}
\begin{proof}
Corollary \ref{Corollary5.1} implies that $h_1$ can be constructed with at most 
$O(d(\xi + d^3 \log q))$ field operations. We shall see in 
Sections \ref{Bray} and 
\ref{Pow} that  generators for $S_-$ and $S_+$ can be constructed
with $O(d^4)$ field operations. Thus the above algorithm requires
$O(d(\xi + d^3 \log q))$ field operations, plus the number of field operations
required in the recursive call. Since the dimension of the matrices
in a recursive call is at most $2d/3$, the total
complexity is as stated.
\end{proof}
Similar results can be obtained for the other classical groups
considered in this paper.

\section{Constructing an involution centraliser}
\label{Bray}

The centraliser of an involution in a black-box group having an order
oracle can be constructed using an algorithm of Bray \cite{Bray}. 
Elements of the centraliser are constructed using the following result.
\begin{theorem}
\label{thm:bray}
If $u$ is an involution in a group $G$, and $g$ is an arbitrary element of $G$,
then $[u,g]$ either has odd order $2k+1$, in which case
$g[u,g]^k$ commutes with $u$, or has even order $2k$, in which case
both $[u,g]^k$ and $[u,g^{-1}]^k$ commute with $u$.
\end{theorem}
That these elements centralise $u$ follows from elementary
properties of dihedral groups. 

Bray \cite{Bray} also proves that if $g$ is uniformly
distributed among the elements of $G$ for which $[u,g]$
has odd order, then $g[u,g]^k$ is uniformly distributed among the
elements of the centraliser of $u$. If the order of $g[u,g]^k$ is even, 
then the elements returned are involutions; but if just
one of these is selected, then the elements returned
within a given conjugacy class of involutions {\it are independently
and uniformly distributed within that class}. 

Let $u \in \SL(d, q)$ and let $E_+$ and $E_-$ denote the 
eigenspaces of $u$.
We apply the Bray algorithm in the following contexts. 
\begin{enumerate}
\item 
We wish to find a generating set for (a subgroup of) the centraliser 
of $u$ that contains $\SL(E_+)\times \SL(E_-)$. 
\item 
The eigenspaces, $E_+$ and $E_-$ have the same dimension. 
We wish to construct the projective centraliser of $u$.
As we observed in Section \ref{cent}, the centraliser 
of $u$ contains an element which interchanges the eigenspaces.
\end{enumerate}
The other contexts are similar, but with $\SL(d,q)$ 
replaced by the other classical groups. 

Parker \& Wilson \cite{PW05} prove that,
in a simple classical group of odd characteristic and Lie rank $r$, 
the probability of the Bray algorithm returning an odd 
order element is at least $O(1/r)$. 
More precisely they prove the following.
\begin{theorem}\label{clasthm}
There is an absolute constant $c$ such that if $G$ is a finite
simple classical group, with natural module
of dimension $d$ over a field of odd order,
and $u$ is an involution in $G$, then $[u,g]$ has odd order
for at least a proportion $c/d$ of the elements $g$ of $G$.
\end{theorem}

Our estimate of the efficiency of Bray's algorithm relies critically 
on their result .

When applying the Bray algorithm we always operate in $\PSL(d,q)$
rather than in $\SL(d,q)$.  This has the effect that an element is
treated as having odd order if its order is odd modulo scalars.
Note that, if a subset of the centraliser of an involution generates
the centraliser modulo involutions then it generates the centraliser.

Hence, by a random search of length at most $O(d)$, we construct
random elements of the centraliser of the involution. The results of 
\cite{lish} imply that (the derived group of) the centraliser is generated 
by a bounded number of elements.   We return to this point later.

It remains to consider a stopping criterion: how can we tell when we
have a subset of the centraliser that generates a sufficiently large
subgroup?  %In the first of our two applications, 
We apply the Niemeyer-Praeger algorithm \cite{NP} to 
the projection of the centraliser onto
each factor to deduce that this 
contains a perfect classical group in its natural representation.
This algorithm, when applied to a subgroup of $\GL(k,q)$,
has complexity at most $O(k^3)$ group operations. If the factors have the
same dimension, there is a small possibility that the given elements
generate a group that contains a diagonal embedding of $\SL(d/2,q)$
in $\SL(d/2,q)\times\SL(d/2,q)$ but does not contain the full direct product.
This case is easily detected. A similar stopping criterion applies for the
second application; we can readily detect when an element
of the centraliser interchanges the eigenspaces.
Again these remarks apply to the other classical groups.

In its black-box application, 
the involution-centraliser algorithm 
assumes the existence of an order oracle.
We do not require such an oracle for linear groups.
If a multiplicative upper-bound $B$ for the
order of $g \in G$ is available,
then we can learn in polynomial time
the {\it exact} power of $2$ (or of any specified prime)
which divides $|g|$.
By repeated division by 2, we write $B = 2^m b$ where
$b$ is odd. Now we compute $h = g^{b}$, and determine
its order by repeated squaring.
In particular, we can determine whether $g$ has even order.
The cost of exponentiating is discussed later.
If $g \in \GL(d, q)$, then a multiplicative upper 
bound of magnitude $O(q^d)$ can be obtained for $|g|$
using the algorithms of \cite{Storjohann98}
and \cite{CLG97} in at most $O(d^3 \log q)$
field operations.  This is considered
further in Section \ref{Pow}. 
Further, as discussed in \cite{Ryba-paper},
the construction of the centraliser
of an involution requires only
knowledge of such an upper bound.

We conclude that our applications of the Bray algorithm have 
complexity $O(d(\xi + d^3 \log q))$ field operations.

\section{The base cases}
\label{base}

Consider the {\it base cases}: $\SX(d, q)$ where $d \leq 4$.
We construct standard generating sets for these groups using
specialised constructive recognition algorithms. 
We summarise the general algorithm for the base cases
and then consider its components in more detail.

\begin{algorithm2e}[H] 
\caption{\tt BaseCase$(X,{\it type}, Complete)$}
\label{alg:base}
\tcc{
$X$ is a generating set for
the perfect classical group $G$
in odd characteristic, of type SL or Sp or SU, in dimension at most 4.
If Complete = {\tt false} then 
return standard generating set for a copy 
of $\SL(2, q) \wr C_{2} \leq G$; 
otherwise return standard generating set for $G$.
Also return the 
SLPs for the elements of the set, and the change-of-basis matrix.
}
\Begin{
 $d$ := the rank of the matrices in $X$; 

  $q$ := the size of the field over which these matrices are defined;   

If {\it type} = SU then $q := q^{1/2}$;  

  \eIf{$d = 2$}
   {
      Apply the SL2 algorithm to construct generating set;
   }{
   \eIf{type={\rm SL} and $d=3$}
    {
     Apply the SL3 algorithm to construct generating set;
  
    }{
       Use centraliser-of-involution algorithm to  construct generating set;
    }
   }
   return standard generating set, SLPs, and change-of-basis;
}
\end{algorithm2e}

\begin{theorem}\label{ryba-alg}
Subject to the availability of a discrete log oracle 
for $\GF(q)$, the standard generators for $\SX(d, q)$ 
for $d \leq 4$ can be constructed  
in $O(\log q)$ field operations.
\end{theorem}

The base case encountered most frequently is $\SL(2,q)$ 
in its natural representation.
An algorithm to construct an element of $\SL(2,q)$ as an SLP  
in an arbitrary generating set is described in \cite{Conderetal05}. 
This algorithm requires $O(\log q)$ field operations, and the 
availability  of discrete logarithms in $\GF(q)$.

For $\SL(3,q)$ we use the algorithm of \cite{sl3q}
to perform the same task. 
It assumes the existence of an oracle
to recognise constructively $\SL(2, q)$
and its complexity is that of the oracle.

\subsection{The involution-centraliser algorithm} 
\label{ryba-base}
We use the involution-centraliser algorithm of \cite{Ryba-paper}
to construct SLPs for elements of $\SU(3,q)$  
and $\SX(4, q)$. We briefly summarise 
this algorithm.  

Assume $G = \langle X \rangle$ is a black-box group
with order oracle. We are given 
$g \in G$ to  be expressed as an SLP in $X$.
In this description  we say
that an element of $G$  is ``found" if it is known as an SLP
in $X$. First find by random search $h\in G$ such that
$gh$ has even order $2\ell$, and $z:=(gh)^\ell$ is a non-central
involution. Now  find, by random search and powering, an involution
$x\in G$ such that $xz$ has even order $2m$, and $y:=(xz)^m$ is a
non-central involution. Note that $x$ has been found, but, at this
stage, neither $y$ nor $z$ has been found. 
Observe that $x$, $y$ and $z$
are non-central involutions. 
We construct their centralisers using the Bray algorithm.
We assume that we can  solve the explicit membership problem 
in these centralisers.
In particular, we find $y$ as an element of the centraliser of $x$, 
and $z$ as an
element of the centraliser of $y$, and $gh$ as an element of the
centraliser of $z$. Having found $gh$, we have found $g$.

In summary, this algorithm reduces the constructive
membership test to three constructive membership  tests  in involution
centralisers;  but  this is an imperfect recursion, since  the 
algorithm may not be applicable to  these centralisers. 
We do not rely on the recursion; instead we construct
explicitly the desired elements of the centralisers, 
since these are (direct products of) $\SL(2,q)$.
In this context, its complexity is that stated
in Theorem \ref{ryba-alg}.

As presented, this is a black-box algorithm  requiring an order oracle.
If $G$ is a linear group, the algorithm does not require 
an order oracle, exploiting instead the multiplicative 
bound for the order of an element which can
be obtained in polynomial time as described in Section \ref{Bray}.

It is instructive to see why this algorithm fails in the case of
$\Sp(4,q)$. This group has only one conjugacy class of
non-central involutions, namely the class consisting of elements with
both the $+1$ and $-1$-eigenspaces being mutually orthogonal
2-dimensional spaces. The centraliser of $x$ is isomorphic to
$\SL(2,q)\times\SL(2,q)$, and this group contains only two non-central
involutions, arising from the unique involution in $\SL(2,q)$. A
similar remark applies, of course, to $z$. Thus, if $y$ is an
involution that  commutes with $x$ and $z$, then $z=\pm x$, and the
probability of this being the case is too low.
Hence $x$ cannot be found efficiently by random search.

In the case of $\SU(4,q)$ or $\SL(4,q)$, this problem does not arise.
The centraliser of the involution $u$ whose matrix  with respect to a
hyperbolic basis is
$$\left(\matrix{-1&0&0&0\cr0&-1&0&0\cr0&0&1&0\cr0&0&0&1\cr}\right)$$
contains many involutions: namely
$$\left(\matrix{0&1&0&0\cr1&0&0&0\cr0&0&0&1\cr0&0&1&0\cr}\right)$$
and its conjugates in the centraliser.

To avoid the problem with $\Sp(4,q)$, we work in the projective
group $\PSp(4,q)$. This gives rise to two cases. Suppose first that
$q\equiv1\bmod 4$, so that $\GF(q)$ has a primitive 4-th root $\omega$ of 1.
Then the projective centraliser of $u$ in $\Sp(4, q)$ contains 
$$\left(\matrix{\omega&0&0&0\cr0&-\omega&0&0\cr0&0&
\omega&0\cr0&0&0&-\omega\cr}\right)$$
which has many conjugates in this centraliser. If $q\equiv3\bmod4$,
then the projective centraliser of $u$ in $\Sp(4, q)$ contains the 
projective involution
$$u=\left(\matrix{0&1&0&0\cr-1&0&0&0\cr0&0&0&1\cr0&0&-1&0\cr}\right).$$

%which acts irreducibly on two blocks of dimension 2. Thus the
%centraliser of $u$ in $\Sp(4,q)$ is contained in $\SL(2,q^2)$, and is
%$\SL(2,q^2)\cap\Sp(4,q)$. It is easy to see that the symplectic
%form  defined  over $\GF(q)$ that is preserved by $\Sp(4,q)$ is a
%hermitian form  when $V$  is regarded as a module over the copy of
%$\GF(q^2)$ in the ring of $4\times4$  matrices over $\GF(q)$ that
%centralise $\SL(2,q^2)$, so the centraliser of $u$ in $G$ is
%$\SU(2,q)=\SL(2,q)$. The centraliser of the image $u$ in $G$ will
%also contain elements represented by matrices that conjugate $u$ to
%its inverse, and hence is $\PSL(2,q).C_2$.
%HAVE I LOST A $C_2$ ALONG THE WAY?
%{\tt PARA NEEDS ATTENTION}

%For an analysis of the general algorithm, we refer to \cite{Ryba};
%in these specialised cases, we deduce the following result.  
%\begin{lemma}\label{ryba-alg}
%Subject to the availability of a discrete log oracle 
%for $\GF(q)$, the standard generators for $\SX(4, q)$ 
%for $d \leq 4$ can be constructed  
%in $O(\log q)$ field operations.
%\end{lemma}

\subsection{The glue element}
In executing either Algorithm {\tt OneMain} or {\tt TwoMain}, 
each pair of recursive calls generates
an instance of the following problem.

\begin{problem}
Let $V$ be the natural module of  $G=\SX(4,q)$, and let
$(e_1,f_1,e_2,f_2)$ be a hyperbolic basis for $V$. Given a generating
set for $X$, and the involution $u$, where $u$ maps $e_1$ to $-e_1$
and $f_1$ to $-f_1$, and centralises the other basis
elements, construct the involution  $j$  that permutes the basis
elements, interchanging $e_1$ with $e_2$, and $f_1$ with $f_2$.
\end{problem}

Of course, $j$ is the permutation matrix used
in each algorithm to ``glue" $v_1$ and $v_2$ together to 
form $v$, the long cycle. 
(See for example l.\ 16 of {\tt OneEven}.)

We use the following algorithm to construct this element. 
\begin{enumerate}
\item 
Construct the projective
centraliser $H$ of $u$ in $\SX(4,q)$, using the Bray algorithm. 

\item 
Since $H$ lies between $\SL(2,q)\wr C_2$ and
$\GL(2,q) \wr C_2$, we find $h\in\SL(2,q)\wr C_2$ that
interchanges the spaces $\langle e_1,f_1\rangle$ and $\langle
e_2,f_2\rangle$. 

\item Then $jh$ lies in $\SL(2,q)\times\SL(2,q)$.  By the
powering algorithm described in Section \ref{Pow}, 
we construct the two direct factors, solve in each
direct factor for the projection of $jh$ and so 
construct $jh$ as an SLP. We can now solve for $j$.
\end{enumerate}
This algorithm requires $O(\log q)$ field operations.

\subsection{The final step}\label{final-step}
We must also perform the final step 
of Algorithm {\tt OneMain} or Algorithm {\tt TwoMain}:
namely, obtain an additional element. 

Consider first the case where $d$ is even. 
For $\SL(d, q)$, the additional element $a$ allows
us to construct the $d$-cycle from two smaller cycles; 
in the other cases, we construct the additional element $t$.
This additional element is found in $\SX(4, q)$.

If $d$ is odd and $G = \SU(d, q)$, 
then we must also find the element
$t$ in $\SU(3, q)$. 

In all cases, we employ the involution-centraliser algorithm described 
in Section \ref{ryba-base}. Theorem \ref{ryba-alg} again applies.

\section{Exponentiation}
\label{Exp}

A frequent step in our algorithms is computing the 
power $g^n$ for some $g\in \GL(d,q)$ and integer $n$.

Sometimes we raise an element to a high power in order to construct an
involution, and we may be able to write down
this involution without performing the calculation. However, if, for example, 
we want to construct elements of one direct factor of a direct product 
of two groups by exponentiation, then we must explicitly
compute the required power. 

The value of $n$ may be as large as $O(q^d)$. We
could construct $g^n$ with $O(\log(n))$ multiplications using the familiar
black-box squaring technique, 
Instead, we describe the following faster algorithm to perform this task.
\begin{enumerate}
\item 
Construct the Frobenius normal form of $g$ and record
the change-of-basis matrix. 

\item 
 From the Frobenius normal form, 
we read off the minimal polynomial
$h(x)$ of $g$, and factorise $h(x)$ 
as a product of irreducible polynomials.

\item 
This form determines a multiplicative upper bound
to the order of $g$. 
If $\{f_i(x):i\in I\}$ is the set of distinct
irreducible factors of $h(x)$, and if $d_i$ is the degree of
$f_i(x)$, then the order of the semi-simple part of $g$ divides
$\prod_iq^{d_i}-1$, and the order of the idempotent part of $g$ can be
read off directly. The product of these two factors  gives the
required upper bound $m$. 

\item If $n>m$ we replace $n$ by $n\bmod m$. 
By repeated squaring we calculate $x^n\bmod h(x)$ 
as a polynomial of degree $d$. 

\item This polynomial is evaluated in $g$ to give $g^n$. 

\item Conjugate $g^n$ by the inverse of the change-of-basis 
matrix to return to the original basis.
\end{enumerate}

We now consider the complexity of this algorithm.
\begin{lemma}Let $g\in\GL(d,q)$ and let $0\le n<q^d$. Then
$g^n$ can be computed using the above algorithm
with $O(d^3 + d^2 \log d \log \log d \log q)$ field operations.
\end{lemma}
\begin{proof}
The Frobenius normal form of $g$ can be computed with
$O(d^3)$ field operations \cite{Storjohann98}
and provides the minimal polynomial.
The minimal polynomial can be factored in 
in $O(d^2 \log q)$ field operations \cite[Theorem 14.14]{vzg}.
Calculating $x^n \bmod h(x)$ requires $O(\log(n))$
multiplications in $\GF(q)[x]/(h(x))$, 
at most $O(d^2 \log d \log \log d \log q)$ 
field operations \cite{vzg}. Evaluating the resultant polynomial in $g$ requires
$O(d)$ matrix multiplications;  but multiplying by   
$g$ only costs $O(d^2)$ field operations, since $g$ is sparse when 
in Frobenius normal form. Finally, conjugating $g$ by the inverse of 
the change-of-basis matrix costs a further $O(d^3)$ field operations.
\end{proof}

One should consider the cost of
dividing $m$ by $n$, even though this does not 
contribute to the number of field operations. 
However, for our applications, the exponent $n$ 
is always less than $q^d$, so
reducing $m$ modulo $n$ is unnecessary.

There is no need to prefer one normal form for $g$ to another,
provided that the normal form can be computed in at most $O(d^3)$ field
operations, the form is sparse, and the minimum polynomial
and multiplicative upper bound for the order of $g$
can be determined readily from the normal form.

This algorithm is similar to 
that of \cite{CLG97} to determine 
the order of an element of $\GL(d, q)$.

\section{Decomposing direct products and computing derived subgroups}\label{Pow}
Given a generating set $X$ of $G=\SX(e,q)\times\SX(d-e,q)$ we
wish to construct a generating set for one or both of the direct factors.
Also, given a generating set for $H$ where $\SX(e,q)\le H\le
\GX(e,q)$ we wish to construct a generating set for the derived
subgroup of $H$.

We solve both problems in the same style, and base our analysis on [NP LMS] and
[NPJAMS].

We start by solving the derived subgroup problem.  

The algorithm of [NP LMS] is a one-sided Monte-Carlo
algorithm, with running time approximately $O(d^3\log^2d\log q)$ 
field operations (for a precise statement see [Bath paper])
[Check Bath paper].
that takes
as input a subset $Y$ of $\GX(d,q)$ and endeavours to prove that
$G:=\langle Y\rangle$ contains $\SX(d,q)$, given that
$G$ is an irreducible subgroup of $\GX(d,q)$ that does not
preserve any bilinear or quadratic form not preserved by $\GX(d,q)$.
Their algorithm
carries out a random search, looking for certain test elements of $G$.  If it
finds a suitable set of test elements then the group generated by these test elements
must contain $\SX(d,q)$.  
In general, the criterion for an element of $G$ to be a test
element is that its order be divisible by a prime satisfying certain conditions.
These primes do not divide $q-1$ (or $q^2-1$ in the case of $\SU(d,q)$), and
so they remain test elements when raised to the power $n$, where $n=q-1$
in the case of $\GL(d,q)$, and is $q+1$ in the case of $\GU(d,q)$.
Thus the $n$-th
powers of the test elements  generate either $\SX(d,q)$, or a reducible group,
or a group that preserves some form not preserved by $\GX(d,q)$.  The number
of test elements required in practice is at most 4.  [Check this].
To find a suitable set of test elements, the expected number of
random elements to be examined is asymptotically
$O(\log\log d)$; see Proposition 7.5, and Theorem 7.6 of [NP LMS].

As a first approximation to computing the derived subgroup of $G$ we may thus
use the N-P algorithm to prove that $G$ contains $\SX(d,q)$, take the set of
test elements that were found, and set $H$ to be the group generated by
the $n$-th powers of the test elements.  Then any subgroup of $\SX(d,q)$ that contains
$H$ is either $\SX(d,q)$, or is reducible, or preserves a form not preserved
by $\SX(d,q)$.

Before dealing with these possibilities we look more closely at the test elements. 
A {\it primitive prime divisor} of $q^e-1$ is a prime divisor of
$q^e-1$ that does not divide $q^i-1$ for any positive integer $i<e$.  Note
that this involves an abuse of notation, in that $q$ cannot be unambiguously
determined by $q^e-1$, but only from its representation in this form.  If
$r$ is a primitive prime divisor of $q^e-1$ then $r\equiv 1\bmod e$, and so
$r\ge e+1$.  $r$ is defined to be a {\it large primitive prime divisor} of
$q^e-1$ if $r$ is a primitive prime divisor of $q^e-1$, and either $r>e+1$
or $r^2$ divides $q^e-1$.  Finally, $r$ is defined to be a {\it basic}
primitive prime divisor of $q^e-1$ if $r$ is a primitive prime divisor of 
$p^{ae}-1$ where $p$ is prime and $q=p^a$.  The fact that this a stronger 
condition than being simply a primitive prime divisor of $q^e-1$ points to
the above mentioned abuse of notation.  A {\it ppd}$(d,q;e)$ element of
$G$ is one whose order is a multiple of a primitive prime divisor of
$q^e-1$, and the set of test elements will in general contain two elements
$h_1$ and $h_2$ that are ppd$(d,q;e_i)$ elements where $e_1\ne e_2$,
and $e_i>d/2$ for $i=1,2$.  Thus our supposed generating set for $G'$ contains
two elements $k_1$ and $k_2$, powers of $h_1$ and $h_2$, that are
ppd$(d,q;e_i)$ elements, and thus act irreducibly on subspaces $W_1$ and $W_2$
of $V$ of dimensions $e_1$ and $e_2$.  These subspaces are readily computed.
In general they will span $V$; but if not then $k_1$ may be replaced by a
suitable $G$-conjugate of $k_1$ so that this condition is satisfied.  Then
$H$ acts irreducibly on $V$.  It remains to see whether $H$ preserves
some form not preserved by $\SX(d,q)$.  In the present context the only
issue is to whether the supposed generators of $\SL(d,q)$ preserve a 
non-degenerate form.
If $\SL(d,q)<G\le\GL(d,q)$ and $d>2$ then the proportion of elements of $G$ that, when
raised to the power $q-1$, give an element that does not preserve a 
non-degenerate form is of the form $1+O(1/q)$, and is always positive, 
and the condition may be checked by considering its characteristic
polynomial of the exponentiated element.  Thus we can easily impose the conditions that our supposed generators of $\SL(d,q)$ generate an irreducible subgroup of $\GL(d,q)$ that 
preserves no non-degenerate bilinear form, and hence, by applying the
analysis of N-P, hope to be able to assert that the given elements do
indeed generate $\SL(d,q)$; and similarly for $\SU(d,q)$, where the
question of an additional  form being preserved does not apply.

Unfortunately the set of test elements required by the N-P algorithm is not always
quite as simple as this.  In [NP LMS] p.159, Case 1, they discuss the
problem of distinguishing subgroups of $Z\times \PGL(2,7)$ from $\SL(3,q)$ when
$q=3.2^s-1$ for some $s\ge2$.  This they do by observing that
$\SL(3,q)$ contains many elements of order a multiple of 8, and $Z\times\PGL(2,7)$
contains none.  But $\GL(3,11)$ contains no element of order 16, so we cannot
find an element of $\SL(3,11)$ of order 8 by raising an element of $\GL(3,11)$
to the power 10.  Also, in [N-P JAMS] p. 248, Table 8, the set of test elements
for $\SU(3,3)$ and $\SU(3,5)$ include a test element of even order, and
elements of the type required cannot be obtained from the corresponding
general unitary groups by raising to the power 4 (in the first case) or 6 (in the
second).  These appear to be the only problem cases in the present context,
which presumes that $q$ is odd, and that we are dealing with the linear and
unitary groups.

A further minor issue is the case $d=2$.  In this case we can find by random
search an element that powers up to an element $g_1$ of determinant 1 
and order $q+1$, take a conjugate $g_2$ of this element that does not
commute with $g_1$, and, if $q=5$, take an element $g_3$ of order 5.  Then these
elements will generate $\SL(d,q)$, and can easily be found;
$g_3$ is needed to exclude the possibility that the elements in question generate
$2.A_4$.

We have proved, subject to a very careful reading of [NP LMS] and
[NP JAMS], and a little more analysis, the following theorem.

\begin {theorem}
Let $Y$ be a subset of $\GX(d,q)$, and let $G=\langle Y\rangle$.  Then a generating set
for the derived group of $G$ can be constructed at the cost of (up to a
constant multiple)  of the [NP] algorithm.
\end{theorem}

\begin{proof}
We have seen that, in general, we require one application of the N-P algorithm in $G$,
but in a few cases we need a second application to check the correctness of the result.
We also need up to four exponentiations, the exponent being $q-1$ or $q+1$.  We
also may need to compute one or more characteristic polynomials to exclude the possibility
that we have constructed a group that preserves a form, and to compute the spaces
$W_1$ and $W_2$. Let $h_1$ have characteristic polynomial $f(x)$.  Then $f(x)=u(x)w(x)$, where $u$ is irreducible of degree $e$ for some $e>d/2$, and $W_1$ is the image of $w(g)$.  Let $w(x) = a_0+a_1x+\cdots+x^{d-e)$, and compute $y := v.w(g)$ in $O(d^3)$ field operations.  
Then $y\in W_1$,
and the probability that $y$ is zero is $1:q^{d-e}$.  If $y$ is non-zero, spin up $y$ under $h_1$
to obtain a basis for $W_1$.  Given the  factorisation of $f(x)$ the total Las Vegas time for 
computing $W_1$ is thus $O(d^3)$.  The same applies to computing $W_2$.  To determine
whether or not $W_1+W_2=V$ costs $O(d^3)$ field operations.  If this is not the case, it is easy
to see that we can replace $k_1$ by a random conjugate and try again in $O(d^3)$ steps.  Or
since the probability that this will occur is less than $1/(q-1)$, we can simply start again.
Computing and factorising characteristic polynomials is part of the [NP] algorithm, and
hence is allowed within the costing of the theorem.
\end{proof}

We now turn to decomposing a direct product.  That is to say, we have a generating set
$X$ for $\SX(e,q)\times\SX(d-e,q)$, and wish to construct generating sets for the
direct factors.

We use essentially the same algorithm as for constructing the derived group of $\GX(d,q)$.
We construct an element of $\SX(e,q)$ by taking an element $(g_1,g_2)$ of $G=\langle X\rangle$, and raise this element to the power $n$, where now $n$ is the order of $g_2$.  We need to
check the probability of our obtaining in this way a test element for $\SX(e,q)$.  In general
a test element be a test element by virtue of having an order that is a multiple of some prime, and we need to assess the probability that the order of $g_2$ will not be a multiple of this prime.
Babai, Palfy and Saxl \cite{BPS} prove the following.
\begin{theorem}\label{bps}
Let $C$ be a finite simple classical group,
with natural module of dimension $d$.
For a prime $p$, the proportion
of $p$-regular elements
of $C$ is greater than $1/{2d}$.
\end{theorem}
However, if our prime is, in a natural sense, large, we can do better than this.
If $p$ is a primitive prime divisor of $q^e-1$, and $e>d/2$, then by  
\ref{Lemma5.5} the proportion of 
elements of $\SX(e,q)$ of order a multiple of $p$ is $1/e+O(1/q)$.
On occasion the [NP] algorithm calls for elements test elements that are
required to be multiples of primes (or even non-primes) that are not
captured by \ref{Lemma5.5}.  But in these cases $d$ will be small,
and so we can rely on \ref{bps} in these cases.

Another problem arises, in that we now have to consider symplectic groups, as
well as linear and unitary groups.


\subsection{Bounded rank}
Babai, Palfy and Saxl \cite{BPS} prove the following.
\begin{theorem}\label{bps}
Let $C$ be a finite simple classical group,
with natural module of dimension $d$.
For a prime $p$, the proportion
of $p$-regular elements
of $C$ is greater than $1/{2d}$.
\end{theorem}

This result underpins a black-box Monte Carlo polynomial-time
algorithm of Babai \& Beals \cite[Claim 5.3]{BabaiBeals99}
to obtain one of the direct factors of a semisimple group.

In our limited context, we can readily convert this
to a Las Vegas algorithm which we now summarise. 
We repeat the following step at most $O(d)$ times.  
For a random element $g \in G$, construct
$B$, a multiplicative upper bound for $|g|$.
For primes $p|B$, construct $h = g^{B/p}$ 
and $N=\langle h \rangle^G$. If $N$ acts
irreducibly on a composition factor of the underlying 
vector space of the appropriate dimension, then decide 
using the algorithm of \cite{NP} whether or not $N$ is the 
appropriate classical group.

\begin{lemma}
If $G=\SX(e,q)\times\SX(d-e,q)$, then
we can construct one of its
direct factors in at most Las Vegas 
$O(d^4)$ group operations.
\end{lemma}
Useful only if $d$ is bounded.

\subsection{Using primitive prime divisors}

Given a generating set $X$ for $G$, where $SX(d,q)\le G\le GX(d,q)$ we need
an efficient algorithm to construct a generating set for $SX(d,q)=G'$ as
a straight line program in $X$.  This is done by taking random elements of
$G$, and raising them to the power $q-1$.
[This is wrong except for the case of $\GL(d,q)$.  Thus for unitary
groups we raise to the power $q+1$.]
 We need to estimate the number
of random elements of $G$ needed.  We deal first with the case of generic
parameters in the sense of \cite{NP}.  
This excludes a few small values of $d$,
see {\it ibidem} Theorem 3.6.  

We reproduce some of the definitions of that
paper.  
 If $G$ is given as a subgroup of
$GX(d,q)$ then $G$ is said to have parameters $(\X,d,q)$. The critical
definition is then as follows.

\begin{definition}
Definition 3.2 of [NP].  Let $G\le GL(d,q)$, and suppose that $G$
acts irreducibly on the underlying vector space $V$, and that $G$ has
parameters $(\X,d,q)$.
\begin{enumerate}
\item[(a)] If the following conditions hold, we shall say that $G$ has {\it generic
parameters}:
\begin{enumerate}
\item[ (i)] there are integers $e_1$ and $e_2$ with $d/2<e_1<e_2\le d$
such that $SX(d,q)$ contains both a ppd$(d,q;e_1)$-element and a
$ppd(d,q;e_2)$-element;
\item[(ii)] there is an integer $e$ with $d/2<e\le d$ such that $SX(d,q)$
contains a large ppd$(d,q,e)$;
\item[(iii)] there is an integer $e$ with $d/2<e\le d$ such that $SX(d,q)$
contains a basic ppd$(d,q;e)$-element.
\end{enumerate}
If at least one of the conditions (i)--(iii) fails we shall say that
$G$ has {\it non-generic parameters}.

\item[(b)] Further, if conditions (a) (i)--(iii) hold, with the group $G$
replacing $\SX(d,q)$, we shall say that $G$ is a {\it generic subgroup}
of $\GL(d,q)$.
\end{enumerate}

\end{definition}

It is a running hypothesis of the paper in question that the type \X of
$G$ has been determined.  Thus if $G$ preserves no non-trivial form
modulo scalars then $G$ is of linear type, and if $G$ does preserve a form then
$G$ is of general symplectic, unitary, or one of the orthogonal types.  Since
$G$ is assumed to be irreducible this goes not give rise to any
ambiguity.

We now quote a simplified version of Theorem 4.8 of \cite{NP}.

\begin{theorem}
Suppose that $G$ has parameters $(\X,d,q)$,
so that $G\le\GX(d,q)$.  Suppose also that $d\ge3$ and $G$ acts irreducibly
on the underlying vector space $V$.  If $G$ is a generic subgroup then one of 
the following holds.
\begin{enumerate}
\item[(a)] Classical examples, $G\ge SX(d,q)$.
\item[(b)] Extension field examples: there is a prime $b$ such that $b$ divides $d$
and $G$ is conjugate to a subgroup of $\GL(d/b,q^b).b$.
\item[(c)] Nearly simple examples: $G$ belongs to a small class of almost simple
groups.
\end{enumerate}
\end{theorem}

The theorem has been simplified as follows.  In case (b) further information
about $b$ is given in the original paper.  In case (ii) the various possibilities for 
$G'$ are
given explicitly.  These derived groups are all simple or central
extensions of simple groups.  For any particular value of $(d,q)$ there
are at most three possibilities for $G'$, this bound being reached in
the case of $(d,q)=(11,2)$, when $G'$ could be $M_{23}$ or $M_{24}$ or
$\PSL(2,23)$.

The critical point now is that if $\SX(d,q)\le G\le\GX(d,q)$, where
these parameters are generic, and if $N$ independent random elements of
$G$ are taken, then the probability that this set does not contain
two ppd$(d,q,e_i)$-
elements for some $(e_1,e_2)$ where $d/2<e_1<e_2\le d$, one of these
being a large ppd element and one being a basic ppd element, is
bounded above by a function of $d$ and $N$, independent of $q$, that
is monotonic decreasing as a function of $d$, and that tends exponentially
to 0 as a function of $N$ if $d>3$.  This information is extracted from
Corollary 6.3, Theorem 6.4 and Lemma 6.5 of \cite{NP} except that some orthogonal
cases are excluded, except in so far as they are covered by the observation
`Since the proof is so technical, we will assume that we are not in the case 
$\O^0$ or $\O^-$.  However, a bound for these cases can be derived analogously'.

We now prove that if we take $N$ random elements of $G$, where
$\SX(d,q)\le G\le\GX(d,q)$, the probability of these elements, when
raised to the power $q-1$, generating $\SX(d,q)$ is bounded above by a 
function with the above properties, provided that $d$ is big enough.

The requirement that $d$ is to big enough allows us to assume that
$G$ has generic parameters.  We may also assume that the set of
random elements contains two ppd elements with the above properties.
Let $H$ denote the subgroup of $G$ generated by the powers in question.

We first prove that $H$ is probably irreducible.

$H$ contains two ppd elements, say $h_1$ and $h_2$, that act irreducibly on
submodules of $V$ of dimensions $e_1$ and $e_2$, where $e_i>d/2$.  Since
$h_1$ and $h_2$ were chosen independently these spaces probably span the
whole of $V$, in which case it follows easily that $V$ will be an irreducible
$H$-module.  By `probably' we mean `with probability bounded away from 0 by a positive
absolute constant', though a much stronger result would obviously hold.

We now need to consider the probability of $H$ lying in one of the
non-classical groups in cases (b) and (c) of the above theorem.
In this case $H$ lies in the centraliser of one of a small number of
explicitly defined proper subgroups of $SX(d,q)$, and it is easy to
see that if $h$ is a random element of $H$, chosen according to a
probability distribution that is constant on conjugacy classes then
$h$ will probably centralise none of these groups, with the same meaning
of `probably'.  This proves the following result.

\begin{theorem}
 Let $\SX(d,q)\le G\le\GX(d,q)$, and let $\GX(d,q)/\SX(d,q)$
have order $k$.  Take $N$ uniformly distributed random elements of
$G$ and raise them to the power $k$.  There is a function $f_\X(N,d)$,
independent of $q$, such that $f_\X(N,d)$ is monotonic decreasing as a
function of $d$, and converges exponentially fast to 0 (for $d>3$) as a
function of $N$; and the probability that the above $k$-th powers
fail to generate $SX(d,q)$ is less than $f_X(N,q)$.
\end{theorem}

In exactly the same way we can can find a generating set for
$H=\SX(e,q)$ from a generating set of $G=\SX(e,q)\times\SX(d-e,q)$ if $e$
is not too small; that is to say, is greater than some absolute
constant.  Thus we take $N$ random elements $(h_i,k_i)$ if $G$, and raise
each to the order of $k_i$.  If $h_i$ is a ppd$(d,q;a)$ element for
some $a$ then so is $h_i$ raised to the power of the order of $k_i$
provided that the prime in question does not divide the order of $k_i$,
and the property of being a large or primitive ppd element will also
not be destroyed in this case.  But the prime in question is unlikely
to divide the order of $k_i$ as we shall see.  A simplified version
of Theorem 5.7 of \cite{NP} is as follows:

\begin{theorem}
 Let $\SX(d,q)\le G\le\GX(d,q)$ where we are not in an
orthogonal case, and let $e$ be even in the symplectic case and be odd
in the unitary case.  Then the proportion ppd$(G,e)$ of ppd$(d,q;e)$ elements
of $G$ satisfies $1/(e+1)\le {\rm ppd}(G,e)\le 1/e$.
\end{theorem}

The theorem has been simplified by omitting the case of orthogonal groups.

It follows that the probability of the ppd property in question being destroyed
by powering is less than $2/d$, as required.  Note also that this remains the
case if we raise $(h_i,k_i)$ to the power given by the over order of $k_i$
rather than the order of $k_i$, thus avoiding problems with integer factorisation.

Seress \cite[Theorem 2.3.9]{Seress03} presents 
a Monte Carlo polynomial time
algorithm to construct a generating set for the 
derived group of a black-box group. 

Remark: There is a problem about only looking for ppd$(q,d;e)$ elements for
odd $e$ when in the unitary case that I need to understand.


\section{Complexity of the algorithms} 
\label{Analysis}
We now analyse the principal
algorithms, and in the next section estimate the length of the SLPs
that express the canonical generators as words in  the given
generators. The time analysis is based on counting the number of
field operations, and the number of calls to 
the discrete logarithm oracles. Use of discrete
logarithms in a given field requires first the setting up of certain
tables, and these tables are consulted for each application. The
time spent in the discrete logarithm algorithm, and the space that it
requires, are  not proportional to the number of applications in a
given field.

We first consider the costs associated with tasks 
not previously discussed.

Babai \cite{Babai91} presented a Monte Carlo algorithm to
construct in polynomial time nearly uniformly distributed 
random elements of a finite group.  An alternative is the 
{\it product replacement algorithm} of Celler
{\it et al.\ }\cite{Celleretal95}.
That this is also polynomial time was
established by Pak \cite{Pak00}.
For a discussion of both algorithms, we refer
the reader to \cite[pp.\ 26-30]{Seress03}.


We now complete our analysis of the main algorithms.
\begin{theorem}\label{Theorem1}  
The number of field operations carried out in
Algorithm {\tt OneEven} is at most  
$O(d (\xi + d^3 \log q)$.
\end{theorem}
\begin{proof} 
The construction of a hyperbolic basis for a vector space with a given
symplectic or hermitian form, as in line 5, can be carried out
in $O(d^3)$ field operations \cite[Chapter 2]{Grove02}.

The proportion of elements of $G$ with the required property in line 6
is at least $k/d$ for some absolute constant $k$, as proved in Section
\ref{Involution}.

The number of field operations required in lines 8 and 14 is 
$O(d(\xi + d^3 \log q))$,
as proved in Section \ref{Bray}.

The recursive calls in  lines 10 and 11 are to cases of dimension at
most $2d/3$, and hence they increase only a constant factor 
the number of field operations. 

The number of field operations required in lines 9 and 13 is at most 
$O(d^3\log q)$, as proved in Section \ref{Pow}. 

The result follows.
\end{proof}

The algorithm is Las Vegas. Thus a more precise statement
would be that the probability of $kd^4$ field operations proving
insufficient tends to zero exponentially as a function of $k$. 
The field operations counted are the operations of elementary
arithmetic. 

We record the number of calls to the $\SL(2, q)$ construct
recognition algorithm and the associated discrete logarithm oracle.
\begin{theorem}\label{Theorem2} 
Algorithm {\tt OneEven} generates at most $4d$ calls to 
the discrete logarithm oracle for $\GF(q)$.
\end{theorem}
\begin{proof}
{\tt Needs consideration.} 
Each call to the constructive recognition oracle for SL2 generates 
three calls to the discrete logarithm oracle for $\GF(q)$ \cite{Conderetal05}.
Let $f(e) =  \alpha\cdot e - 6$ denote the number
of calls generated by applying {\tt OneEven} to $\SL(e, q)$, 
where $\alpha$ is some positive constant.  
Then $f(d) = f(e) + f(d - e) + 2 \cdot 3 = \alpha d - 6$.
There are 9 calls to the discrete log oracle for degree 4.
Hence the number of calls is at most $4d$.
\end{proof}
Similar results hold for Algorithm {\tt OneOdd} and so 
Algorithm {\tt OneMain} also has this complexity.

The results of the analysis of Algorithm {\tt TwoMain} are 
qualitatively similar; however it generates at most $d-1$ calls to the 
constructive recognition algorithm for $\SL(2,q)$.  

\section{Straight Line Programs}\label{SLP}

We now consider the length of the straight line programs (or SLPs) for
the standard generators for $\SX(d, q)$ constructed by our algorithms.

In its simplest form, 
an SLP on a subset $X$ of a group $G$ is a
string, each of whose entries is either a pointer to an element of $X$, 
or a pointer to a previous
entry of the string, or an ordered pair of pointers to not necessarily 
distinct previous entries.
Every entry of the string defines an element of $G$.  An 
entry that points to an element of
$X$ defines that element.  An entry that points to a previous entry defines 
the inverse of the element defined by that entry.  An entry that points to 
two previous entries defines the
product, in that order, of the elements defined by those entries.

Such a simple SLP defines an element 
of $G$, namely the element defined by the last entry, and this element 
can be obtained by computing in turn
the elements for successive entries.
The SLP is primarily used by replacing the elements 
$X$ of $G$ by the elements $Y$ of some group
$H$, where $X$ and $Y$ are in one-to-one correspondence, and then evaluating 
the element of $H$ that the SLP then defines.

It is easy to arrange for our algorithms to 
construct SLPs for the standard generators of 
$\SX(d,q)$ on the given generating set $X$ for $G$.

%The reason for using SLPs rather than words in $X$ is that, as successive
%multiplications are carried out, the length of the corresponding words in 
%$X$ can grow exponentially as a function of the number of multiplications 
%performed, whereas the SLP will only
%grow linearly.

We now identify other desirable features of SLPs. 
\begin{enumerate}
\item 
We need to replace the second type of node, that defines the 
inverse of a previously defined element, by a type of node with two fields, 
one pointing to a previous entry, and one containing 
a possibly negative integer.  The element defined is then the 
element defined by the entry to which
the former field points, raised to the power defined by 
the latter field.  This reflects the fact that we
raise group elements to very large powers, and have 
an efficient algorithm for performing this.
Of course it may be convenient to have nodes corresponding 
to other group-theoretic constructions such as commutators.

\item 
We should regard an SLP as defining a number of elements of $G$,
and not just one element, so a sequence of nodes may be specified as 
giving rise to elements
of $G$.  Thus we wish to return a single straight line program that defines 
all the standard generators of $\SX(d,q)$, rather than one straight line 
program for each of these elements.
This avoids duplication when two or more of the standard generators 
rely on common calculations.

%\item 
%In order to preserve space the structure of an SLP needs to 
%be enhanced to ensure that, when the
%SLP is evaluated in some other group, the element defined by a 
%node is only calculated when it
%will be needed later, and is discarded when it it is no longer needed.  
%Discarding the element of $H$ defined by a node when it is no longer 
%required in an evaluation ensures that the space complexity of evaluating 
%an SLP is at worst proportional to the space complexity of the space 
%required to construct the corresponding element (or elements) of $G$ 
%in the first place, given a bound to the space required to store 
%an element of $H$.

\item 
A critical concern is how the number of trials in a random search for 
a group element effects the length of an SLP that defines that element.  
This requires an
assumption as to the nature of the random process.  We assume that this 
random process
is a stochastic process taking place in a graph whose vertices 
are defined by a seed, the seed
consisting of an array of elements of the group.  We refer to this 
array as {\it the seed}.  
We assume that a random number generator now determines which 
edge adjoining the current vertex in the graph will be followed in 
the stochastic process.
If no effort is made to improve the situation, then the length of the SLP will 
then be increased by a constant
amount for every trial, successful or unsuccessful. This constant can, 
in effect, be regarded as 1 by testing all the group elements constructed in 
updating the seed.  However, this is not good enough for our purposes.  
We propose the following solution to the problem.  {\it When embarking on a 
search that is
expected to require approximately $d$ trials, we record the value 
of the seed, and repeatedly carry
out a random search, using our random process, but returning, after 
every $c\log d$ steps, for some suitable constant $c$, to the 
stored value of the seed, until we succeed.}  If the vertices in the graph
have valency at least $v$ then $c$ should be chosen so that  
$c\log d$ is significantly less than $\log_v(d)$.  
\end{enumerate}

%With simple versions of product replacement this 
%valency will be at least 50, so in practice
%the number of steps taken before returning to the stored 
%seed may be taken as a small constant.

We now turn to our analysis of the SLPs, which we assume 
have these additional features. 

Both algorithms rely on a divide-and-conquer strategy. The first
algorithm produces recursive calls to $\SX(e,q)$ and to  $\SX(d-e,q)$
where $e$ is approximately $d/2$. The second algorithm reduces to the
case when $d$ is a multiple of 4, and then has a single recursive call
to $\SX(d/2,q)$. Since the time complexity of the algorithm is greater
than $O(d^3)$, for fixed $q$, the cost in time of the recursive calls
is unimportant. This is not the case with the length of the
SLPs. The algorithm expends most of its time in
random searches; ignoring the construction and testing of
random elements that fail to pass the required test,
the number of group
operations outside  recursive calls, including exponentiation as a
single operation, becomes constant in the first main algorithm, and 
$O(\log d)$ in
the second, where the involution with eigenspaces of equal dimension 
that is used when $d=4n$ is constructed as a product of $O(\log d)$ involutions.

Thus in Algorithm {\tt OneMain}, where the number of recursive calls 
is $O(d)$, the SLP in question has length approximately $O(d)$.   
In Algorithm {\tt TwoMain}, 
where the number of recursive calls 
is $O(\log d)$, the length of the SLP is approximately $O(\log^2 d)$.  
However, in each case there
are random searches of length $O(d)$ that multiply these estimates 
by another factor of $\log d$.

%XXXX  Small point.  In the SLP section we argue for a single SLP to describe 
%all the standard generators.  Do we want to use SLP here in the singular 
%or in the plural?

We thus arrive at the following result.
\begin{theorem} 
%Counting exponentiation as a single operation, 
Subject to the properties specified above, the lengths of 
the SLP for the standard generators produced by 
{\tt OneMain} is $O(d\log d)$; the length produced by {\tt TwoMain}
is $O(\log^3 d)$.
\end{theorem}

\section{An implementation}
Our implementation of these algorithms is publicly available in {\sc Magma}.
It uses:
\begin{itemize}
\item 
the product replacement algorithm \cite{Celleretal95}
to generate random elements; 
\item our implementations of Bray's algorithm \cite{Bray}
and the involution-centraliser algorithm \cite{Ryba-paper}.
\item our implementations of the algorithm of \cite{Conderetal05}
and \cite{sl3q}.
\end{itemize}

The computations reported in Table \ref{table1} were carried out
using {\sc Magma} V2.13 on a Pentium IV 2.8 GHz processor.
The input to the algorithm is $\SX (d, q)$.
In the column entitled ``Time", we list the CPU time in seconds
taken to construct the standard generators.

\begin{table}[h]
\caption{Performance of implementation for a sample of groups}
\label{table1}
\begin{center}
\begin{tabular}
{|c|r|} \hline
Input  &   Time  \rule{0cm}{2.5ex}\\
\hline
$SL_3(11)$ & $2.1$      \rule{0cm}{2.5ex}\\ \hline
$SL_6(2)$  & 13.1 \rule{0cm}{2.5ex}\\ \hline
$Sp(10,5^{10})$ & 55.4 \rule{0cm}{2.5ex}\\ \hline
$Sp(40,5^{10})$ & 2980.4 \rule{0cm}{2.5ex}\\ \hline
$SU_{8}(3^{16})$ & 22.6 \rule{0cm}{2.5ex}\\ \hline
$SU_{20}(5^{12})$ & 47.6 \rule{0cm}{2.5ex}\\ \hline
$SU_{70}(5^2)$ & 191.3 \rule{0cm}{2.5ex}\\ \hline
\end{tabular}
\end{center}
\end{table}

\begin{thebibliography}{10}

\bibitem{Babai91}
L\'aszl\'o Babai,  
Local expansion of vertex-transitive graphs and
  random generation in finite groups,  {\it Theory of Computing}, (Los
  Angeles, 1991), pp.\ 164--174. Association for Computing Machinery, 
New York, 1991.

\bibitem{BabaiSzemeredi84}
L\'aszl\'o Babai and Endre Szemer\'edi.
\newblock On the complexity of matrix group problems, {I}.
\newblock In {\em Proc.\ $25$th IEEE Sympos.\ Foundations Comp.\ Sci.}, pages
  229--240, 1984.

\bibitem{BabaiBeals99}
L. Babai and R.M. Beals,
\newblock A polynomial-time theory of black-box groups. {I}.
\newblock In {\em Groups St. Andrews 1997 in Bath, I}, volume 260 of {\em
  London Math. Soc. Lecture Note Ser.}, pages 30--64, Cambridge, 1999.
  Cambridge Univ. Press.

\bibitem{BPS} L. Babai, P.\ P{\'a}lfy and J.\ Saxl, On the number
of $p$-regular elements in simple groups, preprint.


\bibitem{Bray} J.N. Bray, An improved method of finding
the centralizer of an involution, {\it Arch. Math. (Basel)}
{\bf 74} (2000), 241--245.

\bibitem{Magma}
Wieb Bosma, John Cannon, and Catherine Playoust,
\newblock The {\sc Magma} algebra system I: The user language,
\newblock {\em J.\ Symbolic Comput.}, {\bf 24}, 235--265, 1997.

\bibitem{Brooksbank03}
P.A. Brooksbank,
\newblock Constructive recognition of classical groups
in their natural representation.
\newblock {\em J. Symbolic Comput.} {\bf 35} (2003), 195--239.

\bibitem{Celleretal95}
Frank Celler, Charles R.\ Leedham-Green, Scott H.\ Murray, Alice C.\
  Niemeyer and E.A.\ O'Brien, Generating random elements of a 
finite group, {\it Comm.\ Algebra}, {\bf 23} (1995), 4931--4948.

\bibitem{CLG97}
Frank Celler and C.R.\ Leedham-Green,
\newblock Calculating the order of an invertible matrix,
\newblock In {\em {Groups and Computation {II}}}, volume~28 of {\em Amer.\
  Math.\ Soc.\ DIMACS Series}, pages 55--60. (DIMACS, 1995), 1997.


\bibitem{CellerLeedhamGreen98}
F.~Celler and C.R. Leedham-Green.
\newblock A constructive recognition algorithm for the special linear group.
\newblock In {\em The atlas of finite groups: ten years on (Birmingham, 1995)},
  volume 249 of {\em London Math. Soc. Lecture Note Ser.}, pages 11--26,
  Cambridge, 1998. Cambridge Univ. Press.

%\bibitem{CMT}
%Arjeh M.\ Cohen, Scott H.\ Murray, and D.E.\ Taylor.
%\newblock Computing in groups of Lie type.
%\newblock {\em Math. Comp.\ }{\bf73}, 1477-1498, 2003.

%\bibitem{ConderLeedhamGreen01}
%Marston Conder and Charles~R. Leedham-Green.
%\newblock Fast recognition of classical groups over large fields.
%\newblock In {\em Groups and Computation, III (Columbus, OH, 1999)}, volume~8
%  of {\em Ohio State Univ. Math. Res. Inst. Publ.}, pages 113--121, Berlin,
%  2001. de Gruyter.

\bibitem{Conderetal05}
M.D.E. Conder, C.R. Leedham-Green, and E.A. O'Brien.
\newblock Constructive recognition of PSL$(2, q)$.
\newblock {\em Trans.\ Amer.\ Math.\ Soc.} {\bf 358}, 1203--1221, 2006.

\bibitem{FNP}
Jason Fulman, Peter M. Neumann, and Cheryl E. Praeger.
A Generating Function Approach to the Enumeration
of Matrices in Classical Groups over Finite Fields.
Mem. Amer. Math. Soc. {\bf 176}, no.\ 830, 2005.

\bibitem{GLS3}
Daniel Gorenstein, Richard Lyons, and  Ronald Solomon.
The classification of the finite simple groups. Number 3. Part I,
American Mathematical Society, Providence, RI, 1998. 

\bibitem{Grove02}
Larry C. Grove. Classical Groups and Geometric Algebra.
AMS Graduate Studies in Math.\ {\bf 39}.

\bibitem{GuralnickLubeck01}
R.M. Guralnick\ and\ F. L\"ubeck.
 On $p$-singular elements in Chevalley groups in characteristic $p$.
\newblock In {\em Groups and Computation, III (Columbus, OH, 1999)}, volume~8
  of {\em Ohio State Univ. Math. Res. Inst. Publ.}, pages 113--121,
Berlin, 2001. de Gruyter.

\bibitem{HoltEickOBrien05}
Derek~F.\ Holt, Bettina Eick, and Eamonn~A.\ O'Brien.
\newblock {\em Handbook of computational group theory}.
\newblock Chapman and Hall/CRC, London, 2005.

\bibitem{Ryba-paper} P.E. Holmes, S.A. Linton, E.A. O'Brien, A.J.E. Ryba and
R.A. Wilson, Constructive membership testing in black-box
groups, preprint.

\bibitem{KantorSeress01}
William~M. Kantor and {\'A}kos Seress.
\newblock Black box classical groups.
\newblock {\em Mem. Amer. Math. Soc.}, {\bf 149}, 2001.
                             
%\bibitem{LG01}
%Charles R.\ Leedham-Green,
%The computational matrix group project, in
%{\it Groups and Computation}, III (Columbus, OH, 1999), 229--247, Ohio
%State Univ. Math. Res. Inst. Publ., {\bf 8}, de Gruyter, Berlin, 2001.
                                                                                
%\bibitem{LiebeckOBrien05}
%Martin~W. Liebeck and E.A.\ O'Brien.
%\newblock Finding the characteristic of a group of Lie type.
%\newblock Preprint, 2005.

\bibitem{lish} M.W. Liebeck and A. Shalev. The probability of generating
a finite simple group, {\it Geom. Ded.} {\bf 56} (1995), 103--113.

\bibitem{sl3q}
F.\ L{\"u}beck, K.\ Magaard, and E.A. O'Brien. 
Constructive recognition of $\SL_3(q)$.
Preprint 2005.

\bibitem{Huppert67}
B.\ Huppert.
\newblock {\em {E}ndliche {G}ruppen {I}}, volume 134 of {\em Grundlehren Math.\
  Wiss.}
\newblock Springer-Verlag, Berlin, Heidelberg, New York, 1967.

\bibitem{NeumannPraeger92}
Peter~M.\ Neumann and Cheryl~E.\ Praeger.
\newblock A recognition algorithm for special linear groups.
\newblock {\em Proc.\ London Math.\ Soc.\ $(3)$}, 65:555--603, 1992.

\bibitem{NP} A.C. Niemeyer and C.E. Praeger.
A recognition algorithm for classical groups over finite fields,
{\it Proc. London Math. Soc.} {\bf 77} (1998), 117--169.

\bibitem{NP2}
Alice~C. Niemeyer and Cheryl~E. Praeger.
\newblock Implementing a recognition algorithm for classical groups.
\newblock In {\em Groups and Computation, II (New Brunswick, NJ, 1995)},
  volume~28 of {\em DIMACS Ser. Discrete Math. Theoret. Comput. Sci.}, pages
  273--296, Providence, RI, 1997. Amer. Math. Soc.

\bibitem{OBrien05}
E.A. O'Brien. Towards effective algorithms for linear groups.
{\it Finite Geometries, Groups and Computation},
(Colorado), pp. 163-190. De Gruyter, Berlin, 2006.

\bibitem{Pak00}
Igor Pak. The product replacement algorithm is polynomial.
In {\it 41st Annual Symposium on Foundations of Computer Science
(Redondo Beach, CA, 2000)}, 476--485,
IEEE Comput. Soc. Press, Los Alamitos, CA, 2000.

\bibitem{PW05}
C.W. Parker and R.A. Wilson.
Recognising simplicity in black-box groups. 
Preprint 2005.

\bibitem{Seress03}
{\'A}kos Seress.
\newblock {\em Permutation group algorithms}, volume 152 of {\em Cambridge
  Tracts in Mathematics}.
\newblock Cambridge University Press, Cambridge, 2003.

\bibitem{Storjohann98}
Arne Storjohann.
An $O(n\sp 3)$ algorithm for the Frobenius normal form. In
{\em Proceedings of the 1998 International Symposium on Symbolic
and Algebraic Computation} (Rostock), 101--104, ACM, New York, 1998.

\bibitem{vzg}
Joachim von zur Gathen and J\"urgen Gerhard,
{\it Modern Computer Algebra}, Cambridge University Press, 2002.
\end{thebibliography}

\begin{tabbing}
\=\hspace{70mm}\=\kill
\>School of Mathematical Sciences \>Department of Mathematics    \\
\>Queen Mary, University of London \>Private Bag 92019, Auckland \\
\>London E1 4NS, United Kingdom   \>University of Auckland     \\
\>United Kingdom                  \> New Zealand     \\
\> C.R.Leedham-Green@qmul.ac.uk   \> obrien@math.auckland.ac.nz
\end{tabbing}

\vspace*{2mm}
\noindent 
Last revised July 2006

\end{document}

