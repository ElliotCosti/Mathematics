\documentclass[12pt]{article}
\usepackage{lscape}
\usepackage{amssymb}
\usepackage{float}
\usepackage[algo2e,ruled,linesnumbered]{algorithm2e} % for algorithms
\hoffset -25truemm
\usepackage{latexsym}
\oddsidemargin=30truemm  %%
\evensidemargin=25truemm %% inner margin 30mm, outer margin 25mm
\textwidth=155truemm  %%
\voffset -25truemm
\topmargin=25truemm   %% top margin of 25mm
\headheight=0truemm   %% no head
\headsep=0truemm   %% no head
\textheight=220truemm  
\renewcommand{\thefootnote}{}
\newtheorem{definition}{Definition}[section]
\newtheorem{lemma}[definition]{Lemma}
\newtheorem{theorem}[definition]{Theorem}
\newtheorem{corollary}[definition]{Corollary}
\newtheorem{remark}[definition]{Remark}
\newtheorem{problem}[definition]{Problem}
\newenvironment{proof}{\normalsize {\sc Proof}:}{{\hfill $\Box$ \\}}

\def\O{{\rm O}}  % We have Oh for the Oh notation, and O for orthogonal.

\def\lcm{{\rm lcm}}
\def\SL{{\rm SL}}
\def\SO{{\rm SO}}
\def\SP{{\rm Sp}}
\def\GL{{\rm GL}}
\def\Oh{O}  % For O notation; in case we change our minds.
\def\U{{\rm U}}
\def\C{{\cal C}}
\def\F{{\cal F}}
\def\PSL{{\rm PSL}}
\def\rad{{\rm rad}}
\def\PSp{{\rm PSp}}
\def\PSU{{\rm PSU}}
\def\GU{{\rm GU}}
\def\GF{{\rm GF}}
\def\Sp{{\rm Sp}}
\def\SU{{\rm SU}}
\def\SX{{\rm SX}}
\def\GX{{\rm GX}}
\def\PX{{\rm PX}}
\def\X{{\rm X}}
\def\X{{\bf X}}
\def\PSX{{\rm PSX}}
\def\PGL{{\rm PGL}}
\def\ADD{Moreover, the number of such elements is
independent of the choice of irreducible factor.}
\def\q{\quad}
\def\centreline{\centerline}
\def\SLP{{\rm SLP}}
\def\SLPs{{\rm SLPs}}
\def\w{\sigma} % For the extra element; in case we change our minds. We did.

\begin{document}

\title{Constructive recognition of classical groups in odd characteristic} 
\author{C.R.\ Leedham-Green and E.A. O'Brien}
\date{}
\maketitle

\begin{abstract}
Let $G = \langle X \rangle \leq \GL(d, F)$ be 
a classical group in its natural representation 
defined over a finite field $F$ of odd characteristic. 
We present Las Vegas algorithms to construct
standard generators for $G$ which permit us 
to write an element of $G$ as a straight-line
program in $X$. The algorithms run in polynomial-time, 
subject to the existence of a discrete logarithm oracle for $F$. 
\end{abstract}

\footnote{This work was supported in part by the Marsden Fund of
New Zealand via grant UOA 412. We thank Peter Brooksbank
and Cheryl Praeger for detailed commentary and discussion on various drafts
of this paper. We thank John Bray,  Bill Kantor, 
Frank L\"ubeck, Alice Niemeyer, and 
Robert Wilson for helpful discussions. 
2000 {\it Mathematics Subject Classification}.
Primary 20C20, 20C40.}

\section{Introduction}
\label{intro}

The goal of the `matrix group recognition project'
is the development of efficient
algorithms for the investigation of 
subgroups of $\GL(d, F)$ where $F$ 
is a finite field. 
A particular aim is to 
construct the composition factors
of $G \leq \GL(d, F)$. If a problem
can be solved for the composition factors,
then it can be frequently be solved for $G$:
examples include constructing the 
Sylow $p$-subgroups of $G$ for given prime $p$.
We refer to the recent survey \cite{OBrien05}
for background related to this work.

One may intuitively think of a {\it straight-line program} (\SLP)
for $g \in G = \langle X \rangle$ as an efficiently stored group word
on $X$ that evaluates to $g$.  
Informally, a {\it constructive recognition algorithm}
constructs an explicit isomorphism
between $G$ and a `standard' (or natural)
copy of $G$, and exploits this isomorphism
to write an arbitrary element of $G$ as
an \SLP\ in its defining generators. 
For a more formal definition, see \cite[p. 192]{Seress03}.

In our context, $\langle X \rangle = G \leq \GL(d, F)$ is a 
classical group of odd 
characteristic in its {\it natural representation},
so one might regard the construction of the 
identity map as an easy exercise!
However, a solution to this problem requires
us to write an arbitrary element of $G$
as an \SLP\ in the given generating set $X$.
In this paper, we define {\it standard generators} 
for $G$ from which \SLPs\ to arbitrary
elements of $G$ may readily be constructed, and 
then devise algorithms to construct
these canonical generators as \SLPs\ in $X$.

We comment briefly on the significance of our work. As a doubly parameterised
family, the classical groups in their natural representation are the
most ubiquitous and challenging of all linear groups. 
The constructive recognition problem is fundamental:
its solution is key to a number of other hard problems,
including conjugacy testing of subgroups and elements, and 
construction of maximal subgroups.
The algorithm we present here to solve the problem
is both provably theoretically efficient 
and also eminently practical: its structure was influenced
by this latter concern. 
Subject to the existence of a discrete logarithm oracle,  
its complexity is 
$\Oh(d (\xi + d^3 \log d + d^2\log d \log\log d\log q))$ measured
in field operations, where $q$ is the field size
and $\xi$ is the cost of constructing a random element. 
%If we assume that the cost of matrix multiplication is $\Oh(d^3)$, 
%then our algorithms achieve essentially optimal theoretical efficiency.

Central to our work are centralisers of involutions, 
long of theoretical importance. As part of our analysis, 
we obtain general theoretical results 
of independent interest about the proportions of certain
kinds of elements in the classical groups. 

Another striking aspect of our work is the short length of 
the \SLPs\ in $X$ we construct to encode the canonical generators of $G$.
Subject to certain 
assumptions about the behaviour of the algorithm used to 
generate random elements, one family of algorithms 
constructs \SLPs\ of length $\Oh(\log^3 d)$, which is polynomial in
$\log\log|G|$; by contrast,
Babai \& Szemer{\'e}di \cite{BabaiSzemeredi84}
prove that an arbitrary element of
$G$ has an \SLP\ in $X$ of length $\Oh(\log^2|G|)$.
%Usually, the length of an \SLP\ is of greater importance than
%the time taken to construct~it.

\subsection{The groups}
We divide the groups of principal interest into three overlapping classes.

The first class consists of the following groups.
In all cases $d > 1$, $q$ is odd, and $V$ denotes the underlying vector space.

\begin{itemize}
\item $\GL(d,q)$, the group of all invertible $d\times d$ matrices 
over $\GF(q)$.

\item $\Sp(d,q)$, the group of all elements of $\GL(d,q)$ that preserve 
a given non-degenerate
alternating bilinear form on $V$.  The existence of such a form 
implies that $d$ is even.

\item $\U(d,q)$, the group of all elements of $\GL(d,q^2)$ that 
preserve a given non-degenerate
hermitian form on $V$.

\item $\O^+(d,q)$, the group of all elements of $\GL(d,q)$ that 
preserve a given 
non-degenerate symmetric bilinear form 
on $V$ of $+$ type.  This implies that $d$ is even.

\item $\O^-(d,q)$ is defined  in the same way, except that the 
form is of $-$ type; again $d$ is even.

\item $\O^0(d,q)$, the group of all elements of $\GL(d,q)$ that preserve 
a given non-degenerate symmetric bilinear form on $V$, where $d$ is odd.
\end{itemize}

The definition of all of these groups, except for the first, 
depends on the choice of form.  However, the groups defined by 
two different forms of the same type 
are conjugate in the corresponding general linear group. 
We use the notation $\GX(d,q)$ to represent any one of the above groups.

The second class of groups is obtained from the first by replacing
each group by the subgroup consisting of the elements of determinant 1.
All elements of $\SP(d,q)$ have determinant 1.  The subgroups of the other
groups thus defined are denoted respectively by
$\SL(d,q)$, $\SU(d,q)$, $\SO^+(d,q)$, $\SO^-(d,q)$ 
and $\SO^0(d,q) = \SO(d, q)$.
Thus $\SP(d,q)$ belongs to both classes.
We use the notation $\SX(d,q)$ to represent any group in the second class.

All of the groups in the second class are perfect with the exception 
of $\SX(2, 3)$ and the orthogonal
groups; the latter contain a unique subgroup of index 2, denoted respectively 
by $\Omega^+(d,q)$, $\Omega^-(d,q)$ and $\Omega^0(d,q)$; with two 
exceptions, $\Omega(3, 3)$ and $\Omega^+(4, 3)$, these groups are 
perfect for $d>2$. 
The third class consists of these three families
together with the non-orthogonal groups in the second class.
We sometimes use the notation $\Omega X(d,q)$ to represent any 
group in the third class.

Let $\C$ denote the union of the second and third classes
of subgroups of $\GL(d, q)$, so it consists of classical
groups in their natural representations.
We say that $\SX(d, q)$ and $\Omega^\epsilon(d, q)$ have 
{\it type} {\bf SL}, {\bf Sp}, {\bf SU}, $\mathbf{SO^\epsilon}$, 
or $\mathbf{\Omega^\epsilon}$,
where $\epsilon \in \{-,0,+\}$, 
and {\it parameters} (type, $d, q$).

We may regard each of these groups as a group of automorphisms 
of a vector space $V$ of
dimension $d$ over $\GF(q)$ (or over $\GF(q^2)$ in the case of unitary groups); 
and we replace $(d,q)$ by $V$ when we wish to specify the vector space. 
If $V$ has an associated non-degenerate form, then we 
may write $\O(V)$ rather than, for example, $\O^-(V)$, allowing 
the type of the form to determine the type of the group.

\subsection{The primary result}
Let $G$ be a classical group in its natural representation 
contained in $\C$.
We present and analyse two Las Vegas algorithms that
take as input a generating subset $X$ of $G$ and 
the form preserved by $G$, 
and return as output {\it standard generators} of $G$ as 
\SLPs\ in $X$.  (These standard generators are defined
in Section \ref{standard}.) Usually, these  
generators are defined with respect to a 
basis different to that for which $X$ was defined, 
and a change-of-basis 
matrix is also returned to relate these bases.

Let $\xi$ denote an upper bound to the number of field operations 
needed to construct an independent
(nearly) uniformly distributed random element of a group,
and let $\chi(q)$ denote an upper bound to the 
number of field operations
equivalent to a call to a discrete logarithm oracle for $\GF(q)$.

Our principal result is the following.
\begin{theorem} \label{main}
There is a Las Vegas algorithm that takes as input a subset
$X$ of bounded cardinality of $\GL(d,q)$, where $X$ generates 
a group $G$ in $\C$, and 
returns standard generators for
$G$ as \SLPs\ of length $\Oh(\log^3 d)$ in $X$.
The algorithm has complexity 
$\Oh(d (\xi + d^3 \log d + d^2\log d \log\log d\log q+ \chi(q)))$ measured
in field operations if
$G$ is neither of type $\mathbf{SO^-}$ or $\mathbf{\Omega^-}$.  
Otherwise the complexity is 
$\Oh(d (\xi + d^3 \log d + d^2\log d\log\log d \log q + \chi(q))+\chi(q^2))$ 
measured in field operations.
\end{theorem}

We prove this theorem by  exhibiting  an  algorithm with the given
specifications; more precisely, we exhibit two  algorithms for each of the
given types of group. For each type, the first algorithm is designed
to run fast, and the second  to produce shorter straight line programs.
The first algorithm spends less time in the parent group; the second
spends more time in the parent  group, but generates fewer
recursive calls. The bound of $\Oh(\log^3 d)$ for the length of the
\SLPs\ is achieved only by the second algorithm;
both have the stated time complexity.  
The second algorithm does not apply directly to orthogonal groups 
that are not of $+$ type;
but, when dealing with the other orthogonal groups in large 
dimensions, most of the work is
carried out in an orthogonal subgroup of $+$ type, and this subgroup 
can be processed using the second algorithm.

If we {\it assume} that a random element of the group can be 
constructed in $\Oh(d^3)$ field operations, then, for fixed $q$,
and subject to the existence of a discrete logarithm oracle,
both algorithms require $\Oh(d^4\log d)$ field operations to construct 
the standard generators. 

Our estimate of the complexity contains the term $d\xi$.  This encodes the
fact that $\Oh(d)$ random elements of the group are constructed outside
the recursive calls. Random elements are also constructed
in the recursive calls: the total number of random elements 
constructed is $\Oh(d\log d)$.  If $\xi$ is at least $d^3$, 
then Lemma \ref{d-a-c} implies that the cost of constructing 
random elements in the recursive calls does not affect 
the complexity estimate.

Once we have constructed these standard generators for $G$,
a standard `generalised echelonisation' algorithm can 
be used to write a given element of $G$ 
as an \SLP\ in these generators.
We do not consider this task here, but refer 
the interested reader to the algorithm
of \cite[Section 5]{Brooksbank03},
which performs this task in 
$\Oh(d^3\log q + \log^2 q)$ field operations. 

\subsection{Related work}
Already constructive recognition algorithms exist 
for various families of groups. 

Brooksbank's algorithms \cite{Brooksbank03} to construct
(different and larger) canonical generating sets for the natural 
representation of each of 
$\Sp(d, q)$, $\SU(d, q)$, and $\Omega^\epsilon(d, q)$ 
have complexity 
$$\Oh(d^3 \log q (d + \log d \log^3 q) + (d + \log \log q)\xi + d^5 \log^2 q
 + (\log q)\chi (q^2)),$$
and again assumes the existence of a discrete logarithm oracle.
The algorithm of Celler \& Leedham-Green \cite{CellerLeedhamGreen98}
for $\SL(d, q)$ has complexity $\Oh(d^4  q)$.

Kantor \& Seress \cite{KantorSeress01} have developed  
{\it black-box} constructive recognition algorithms 
(see \cite[p.\ 17]{Seress03}) for 
the classical groups.  These algorithms do not run
in time polynomial in the size of the input:
their complexity involves $q$.
However, Brooksbank \& Kantor \cite{BK} demonstrate that
the complexity of these algorithms can be
made polynomial in $\log q$ given an oracle for explicit 
membership testing in $\SL(2,q)$ and
(in some cases) in $\SL(2,q^2)$.  
Subject to a fixed number of calls to a discrete logarithm oracle
for $\GF(q)$, Conder \& Leedham-Green \cite{ConderLeedhamGreen01} and
Conder, Leedham-Green \&
O'Brien \cite{Conderetal05} present a Las Vegas algorithm
which constructively recognises $\SL(2, q)$
as a linear group in defining characteristic
in time polynomial in the size of the input.
Brooksbank \cite{Brooksbank03a} and
Brooksbank \& Kantor \cite{BK, BK06} have exploited
this work to produce better constructive
recognition algorithms for black-box classical groups.

Other constructive recognition algorithms include those  of 
B\"a\"arnhielm \cite{HB} for the Suzuki groups and of 
Beals {\it et al.\ }\cite{BLNPS1} 
for black-box representations of alternating groups. 

\subsection{Other directions}
We plan to develop similar constructive recognition algorithms for 
classical groups of characteristic 2 in their natural representation.
We will also generalise these algorithms to deal with 
an arbitrary representation of a classical
group in the defining characteristic.  Now the 
problem of writing group elements as
\SLPs\ in the standard generators is not so easy; work on 
this problem is in progress.

\subsection{The content of the paper} 
In Section \ref{cent} we review some background material
on forms, and summarise
the structure of involution centralisers
for elements of classical groups in odd characteristic.
In Section \ref{standard} we define
standard generators for the classical groups.
In Sections \ref{Alg1}--\ref{Alg4} 
the algorithms are described: the non-orthogonal groups 
are first presented uniformly. 
The algorithms use involutions whose $-1$-eigenspaces  have
dimensions in a prescribed range.  The cost 
of finding and constructing such involutions 
is analysed in Sections \ref{Involution} and \ref{Equal}.
We frequently compute high powers of elements of linear groups; 
an algorithm to do this
efficiently is described in Section \ref{Exp}. 
In Section \ref{Pow} we discuss how to  
construct the perfect quotients of the factors of a 
direct product of two classical groups. 
The centraliser of an involution is constructed using an 
algorithm of Bray \cite{Bray}; this is considered
in Section \ref{Bray}. The base cases of the
algorithms (when $d \leq 6$) are discussed in Sections \ref{base}
and \ref{base-omega}. 
The complexity of the algorithms and the 
length of the resulting \SLPs\ for the standard generators
are discussed in Section \ref{Analysis} and \ref{SLP}.  
Finally we report on our implementation of the algorithm, 
publicly available in {\sc Magma} \cite{Magma}.

\section{Notation and background}\label{cent} 
Throughout the paper $q$ denotes an odd prime power.

We assume familiarity with most basic results in the theory of
classical groups; all can be found in \cite{Taylor}.  
Recall that the spinor norm (see \cite[p.\ 163]{Taylor})
is a homomorphism from
$\SO^\epsilon(d,q)$ to $\{\pm1\}$ with kernel $\Omega^\epsilon(d,q)$. 
We use Witt's Theorem in the following form. 
\begin{theorem}  \label{Witt}
Let $V$ be a  finite  dimensional  vector  space that  supports  a
non-degenerate bilinear or hermitian form.  
Let $U$  and $W$ be subspaces of $V$,
and let $g$ be  a linear isometry from $U$  to $W$.  Then there is a
linear isometry $f$ from $V$ to $V$ such that  $uf=ug$ for all $u\in U$.
\end{theorem}
For  a proof see \cite[Theorem 7.4]{Taylor}; we have specialised the
quoted theorem
to the case where the form is non-degenerate. 
If the bilinear form restricted to $U$ is non-degenerate, then 
$f$ can be chosen to have determinant 1. 
If, in addition, the form is symmetric, and 
$U$ has codimension at least 2 in $V$, we 
can choose $f$ to have determinant 1 and
spinor norm 1. This is because $V=W\oplus  W^\perp$,  and a  linear
isometry $h$ can be constructed from $V$ to  $V$ that maps $W$ to $W$
as the identity, and that maps  $W^\perp$ to $W^\perp$ with the  same
determinant and (when relevant) the same  spinor norm as $h$, and then
$fh^{-1}$ will be the required isometry of $V$.

If $V$ has a bilinear form, then we denote the image of 
the ordered pair of vectors $(u,v)$ in $V\times V$ under the form by $u.v$.

Let $g\in G \leq \GL(d, q)$, 
let $\bar{G}$ denote $G / G \cap Z$
where $Z$ denotes the centre of $\GL(d, q)$,
and let $\bar{g}$ denote the image of $g$ in $\bar{G}$.
The {\it projective centraliser} of $g \in G$
is the preimage in $G$ of $C_{\bar{G}} (\bar{g})$.
Further, $g \in G$ is a {\it projective involution} 
if $g^2$ is scalar, but $g$ is not.

Involution centralisers are fundamental to our algorithms.
We briefly review the structure of involution
centralisers in (projective) classical groups defined  over fields of
odd characteristic. A detailed account can be  found in \cite[4.5.1]{GLS3}. 

If $h$ is an involution in a 
classical group $G$, then we denote its  
$+1$ and $-1$-eigenspaces  by $E_+$ and $E_-$ respectively.  
Observe that the dimension of the $-1$-eigenspace of 
an involution in $\SX(d,q)$ is always even, since the 
involution must have determinant 1.  

If $G$ preserves a non-degenerate form, then 
$E_+$ and $E_-$ are mutually orthogonal, and the form restricted to
each of these spaces is non-degenerate. If $G \in \C$, then 
$C_G(u)=(\GX(E_+)\times\GX(E_-))\cap G$.  The
centraliser of the image of $u$ in the central quotient $\overline{G}$
of $G$ is the image of $C_G(u)$ in $\overline{G}$ if $E_+$ and $E_-$
are of different dimensions or (in the orthogonal case) of different
types. Otherwise $E_+$ and $E_-$ are isometric and 
the centraliser is the image of $(\GX(E_+)\wr C_2)\cap G$ in $\overline{G}$.

A subgroup of $\GL(U)$, where $U$ is a subspace of
$V$ that supports a non-degenerate form, 
is regarded  as a subgroup of $\GL(V)$ centralising $U^\perp$. With
this convention, the base of the wreath product $\GX(E_+)\wr C_2$ is
$\GX(E_+)\times\GX(E_-)$.  Similarly, if $E_+$  and $E_-$ are the
eigenspaces of an involution in $\GL(V)$,  then a subgroup of $\GL(E_+)$ is
regarded  as a subgroup of $\GL(V)$ that centralises $\GL(E_-)$; and
{\it mutatis mutandis} the same applies to a subgroup of $\GL(E_-)$.

We denote the subgroup 
of $\SO^\epsilon (m, q) \times \SO^\epsilon (n, q)$
consisting of those pairs of elements 
whose spinor norms are equal by $\SO^\epsilon(m, q)
\times_{C_2}\SO^\epsilon(n, q)$. 

We summarise some observations about 
symmetric bilinear forms of $+$ and $-$ type.
\begin{lemma} \label{form-type}
Let $E_+$ and $E_-$ denote the $+1$ and $-1$-eigenspaces  of 
an involution $h \in 
\Omega^\epsilon(d,q)$, where $E_-$ has dimension $e$.
\begin{enumerate}
\item[(i)]
The form supported by $E_-$ is of $-$ type if and only if
both $q\equiv3\bmod4$ and $e\equiv2\bmod4$.

\item[(ii)]
The restrictions of the symmetric bilinear form 
preserved by $\Omega^\epsilon(d,q)$ to the two eigenspaces 
of $h$ are of the same type if 
$\epsilon=+$, and are of opposite
types if $\epsilon = -$.
\end{enumerate}
\end{lemma}
The proof  of these assertions is elementary:
$-I_2\in\O^+(2,q)$ has spinor norm $+1$  if $q\equiv1\bmod4$, and  has
spinor norm $-1$  if $q\equiv3\bmod4$; whereas  $-I_2\in\O^-(2,q)$ has
spinor norm $-1$ if $q\equiv1\bmod4$, and has spinor norm $+1$  if
$q\equiv3\bmod4$.

To distinguish readily between symmetric bilinear forms of $+$ and $-$ type,
we use the following observation.
\begin{lemma}\label{square-det}
If $A$ is the $2n$-dimensional matrix of a symmetric bilinear form, 
then the form is of $+$ type if $(-1)^{n} \det (A)$ is a square,
otherwise the form is of $-$ type.
\end{lemma}

\subsection{Las Vegas algorithms and complexity}
We use the `big \Oh' notation in the following way.  
If $f$ and $g$ are real valued
functions, defined on all sufficiently large integers, then we write
$f(n)=\Oh(g(n))$ to mean $|f(n)|<C|g(n)|$ for some positive
constant $C$ and all sufficiently large $n$.
The modulus here will be relevant only when $g(n)$ tends 
to $0$ with $n$. %, as in $f(n)=\Oh(1/n)$.

A {\it Las Vegas} algorithm is a randomised algorithm 
which never returns an incorrect answer, but may report 
failure with probability less than some specified value.

Our algorithms usually search for elements of $G$ having 
a specified type.  As part of the analysis of these 
algorithms, we determine a lower bound, say $1/k$, 
for the proportion of such elements in $G$. 
It is now an easy exercise to prescribe the 
probability of failure of the corresponding algorithm. 
Namely, to find such an element by random
search with a probability of failure less 
than a given $\epsilon \in (0, 1)$
it suffices to choose (with replacement) a sample 
of uniformly distributed random elements in $G$ 
of size at least $\lceil -\log_e(\epsilon) \rceil k$.
Hence we do not include such estimates as 
part of each theorem.

We record an elementary observation that is frequently used to estimate
the cost of our `divide-and-conquer' algorithms.
\begin{lemma}\label{d-a-c}
Let $f$ be a real valued function defined on the set of 
integers greater than $1$.
Suppose that 
$$\exists k>1\q\exists c>0\q\forall d\ge4\q\exists e\in(d/3,2d/3]\q 
f(d)\le f(e)+f(d-e)+cd^k.$$
Then $f(d)=\Oh(d^k)$.
\end{lemma}
\begin{proof}
Let $m=\max(c/(1-(1/3)^k-(2/3)^k),f(2)/2^k,f(3)/3^k)$.  
We prove, by induction on $d$, that
$f(d)\le md^k$ for all $d> 1$. This is obvious
for $d = 2, 3$. Suppose that $d\ge4$, that $e$ is 
as in the statement of the lemma,
and that $f(n)\le mn^k$ for all $n<d$.  Then
\begin{eqnarray*}
f(d) & \le & f(e)+f(d-e)+cd^k \\
 & \le & me^k+m(d-e)^k+cd^k \\
 & = & md^k\left(\left({e\over d}\right)^k+\left({d-e\over d}\right)^k\right)+cd^k \\
 & \le & md^k\left(\left({1\over3}\right)^k+\left({2\over3}\right)^k\right)+cd^k \\
 & \le & md^k. 
\end{eqnarray*}
The result follows.
\end{proof}

This lemma demonstrates that the cost of the recursive
calls in a `divide-and-conquer' algorithm of the type 
we employ does not affect the degree of complexity 
of the overall algorithm.  The condition $k>1$ is 
required to ensure that $1-(1/3)^k-(2/3)^k>0$.

\subsection{The pseudo-order of a matrix} \label{pseudo-order}
While the precise order of an arbitrary $g \in \GL(d, q)$ cannot
be determined in polynomial time, because of problems with
integer factorisation, we can readily compute
a ``good" multiplicative upper bound for $|g|$,
which we shall call its \emph{pseudo-order}.

Let the factorisation of the minimal polynomial polynomial $f(x)$
of $g$ into powers of distinct irreducible monic polynomials be given by
$f(x) = \prod_{i = 1}^t f_i(x)^{n_i}$,
where $\deg(f_i) = e_i$. Then $|g|$ divides
$B = {\lcm} (q^{e_1} - 1, \ldots, q^{e_t} - 1) \times p^\beta$,
where $\beta = \lceil \log_p \max n_i \rceil$ and $\GF(q)$
has characteristic $p$.

>From $B$, we can readily learn in polynomial time
the {\it exact} power of any specified prime
that divides $|g|$.  In particular, we can
determine if $g$ has even order.

Recall from \cite{NP} that
a {\it primitive prime divisor} of $q^e-1$ is a prime divisor of
$q^e-1$ that does not divide $q^i-1$ for any positive integer $i<e$.

\begin{definition}
Using the above notation, let $u_1 < u_2 < \ldots <u_s$
be the factors of the distinct degrees
of the irreducible factors of $f(x)$.
The pseudo-order of $g$ is defined to be
$n := p^\beta \cdot \prod_{k= 1}^s r_{k} \cdot \prod_{j \in J} p_j$
 where:
\begin{enumerate}
\item[\rm (i)] $\{p_j : j \in J\}$ is the set of primes,
with multiplicities, that divide $|g|$ and that are at most $d + 1$;
\item[\rm (ii)] $r_k \neq 1$ if and only if
$|g|$ is a multiple of a primitive prime divisor of $q^{u_k}-1$
greater than $d+1$.
In this case $r_k$ is the product of
all the primitive prime divisors of $q^{u_k} - 1$,
with multiplicities, that are greater than $d + 1$.
$($Here, the multiplicity of a prime is the multiplicity with
which it divides $q^{u_k} - 1.)$
\end{enumerate}
\end{definition}
Clearly $|g|$ divides the pseudo-order of $g$.
If $r_k\ne 1$ then $r_k$ is a {\it pseudo-prime divisor} of $|g|$.

\begin{lemma} \label{factorise} The following algorithm returns the
product $m$ of the primitive prime divisors of $q^e-1$, multiplied
by powers of certain primes at most $e$.

\noindent
{
$m := q^e-1$; for $i=1$ to $e - 1$ do
if $i$ divides $e$ then $m:=m/\gcd(m, q^i-1)$;  return $m$.
}
\end{lemma}
\begin{proof}
Let $\ell$ be a prime dividing the returned value of $m$.  Since $\ell$
divides $m$, it follows that $\ell$ is a primitive prime divisor of
$q^i-1$ for some $i$ dividing $e$.  If 
the multiplicity of $\ell$, as a prime divisor of $q^e-1$, is
greater than its multiplicity as a prime divisor of $q^i-1$, 
then $\ell$ divides $(q^e-1)/(q^i-1)=1+q^i+\cdots+q^{e-i}$,
and hence divides $e$.  
The result follows.
\end{proof}

The greatest common divisors used in the
algorithm can be calculated readily
using the following observations:  $\gcd(q^i-1,q^j-1)=q^k-1$,
where $k=\gcd(i,j)$, and $\gcd(n/a,b)=\gcd(n,b)/\gcd(a,b)$.

Thus we can factorise $B$ as
$\prod_{k=1}^sr_k\prod_{j\in J}p_j$, where, for all $j$,
$p_j$ is a prime at most $d+1$, and $r_k$ is the product of those
 primitive prime divisors (with multiplicities) of
$q^{u_k}-1$ that are greater than $d+1$.

For easy reference, we summarise the costs of certain basic operations.
\begin{lemma}\label{pseudo}
\mbox{}
\begin{enumerate}
\item[\rm (i)]
The cost of multiplication and division operations for polynomials of
degree $d$ defined over $\GF(q)$ is $\Oh(d\log d\log\log d)$ field
operations.
Such a polynomial can be factored into its irreducible factors in
$\Oh(d^2 \log d \log\log d\log q)$ field operations.

\item[\rm (ii)] Using Las Vegas algorithms, both
the characteristic and minimal polynomial of
$g \in \GL(d, q)$ can be computed
in $\Oh(d^3 \log d)$ field operations.

\item[\rm (iii)] Using a Las Vegas algorithm,
the multiplicative upper bound $B$
to the order of $g \in \GL(d, q)$ can be computed in
$\Oh(d^3 \log d + d^2 \log d \log\log d \log q)$ field operations.

\item[\rm (iv)] Using a Las Vegas algorithm,
the pseudo-order of $g \in \GL(d, q)$ can be computed in
$\Oh(d^3 \log d + d^2 \log d \log\log d \log q)$ field operations.
\end{enumerate}
\end{lemma}
\begin{proof}
For the cost of polynomial operations, see
\cite[\S 8.3, \S 9.1, Theorem 14.14]{vzg}.

The characteristic and minimal polynomials of $g$ can be computed
in the claimed time using
the Las Vegas algorithms of \cite{Ambrose,KG}
and \cite{Giesbrecht} respectively.

Hence $B$ can be obtained in the claimed time.

Using Lemma \ref{factorise}, we can express $B$ as the product of
at most $2d + 1$ factors, each of which is either
a pseudo-prime factor of $B$, or a prime-power factor of $B$.
To compute the pseudo-order of $g$ from this information
requires $\Oh(\log |g| \log d)$ operations in the
ring $\GF(q)[t]/(f(t))$, as in \cite{CLG97}.
\end{proof}

We choose the bound of $d+1$ on the primes being extracted
in our definition of pseudo-order for two reasons: 
Lemma \ref{factorise} shows that bound must be at least $d$;
the algorithm of \cite{NP} requires knowledge of
the precise prime divisor in question if this is $d+1$ (in the
definition of a large primitive prime divisor).  Of course,
the upper bound could be enormously increased without problems of
integer factorisation arising.

Observe that the concept of primitive prime divisor is only
well-defined if one regards $q$ as part of the data.
If $q=p^f$, then a primitive prime divisor of $q^e-1$
need not be a primitive prime divisor of $p^{ef}-1$, 
since the prime in question might divide $p^n-1$ for some
$n<ef$, but not $q^m-1$ for any $m<e$.  If the prime does not divide
$p^n-1$ for any $n<ef$, then it is a \emph{strong} primitive prime 
divisor of $q^e-1$, and in some cases this
is a requirement for the algorithm of \cite{NP}.  To accommodate
this condition, we need to factorise $B$ accordingly, 
this being achieved by the same algorithm that was used above, 
but with the parameters $q$ and $e$ replaced by $p$ and $fe$
respectively.  These variations do not affect the complexity analysis.

\section{Standard generators for classical groups}
\label{standard}
We now describe {\it standard generators} for
the groups $\SX(d,q)$ for odd $q$.

Recall that $V$ is the natural module for $G = \SX(d, q)$.
The standard generators for $G$ are defined with respect to 
a hyperbolic basis for $V$, which in turn is defined 
in terms of the given basis by a change-of-basis matrix. 
We define a {\it hyperbolic} basis for $V$ as follows. 
\begin{enumerate}
\item 
If $V$ does not support a classical form, then any ordered basis, 
say $(e_1, \ldots, e_d)$, is hyperbolic. 

\item 
If the form supported by $V$ is symplectic of rank $2n$, then 
a hyperbolic basis for 
$V$ is an ordered basis $(e_1,f_1,\ldots,e_n,f_n)$, where
$e_i.e_j=f_i.f_j=0$ for all $i,j$ (including the case $i=j$), and
$e_i.f_j=0$ for $i\ne j$, and $e_i.f_i=-f_i.e_i=1$ for all $i$. 

\item 
If the form supported by $V$ is hermitian of rank $2n$, then a hyperbolic 
basis for $V$ is exactly as
for $\Sp(2n,q)$ except that, the form being hermitian, the
condition  $e_i.f_i=-f_i.e_i=1$ for all $i$ is replaced by the
condition $e_i.f_i=f_i.e_i=1$ for all $i$. 

\item 
If the form supported by $V$ is hermitian of rank $2n + 1$, 
then a hyperbolic basis for $V$ is 
an ordered basis of the form 
$(e_1,f_1,\ldots,e_n,f_n,w)$, where the above equations hold, and in
addition $e_i.w=f_i.w=0$ for all $i$, and $w.w=1$. 

\item 
If the form supported by $V$ is symmetric bilinear 
of $+$ type and of rank $2n$, 
then a hyperbolic basis  for  $V$ is  an
ordered  basis  of the  form $(e_1,f_1,\ldots,e_n,f_n)$, where the
equations used to define the form for $\SU(2n,q)$ again apply.

\item  
If the form supported by $V$ is symmetric bilinear of  
$-$ type and of rank $2n$, 
then  a hyperbolic  basis for $V$  is an
  ordered  basis  of  the form
  $(e_1,f_1,\ldots,e_{n-1},f_{n-1},w_1,w_2)$,  where  the  above
  relations hold for  $i, j <n$; in addition
  $w_1.e_i=w_1.f_i=w_2.e_i=w_2.f_i=w_1.w_2=0, w_1.w_1=-2$,  and
  $w_2.w_2=2\omega$  where  $\omega$ is  a primitive  element of
  $\GF(q)$.  Since $\omega$ is not a square in
  $\GF(q)$, this defines a form of $-$ type (see Lemma \ref{square-det}).

\item  
If $V$ has dimension $2n+1$, then there are two equivalence classes
of non-degenerate symmetric bilinear forms on $V$, distinguished by their
discriminants. To convert a form in one class to a form in
the other class, we multiply it by a non-square scalar, 
thus obtaining an inequivalent form preserved by the same group.  
A hyperbolic
basis is an ordered basis of the form $(e_1,f_1,\ldots,e_n,f_n,w)$, where
again the relations in 3 hold, and in addition $w.e_i=w.f_i=0$, and
$w.w=-1/2$.  
If necessary, we multiply the form for the input group by a non-square scalar.
\end{enumerate}

For uniformity of exposition, we sometimes
label the ordered basis for 
$\SL(2n, q)$ as $(e_1,f_1,\ldots,e_n,f_n)$
and that for $\SL(2n + 1, q)$ 
as $(e_1,f_1,\ldots,e_n,f_n, w)$.

Subject to the following conventions,
the standard generators for the non-orthogonal
groups $\SX(d, q)$ are defined in Table 1, 
and for $\SO^\epsilon(d,q)$ in Table 2. 

\begin{enumerate}
\item 
$\gamma$ is a specified primitive element for $\GF(q^2)$,
and $\alpha = \gamma^{(q + 1)/2}$, and
$\omega=\alpha^2$ is a primitive element for $\GF(q)$.

\item In all  but one case,  we  describe $v$ as a  signed permutation
  matrix acting on the  hyperbolic  basis for  $V$. We  adopt  the
  following notation.  Given a basis for $V$, a signed permutation
  matrix with respect  to this  basis will be  given as  a product  of
  disjoint signed cyclic  permutations of the  basis elements.  Such a
  cycle  either  permutes the  vectors in  the  cycle,  no sign  being
  involved, or it  sends each vector in  the cycle to the next, except
  for the  last vector which  is sent to minus the first vector.  In
  this  case the  cycle is adorned with the superscript $-$, as in
  $(e_1,e_2,\ldots,e_n)^-$.  The superscript $+$ has no effect,
  so that $(e_1,e_2,\ldots,e_n)^+=(e_1,e_2,\ldots,e_n)$. 
  If we use the notation $(e_1,e_2,\ldots,e_n)^{\epsilon_n}$, then 
  $\epsilon_n=+$ if $n$ is odd, and $\epsilon_n=-$ if $n$ is even. 

\item For $\SU(2n+1, q)$, the matrices $x$ and $y$
normalise the subspace $U$ having ordered basis $B = (e_1,w,f_1)$ and
centralise $\langle e_2, f_2, \ldots, e_{n}, f_{n}\rangle$.
We list their action on $U$ with respect to basis $B$.

\item The remaining generators, other than $v$,  
of groups in Table 1 normalise a subspace $U$
having ordered basis $B$, where $B=(e_1, f_1)$ or 
$B=(e_1, f_1, e_2, f_2)$, and centralise the 
space spanned by the remaining basis
vectors.  We write the action of a generator on $U$ with respect to
basis $B$.

\item
We assume $n>1$ for the groups $\SO^\epsilon(2n,q)$. 
In Table 2 the generators of $\SO^+(2n,q)$ 
given as $4\times4$ matrices normalise a subspace $U$ having ordered basis $B$,
where $B=(e_1,f_1,e_2,f_2)$, and centralise the subspace spanned by the 
remaining basis vectors.
We write the action of a generator on $U$ with respect to basis $B$. 
For $\SO^-(2n,q)$ the same applies but with 
$B=(e_1,f_1,w_1,w_2)$. 
For $\SO(2n+1,q)$ we write the action
of matrices with respect to basis $B=(e_1,f_1,w)$. 

\item In the definition for $\SO^-(2n, q)$, the
variables $A, B, C$ have the following values:
\begin{eqnarray*}
A & = & {1\over2}(\gamma^{q-1}+\gamma^{-q+1}) \\
B & = & {1\over2}\alpha(\gamma^{q-1}-\gamma^{-q+1}) \\
C &=& {1\over2}\alpha^{-1}(\gamma^{q-1}-\gamma^{-q+1}).
\end{eqnarray*}

\item For $\SO^\epsilon(d, q)$, the generator
$\w$ has 
spinor norm $-1$; the others are the standard 
generators for the corresponding $\Omega^\epsilon(d, q)$.
For $\epsilon = 0, +$, the value of $b$ is determined by 
$q - 1 = 2^a \cdot b$ where $b$ is odd; 
$\lambda = (-1)^{(q - 1) / 2}$.

\item 
To facilitate uniform exposition, we introduce trivial generators. 
If the dimension required to define a generator is greater than 
the dimension of the group, then the generator is assumed to be trivial.
\end{enumerate}

By analogy with the general case,
we assume that $\SO^+(2, q)$
has the same sequence of nine standard generators, 
where the only non-trivial elements are:
\begin{eqnarray*}
\delta = \left(\matrix{\omega^2 & 0 \cr 0 & \omega^{-2}} \right)& 
\sigma = \left(\matrix{ \omega^b & 0 \cr 0 & \omega^{-b}}\right); 
\end{eqnarray*}
of course, $\Omega^+(2, q) = \langle \delta \rangle$.

Once a hyperbolic basis has been chosen for $V$, the Weyl group of $G$
can  be defined as a section  of $G$, namely  as the group of monomial
matrices in $G$ modulo diagonal  matrices, thus defining a subgroup of
the symmetric group  $S_d$. The Weyl  group of $\SL(d,q)$ is  $S_d$.
The Weyl group of $\Sp(2n,q)$ is  the subgroup of $S_{2n}$ that
preserves the  system of imprimitivity with blocks $\{e_i,f_i\}$ for
$1\le i\le  n$, and is  thus $C_2\wr S_n$.  The Weyl group  of each of
$\SU(2n,q)$ and $\SU(2n+1,q)$ is also $C_2\wr S_n$.  The Weyl group of
$\Omega^+(2n,q)$ is the  subgroup of $C_2\wr S_n$
consisting of even permutations.  The Weyl group of $\Omega^-(2n,q)$
is $C_2\wr S_{n-1}$, and that of $\Omega(2n+1,q)$ is $C_2\wr S_n$.

%{\tt EOB -- Does Weyl group make sense for $SO$?}
%{\tt CRLG -- Yes, the Weyl group is $N/(B\cap N)$, and the extra generators lie
%in $B\cap N$, so the Weyl groups for $\Omega$ and $O$ are the same. }

If $G$ is $\SL(d, q)$ or $\Sp(d,q)$, then its standard generators 
have the property that it is easy to construct from them
any of its root groups, and consequently we deduce that
they generate $G$. The root
groups are defined with respect to a maximal split torus, the group of
diagonal matrices in  $\SX(d, q)$; for  a detailed  description  see
\cite{Carter}.  The situation  is  similar  for  $\SU(d, q)$ and the
orthogonal groups, as we now show.
\begin{lemma}
Let $G = \SU(d, q)$ for $d \geq 2$.  
Then $G = \langle s,t, \delta, u,v, x, y\rangle$. 
\end{lemma}
\begin{proof}
If $d = 2$, then $u, v, x$ and $y$ are by convention trivial,
and $\langle s, t, \delta \rangle$ is isomorphic
to $\SL(2, q) \cong \SU(2, q)$.
If $d = 2n + 1$, then a direct computation shows that 
$$x^y = 
\left(\matrix{
1 & \omega^{q - 2} & -\omega^{-(q + 1)} /2 \cr
0 & 1  & \omega^{-2q + 1} \cr
0 & 0  & 1 
\cr}\right).
$$
Observe that $y$ has order $q^2 - 1$.
Thus $S = \langle x^{y^k} : 1 \leq k \leq q^2-1 \rangle$
is non-abelian of order $q^3$, having
derived group and centre of order $q$. 
A similar calculation for $d = 2n$ where $n > 1$ shows that 
$\langle x^{y^k} : 0 \leq k < q^2-1 \rangle$ has order $q^2$.
These groups correspond to the subgroups $X_S^1$  of \cite{Carter}
and the result follows from \cite[Proposition 13.6.5]{Carter}. 
\end{proof}

\begin{landscape}
\begin{table} \label{standard-table}\tiny 
\begin{center}
\begin{tabular}{|r||c|c|c|c|c|c|c|} 
\hline 
Group & $s$ & $t$ & $\delta$ & $u$ & $v$ & $x$ & $y$ 
\rule{0cm}{3.0ex}\\ \hline

$\SL(2n, q)$ 
& 
$\begin{array}{cc} \left(\matrix{0&1\cr-1&0\cr}\right) \end{array} \;$
& 

$\begin{array}{cc} \left(\matrix{1&1\cr0&1\cr}\right) \end{array} \;$

& 
$\left(\matrix{\omega&0\cr0&\omega^{-1}\cr}\right)$
& 
$I_2$
& 

$ (e_1, e_2, \ldots, e_{n})(f_1,f_2,\ldots, f_n) $

& 

$\left(\matrix{0&1&0&0\cr0&0&1&0\cr0&0&0&1\cr-1&0&0&0\cr}\right)$

& 

$I_4$

\rule{0cm}{3.0ex}\\ \hline

$\SL(2n + 1, q)$ & 
$\begin{array}{cc} \left(\matrix{0&1\cr-1&0\cr}\right) \end{array} \;$
& 

$\begin{array}{cc} \left(\matrix{1&1\cr0&1\cr}\right) \end{array} \;$

& 
$\left(\matrix{\omega&0\cr0&\omega^{-1}\cr}\right)$
& 
$I_2$
& 

$\left(\matrix{ 0 & 1 \cr -I_{2n} & 0 \cr }\right)$
& 

$I_{4}$

& 

$I_4$

\rule{0cm}{3.0ex}\\ \hline

$\Sp(2n, q)$ & 
$\left(\matrix{0&1\cr-1&0\cr}\right)$
& 

$ \begin{array}{cc} \left(\matrix{1&1\cr0&1\cr}\right) \end{array} $

& 
$\left(\matrix{\omega&0\cr0&\omega^{-1}\cr}\right)$
& 
$\left(\matrix{0&0&1&0\cr0&0&0&1\cr1&0&0&0\cr0&1&0&0\cr}\right)$
& 

$ (e_1, e_2, \ldots, e_{n})(f_1,f_2,\ldots, f_n) $

& 

$\left(\matrix{1&0&0&0\cr0&1&1&0\cr0&0&1&0\cr1&0&0&1\cr}\right)$

& 

$I_4$

\rule{0cm}{3.0ex}\\ \hline

$\SU(2n, q)$ & 

$\left(\matrix{0&\alpha\cr \alpha^{-q}&0\cr} \right)$

& 

$\left(\matrix{1&\alpha\cr0&1\cr}\right)$

& 

$\left(\matrix{\gamma^{q + 1}&0\cr0&\gamma^{-(q+1)}\cr}\right)$

& 
$\left(\matrix{0&0&1&0\cr0&0&0&1\cr1&0&0&0\cr0&1&0&0\cr}\right)$

& 
$ (e_1, e_2, \ldots, e_{n})(f_1,f_2,\ldots, f_n) $
& 

$\left(\matrix{1&0&1&0\cr0&1&0&0\cr0&0&1&0\cr0&-1&0&1\cr}\right)$

& 

$\left(\matrix{\gamma&0&0&0 \cr 0&\gamma^{-q} & 0 & 0\cr 
  0 & 0 & \gamma^{-1} & 0 \cr 0 & 0 & 0 & \gamma^q\cr}\right)$

\rule{0cm}{3.0ex}\\ \hline

$\SU(2n + 1, q)$ & 

$\left(\matrix{0&\alpha\cr \alpha^{-q}&0\cr} \right)$

& 

$\left(\matrix{1&\alpha\cr0&1\cr}\right)$

& 

$\left(\matrix{\gamma^{q + 1}&0\cr0&\gamma^{-(q+1)}\cr}\right)$

& 

$\left(\matrix{0&0&1&0\cr0&0&0&1\cr1&0&0&0\cr0&1&0&0\cr}\right)$

& 
$ (e_1, e_2, \ldots, e_{n})(f_1,f_2,\ldots, f_n) $

& 

$\left(\matrix{1& 1 & -1/2 \cr 0&1 & -1 \cr 0 & 0 & 1\cr}\right)$

& 

$\left(\matrix{\gamma &0&0\cr 0&\gamma^{q-1} & 0 \cr 0 & 0 & \gamma^{-q}\cr}\right)$

\rule{0cm}{3.0ex}\\ \hline

\end{tabular}
\end{center}
\caption{Standard generators for non-orthogonal classical groups}
\end{table}
\end{landscape}

\begin{landscape}
\begin{table} \label{orthog-table}\tiny 
\begin{center}
\begin{tabular}{|r||c|c|c|c|c|c|c|} 
\hline 
%Group & $s$ & $t$ & $\delta$ & $u$ & $v$ & $x$ & $y$ 
%\rule{0cm}{3.0ex}\\ \hline
Group&$s$&$t$&$\delta$&$u$&$v$& $\w$ \rule{0cm}{3.0ex}\\ \hline
$\SO^+(2n,q)$
&
$\left(\matrix{0&0&0&-1\cr0&0&-1&0\cr0&1&0&0\cr1&0&0&0}\right)$
&
$\left(\matrix{1&0&0&-1\cr0&1&0&0\cr0&1&1&0\cr0&0&0&1}\right)$
&
$\left(\matrix{\omega&0&0&0\cr0&\omega^{-1}&0&0\cr0&0&\omega&0\cr0&0&0&\omega^{-1}}\right)$
&
$I_4$
&
$(e_1,e_2,\ldots,e_n)^{\epsilon_n}(f_1,f_2,\ldots,f_n)^{\epsilon_n}$
&
%$\left(\matrix{-I_2&0\cr0&I_2}\right)$
$\left(\matrix{ \omega^b & 0 & 0 & 0 \cr
                   0     & \omega^{-b} & 0 & 0 \cr
                   0 & 0 & 1 & 0 \cr
                   0 & 0 & 0 & 1 \cr }\right)$
\rule{0cm}{3.0ex}\\ \hline
&$s'$&$t'$&$\delta'$& & &  
\rule{0cm}{3.0ex}\\ \hline
&
$\left(\matrix{0&0&1&0\cr0&0&0&1\cr-1&0&0&0\cr0&-1&0&0}\right)$
&
$\left(\matrix{1&0&1&0\cr0&1&0&0\cr0&0&1&0\cr0&-1&0&1}\right)$
&
$\left(\matrix{\omega&0&0&0\cr0&\omega^{-1}&0&0\cr0&0&\omega^{-1}&0\cr0&0&0&\omega}\right)$
&
&
&
\rule{0cm}{3.0ex}\\ \hline
Group&$s$&$t$&$\delta$&$u$&$v$& $\w$ 
\rule{0cm}{3.0ex}\\ \hline
$\SO^-(2n,q)$
&
$\left(\matrix{0&1&0&0\cr1&0&0&0\cr0&0&-1&0\cr0&0&0&1}\right)$
&
$\left(\matrix{1&1& 1 &0\cr0&1&0&0\cr0& 2 &1&0\cr0&0&0&1}\right)$
&
$\left(\matrix{\omega&0&0&0\cr0&\omega^{-1}&0&0\cr0&0&A&
B\cr0&0&C&A}\right)$
&
%$\left(\matrix{0&I_2\cr-I_2&0}\right)$
$(e_1, e_2)^{-} (f_1, f_2)^{-}$
%$I_4$
&
$(e_1,\ldots,e_{n-1})^{\epsilon_{n-1}}(f_1,\ldots,f_{n-1})^{\epsilon_{n-1}}$
&
%$\left(\matrix{-I_2&0\cr0&I_2}\right)$
$\left(\matrix{ \lambda I_2&0\cr0& -\lambda I_2}\right)$

\rule{0cm}{3.0ex}\\ \hline
Group&$s$&$t$&$\delta$&$u$&$v$& $\w$ 
\rule{0cm}{3.0ex}\\ \hline

$\SO(2n+1,q)$
&
$\left(\matrix{0&1&0\cr1&0&0\cr0&0&-1}\right)$
&
$\left(\matrix{1&1&2\cr0&1&0\cr0&1&1\cr}\right)$
&
$\left(\matrix{\omega^2&0&0\cr0&\omega^{-2}&0\cr0&0&1}\right)$
&
$(e_1, e_2)^{-} (f_1, f_2)^{-}$
%$I_4$
%$\left(\matrix{0&I_2\cr-I_2&0}\right)$
&
$(e_1,\ldots,e_n)^{\epsilon_n}(f_1,\ldots,f_n)^{\epsilon_n}$
&
$\left(\matrix{\omega^b & 0 & 0  \cr
                  0     &  \omega^{-b} & 0 \cr
                   0  & 0 & 1}\right)$
%$\left(\matrix{-I_2&0\cr0& 1}\right)$
\rule{0cm}{3.0ex}\\ \hline
\end{tabular}
\end{center}
%\centerline{$A={1\over2}(\gamma^{q-1}+\gamma^{-q+1})\quad B={1\over2}\alpha(\gamma^{q-1}-\gamma^{-q+1})
%\quad C = {1\over2}\alpha^{-1}(\gamma^{q-1}-\gamma^{-q+1})$ and 
%$\w$ is only defined if $q\equiv3\bmod4$ and, in the case 
%of $\Omega^-(2n,q)$, if $n>1$.}
\caption{Standard generators for orthogonal groups}
\end{table}
\end{landscape}

\begin{lemma} \label{omegap4}
Let $G=\Omega^+(2n,q)$ for $n\ge 2$. Then
$G=\langle s,s',t,t',\delta,\delta',v\rangle$.
\end{lemma}
\begin{proof}
  If  $n=2$ then $G$ is  the  central  product  of  two copies of
  $\SL(2,q)$ (see \cite[Corollary 12.39]{Taylor}). 
  Let the natural modules  for these copies of $\SL(2,q)$
  be $U_1$ and $U_2$, and let these  modules  have  ordered  bases
  $(a_1,b_1)$  and $(a_2,b_2)$  respectively.  Define  an  alternating
  bilinear form on $U_i$  by  $a_i.b_i=1$ for  $i=1,2$.  This form  is
  preserved by the  respective  copies  of $\SL(2,q)$.  Now  define  a
  bilinear form on $V=U_1\otimes U_2$ by $(u_1\otimes u_2).(v_1\otimes
  v_2)=  u_1.v_1\times u_2.v_2$.  This defines  a  non-degenerate
  symmetric  form on $V$. A hyperbolic basis for $V$ is  then  given by
  $(a_1\otimes a_2, b_1\otimes b_2, a_1\otimes b_2, -b_1\otimes a_2)$.
  Let $s, t, \delta$ in $\SL(U_1)$ be defined, with respect to
  the basis $(a_1,b_1)$, by the matrices
$$\left(\matrix{0&1\cr-1&0}\right)\q\q
\left(\matrix{1&1\cr0&1}\right)\q\q
\left(\matrix{\omega&0\cr0&\omega^{-1}}\right),$$  and  let
$s', t', \delta'$ denote the corresponding  elements  of
$\SL(U_2)$. Now $\Omega^+(4,q)$ is the  central product of these two
copies of $\SL(2,q)$. Abusing notation by writing $s$, $t$ and
$\delta$  for  the images of  $(s,I_2)$, $(t,I_2)$ and
$(\delta,I_2)$ in $\Omega^+(4,q)$,  and $s', t', \delta'$ for
the images of $(I_2,s')$, $(I_2,t')$ and $(I_2,\delta')$, we obtain
the first six given generators.  Observe that $v = s'$ in dimension $4$. 

We next prove that the spinor norm of $v$ is $+1$ if $n > 2$.  
If $n$ is odd, this follows since $v$ is of odd order.  
If $n > 2$ is even, then 
$$(e_{n-1},e_n)^-(f_{n-1},f_n)^-(e_1,\ldots,e_{n-1})^{\epsilon_{n-1}}
(f_1,\ldots,f_{n-1})^{\epsilon_{n-1}}=
(e_1,\ldots,e_n)^{\epsilon_n}(f_1,\ldots f_n)^{\epsilon_n}$$ 
and the result now follows from that for $n = 2$ and odd $n > 2$.

Since the Weyl group of $\Omega^+(2n,q)$ is 
generated modulo diagonal elements by  $\{s, s', v\}$, 
the claim follows. 
\end{proof}

\begin{lemma} \label{omegam4}
Let $G=\Omega^-(2n,q)$ for $n\ge 2$.  Then
$G=\langle s,t,\delta,u,v\rangle$.
\end{lemma}
\begin{proof}
If $n=2$ then $G$ is isomorphic to $\PSL(2,q^2)$
(see \cite[Corollary 12.43]{Taylor}).
This isomorphism
arises as follows. Take the natural module $U$ for $\SL(2,q^2)$,
and let $W$ be $U$ twisted by the automorphism of $\GF(q^2)$ given
by $a\mapsto a^q$. Then $U\otimes W$ gives rise to a representation
of $\PSL(2, q^2)$ over $\GF(q^2)$.  If $(a_1,b_1)$  is a basis for
$U$, and $(a_2,b_2)$  is  a basis for  $W$,  then the resulting
representation of $\PSL(2,q^2)$ on  $U\otimes W$ with respect to the
ordered basis $(a_1\otimes a_2, a_1\otimes  b_2,  b_1\otimes
a_2,b_1\otimes b_2)$ preserves the symmetric non-degenerate bilinear form
$$\left(\matrix{0&J\cr-J&0}\right),$$
where
$$J=\left(\matrix{0&1\cr-1&0}\right).$$
Now let $\gamma$ be a primitive  element  of  $\GF(q^2)$, and  let
$\alpha=\gamma^{{1\over2}(q+1)}$,  so that  $\alpha^2$ is a  primitive
element $\omega$ of $\GF(q)$.  Conjugating by the matrix
$$\left(\matrix{1&0&0&0\cr 0&\alpha&1&0\cr 0&-\alpha&1&0\cr 0&0&0&1}\right)$$
transforms the above image of $\PSL(2,q^2)$ into a subgroup of $\SL(4,q)$.  
Interchanging the second and fourth basis vectors now transforms this 
image into a group that preserves the form
$$\left(\matrix{0&1&0&0\cr 1&0&0&0\cr 0&0&-2&0\cr 0&0&0&2\omega}\right),$$
and thus into our chosen copy of $\Omega^-(4,q)$.  
It is straightforward to check
that the given generators $s, t, \delta$ are the 
images of the matrices
$$\left(\matrix{0&1\cr-1&0}\right)\q\q
\left(\matrix{1&1\cr 0&1}\right)\q\q
\left(\matrix{\gamma&0\cr 0&\gamma^{-1}}\right),$$
and hence generate $\Omega^-(4,q)$.  

A similar argument to that in Lemma \ref{omegap4}
shows that $v$ has spinor norm +1. 
Since the Weyl group of $\Omega^{-}(2n,q)$ is 
generated modulo diagonal elements by  $\{s, u, v\}$, 
the claim follows.
\end{proof}

\begin{lemma} \label{omega03}
Let $G=\Omega(2n+1,q)$ for $n\ge 1$.  Then
$G=\langle s,t,\delta,u,v\rangle$.
\end{lemma}
\begin{proof}
If $n=1$ then $G$ is isomorphic to $\PSL(2,q)$
(see \cite[Theorem 11.6]{Taylor}).  
This isomorphism
arises as follows.  Take the natural module  $U$ for $\SL(2,q)$, and
let $V$ be the  symmetric square of $U$. If $(a,b)$ is a basis for
$U$  then, with respect to  the ordered basis $(a^2,b^2,ab)$ of $V$, the form
$$\left(\matrix{0&1&0\cr1&0&0\cr 0&0&1/2}\right)$$
is preserved by $G$.  This exhibits $\PSL(2,q)$ as $\Omega(3,q)$.  
The generators
$s$, $t$, $\delta$ correspond, respectively, to the matrices
$$\left(\matrix{0&1\cr -1&0}\right)\q\q\left(\matrix{1&1\cr 0&1}\right)\q\q
\left(\matrix{\omega&0\cr0&\omega^{-1}}\right).$$
A similar argument to that in Lemma \ref{omegap4}
shows that $v$ has spinor norm +1. 
Since the Weyl group of $\Omega(2n+1,q)$ is 
generated modulo diagonal elements by  $\{s, u, v\}$, 
the claim follows.
\end{proof}

\begin{lemma} \label{spinor-1}
The standard generator $\w$ lies in 
$\SO^\epsilon(d,q)\setminus\Omega^\epsilon(d,q)$.
\end{lemma}
\begin{proof}
Clearly in all cases $\w\in\SO^\epsilon(d,q)$, so it remains to
compute the spinor norm of $\w$.

If $\epsilon \in \{+,0\}$, then the spinor norm of $\w$ is
$\omega^b$ modulo the subgroup of squares 
of $\GF(q)^\times$ (see \cite{Zassenhaus} and 
the proof of Lemma \ref{spinor-norm}).
Since $b$ is odd and $\omega$ is a primitive element of $\GF(q)$, 
the result follows.

Now assume $\epsilon = -$.
Observe that $\w$ acts as $-1$ on 
a 2-dimensional subspace that 
supports a form of $+$ type if $q\equiv3\bmod4$, and 
of $-$ type if $q\equiv1\bmod4$, and $\w$ acts as $+1$ on 
the orthogonal complement of this 2-dimensional subspace.
The conclusion now follows since 
$\Omega^+(2,q)$ has odd
order $(q-1)/2$ if $q\equiv3\bmod 4$, 
and $\Omega^-(2, q)$ has 
odd order $(q+1)/2$ if $q\equiv 1\bmod4$.
\end{proof}

Note that we could have 
taken $\w\in\SO^\epsilon(2n,q)$ to be $-I_{2n}$ if $\epsilon=-$ and 
either $q\equiv1\bmod4$ or $n$ is even; or if 
$\epsilon=+$ and $n$ is odd and $q\equiv3\bmod 4$.
In these cases
$\SO^\epsilon(2n,q)\cong\Omega^\epsilon(2n,q)\times\langle-I_{2n}\rangle$.

We conclude with the following observation
which influences the algorithms we develop in Section \ref{Alg1}.
\begin{lemma} \label{wr2}
Let $G=\langle s,t,\delta,u,v,x,y\rangle \leq \GL(2n, q)$  
and let $H=\langle s,t,\delta,u,v\rangle$.
If $G$ is $\SL(2n, q)$ or $\Sp(2n,q)$ or $\SU(2n, q)$, then 
$H=\SL(2,q)\wr C_n$, or $H=\SL(2,q)\wr S_n$ or $H=\SU(2,q)\wr S_n$ respectively.
\end{lemma}
%Observe that $vx$ acts as a $2n$-cycle on the signed hyperbolic 
%basis for $\SL(2n, q)$. 

\section{Algorithm {\tt One} for non-orthogonal groups}
\label{Alg1}

Let $G=\SX(d, q)$ denote a non-orthogonal group in $\C$.
Algorithm {\tt One} takes as input a generating set $X$ for
$G$ and the classical form preserved by $G$, and returns standard 
generators for $G$ as \SLPs\ in $X$.
The standard generators are written 
with respect to a hyperbolic basis for the natural
module $V$.  The change-of-basis matrix from the given basis to 
the hyperbolic basis is also returned.

The algorithm employs a `divide-and-conquer' strategy. 
\begin{definition}
A {\it strong involution} in $\SX(d,q)$ for $d > 2$ is an involution whose
$-1$-eigenspace has dimension in the range $(d/3,2d/3]$. 
\end{definition}

The main algorithm {\tt OneMain} has two subcases,
according to the parity of the input dimension $d$:
algorithms {\tt OneEven} and {\tt OneOdd} 
address the case of even and odd $d$, respectively.
If $d = 2n$,  then Lemma \ref{wr2} shows that
$Y_0 := \{ s, t, \delta, u, v \}$
generates $\SX(2, q) \wr C_n$  or $\SX(2, q) \wr S_n$ according to the 
type of the input group.
If $d$ is even, then, as the first and major task of the 
main algorithm, 
{\tt OneEven} constructs $Y_0$; as a final step, 
{\tt OneMain} constructs the 
additional elements $x, y$.
This reduces the time spent in more difficult base cases.

\begin{algorithm2e}[H] 
\caption{\tt OneEven $(X,{\it type}, \F)$ \rm \ for\ non-orthogonal\ groups}
\label{alg1:even}
\tcc{
$X$ is a generating set for the classical group $G \in \C$
in odd characteristic, of type {\bf SL} or {\bf Sp} or {\bf SU}, 
in even dimension.
The classical form preserved by $G$ is $\F$. 
Return the standard generating set $Y_0$ for 
$\SL(2, q) \wr C_{d/2}$ if {\it type} is {\bf SL}, 
otherwise for $\SX(2, q) \wr S_{d/2}$, as subgroup of  $G$, 
the \SLPs\ for the elements of $Y_0$, and the change-of-basis matrix.
}
\Begin{
 $d$ := the rank of the matrices in $X$; 

if $d = 2$ then return {\tt BaseCase} $(X, {\it type}, \F)$;  

Find by random search $g \in G:=\langle X\rangle$ of 
even order such that $g$ powers to 
a strong involution $h$;

Let $E_+$ of dimension $2k$ and $E_-$ be the eigenspaces of $h$;

Find generators for the centraliser $C$ of $h$ in $G$;

Rewrite with respect to the concatenation of bases 
for $E_+$ and $E_-$;

In $C$ find generating sets $X_1$ and $X_2$ 
for $\SX(E_+)$ and $\SX(E_-)$;

$\bigl((s_1,t_1,\delta_1,u_1,v_1),B_1\bigr)$ := 
{\tt OneEven} $(X_1,{\it type}, \F|_{E_+})$;

$\bigl((s_2,t_2,\delta_2,u_2,v_2),B_2\bigr)$ := 
{\tt OneEven} $(X_2,{\it type}, \F|_{E_-})$;

Let $B=(e_1,f_1,\ldots,e_k,f_k,e_{k+1},f_{k+1},\ldots,e_{d/2},f_{d/2})$
be the concatenation of the hyperbolic bases defined by $B_1$ and $B_2$;

$a := (s_1^2)^{v_1^{-1}}(s_2^2)$;

Find generators for the centraliser $D$ of $a$ in $G$;

In $D$ find a generating set $X_3$ for 
$\SX(\langle e_k,f_k,e_{k+1},f_{k+1}\rangle)$;

In $\langle X_3\rangle$ find the permutation matrix 
$b=(e_k,e_{k+1})(f_k,f_{k+1})$;

$v := v_2 b v_1$;

return $(s_1,t_1,\delta_1,u_1,v)$ and the change-of-basis matrix for $B$;
}
\end{algorithm2e}

\medskip
\medskip
If the type is $\mathbf{SL}$, then the centraliser of $h$ is 
$(\GL(E_+)\times\GL(E_-))\cap\SL(d,q)$ where $E_+$ and $E_-$ are the
eigenspaces  of $h$. If the type is $\mathbf{Sp}$, it is
$\Sp(E_+)\times\Sp(E_-)$; if the type is $\mathbf{SU}$, it is
$(\U(E_+)\times\U(E_-))\cap\SU(d,q)$. 
Thus, if the
eigenspaces have dimensions $e$ and $d-e$, then the derived group of the
centraliser of $h$ in $\SX(d,q)$ is $\SX(e,q)\times\SX(d-e,q)$. 

We make the following observations on Algorithm {\tt OneEven}. 
\begin{enumerate}
\item 
The \SLPs\ that express the standard generators 
in $X$ are also returned.

\item 
Generators for the involution centralisers in lines 
6 and 14 are constructed using the algorithm of Bray \cite{Bray},
see Section \ref{Bray}. 
We need only a subgroup of the centraliser that 
contains the derived group. 

\item 
The generators for the direct factors in line 8
are constructed using the algorithm described in Section \ref{Pow}.

\item 
The algorithms for the {\tt BaseCase} call in line 3 are 
discussed in Section \ref{base}.
In summary, {\tt BaseCase} $(X, {\it type}, \F)$ returns
the standard generators, the associated \SLPs, 
and the corresponding change-of-basis matrix
for the classical group $\langle X \rangle$ of the specified type
having associated form $\F$. 

\item 
The search in line 4 for an element that powers to a strong
involution is discussed in Section \ref{Involution}.

\item 
The recursive calls in lines 9 and 10 are
in smaller dimension. 
As shown in Lemma \ref{d-a-c}, these only affect 
the time and space complexity of the
algorithm up to a constant multiple; however they contribute 
to the length of the \SLPs\ produced. 
We consider these issues in 
Sections \ref{Analysis} and \ref{SLP}.  

\item 
In line 12, $a$ is an involution  with $-1$-eigenspace
$\langle e_k,f_k,e_{k+1},f_{k+1}\rangle$. 

\item 
The element $b$ is the {\it glue},
used in the assignment $v := v_2bv_1$ to `glue' the elements 
$v_1$ and $v_2$. 
We discuss how to find $b$ as an element of 
$\langle X_3\rangle$ in Section \ref{glue-element}.
\end{enumerate}

Algorithm {\tt OneOdd}, for the odd degree case, 
is similar to Algorithm {\tt OneEven} and 
much of this commentary also applies. 

\begin{algorithm2e}[H]
\caption{\tt OneOdd $(X,{\it type}, \F)$ \rm \ for\ non-orthogonal\ groups}
\label{alg1:odd}
\tcc{
$X$ is a generating set for
the classical group $G \in \C$
in odd characteristic and odd dimension, of type {\bf SL} or {\bf SU}.
The classical form preserved by $G$ is $\F$. 
Return the standard generating set for $G$,
the \SLPs\ for elements of this generating set, 
and the change-of-basis matrix.
}

\Begin{
$d$ := the rank of the matrices in $X$;

if $d = 3$ then return {\tt BaseCase} $(X, {\it type}, \F)$;

Find by random search $g \in G:=\langle X\rangle$ of even order
such that $g$ powers to a strong involution $h$;

Let $E_+$ and $E_-$ be the eigenspaces of $h$;

Find generators for the centraliser $C$ of $h$ in $G$;

Rewrite with respect to the concatenation of 
bases for $E_+$ and $E_-$;

In $C$ find generating sets $X_1$ and $X_2$ 
for $\SX(E_+)$ and $\SX(E_-)$;

$\bigl((s_1,t_1,\delta_1,u_1,v_1,x,y),B_1\bigr)$ := 
{\tt OneOdd} $(X_1,{\it type}, \F|_{E_+})$;

$\bigl((s_2,t_2,\delta_2,u_2,v_2),B_2\bigr)$ := 
{\tt OneEven} $(X_2,{\it type, \F|_{E_-}})$;

If $B_1=(e_1,f_1,\ldots,e_k,f_k,w)$ and 
$B_2=(e_{k+1},f_{k+1},\ldots,e_{(d-1)/2},f_{(d-1)/2})$,
then let $B=(e_1,f_1,\ldots,e_k,f_k,e_{k+1},f_{k+1},\ldots,
      e_{(d-1)/2},f_{(d-1)/2},w)$;

$a := (s_1^2)^{v_1^{-1}}(s_2^2)$;

Find generators for the centraliser $D$ of $a$ in $G$;

In $D$ find a generating set $X_3$ for 
$\SX(\langle e_k,f_k,e_{k+1},f_{k+1}\rangle)$;

In $\langle X_3 \rangle$ find the permutation matrix 
$b=(e_k,e_{k+1})(f_k,f_{k+1})$;

$v := v_2 b v_1$;

return $(s_1,t_1,\delta_1,u_1,v,x,y)$ and the change-of-basis matrix for $B$;
}
\end{algorithm2e}

\medskip
\medskip
We summarise the main algorithm for non-orthogonal 
groups as Algorithm {\tt OneMain}. 

\begin{algorithm2e}[H]
\caption{\tt OneMain $(X,{\it type}, \F)$ \rm \ for\ non-orthogonal\ groups}
\label{alg1:main}
\tcc{
$X$ is a generating set for the classical group $G \in \C$
in odd characteristic, of type {\bf SL} or {\bf Sp} or {\bf SU}.
The classical form preserved by $G$ is $\F$. 
Return the standard generating set for $G$, 
the \SLPs\ for elements of this generating set, 
and the change-of-basis matrix.}

\Begin{
 
$d$ := the rank of the matrices in $X$;

%if $d =4$ then return {\tt BaseCase} $(X, {\it type}, \F)$; 

  \eIf{$d$ is odd}
 {
  $\bigl((s, t, \delta, u, v, x, y), B\bigr)$ := {\tt OneOdd}
  $(X,{\it type}, \F)$;

 }
 {
  $\bigl((s,t,\delta, u, v), B\bigr)$ := {\tt OneEven} $(X,{\it type}, \F)$;

  Construct additional elements $x$ and $y$; 
 }


return $(s, t,\delta,u,v,x,y)$ and the change-of-basis matrix for $B$;
}
\end{algorithm2e}

\medskip
\medskip
The correctness and complexity of this algorithm, 
and the lengths of the resulting \SLPs\ for the 
standard generators, are discussed in 
Sections \ref{Involution} and \ref{Analysis}--\ref{SLP}.
The construction of $x$ and $y$ is discussed in Section \ref{base}.

\section{Algorithm {\tt Two} for non-orthogonal groups} 
\label{Alg2}

We present a variant of the algorithms in Section \ref{Alg1} based on  
one recursive call rather than two. Again we denote 
the groups $\SL(d,q)$, $\Sp(d,q)$
and $\SU(d,q)$ by $\SX(d,q)$, and the corresponding projective group
by $\PX(d,q)$.

The key idea is as follows. Suppose that $d$ is a multiple of 4.  
We find $g \in \SX(d, q)$ of order $2m$ and an 
involution $h := g^m$, as in line 4 of {\tt OneEven},
but insist that both eigenspaces of $h$ have dimension $d/2$. 

Let $\bar{h}$ be the image of $h$ in $\PX(d,q)$.
The centraliser of $\bar{h}$ in $\PX(d,q)$
interchanges the eigenspaces $E_+$ and $E_-$ of $h$.
We construct the
projective centraliser of $h$ in $\SX(d, q)$ by applying the algorithm 
of \cite{Bray} to construct the centraliser of $\bar{h}$ in $\PX(d, q)$,
and taking its preimage $C$. 
We identify $c \in C$ that interchanges the two eigenspaces.

If we now find recursively the subset $Y_0$ of
standard generators for $\SX(E_+)$ with respect the basis $\cal B$,
then $Y_0^c$ is a set of standard generators for $\SX(E_-)$ with
respect to the basis ${\cal B}^g$. We now use these 
to construct standard generators for 
$\SX(d,q)$ exactly as in Algorithm {\tt One}.

If $d$ is an odd multiple of 2, then we find an involution with one
eigenspace of dimension exactly 2. The centraliser of this
involution allows us to construct $\SX(2,q)$ and $\SX(d-2,q)$. The $d-2$
factor is now processed as above, since $d-2$ is a multiple of 4, and
the 2 and $d-2$ factors are combined as in the first algorithm. Thus
the algorithm deals with $\SX(d,q)$, for even  values of $d$, in a way
that is similar in outline to the familiar method of powering, that
computes $a^n$, by recursion on $n$, as $(a^2)^{n/2}$ for even $n$ and
as $a(a^{n-1})$ for odd $n$.

Algorithms {\tt TwoTimesFour} and {\tt TwoTwiceOdd} 
describe the case of even $d$. 
Algorithm {\tt TwoTimesFour} calls no new procedures except in line 5,
where we construct an involution with eigenspaces of equal dimension.
This construction is discussed in Section \ref{Equal}.
Algorithm {\tt TwoEven}, which summarises the even degree case,
returns the generating set $Y_0$ defined in Section \ref{Alg1}. 
We complete the construction of $Y$ exactly as in Section \ref{Alg1}.

\begin{algorithm2e}[H]
\caption{\tt TwoTimesFour$(X,{\it type}, \F)$ \rm \ for\ non-orthogonal\ groups}
\label{alg2:even-b}
\tcc{ $X$ is a generating set for
the classical group $G \in \C$
in odd characteristic, of type {\bf SL} or {\bf Sp} or {\bf SU}, 
in dimension a multiple of 4.
The classical form preserved by $G$ is $\F$. 
Return the standard generating set $Y_0$ for 
$\SL(2, q) \wr C_{d/2}$ if {\it type} is {\bf SL}, 
otherwise for $\SX(2, q) \wr S_{d/2}$, as subgroup of  $G$, 
the \SLPs\ for the elements of $Y_0$, and the change-of-basis matrix.
}

\Begin{

$d$ := the rank of the matrices in $X$;

if $d=4$ return {\tt OneEven} $(X, type, \F)$;

$k$ := $d/4$;

Find by random search $g \in G:=\langle X\rangle$ of even order
such that $g$ powers to an involution $h$
with eigenspaces of dimension $2k$;
 
Let $E_+$ and $E_-$ be the eigenspaces of $h$;

Find generators for the projective centraliser $C$ of $h$ in $G$
and identify an element $c$ of $C$ that interchanges the two eigenspaces;

Rewrite with respect to the concatenation of bases 
for $E_+$ and $E_-$;

In $C$ find a generating set $X_1$ for $\SX(E_+)$;

$\bigl((s_1,t_1,\delta_1,u_1,v_1), B_1\bigr)$ := 
{\tt TwoEven}$(X_1,{\it type}, \F|_{E_+})$;

$s_2$ := $s_1^c$;

Let $B=(e_1,f_1,\ldots,e_k,f_k,e_{k+1},f_{k+1},\ldots,e_{2k},f_{2k})$ be the 
concatenation of the bases defined by $B_1$ and $B_1^c$;

$a := (s_1^2)^{v_1^{-1}}(s_2^2)$;

Find generators for the centraliser $D$ of $a$ in $G$;

In $D$ find a generating set $X_3$ 
for $\SX(\langle e_k,f_k,e_{k+1},f_{k+1}\rangle)$;

In $\langle X_3\rangle$ find the permutation 
matrix $b=(e_k,e_{k+1})(f_k,f_{k+1})$;

$v := v_2 b v_1$;

return $(s_1,t_1,\delta_1,u_1,v)$ and the change-of-basis matrix for $B$;
}
\end{algorithm2e}

\begin{algorithm2e}[H]
\caption{\tt TwoTwiceOdd$(X,{\it type}, \F)$ \rm \ for\ non-orthogonal\ groups}
\label{alg2:even-a}
\tcc{ $X$ is a generating set for
the classical group $G \in \C$
in odd characteristic, of type {\bf SL} or {\bf Sp} or {\bf SU}, 
in dimension $d = 2(k + 1)$ for even $k$.
The classical form preserved by $G$ is $\F$. 
Return the standard generating set $Y_0$ for 
$\SL(2, q) \wr C_{d/2}$ if {\it type} is {\bf SL}, 
otherwise for $\SX(2, q) \wr S_{d/2}$, as subgroup of $G$, 
the \SLPs\ for the elements of $Y_0$, and the change-of-basis matrix.
}

\Begin{

$d$ := the rank of the matrices in $X$; 

If $d \leq 8$ return {\tt OneEven $(X,type, \F)$};

Find, by random search, $g \in G:=\langle X\rangle$ of even order
such that $g$ powers to an involution $h$ with eigenspaces of dimensions
2 and $d-2$;

Let $E_1$ and $E_2$ be the eigenspaces of $h$, of dimensions $d-2$ and $2$
respectively;

Find generators for the centraliser $C$ of $h$ in $G$;

Rewrite with respect to the concatenation of 
bases for $E_1$ and $E_2$;

In $C$ find generating sets 
$X_1$ and $X_2$ for $\SX(E_1)$ and $\SX(E_2)$ respectively;

$\bigl((s_{1},t_1, \delta_1,u_1,v_1), B_1\bigr)$ := 
{\tt TwoTimesFour} $(X_1,{\it type}, {\F}|_{E_1})$;

$\bigl((s_{2},t_2, \delta_2, u_2, v_2), B_2\bigr)$ := 
{\tt BaseCase} $(X_2,{\it type}, {\F}|_{E_2})$;

Let $B=(e_1, f_1, \ldots, e_k, f_k, e_{k + 1}, f_{k+1})$
be the concatenation of the hyperbolic bases $B_1$ and $B_2$;

$a := (s_1^2)^{v_1^{-1}}(s_2^2)$;

Find generators for the centraliser $D$ of $a$ in $G$;

In $D$ find a generating set $X_3$ 
for $\SX(\langle e_k,f_k,e_{k+1},f_{k+1}\rangle)$; 

In $\langle X_3\rangle$ find the permutation 
matrix $b=(e_k,e_{k+1})(f_k,f_{k+1})$;

$v := b v_1$;

%return $(s_1,t_1,\delta_1,u_1,v,x_1,y_1)$ and the 
return $(s_1,t_1,\delta_1,u_1,v)$ and the 
change-of-basis matrix for $B$.

}
\end{algorithm2e}

\begin{algorithm2e}[H]
\caption{\tt TwoEven$(X,{\it type}, \F)$ \rm \ for\ non-orthogonal\ groups}
\label{alg2-main:even}
\tcc{ $X$ is a generating set for
the classical group $G \in \C$
in odd characteristic, of type {\bf SL} or {\bf Sp} or {\bf SU}, 
in even dimension $d$.
The classical form preserved by $G$ is $\F$. 
Return the standard generating set $Y_0$ for 
$\SL(2, q) \wr C_{d/2}$ if {\it type} is {\bf SL}, 
otherwise for $\SX(2, q) \wr S_{d/2}$, as subgroup of  $G$, 
the \SLPs\ for the elements of $Y_0$, and the change-of-basis matrix.
}
\Begin{
$d$ := the rank of the matrices in $X$;

\eIf {$d\bmod4=2$} 
  {
 
 return {\tt TwoTwiceOdd}$(X,{\it type}, \F)$;
  }
  {
 return {\tt TwoTimesFour}$(X,{\it type}, \F)$;
  }
}
\end{algorithm2e}

\medskip
\medskip
If $d$ is odd, then we find an involution whose $-1$-eigenspace has
dimension $d - 3$, thus splitting $d$ as $(d-3)+3$. Since $d-3$ is even, we
apply the odd case {\it precisely once}.

The resulting {\tt TwoOdd} is otherwise the same as {\tt OneOdd},
except that it calls {\tt TwoEven} rather than {\tt OneEven};
similarly {\tt TwoMain} 
is the same as {\tt OneMain}, except that it 
calls {\tt TwoOdd} or {\tt TwoEven} rather than
{\tt OneOdd} or {\tt OneEven}. 

The primary advantage of the second algorithm 
lies in its one recursive call. 
As we show in Section \ref{SLP}, 
this significantly reduces the lengths of the 
\SLPs\ for the standard generators.

\section{Algorithm {\tt One} for orthogonal groups}
\label{Alg3}
The algorithms for orthogonal groups are more complex
in design than those for other classical groups.

If $q\equiv 3 \bmod 4$, then $\Omega^+(2,q)$ has odd order 
and so does not contain $-I_2$.  Hence we must use 
a new strategy to construct the involution
whose centraliser contains the `glue' element.
In particular, the algorithm for $\Omega^\epsilon(d, q)$
depends both on the type of form preserved 
and on the residue of $q \bmod 4$.

For each of the form types, we present three algorithms: 
for $\Omega^\epsilon(d, q)$ when $q \equiv 1 \bmod 4$, then
for $\SO^\epsilon(d, q)$ for all odd $q$,
and finally for $\Omega^\epsilon(d, q)$ when $q \equiv3\bmod4$.

The base cases for the orthogonal groups are discussed
in Section \ref{base-omega} and are realised
via {\tt OrthogonalBaseCase}.

\subsection{Groups preserving forms of $+$ type}
\subsubsection{$\Omega^+(2n,q)$ for $q \equiv 1 \bmod 4$} \label{omega+}
This case is similar to 
Algorithm {\tt One} for the other classical groups. 
Let $G = \Omega^+(2n,q)$ when $q \equiv 1 \bmod 4$,
and let $V$ denote the underlying vector space.
An involution of $G$ is {\it suitable} 
if it is strong and has the additional property that 
the symmetric bilinear form preserved by $G$, when 
restricted to each of its eigenspaces, is of $+$ type.  
Algorithm {\tt OneOmegaPlus} summarises the construction
of standard generators for~$G$. 

\begin{algorithm2e}[ht] 
\caption{\tt OneOmegaPlus $(X, \F)$}
\label{alg-omega+}
\tcc{ $X$ is a generating set for the orthogonal group $G$ of type $+$ 
defined over a field of odd characteristic and 
size $q \equiv 1 \bmod 4$. 
The classical form preserved by $G$ is $\F$. 
Return the standard generating set $Y$ for $G$,
the \SLPs\ for the elements of $Y$, and the change-of-basis matrix.
}
\Begin{
 $d$ := the rank of the matrices in $X$; 

if $d \leq 4$ then return {\tt OrthogonalBaseCase} $(X, \F)$;  

Find by random search $g \in G:=\langle X\rangle$ of 
even order such that $g$ powers to a suitable involution $h$;

Let $E_+$ be the $+1$-eigenspace of $h$ having dimension $2k$ and 
let $E_-$ be its $-1$-eigenspace;

Find generators for the centraliser $C$ of $h$ in $G$;

Rewrite with respect to the concatenation of bases 
for $E_+$ and $E_-$;

In $C$ find generating sets $X_1$ and $X_2$ 
for $\Omega^+(E_+)$ and $\Omega^+(E_-)$;

$\bigl((s_1,t_1,\delta_1,u_1,v_1, s_1', t_1', \delta_1'),B_1\bigr)$ := 
{\tt OneOmegaPlus} $(X_1, {\F}|_{E_+})$;

$\bigl((s_2,t_2,\delta_2,u_2,v_2, s_2', t_2', \delta_2'),B_2\bigr)$ 
:= {\tt OneOmegaPlus} $(X_2, {\F}|_{E_-})$;

Let $B=(e_1,f_1,\ldots,e_k,f_k,e_{k+1},f_{k+1},\ldots,e_{d/2},f_{d/2})$
be the concatenation of the hyperbolic bases defined by $B_1$ and $B_2$;

$m := (q - 1) / 4$;

$a := ((\delta_1 \delta_1')^m)^{v_1^{-1}} (\delta_2 \delta_2')^m$;

Find generators for the centraliser $D$ of $a$ in $G$;

In $D$ find a generating set $X_3$ for 
$\Omega^+(\langle e_k,f_k,e_{k+1},f_{k+1}\rangle)$;

In $\langle X_3\rangle$ find the permutation matrix 
$b=(e_k,e_{k+1})^{-}(f_k,f_{k+1})^{-}$;

$v := v_2 b v_1$;

return $(s_1,t_1,\delta_1,u_1,v, s_1', t_1', \delta_1')$ 
and the change-of-basis matrix for $B$;
}
\end{algorithm2e}

\subsubsection{$\SO^+(2n, q)$}
\label{so+}

The definition of a {\it suitable} involution is as in Section \ref{omega+}.
The centraliser in $\SO^+(2n,q)$ of a suitable involution 
contains the direct product of $\SO(E_+)$ and $\SO(E_-)$.
We construct each group as a subgroup of the
centraliser, and proceed recursively.

We modify {\tt OneOmegaPlus} to obtain the 
resulting algorithm, {\tt OneSpecialPlus}, 
by making the following changes:
\begin{itemize}
\item 
the recursive calls are to {\tt OneSpecialPlus},
and so construct the additional standard 
generator needed to generate $\SO^+(2n,q)$;
\item in line 12, $m := (q - 1) / 2$ if $q \equiv 1 \bmod 4$, otherwise
$m := 1$; in line 13, $a := (\w_1^m)^{v_1^{-1}} (\w_2^m)$.
\end{itemize}

\subsubsection{$\Omega^+(2n,q)$ when $q\equiv3\bmod4$}
The algorithm for $G=\Omega^+(2n,q)$ when 
$q\equiv3\bmod4$ is more elaborate than 
when $q\equiv1\bmod4$: 
now $\Omega^+(2,q)$ has odd order $(q - 1) / 2$ and so does
not contain $-I_2$. To construct the involution
whose centraliser contains the `glue' element,  we 
must move outside $\Omega(E)$ to $\SO(E)$ where 
$E$ is a particular eigenspace.

We outline the steps of the algorithm, {\tt OneOmegaPlus3},
which applies when $n > 2$. The remaining cases
are considered in Section \ref{base-omega}.
\begin{enumerate}
\item  
Find, by random search, an element of $G$ that powers to 
a strong involution $i$ having eigenspaces $E$ and $F$,
with the additional property that the symmetric bilinear form 
preserved by $G$, 
when restricted to these eigenspaces, is of $+$ type.  
%Lemma \ref{form-type} shows that the $-1$-eigenspace, $F$ say, 
%of $i$ has dimension a multiple of $4$. 

\item 
Construct a generating set for the centraliser
$H=\SO(E)\times_{C_2}\SO(F)$ of $i$ in $G$, 
and hence generating sets $X$ and $Y$ for $\Omega(E)$ and $\Omega(F)$ as
subgroups of $H$.  

\item 
Find, by random search within $H$, an element $g=(g_1,g_2)$, where
$g_1\in\SO(E)$ and $g_2\in\SO(F)$, and the spinor norms of 
$g_1$ and of $g_2$ are both $-1$. Hence both have even order.
We also require one of the $g_j$, say $g_2$, to have twice odd order.  
Hence $|g_1| = 2^s k_1$ and $|g_2| = 2 k_2$ where $k_i$ is odd and $s \geq 1$. 
Assign $g := g^{k_1 k_2}$; now $g$ has order a 
power of 2, and $g_2$ is an involution.  

\item Let $A = \langle X, g \rangle$.
Its projection onto $E$ is $\SO(E)$.
Using {\tt OneSpecialPlus},
construct \SLPs\ in the generators of $A$ that map onto
standard generators for $\SO(E)$.

\item 
If $z$ is an element of $\SO(E)$, then $z$ is the projection onto $E$ of
an \SLP\ on $X\cup\{g\}\subset\SO(E)\times\SO(F)$.  Evaluating this
\SLP\ gives rise to an element $(z,z_1)$, where $z_1\in\SO(F)$.  
%We shall see that $z_1$ depends only on $z$, and not on the $\SLP$.
Since $X$ centralises $F$, $z_1$ is a power of $g_2$, and hence is 
either the identity
or $g_2$. But the spinor norms of $z$ and $z_1$ are equal; so $z_1=g_2$ if
the spinor norm of $z$ is $-1$, and $z_1=1$ otherwise.  Thus, from
Step 4, we obtain $h=(\sigma_E,g_2)$ where $\sigma_E$ is the standard
generator of $\SO(E)$ with spinor norm $-1$.  Note that $\sigma_E$ is an
involution since $q\equiv3\bmod4$.

\item  Let $B=\langle Y,h \rangle$.  Its projection onto $F$ is $\SO(F)$.
Using {\tt OneSpecialPlus}, construct $\SLPs$ in the generators of $B$ 
that map onto standard generators for $\SO(F)$.  Now apply Step 5 with 
$E$ and $F$, and also $\sigma_E$ and $g_2$, interchanged.  
We thus construct $(\sigma_E,\sigma_F)$.

\item  Now conjugate $(\sigma_E, \sigma_F)$ by a suitable
power of $v_1$ to obtain $a$, the involution
in whose centraliser the `glue' element can be found.
\end{enumerate}

The remaining steps of the algorithm are identical
to those described in {\tt OneOmegaPlus} when $q\equiv1\bmod4$.
Namely, we find generators for the centraliser $D$ of $a$ in $G$;
construct a generating set $X_3$ 
for $\Omega^+(\langle e_k,f_k,e_{k+1},f_{k+1}\rangle)$;
in $\langle X_3\rangle$ find the permutation matrix 
$b=(e_k,e_{k+1})^{-}(f_k,f_{k+1})^{-}$;
and finally construct the standard generator $v := v_2 b v_1$.

\subsection{Groups preserving forms of $-$ type}
\subsubsection{$\Omega^-(2n,q)$ when $q\equiv 1\bmod 4$}\label{omega-}

In summary, we construct an involution in $G = \Omega^-(2n,q)$ whose centraliser
contains a direct product of $\Omega^+(2n-4,q)$ and $\Omega^-(4,q)$. 
We then recursively construct standard generators for each factor. 
Within the centraliser of an involution of $+$ type,
we find the `glue' element. 

An involution of $G$ is {\it suitable} 
if it has one eigenspace 
of dimension 4 supporting a form of $-$ type; and its 
other eigenspace, consequently of dimension $2n-4$,
supports a form of $+$ type.  

Algorithm {\tt OneOmegaMinus} summarises the construction
of standard generators for~$G$.

\begin{algorithm2e}[H] 
\caption{\tt OneOmegaMinus $(X, \F)$}
\label{alg-omega-}
\tcc{
$X$ is a generating set for the orthogonal group $G$ of type $-$ 
defined over a field of odd characteristic and 
size $q \equiv 1 \bmod 4$. 
The classical form preserved by $G$ is $\F$. 
Return the standard generating set $Y$ for $G$,
the \SLPs\ for the elements of $Y$, and the change-of-basis matrix.
}
\Begin{
$d$ := the rank of the matrices in $X$; 

if $d = 4$ then return {\tt OrthogonalBaseCase} $(X, \F)$;  

Find by random search $g \in G:=\langle X\rangle$ of 
even order such that $g$ powers to 
a suitable involution $h$; 

Let $E$ be the eigenspace of $h$ of dimension $d-4$ and let $F$ be the 
eigenspace of $h$ having dimension $4$;

Find generators for the centraliser $C$ of $h$ in $G$;

Rewrite with respect to the concatenation of bases for $E$ and $F$;

In $C$ find generating sets $X_1$ and $X_2$ 
for $\Omega^+(E)$ and $\Omega^-(F)$;

$\bigl((s_1,t_1,\delta_1,u_1,v_1, s_1', t_1', \delta_1'),B_1\bigr)$ := 
   {\tt OneOmegaPlus} $(X_1,\F|_E)$;

$\bigl((s_2,t_2,\delta_2,u_2,v_2),B_2\bigr)$ 
:= {\tt OrthogonalBaseCase} $(X_2, \F|_F)$;

$k := (d - 4) / 2$;

Let $B=(e_1,f_1,\ldots,e_k,f_k,e_{k+1},f_{k+1},e_{d/2},f_{d/2})$
be the concatenation of the hyperbolic bases defined by $B_1$ and $B_2$;

$m := (q - 1) / 4$;

$a := ((\delta_1 \delta_1')^m)^{v_1^{-1}} \delta_2^{m (q + 1)}$;

Find generators for the centraliser $D$ of $a$ in $G$;

In $D$ find a generating set $X_3$ for 
$\Omega^+(\langle e_k,f_k,e_{k+1},f_{k+1}\rangle)$;

In $\langle X_3\rangle$ find the permutation matrix 
$b=(e_k,e_{k+1})^{-}(f_k,f_{k+1})^{-}$;

$v := b v_1$;

return $(s_2,t_2,\delta_2,u_2,v)$ and the change-of-basis matrix for $B$;
}
\end{algorithm2e}

\subsubsection{$\SO^-(d, q)$}
The definition of a {\it suitable} involution is as in Section \ref{omega-}.
The centraliser in $\SO^-(2n,q)$ of a suitable involution 
contains the direct product of $\SO^+(E)$ and $\SO^-(F)$.
We construct each group as a subgroup of the
centraliser, and proceed recursively.
By analogous modifications to those outlined in Section \ref{so+}, 
we modify {\tt OneOmegaMinus} to obtain {\tt OneSpecialMinus}.

\subsubsection{$\Omega^-(2n,q)$ when $q\equiv3\bmod4$}
In summary, we construct an involution in $G=\Omega^-(2n,q)$ 
whose centraliser
contains a direct product of $\Omega^+(2n-2k,q)$ and $\Omega^-(2k,q)$,
where $k$ is 2 or 3, depending on the parity of $n$. 
We then recursively construct standard generators for each factor. 
As in the corresponding case of $\Omega^+(2n,q)$, 
we must move from $\Omega^\epsilon$ to 
the corresponding $\SO^\epsilon$ to find
the involution whose centraliser contains the `glue' element. 

However, the definition of a suitable involution is now more complex.
\begin{itemize}
\item 
If $n > 3$ is even, then an involution is {\it suitable} if its 
$+1$-eigenspace has dimension 4 and supports a form of $-$ type, and 
its $-1$-eigenspace of dimension $2n-4$ supports a form of $+$ type.  
\item 
If $n>3$ is odd, then 
an involution is {\it suitable} if it has one eigenspace of dimension 6 
that supports a form of $-$ type, and its other eigenspace of dimension
$2n-6$ supports a form of $+$ type.
\end{itemize}

We now outline the steps of the algorithm {\tt OneOmegaMinus3}.
Similar in structure to {\tt OneOmegaPlus3}, it applies only when $n > 3$.

\begin{enumerate}
\item 
Find, by random search, an element of $G$ that powers to a 
suitable involution $i$. 
Let $E$ and $F$ denote the eigenspaces of $i$ which support
the forms of $+$ and $-$ type respectively.

\item 
Construct a generating set for the centraliser
$H=\SO^+(E)\times_{C_2}\SO^-(F)$ of $i$ in $G$, 
and hence generating sets $X$ and $Y$ for $\Omega(E)$ and $\Omega(F)$ as
subgroups of $H$.  

\item 
Find, by random search 
within $H$, an element $g=(g_1,g_2)$, where
$g_1\in\SO^+(E)$, and $g_2\in\SO^-(F)$, and the spinor norms of 
$g_1$ and of $g_2$ are both $-1$.  We also require one of the $g_j$
to have twice odd order.  
The proportion of elements of $H$ with this property
is the proportion of elements of 
$\SO^+(E)$ (if $j=1$) or of $\SO^-(F)$ (if $j=2$)
of twice odd order, and of spinor norm $-1$.  
For ease of exposition we assume that $j=2$.
Hence $|g_1| = 2^s k_1$ and $|g_2| = 2 k_2$ where $k_i$ is odd and $s \geq 1$. 
Assign $g := g^{k_1 k_2}$; now $g$ has order a 
power of 2, and $g_2$ is an involution.  

\item Let $A = \langle X, g \rangle$.
Its projection onto $E$ is $\SO^+(E)$.
Using {\tt OneSpecialPlus},
construct \SLPs\ in the generators of $A$ that map onto 
standard generators for $\SO^+(E)$.

\item 
If $z$ is an element of $\SO(E)$, then $z$ is the projection onto $E$ of
an \SLP\ on $X\cup\{g\}\subset\SO(E)\times\SO(F)$.  Evaluating this
\SLP\ gives rise to an element $(z,z_1)$, where $z_1\in\SO(F)$.  
%We shall see that $z_1$ depends only on $z$, and not on the $\SLP$.
Since $X$ centralises $F$, $z_1$ is a power of $g_2$, and hence is 
either the identity
or $g_2$. But the spinor norms of $z$ and $z_1$ are equal; so $z_1=g_2$ if
the spinor norm of $z$ is $-1$, and $z_1=1$ otherwise.  Thus, from
Step 4, we obtain $h=(\sigma_E,g_2)$ where $\sigma_E$ is the standard
generator of $\SO(E)$ with spinor norm $-1$.  Note that $\sigma_E$ is an
involution since $q\equiv3\bmod4$.

\item Let $B=\langle Y,h\rangle$.  Its projection onto $F$ is $\SO(F)$.
Using {\tt OneSpecialMinus}, construct $\SLPs$ in the generators of $B$ 
that map onto standard generators for $\SO(F)$.  Now apply Step 5 
with $E$ and $F$,
and also $\sigma_E$ and $g_2$, interchanged.  We thus construct
$(\sigma_E,\sigma_F)$.

\item  Now conjugate
$(\sigma_E, \sigma_F)$ by a suitable
power of $v_1$ to obtain $a$, the involution
in whose centraliser the `glue' element can be found.

\end{enumerate}

The remaining steps of the algorithm are identical
to those described in {\tt OneOmegaPlus}
when $q\equiv1\bmod4$.

If $n = 3$ then the non-central involutions in $\Omega^-(6,q)$
have centralisers containing $\Omega^+(4,q)\times\Omega^-(2,q)$. 
Algorithms for $\Omega^-(4,q)$ and $\Omega^-(6, q)$ are presented  
in Section~\ref{base-omega}.

\subsection{Groups preserving forms of $0$ type}

\subsubsection{$\Omega(2n+1,q)$ when $q\equiv 1\bmod4$}\label{omega0}
In summary, we construct an involution in $G = \Omega(2n+1,q)$ 
whose centraliser contains $\Omega^+(2n-2,q) \times \Omega(3,q)$.
We then recursively construct standard generators for each factor. 
Within the centraliser of an involution of $+$ type,
we find the `glue' element. 

An involution of $G$ is {\it suitable} if its $-1$-eigenspace has 
dimension $2n - 2$ and supports a form of $+$ type.

Algorithm {\tt OneOmegaCircle} summarises the construction
of standard generators for~$G$.

\begin{algorithm2e}[ht] 
\caption{\tt OneOmegaCircle $(X, \F)$}
\label{alg-omega0}
\tcc{
$X$ is a generating set for the orthogonal group $G$ of type $0$ 
defined over a field of odd characteristic and 
size $q \equiv 1 \bmod 4$. 
The classical form preserved by $G$ is $\F$. 
Return the standard generating set $Y$ for $G$,
the \SLPs\ for the elements of $Y$, and the change-of-basis matrix.
}
\Begin{
 $d$ := the rank of the matrices in $X$; 

if $d = 3$ then return {\tt OrthogonalBaseCase} $(X, \F)$;  

Find by random search $g \in G:=\langle X\rangle$ of 
even order such that $g$ powers to 
a suitable involution $h$;

Let $E$ be the eigenspace of $h$ of dimension $d-3$ and let $F$ be the 
eigenspace of $h$ having dimension $3$;

Find generators for the centraliser $C$ of $h$ in $G$;

Rewrite with respect to the concatenation of bases for $E$ and $F$;

In $C$ find generating sets $X_1$ and $X_2$ 
for $\Omega^+(E)$ and $\Omega^0(F)$;

$\bigl((s_1,t_1,\delta_1,u_1,v_1, s_1', t_1', \delta_1'),B_1\bigr)$ := 
   {\tt OneOmegaPlus} $(X_1, \F|_E)$;

$\bigl((s_2,t_2,\delta_2,u_2,v_2),B_2\bigr)$ := 
{\tt OrthogonalBaseCase} $(X_2, \F|_F)$;

$k := (d - 3) / 2$;

Let $B=(e_1,f_1,\ldots,e_k,f_k,e_{k+1},f_{k+1},w)$
be the concatenation of the hyperbolic bases defined by $B_1$ and $B_2$;

$m := (q - 1) / 4$;

$a := ((\delta_1 \delta_1')^m)^{v_1^{-1}} \delta_2^{m}$;

Find generators for the centraliser $D$ of $a$ in $G$;

In $D$ find a generating set $X_3$ for 
$\Omega^+(\langle e_k,f_k,e_{k+1},f_{k+1}\rangle)$;

In $\langle X_3\rangle$ find the permutation matrix 
$b=(e_k,e_{k+1})^{-}(f_k,f_{k+1})^{-}$;

$v := b v_1$;

return $(s_2,t_2,\delta_2,s_1',v)$ and the change-of-basis matrix for $B$;
}
\end{algorithm2e}

\subsubsection{$\SO(2n+1,q)$}
The definition of a {\it suitable} involution is as in Section \ref{omega0}.
The centraliser in $\SO(2n+1,q)$ of a suitable involution 
contains the direct product of $\SO^+(E)$ and $\SO^-(F)$.
We construct each group as a subgroup of the centraliser, and 
proceed recursively.
By analogous modifications to those outlined in Section \ref{so+}, 
we modify {\tt OneOmegaCircle} to obtain {\tt OneSpecialCircle}.

\subsubsection{$\Omega(2n+1,q)$ when $q\equiv3\bmod4$}
In summary, we construct an involution in $\Omega(2n+1,q)$ 
whose centraliser
contains a direct product of $\Omega^+(2n-2k,q)$ and $\Omega(2k + 1,q)$,
where $k = 1$ or $k=2$ according as $n$ is odd or even.
We then recursively construct standard generators for each factor. 
Within the centraliser of an involution of $+$ type,
we find the `glue' element. 

\begin{itemize}
\item 
If $n>2$ is odd, then 
an involution $i$ is {\it suitable} if it has 
a $-1$-eigenspace $E\_$ of dimension $2n - 2$ which
supports a form of $+$ type.

\item 
If $n>2$ is even, 
then an involution $i$ is {\it suitable} if it has 
a $-1$-eigenspace $E\_$ of dimension $2n - 4$ which
supports a form of $+$ type.
\end{itemize}
Our algorithm, {\tt OneOmegaCircle3}, is similar to {\tt OneOmegaPlus3}
and applies when $n > 2$.
We construct the subgroup 
$H := \SO(E_-) \times_{C_2} \SO(E_+)$ of the centraliser 
of $i$, and call 
{\tt OneSpecialPlus} and {\tt OneSpecialCircle} to construct
the involution whose centraliser contains 
the `glue' element. 

Algorithms for $\Omega (3, q)$ and $\Omega(5, q)$ 
are presented in Section~\ref{base-omega}.

\section{Algorithm {\tt Two} for orthogonal groups}
\label{Alg4}

If $G=\Omega^+(d, q)$ and $q\equiv1\bmod4$, or if $G=\SO^+(d,q)$ with
no such restriction on $q$, then Algorithm {\tt Two} is essentially the 
same as that presented for non-orthogonal groups.

If $G=\Omega^+(d,q)$ and $q\equiv3\bmod4$, then the $-1$-eigenspace of 
an involution in $G$ has dimension a multiple of 4 if
it supports a form of $+$ type (see Lemma \ref{form-type}).  

Hence, if $d$ is a multiple of 8, then we 
find an involution whose eigenspaces are of equal dimension,
and which support forms of $+$ type.  We next find generators for
the centraliser of this involution, and call Algorithm {\tt Two} for
$\SO^+(d/2,q)$ acting on one of the eigenspaces.  We then proceed
as in Algorithm {\tt Two} for non-orthogonal groups.

If $d\equiv e\bmod8$,
where $e\in\{2,4,6\}$,  and $d>8$, then we find an involution 
with one eigenspace of dimension $e$ and one of
dimension $d-e$, construct generating sets for 
$\SO^+(e,q)$ and $\SO^+(d-e,q)$, 
apply Algorithm {\tt One} to the former, and 
Algorithm {\tt Two} to the latter, and glue.

For $\epsilon \in \{-,0\}$, we process
$\Omega^\epsilon (d, q)$ as in Algorithm {\tt One}, 
but apply Algorithm {\tt Two},
rather than Algorithm {\tt One}, in the call 
that processes a copy of $\Omega^+(d-e,q)$.

\section{Finding and constructing involutions}
\label{Involution}

Our principal algorithms require the construction, as an \SLP\ in a 
generating set of a group $G$,
of an involution with various properties; so we search randomly in 
$G$ for an element that powers to an involution with
these properties.  In this section, we estimate the proportion of 
elements of $G$ that power
to such an involution; we also consider how we determine whether a 
given element of $G$ has this property, and estimate 
the time taken to construct the involution.  

We first consider the proportion of elements with the given property.
In Table \ref{inv-table} we summarise the types of involution required, 
including those needed for certain base cases.
Recall from Section \ref{cent} that 
the eigenspaces are denoted by $E$ and $F$, where $E$ has 
dimension $e$, or by $E_+$
and $E_-$ where these are respectively the $+1$ and $-1$-eigenspaces 
of the involution.
If $e$ is required to be the dimension of  $E_-$, then we write $e_-$ for $e$.
If $G$ is a symplectic group or an orthogonal group of $+$ or $-$ type, 
then clearly $d$ and
$e$ must be even; also $e_-$ is always even 
as the involution must have determinant 1.  
If $G$ is an orthogonal group of type $0$, then $d$ is odd. 
We assume $d>e$. These
restrictions on $d$ and $e$ are omitted from Table \ref{inv-table}.  
The type of an eigenspace in an orthogonal group is the 
type $+$, $-$ or $0$, of the form restricted to
the eigenspace.

The first entry in Table \ref{inv-table} identifies the 
group $G$, the second entry gives restrictions on the involution, 
and the third entry gives a lower bound to the proportion of elements of $G$
that power to an involution satisfying these restrictions.  This lower 
bound is generally conservative.  If $e_-$ is required
to lie in the range $(d/3,2d/3]$, then the proportion is a lower bound to 
the proportion of elements of $G$ that power to an involution
with $e_-$ taking any non-specified value in this range, this being 
the property required of our algorithms.  However the bound is
proved by proving that this bound applies to one specific value of 
$e_-$, namely the unique power of $2$ in this range.

Since the precise lower bounds for many of the entries
are complex, we summarise these in 
Table \ref{inv-table} using notation of the form $(c / d)(1 + \Oh(1/q))$
where $c$ is a specified constant.
However, we stress that the actual results are in all cases strictly
positive, and more precise bounds are specified
in the corresponding statements, or can readily be derived from these.

\begin{table}[p]
\begin{center}
\begin{tabular}
{|c|r|r|} \hline
Group       & Conditions         & Proportion \rule{0cm}{3.0ex}\\ \hline
$\SL(d,q)$  & $e_-\in(d/3,2d/3]$  & $(1/(2d))(1+\Oh(1/q))$ \\ 
\hline 

$\SL(d,q)$ & $d\equiv2\bmod4$, $e=2$ & $(1/(2d))(1+\Oh(1/q))$  
\\ \hline 

$\SL(d,q)$ & $d$ odd,  $e=3$  & $(1/(3d))(1+\Oh(1/q))$  %TwoOdd
\rule{0cm}{3.0ex}\\ \hline

$\Sp(d,q)$ & $d$ even, $e_- \in (d/3,2d/3]$ & $(3/(4d))(1 + \Oh(1/q))$  
\rule{0cm}{3.0ex}\\ \hline
% OneEven  non-orthog

$\Sp(d,q)$ & $d\equiv2\bmod4$, $e=2$  &  $(1/(2d))(1+\Oh(1/q))$   
%TwoTwiceOdd non-orthog
\rule{0cm}{3.0ex}\\ \hline

$\Omega^ +(d,q)$ & $q\equiv1\bmod4$, $e_- \in (d/3,2d/3]$
& $(3/(4d))(1 + \Oh(1/q))$  \\ 
& $E$ and $F$ of $+$ type & 
\rule{0cm}{3.0ex}\\ \hline
%OneOmegaPlus 

$\Omega^+(d,q)$ &  $q\equiv3\bmod4$,  
$e_- \in (d/3,2d/3]$
& $(3/(4d))(1 + \Oh(1/q))$  \\ 
& $e_-\equiv0\bmod4$, $E$ and $F$ of $+$ type &  
\rule{0cm}{3.0ex}\\ \hline
%OneOmegaPlus3

$\Omega^+(d,q)$ & $q\equiv3\bmod4$, $d\bmod8\ne0$ &
$(1/(2d))(1 + \Oh(1/q))$ \\ 
%Two
& $e=d\bmod8$, $E$ and $F$ of $+$ type &  
\rule{0cm}{3.0ex}\\ \hline

$\Omega^-(d,q)$ & $q\equiv1\bmod4$, $d\ge6$,  $e=4$ & 
$(1/(4d))(1+\Oh(1/q))$  \\ 
%OneOmegaMinus
& $F$ of $+$ type  &  
\rule{0cm}{3.0ex}\\ \hline

$\Omega^-(d,q)$ & $q\equiv3\bmod4$, $d\equiv0\bmod4$,  $e_+=4$ &
$(1/(4d))(1+\Oh(1/q))$ \\ %OneOmegaMinus3
 & $F$ of $+$ type  & 
\rule{0cm}{3.0ex}\\ \hline

$\Omega^-(d,q)$ & $q\equiv3\bmod4$, $d\equiv2\bmod4$,  $e=6$ 
& $(1/(4d))(1+\Oh(1/q))$   \\
& $F$ of $+$ type & 
 %OneOmegaMinus3
\rule{0cm}{3.0ex}\\ \hline

$\Omega^-(6,q)$ & $q\equiv3\bmod4$,  $e=2$, $F$ of $+$ type & 
$1/8+\Oh(1/q)$  
 %Section 8.2 
\rule{0cm}{3.0ex}\\ \hline

$\Omega^0(5,q)$ & $q\equiv3\bmod4$, $q>3$,  $e_+=1$ & $3/4+\Oh(1/q)$  \\
%Section 8.3
&  $F$ of $+$ type & 
\rule{0cm}{3.0ex}\\ \hline

$\Omega^0(d,q)$ & $q\equiv1\bmod4$ or $d\equiv3\bmod4$,  $e=3$  &
$(1/(4d))(1 +\Oh(1/q))$   \\
%OneOmegaCircle OneOmegaCircle3
& $F$ of $+$ type  &  
\rule{0cm}{3.0ex}\\ \hline

$\Omega^0(d,q)$ & $q\equiv3\bmod4$, $d\equiv1\bmod4$, 
$e=5$ & $(3/(16d))(1+\Oh(1/q))$ \\ %OneOmegaCircle3
& $F$ of $+$ type   & 
\rule{0cm}{3.0ex}\\ \hline

$\SO^+(d,q)$ & $e_- \in (d/3,2d/3]$, 
$E$ and $F$ of $+$ type  & 
$(3/(4d))(1 + \Oh(1/q))$  
%OneSpecialPlus
\rule{0cm}{3.0ex}\\ \hline

$\SO^-(d,q)$ &  $e=4$, $F$ of $+$ type  & $(3/(8d))(1+\Oh(1/q))$  
%OneSpecialMinus
\rule{0cm}{3.0ex}\\ \hline

$\SO^0(d,q)$ & $e=3$, $F$ of $+$ type & $(1/(8d))(1+\Oh(1/q))$  
%OneSpecialCircle
\rule{0cm}{3.0ex}\\ \hline

$\SU(d,q)$  & $e_-\in(d/3,2d/3]$ & $(3/(4d))(1+\Oh(1/q))$  
 %OneEven non-orthog
\rule{0cm}{3.0ex}\\ \hline

$\SU(d,q)$ & $d\equiv2\bmod4$,  $e=2$  & $(1/(2d))(1+\Oh(1/q))$  
 %TwoTwiceOdd non-orthog
\rule{0cm}{3.0ex}\\ \hline

$\SU(d,q)$ & $d$ odd, $e=3$ & $(1/(6d))(1+\Oh(1/q))$  
%TwoOdd
\smallskip
\rule{0cm}{3.0ex}\\ \hline

\end{tabular}
\end{center}
\caption{Elements of even order and lower bounds on proportions}
\label{inv-table}
\end{table}

The first objective of this section is to prove the following theorem.
\begin{theorem}  \label{proportions} The proportion of elements 
of the group named in the first entry of 
any row in Table $\ref{inv-table}$ that 
are of even order, and power to an involution whose eigenspaces satisfy 
the conditions imposed in the second entry, is at least 
the value given in the third entry, and is strictly positive.
\end{theorem}
The theorem will be proved in stages. We commence
our analysis with $\GL(d, q)$. 

\subsection{The general linear group} \label{prop-gl} 
We estimate the proportion of elements of
$\GL(d,q)$ that power to an involution having an eigenspace
of specified dimension within a given range.

\begin{lemma}\label{monic} 
The number of irreducible monic polynomials of degree
$e>1$ with coefficients in $\GF(q)$ is $k$ where
$(q^e-1)/e>k\ge q^e(1-q^{-1})/e$.
\end{lemma}
\begin{proof}
Let $k$ denote the number of such polynomials.
We use the inclusion-exclusion principle to count the
number of elements of $\GF(q^e)$ that do not lie in any  maximal
subfield containing $\GF(q)$, and divide this number by $e$, since
every irreducible monic polynomial of degree $e$ over $\GF(q)$ corresponds
to exactly $e$ such elements.  Thus 
$$k = {q^e-\sum_iq^{e/p_i}+\sum_{i<j}q^{e/p_ip_j}-\cdots\over e}$$ 
where $p_1<p_2<\cdots$ are the distinct prime divisors of $e$. 
The inequality $(q^e-1)/e >k$ is obvious.  If $e$ is a prime,
then $k=(q^e-q)/e\ge q^e(1-q^{-1})/e$, with equality if $e=2$.
Now suppose that $e$ is composite, and let $\ell$ denote the largest prime
dividing $e$.  Hence, from the above formula, 
$$ek \ge q^e-q^{e/\ell}-q^{(e/\ell)-1}-\ldots -1>q^e - q^{e-1}.$$
The result follows.
\end{proof}

\begin{lemma}\label{monic-norm}
The number of irreducible monic polynomials of degree
$e>1$ with coefficients in $\GF(q)$, and specified non-zero constant
term $a\in\GF(q)^\times$, is $k(a)$, where
$(q^e-1)/e\ge(q-1)k(a)\ge q^e(1-q^{-1})/e$ if $e > 2$.
If $e = 2$, then $k(a) = (q \pm 1)/2$.
\end{lemma}
\begin{proof}
Suppose first that $e=2$.  Then $2k(a)$ is the number of elements of
$\GF(q^2)\setminus \GF(q)$ of norm $a$.  The number of elements of
$\GF(q^2)$ of norm $a$ is $q+1$, and either 2 or 0 of these lie in $\GF(q)$, 
depending on whether or not $a$ is a square in $\GF(q)$.  
It follows that $k(a)=(q\pm1)/2$.

Now suppose that $e>2$.  If $e$ is prime, then the number of elements of
$\GF(q^e)$ of norm $a$ is $(q^e-1)/(q-1)$, and the number of elements
of $\GF(q)$ of norm $a$ lies between 0 and $q-1$.  It follows easily that
$k(a)$ lies between the given bounds.  
If $e$ is composite then, with the notation
of Lemma \ref{monic}, 
$$(q^e-1)/(q-1)>ek(a)>(q^e-1)/(q-1)-
\sum_i q^{e/p_i}+\sum_{i<j}q^{e/p_ip_j}-\cdots$$
For the lower bound, we take the number of elements 
of $\GF(q^e)$ of norm $a$, and subtract the
number of elements in the proper subfields of 
$\GF(q^e)$ containing $\GF(q)$, regardless
of their norm.  
Since $(q^e-1)/(q-1) = q^{e-1}+q^{e-2}+\cdots+1$, it follows that
$ek(a)\ge q^{e-1}>q^{e-1}(1-q^{-1})$, giving the required lower bound. 
\end{proof}

\begin{lemma}\label{Lemma5.3} Let $e>d/2$ and $d \geq 4$. 
The proportion of elements of
$\GL(d,q)$ whose characteristic polynomial has an irreducible factor
of degree $e$ lies between $(1/e)(1-q^{-1})$ and $1/e$, and 
is independent of $d$. \ADD
\end{lemma}
\begin{proof}
Let the characteristic polynomial of $g\in\GL(d,q)$ have an
irreducible factor $h(x)$ of degree $e$. Then the kernel of $h(g)$ is
a subspace of $V$ of dimension $e$. It follows that the number of
elements of $\GL(d,q)$ of the required type  is $k_1k_2k_3k_4k_5$
where $k_1$ is the number of subspaces of $V$ of dimension $e$, 
$k_2$ is the number of irreducible monic polynomials of degree $e$
over $\GF(q)$, $k_3$ is the number of elements of $\GL(e,q)$ that
have a given irreducible characteristic polynomial, $k_4$ is the
order of $\GL(d-e,q)$, and $k_5$ is the number of complements in
$V$ to a subspace of dimension $e$.
In more detail: 
\begin{eqnarray*}
k_1 & = & (q^d-1)(q^d-q)\cdots(q^d-q^{e-1})\over(q^e-1)(q^e-q) \cdots (q^e-q^{e-1}) \\
k_3 & = & (q^e-q)(q^e-q^2)\cdots(q^e-q^{e-1}) \\
k_4 & = & (q^{d-e}-1)(q^{d-e}-q) \cdots (q^{d-e}-q^{d-e-1}) \\
k_5 & = & q^{e(d-e)}.
\end{eqnarray*}
The formula 
for $k_3$ arises by taking
the index in $\GL(e,q)$ of the centraliser of an irreducible element,
this centraliser being cyclic of order $q^e-1$. 
The formula for $k_2$ is given in Lemma \ref{monic}. 
Hence $k_1 k_2 k_3 k_4 k_5 = \vert\GL(d,q)\vert\times k_2/(q^e-1)$. 
The result follows.  
\end{proof}
\begin{lemma}\label{Lemma5.4} Let $e\in(d/3,d/2]$ and $d\ge 4$.  
The proportion
of elements of $\GL(d,q)$ that have a characteristic polynomial with exactly
one irreducible factor of degree $e$ lies in the 
interval $[e^{-1}(1-q^{-1})-e^{-2}(1-q^{-1})^2,e^{-1}-e^{-2}]$. \ADD
\end{lemma}
\begin{proof} This proportion may be estimated as in the 
proof of Lemma \ref{Lemma5.3},
but $k_4$ must be replaced by the number of elements 
of $\GL(d-e,q)$ whose characteristic
polynomial does not have an irreducible factor of degree $e$.  
Thus the proportion required
is $(1/e)(1-c/q)-(1/e^2)(1-c/q)^2$, where $c$ lies in the interval $[0,1]$.
\end{proof}

\begin{lemma}\label{Lemma5.5b}
The proportion of elements of $\GL(d,q)$ whose characteristic polynomial is
irreducible, and with a specified determinant, lies in the interval  
$\big((dq)^{-1}, d^{-1}(q - 1)^{-1}\big]$ if $d >2$ and
is $d^{-1}(q \pm 1)^{-1}$ if $d = 2$.  Moreover, the 
number of such elements is independent of the choice
of characteristic polynomial.
%$\big((dq)^{-1}(1-q^{-1}), (dq)^{-1}(1-q^{-1})^{-1}\big]$.
\end{lemma}
\begin{proof}
The proportion is $k(a)/(q^d-1)$, where $a$ is the determinant 
in question, and $k(a)$ is defined and
estimated in Lemma \ref{monic-norm}.  
%Thus the proportion lies in the given interval.
\end{proof}
\subsection{The special linear group}
We now show how the results of Section \ref{prop-gl} 
must be adjusted if $\GL(d,q)$ is replaced by $\SL(d,q)$.

\begin{lemma}\label{Lemma5.5a} 
The proportion in Lemma $\ref{Lemma5.3}$ is unaltered if
$\GL(d,q)$ is replaced by $\SL(d,q)$, provided that $e<d$. 
\end{lemma}
\begin{proof}
Since $e<d$, the number of elements of $\SL(d,q)$ of the required type
may be obtained by replacing $k_4$ with
the number of elements of $\GL(d-e,q)$ of a specified determinant.
But the number of such elements is exactly the number of elements of
$\GL(d-e,q)$ divided by $q-1$; so the result follows.
\end{proof}

\begin{lemma}\label{Lemma5.5c}
The proportion in Lemma $\ref{Lemma5.4}$ is unaltered if
$\GL(d,q)$ is replaced by $\SL(d,q)$, provided that $e<d/2$. 
\end{lemma}
\begin{proof}
The proof is similar to that of Lemma \ref{Lemma5.5a}.
\end{proof}

\begin{lemma}\label{Lemma5.5d}
The proportion of elements of $\SL(2e,q)$ whose characteristic 
polynomial has a unique irreducible factor
of degree $e$ lies in the interval 
$\big[e^{-1}(1-q^{-1})-e^{-2}(1-q^{-1}), e^{-1}-e^{-2}(1-q^{-1})^2\big)$.
\ADD
\end{lemma}
\begin{proof}
The proportion in question is $\alpha(1-(q-1)\beta)$, 
where Lemma \ref{Lemma5.3} implies that 
$\alpha\in\big[e^{-1}(1-q^{-1}), e^{-1}\big]$, and
Lemma \ref{Lemma5.5b} implies that 
$\beta\in\big(1/(eq), 1/(e(q - 1))\big]$ if $e > 2$ 
and $\beta = 1/(e(q \pm 1))$ if $e = 2$.
Thus the proportion lies in the given interval.
\end{proof}

If $n$ is an integer, then we write $v_2(n)$ for the $2$-adic value of 
$n$; so $2^{v_2(n)}$ is the
largest power of $2$ that divides $n$.
\begin{lemma}\label{Lemma5.6} 
If $v_2(m)=v_2(n)$ then $v_2(q^m-1) = v_2(q^n-1)$.
\end{lemma}
\begin{proof} 
 It suffices to consider the case where $m=kn$ and $k$ is odd.
Now $(q^m-1)/(q^n-1)$ is the sum of $k$ powers of $q^n$, and so is odd.
\end{proof}
\begin{lemma}\label{Lemma5.7} If $u<v$ then $v_2(q^{2^u}-1)<v_2(q^{2^v}-1)$, 
and if $u>0$ then $v_2(q^{2^u}-1) =v_2(q^{2^{u+1}}-1)-1$.
\end{lemma}
\begin{proof} 
Observe that $(q^{2^{u+1}}-1)/(q^{2^u}-1)=q^{2^u}+1$ which is even. Now
$v_2(q^{2^u}-1)>1$ if $u>0$. It follows that $v_2(q^{2^u}+1)=1$.
\end{proof}

We now obtain a lower bound to the proportion of 
$g\in\SL(d,q)$ such that $g$ has even order $2n$, and $g^n$
has an eigenspace with specified dimension in a given range. 
\begin{theorem}\label{Theorem5.1}  
Let $d\ge4$.  The proportion of elements of $\SL(d,q)$ that power 
to an involution
whose $-1$-eigenspace lies in the range $(d/3,2d/3]$ is greater than
$$\left({1\over 2d}\right)\left(1-{1\over q}\right).$$
\end{theorem}
\begin{proof}
Let $2^k$ be the unique power of $2$ in the range $(d/3,2d/3]$.  If the
characteristic polynomial of $g\in\SL(d,q)$
has a unique irreducible factor of degree $2^k$, and the order of 
the restriction
of $g$ to the corresponding block of dimension 
$2^k$ has order a multiple of $v_2(q^{2^k}-1)$,
then by the previous two lemmas $g$ will power to an 
involution whose $-1$-eigenspace has
dimension $2^k$.  We prove the theorem by estimating the 
proportion of elements of $\SL(d,q)$
of this type.  

By Lemma \ref{Lemma5.5a}, the proportion of elements 
of $\SL(d,q)$ whose characteristic
polynomials have exactly one irreducible factor of 
degree $e=2^k$ is at least $e^{-1}(1-q^{-1})$ if
$e>d/2$, and, by Lemma \ref{Lemma5.5c}, is at least 
$(e^{-1}-e^{-2})(1-q^{-1})$ if $d/2>e>d/3$.  
If $e=d/2$, then, by Lemma \ref{Lemma5.5d},
the proportion is at least  
$(e^{-1}-e^{-2})(1-q^{-1}) \geq d^{-1}(1-q^{-1})$.
%with equality if $k=1$.  
Thus the proportion is at least $d^{-1}(1-q^{-1})$ in all cases.
  
Suppose now that the characteristic polynomial of $g$ has 
exactly one irreducible factor of degree $2^k$. 
Set $x=v_2(q^{2^k}-1)$. We now prove that the
probability that the order of $g$ is a 
multiple of $2^x$ is greater than $1/2$. 

The action of $g$ on the $g$-invariant block $W$ of dimension $2^k$ 
can be used to map $g$ into
$T=\GF(q^{2^k})\setminus U$, where $U$ is the union of all 
proper subfields of
$\GF(q^{2^k})$ that contain $\GF(q)$: namely, we map $g$ to a 
zero of the characteristic polynomial of $g$ 
restricted to $W$.  This mapping is not unique.
The Galois group of $\GF(q^{2^k})$ over $\GF(q)$ acts regularly on $T$,
and the image of $g$ is determined up to the action of this Galois group.
Since we do not distinguish among elements of the
same orbit of this Galois group on $T$, we may assume that the image of
$g$ is uniformly distributed in $T$.  But exactly half the elements of
$\GF(q^{2^k})^\times$ have order a multiple of $2^x$, and none of the 
elements of $U$
has order a multiple of $2^x$.  Thus more than half of the elements of $T$ 
have order a multiple of $2^x$.  

The result follows, and covers the first row of Table \ref{inv-table}.
\end{proof}

We now deal with the other cases of $\SL(d,q)$ in Theorem \ref{proportions}.
\begin{lemma}
Let $1<e<d - 1$, where $v_2(e)\ne v_2(d-e)$.  Of the elements of $\SL(d,q)$, 
the proportion that are of even order and power to an involution
with an eigenspace of dimension $e$ is at least  
$(1-q^{-1})^2/(e(d - e))$ if $2 < e < d - 2$
and is at least $(1 - q^{-1}) (1 - 2(q + 1)^{-1}) / (e(d - e))$
if $e \in \{2, d - 2\}$.
%$(1-q^{-1})^4/(de)$.
\end{lemma}
\begin{proof}
We look for elements of $\SL(d,q)$ with one irreducible factor of 
degree $e$, and one of degree $d-e$.  The proportion
of elements of $\SL(d,q)$ with this property is 
$\pi := (q-1)\sum_{a\in\GF(q)^\times}\alpha(a)\beta(a)$, 
where $\alpha(a)$ is the proportion of elements of $\GL(e,q)$ that have
an irreducible characteristic polynomial with constant term $a$, and 
$\beta(a)$ is the proportion of elements of $\GL(d-e,q)$
that have an irreducible characteristic polynomial with 
constant term $a^{-1}$.  Lemma \ref{Lemma5.5b} implies that 
$\pi > (1-q^{-1})^2/(e(d - e))$ if $2 < e < d - 2$
and  $\pi > (1 - q^{-1}) (1 - 2(q + 1)^{-1}) / (e(d - e))$
if $e \in \{2, d - 2\}$.
%$\pi>(1-q^{-1})^4/(de)$.  
If $g\in\SL(d,q)$ has this property, and if $u$ and $w$ are eigenvalues 
(in an algebraic closure of $\GF(q)$) of the restriction 
of $g$ to the $e$ and $d-e$-dimensional $g$-invariant subspaces of $V$, then 
$u^{(q^e-1)/(q-1)}=a$ and $w^{(q^{d-e}-1)/(q-1)}=a^{-1}$,
for some $a\in\GF(q)^\times$.  Since $v_2(e)\ne v_2(d-e)$, 
the orders of the restriction of $g$ to these two spaces have
unequal $2$-adic values; and so $g$ powers to an involution whose 
$-1$-eigenspace has dimension $e$ if $v_2(e)>v_2(d-e)$, and 
dimension $d-e$ if $v_2(e)<v_2(d-e)$. 
This covers the second and third entries in Table \ref{inv-table}.
\end{proof}

\subsection{The symplectic and orthogonal groups}
We now turn to the symplectic and orthogonal groups.
If $h(x)\in\GF(q)[x]$ is a
monic polynomial with non-zero constant term, then let
$\tilde{h}(x)\in\GF(q)[x]$ be the monic polynomial whose zeros in the
algebraic closure of $\GF(q)$ are the
inverses of the zeros of $h(x)$. Hence the multiplicity of a zero of
$h(x)$ is the multiplicity of its inverse in $\tilde{h}(x)$, and 
$h(x)\tilde{h}(x)$ is a symmetric  polynomial. 
We call $\tilde{h}$ the \emph{reverse} of $h$.

\begin{lemma}\label{spinor-norm}
Let $g\in\SL(2n,q)$, where $n>1$, have characteristic polynomial 
$f(x)=h(x)\tilde{h}(x)$, where $h(x)\ne\tilde{h}(x)$ is
irreducible.  Let $c$ be the constant term of $h(x)$.
Then $g$ preserves a non-degenerate orthogonal form on the 
underlying space, and every such form
is of $+$ type. 
As an element of the corresponding orthogonal group, 
$g$ has spinor norm $c\bmod U^2$, where
$U$ is the multiplicative group of $\GF(q)$.
\end{lemma}
\begin{proof}
Clearly $g$ preserves an orthogonal form, since $\tilde{f}=f$.  
Choose one such form.
The null spaces of $h(g)$ and $\tilde{h}(g)$ are orthogonal complements, 
and the form restricted
to each of these is the null form, 
as $h(x)\ne\tilde{h}(x)$, so the form is of $+$ type.
The spinor norm of $g$ may be calculated using the 
definition in \cite[p.\ 444]{Zassenhaus}.
This definition gives the spinor norm as the product of two 
terms in $U/U^2$.
The first term is the discriminant of the quadratic form restricted 
to the maximum subspace 
$W$ of $V$ on which $1+g$ acts nilpotently.  Since, by hypothesis,
$-1$ is not an eigenvalue of $g$, this term vanishes.  
The second term is $\det((1+g)/2)$ restricted to the
orthogonal complement of $W$, modulo $U^2$; but here $W=0$.  
Since the dimension is even, the factor of $1/2$ does not
make any contribution.  Let $a$ be a zero of $h(x)$ in $\GF(q^n)$, 
so $1/a$ is a zero of $\tilde{h}(x)$.
Let $N$ denote the norm map from $\GF(q^n)$ to $\GF(q)$.  Thus 
$\det(1+g)=N(1+a)N(1+a^{-1})\,U^2=N(1+a)^2N(a^{-1})\,U^2 =c\,U^2$.
\end{proof}
\begin{corollary}\label{Comp}
The proportion of elements of $\SO^+(2n,q)$, 
for $n>1$, and $q>3$ if $n=2$, whose 
characteristic polynomial is the product of two distinct irreducible 
polynomials,
each the reverse of the other, divided by the proportion of 
such elements in $\Omega^+(2n,q)$, is $1$ if $n$ is odd, 
lies in the interval $(1, 1+2/(q^{n/2}-3))$ if $n$ is a power of $2$,
and in the interval $(1, 1+2/(q^{n/2}-6))$ otherwise.
%and lies in the interval $(1, 1+2/(q^{n/2}+5))$ if $n$ is even.
\end{corollary}
\begin{proof}
Lemma \ref{spinor-norm} implies that the ratio in question
equals the number of irreducible polynomials of degree
$n$ over $\GF(q)$ not equal to their reverses, divided by twice the 
total number of  such polynomials whose constant terms are squares. 

Suppose first that $n$ is odd.  An irreducible polynomial of 
odd degree (greater than 1) cannot be equal to its reverse; so
this ratio is the number of elements of $\GF(q^n)$ that lie in no 
proper subfield containing $\GF(q)$ divided by twice the number of 
such elements
whose norm (under the norm map from $\GF(q^n)$ to $\GF(q)$) is a square.  
But exactly half the non-zero elements of every subfield of $\GF(q^n)$
containing $\GF(q)$ are mapped to squares, since $n$ is odd, and the 
result follows in this case.

Now suppose that $n$ is a power of 2.  The proportion is now changed, since 
every element of $\GF(q^{n/2})$ has square norm, as does
every element of $\GF(q^n)$ whose minimum polynomial is equal to its reverse, 
these latter being the elements of order dividing $q^{n/2}+1$. 
The set of elements of $\GF(q^n)^\times$ that do not lie in $\GF(q^{n/2})$, 
and whose order does not divide $q^{n/2}+1$, is of cardinality 
$q^n-2q^{n/2} + 1$.
Since all elements of $\GF(q^n)$ of non-square norm (that is to say, 
elements that are themselves not squares) lie in this set, the number of
squares in this set is $q^n-2q^{n/2}+1-(q^n-1)/2=(q^n-4q^{n/2}+3)/2$.  
Thus the ratio in question is 
$$q^n-2q^{n/2} + 1:q^n-4q^{n/2}+3=
q^{n/2}-1:q^{n/2}-3=1+2/(q^{n/2}-3).$$
Note that $\Omega(4,3)$ has no elements of the required type, reflecting
the fact that $q^{n/2}-3=0$ if $n=2$ and $q=3$.

Now suppose that $n$ is even, but not a power of two.  
Consider the following
disjoint subsets of $\GF(q^n)^\times$:  
\begin{itemize}
\item $A$ is the subset of elements that lie in
$\GF(q^{n/r})$ for some odd prime $r$;  
\item $B=\GF(q^{n/2})^\times \setminus A$;
%\cap \GF(q^{(n/2)}$.
\item $C$ is the subset of elements of order dividing 
$q^{n/2}+1$ that do not lie in $A\cup B$.
\end{itemize}
Half the elements of $A$ have spinor norm 1, but all the 
elements of $B$ and $C$ have 
spinor norm 1.  Thus the proportion in question is
$${|\GF(q^n)^\times|-|A|-|B|-
|C|\over2({1\over2}|\GF(q^n)^\times|-{1\over2}|A|-|B|-|C|)}
=1+{|B|+|C|\over|\GF(q^n)^\times|-|A|-2|B|-2|C|}.$$
Since $A$, $B$ and $C$ all have fewer than $q^{n/2}$ elements, 
and $n\ge6$, this proportion is less than 
$1 + 2/(q^{n/2}-6)$. The result follows.
\end{proof}

The following result is an analogue of Lemma \ref{Lemma5.3}.
\begin{lemma}\label{Lemma5.8} 
Let $G$ be one of the groups 
$\Sp(2n,q)$, $\SO^+(2n,q)$, $\SO^-(2n,q)$, $\SO(2n+1,q)$.
Let $n\ge m>n/2$ where $n\ge2$, and $n>m$ if $G=\SO^-(2n,q)$.
The proportion of elements of
$G$ whose characteristic polynomial has an irreducible factor of degree $m$
that is not equal to its reverse 
lies in the interval 
$$\bigl(m^{-1}(1-q^{-1})/2-q^{-\lceil m/2\rceil}/2,m^{-1}/2\bigr),$$
is independent of $n$, and hence is strictly positive.
If $q=3$ and $m=2$, then the proportion is $1/16$.
\ADD
\end{lemma}
\begin{proof}
Let $g\in G$ act on the natural module $V$, and let
$h(x)$ be an irreducible factor of degree $m$
of the characteristic polynomial
$f(x)$ of $g$ not equal to its reverse. 
Let $V_0$ be the kernel of $h(g)$. Since    
$h(x)\ne \tilde{h}(x)$, and $g$ acts irreducibly on $V_0$,
it follows that $V_0$ is totally isotropic. Also
$\tilde{h}(x)$ is a factor of $f(x)$ since $f(x)=\tilde{f}(x)$,  
and if  $V_1$ is the kernel of
$\tilde{h}(g)$ then $V_1$ is totally isotropic. Since $h(x)$ and
$\tilde{h}(x)$ divide $f(x)$ with multiplicity 1, $V_0$ and $V_1$ are
uniquely determined, and the form restricted to $V_2=V_0\oplus V_1$ is
non-degenerate.

Thus the number of possibilities
for $g$ is the product $\ell_1\ell_2\ell_3\ell_4\ell_5/2$, where $\ell_1$ is
the number of choices for $V_2$, $\ell_2$ is the number of choices
for $V_0$ given $V_2$, $\ell_3$ is the number of irreducible monic 
polynomials $h(x)$ of degree $m$ over $\GF(q)$ such that $h(x)\ne
\tilde{h}(x)$, $\ell_4$ is the number of elements of $\GL(m,q)$ with a
given irreducible characteristic polynomial, and $\ell_5$ is the order of
$\SX(V_2^\perp)$. The factor $1/2$ in the above expression arises 
since the symmetry between $h(x)$ and $\tilde{h}(x)$ ensures 
that every such element $g$ is counted twice.
In more detail: 
\begin{eqnarray*}
\ell_1 & = & \vert \GX(V)\vert/\vert\GX(V_2)\times\GX(V_2^\perp)\vert \\
\ell_2 & = & \vert\GX(V_2)\vert/\vert\GL(V_0)\vert \\
\ell_3 & \sim & q^m/m \\
\ell_4 & = & \vert\GL(V_0)\vert/(q^m-1) \\
\ell_5 & = & \vert\SX(V_2^\perp)\vert. 
\end{eqnarray*}
%{\bf A more precise estimate for $\ell_3$ is given later. IS THIS CORRECT?}

These results are obtained as follows. By Witt's Theorem 
(see Theorem \ref{Witt}), $\GX(V)$ acts transitively
on the subspaces of $V$ that are isometric to $V_2$, and the normaliser of
$V_2$ in $\GX(V)$ is $\GX(V_2)\times \GX(V_2^\perp)$.  
Similarly $\GX(V_2)$ acts
transitively on the maximal totally isotropic subspaces of $V_2$, and the
normaliser of $V_0$ in $\GX(V_2)$ is isomorphic to $\GL(V_0)$.  Thus $\ell_1$
and $\ell_2$ are as stated.
We observe that $\ell_3$ is the number of orbits of the Galois group of
$\GF(q^m)$ over $\GF(q)$ acting on those $a \in \GF(q^m)$ that 
do not lie
in a proper subfield containing $\GF(q)$, and have the property that 
the orbit of $a$ does not contain $a^{-1}$.  This last condition is 
equivalent to the statement that $h(x)\ne \tilde{h}(x)$.   
(If $h(x)$ is irreducible and of degree $m$,
then $h(x)=\tilde{h}(x)$ if and only if
$m$ is even, and $a^{-1}=a^{q^{m/2}}$
for every zero $a$ of $h(x)$ in $\GF(q^m)$. This could
be used to obtain an exact formula for $\ell_3$.)
The estimate for $k(a)$ in Lemma \ref{monic} becomes an estimate for $\ell_3$
once we subtract (at least from the lower bound) the 
number of monic irreducible
symmetric polynomials of degree $m$ over $\GF(q)$.  
The number of monic symmetric
polynomials of degree $m$ over $\GF(q)$ is $q^{\lfloor m/2\rfloor}$,
and at least one of these vanishes at 1, and hence is reducible.
Thus $m^{-1}(q^m-1)>\ell_3\ge m^{-1}q^m(1-q^{-1})-q^{\lfloor m/2\rfloor}+1$.
The product of the $\ell_i$ is
$\ell_3\vert G\vert/(q^m-1)$, and the result follows. 
\end{proof}

The detail of adding 1 to the lower bound, proved 
by observing that at least
one of these polynomials is reducible, ensures that the 
stated lower bound is strictly positive in all cases: it
is the precise value, namely 1, 
when $q=3$ and $m=2$, the polynomial in question being $x^2+x+2$.

\begin{lemma}\label{Lemma5.9} 
Let $G$ be as in the previous lemma, and let $m\in(n/3,n/2]$, and $m<n/2$
if $G$ is $\SO^-(2n,q)$. 
Let $S$ denote the set of elements of $G$ whose characteristic polynomial
has exactly two distinct irreducible factors of degree $m$, each the 
reverse of the other.
Then 
$${\vert S\vert\over\vert G\vert}=
{1\over2}{\ell_3\over q^m-1} - {1\over4}\left({\ell_3\over q^m-1}\right)^2$$
where $m^{-1}(q^m-1)>\ell_3\ge m^{-1}q^m(1-q^{-1})-q^{\lfloor m/2\rfloor}+1$.
In particular,
$${\vert S\vert\over\vert G\vert}=
\left({1\over2m}-{1\over4m^2}\right)(1+\Oh(1/q)).$$
If $G=\SO^-(2n,q)$ and $m = n/2$, so $n$ is even, then 
$${\vert S\vert\over\vert G\vert}=
{1\over2}{\ell_3\over q^m-1} = 
\left({1\over2m}\right)(1+\Oh(1/q)).$$
\end{lemma}
\begin{proof} 
The proof is similar to that of Lemma \ref{Lemma5.4}.  
The case $G=\SO^-(2n,q)$
and $m/2$ is exceptional: $G$ cannot have two 
pairs of distinct irreducible
mutually reverse factors of degree $n/2$.
\end{proof}

\begin{lemma}\label{Omega}  
If $m < n$, then 
Lemmas $\ref{Lemma5.8}$ and $\ref{Lemma5.9}$ 
apply essentially unchanged when $\SO^{\pm}(2n, q)$
is replaced by $\Omega^{\pm}(2n,q)$.
\end{lemma}
\begin{proof}  
Suppose first that $m>n/2$.
In the notation of Lemma \ref{Lemma5.8} let $G=\Omega(V)$.  The restriction
of $g\in G$ to $V_2$ and to $V_2^\perp$ must have equal spinor norms.
But exactly half the elements of $\SO(V_2^\perp)$ have spinor norm $1$,
so the proportion of elements $g$ satisfying the required condition is
exactly the same in $\Omega(V)$ as in $\SO(V)$.
Similarly, the proportions are exactly equal if $n/3<m<n/2$.  

This leaves the case $m/2$, so $n$ is even.
If $G = \Omega^-(2n, q)$, then 
the above argument still applies, for the same
reason that $\SO^{-}(2n, q)$ was an exceptional
case in Lemma \ref{Lemma5.9}.
If $G = \Omega^+(2n,q)$, then 
we need to exclude from our count those elements of 
$\Omega^+(2n,q)$ whose restriction
to $V_2^\perp$ has a characteristic polynomial that is 
the product of two distinct irreducible factors,
each the reverse of the other.  
Corollary \ref{Comp} implies that the required
proportion is obtained by dividing 
the proportion given in Lemma \ref{Lemma5.9} by 
a factor in 
the interval $(1, 1+2/(q^{n/2}-3))$ if $n$ is a power of $2$,
and in $(1, 1+2/(q^{n/2}-6))$ otherwise.
\end{proof}
%We now need to exclude from our count those elements of 
%$\Omega^+(2n,q)$ whose restriction
%to $V_2^\perp$ has a characteristic polynomial that is 
%the product of two distinct irreducible factors,
%each the reverse of the other.  
%The result follows from Corollary \ref{Comp}.

We now obtain the analogue of Theorem \ref{Theorem5.1}.
\begin{theorem}\label{Theorem5.2}  Let $G$ be one of the 
groups $\SP(2n,q)$, $\SO^{\pm}(2n,q)$, $\SO(2n+1,q)$,
$\Omega^{\pm}(2n,q)$, $\Omega(2n+1,q)$, where $n\ge 3$.
The proportion of elements of $G$ that power to an 
involution with $-1$-eigenspace having dimension
in the range $(2n/3,4n/3]$ is greater than
$m^{-1} (1 - q^{-1}) / 4 - q^{-\lceil m/2 \rceil /4}$
where $m = \lfloor 2n / 3 \rfloor$, and is always positive.
If $G$ is orthogonal, then the $-1$-eigenspace of the 
involution supports a form of $+$ type.
%, since the form restricted to the kernel of $h(g)$ 
%(or of $\tilde{h}(g)$) is null.  
%$(3/(8n))(1 + \Oh(1/q))$, and is always positive.
\end{theorem}
\begin{proof} 
Using Lemmas \ref{Lemma5.8}, \ref{Lemma5.9} and \ref{Omega}, 
the proof is similar to that of Theorem \ref{Theorem5.1}.   

Let $2^k$ be the unique power of $2$ in the range $(2n/3,4n/3]$, 
so $k\ge2$. We look for an element $g$ of $G$ whose characteristic
polynomial has a unique pair of factors $h(x)$ and $\tilde{h}(x)$, 
where $h(x)\ne\tilde{h}(x)$ is irreducible of degree $2^{k-1}$, and
consider the probability that $g$ will power to an involution whose 
$-1$-eigenspace has
dimension $2^k$.  If $U$ is the null space of $h(g)\tilde{h}(g)$,
then the restriction of $g$ to $U$ has order 
dividing $q^{2^k}-1$, and so, with probability slightly greater than 
$1/2$, the $2$-adic value of the order of $g$
restricted to $U$ will be $v_2(q^{2^k}-1)$.  Now $V$, regarded as a 
module for $\GF(q)[C]$, where $C$ is the cyclic group generated
by $g$, has a series $V=V_1>V_2>\cdots$, where the characteristic polynomial 
of $g$ acting on $V_i/V_{i+1}$ is either
the product of two distinct irreducible factors $h_i(x)$ and $\tilde{h_i}(x)$,
or an irreducible polynomial $f_i(x)$ with $f_i(x)=\tilde{f_i}(x)$. 
Let $n_i$ denote the dimension of  $V_i/V_{i+1}$.  In the former 
case $n_i$ is even and the order of $g$ acting on 
$V_i/V_{i+1}$ divides $q^{n_i/2}-1$.
Also, by assumption, $n_i\ne 2^k$, and so $v_2(n_i)<v_2(2^k)$, and 
$v_2(q^{n_1/2}-1)<v_2(q^{2^{k-1}}-1)$.  In the latter case $n_i$ is even
or $n_i=1$.  If $n_i$ is even, then the order of $g$ acting 
on $V_i/V_{i+1}$ divides $q^{n_i/2}+1$, and 
if $n_i=1$ this order is $\pm1$.  
Hence, in any case, the $2$-adic value of this order is 
less than $v_2(q^{2^{k-1}}-1)$.  It follows that $g$ will power 
to an involution with $-1$-eigenspace equal to $U$ if the order of 
the restriction of $g$ to $U$ has $2$-adic value equal to 
$v_2(q^{2^{k-1}}-1)$.  The proportion
of elements $g$ of $G$ satisfying the conditions now imposed 
on $g$ may be estimated using Lemmas \ref {Lemma5.8} and
\ref{Lemma5.9}.  The proportion given by these lemmas, 
for $m\in(n/3,2n/3]$, is least when $m$ is the 
integral part of $2n/3$.
Thus the proportion of elements $g$ of $G$ satisfying all the 
conditions imposed on $g$ is greater 
than $m^{-1}(1-q^{-1})/4-q^{-\lceil m/2\rceil}/4$
where $m=\lfloor 2n/3\rfloor$.

Note that the proportion of elements satisfying the conditions 
imposed on $g$ if $G=\SO(V)$ is exactly the same
as the proportion if $G=\Omega(V)$.  The restriction of 
$g$ to $U^\perp$ must be chosen to have the same 
spinor norm as the restriction of $g$ to
$U$, and half the elements of $\SO(U^\perp)$ will have this property.

If $G$ is orthogonal, then the $-1$-eigenspace of the 
involution obtained by powering $g$ supports a form of
$+$ type, since the form restricted to the kernel of $h(g)$, 
or of $\tilde{h}(g)$, is null.  

Thus the entries in Table \ref{inv-table} 
for orthogonal and symplectic groups 
that require $e_-$ to lie in the range $(d/3,2d/3]$ are valid.
\end{proof}

Observe that the dimension of $U$ in the proof 
is a power of $2$, and is at least $4$.  Thus the theorem is 
compatible with the fact that
$\Omega^\epsilon(2n,q)$ does not have an involution whose 
$-1$-eigenspace is an odd multiple of $2$ if both 
$q\equiv3\bmod4$ and $\epsilon=+$,
or if both $q\equiv1\bmod4$ and $\epsilon=-$.  

%In the case of orthogonal groups, the $-1$-eigenspace of the 
%involution obtained by powering $g$ supports a form of
%$+$ type, since the form restricted to the kernel of $h(g)$ 
%(or of $\tilde{h}(g)$) is null.  
%Thus the entries in Table \ref{inv-table} 
%for orthogonal and symplectic groups 
%that require $e_-$ to lie in the range $(d/3,2d/3]$ are valid.

\begin{theorem}
The remaining entries in Table $\ref{inv-table}$ 
for orthogonal and symplectic groups are valid.
\end{theorem}
\begin{proof}
Consider first the case where $d-e$ is even and $d>2e$. 
Let $S$ be the set of elements of such a group $G$ whose 
characteristic polynomial contains two distinct irreducible factors $h(x)$ 
and $\tilde{h}(x)$, where $\tilde{h}(x)$, the reverse of $h(x)$, is not
equal to $h(x)$, and where $h$ has degree $(d-e)/2$.  
% EOB -- Last condition explains requirement that d - e is even.
Lemma~\ref{Lemma5.8} implies that 
the proportion of elements of $G$ with this property is
$(1/(d-e))(1+\Oh(1/q))$ and is positive for all values of $q$.  
It is a straightforward, if tedious, exercise 
to use the explicit lower bound given there 
to obtain explicit bounds for the proportions stated here.

It remains to estimate the probability that the 
$2$-adic value of the order of such an element $g$ 
restricted to the null space
$U$ of $h(g)\tilde{h}(g)$ is greater than the $2$-adic value of 
the order of its 
restriction to $U^\perp$, since in this case $g$ will power to an involution
with $-1$-eigenspace $F=U$ and $+1$-eigenspace $E=F^\perp$.
If the form is orthogonal, then the restriction of the form to 
$F$ is either required or permitted to be of $+$ type.

If $G=\Omega(V)$, then the spinor norms of $g$ restricted to $E$ and to $F$ 
must be equal.  It is easy to see that
the proportion of elements of $G$ that satisfy the conditions 
imposed on $g$ is higher when $g$ is required to have spinor norm
$-1$ in both $E$ and $F$ than when $g$ is required to have spinor 
norm $+1$ in these spaces. This is because the condition that $h$ 
be irreducible
and not equal to its reverse excludes a higher proportion of polynomials 
whose constant terms are squares than of general polynomials;
more significantly, the $2$-adic value
of the order of such an element (restricted to $F$) takes its 
maximum value when the constant term of $h(x)$ is not a square.

Thus, if $G=\Omega(V)$, then we define $T$ to be the subset of $S$ 
consisting of elements that act on $E$ and on $F$ with spinor norm $+1$, 
and estimate the proportion of elements of $T$ that 
power to a suitable involution.  

Note that the order of the restriction of $g$ to $F$ has $2$-adic value at 
most $v_2(q^{(d-e)/2}-1)$, and at most $v_2(q^{(d-e)/2}-1)-1$ 
in the orthogonal case if $g$ restricted to $F$ has spinor norm $+1$.  
Moreover, the proportion of elements of $S$ or of $T$ for which this value
is achieved is greater than $1/2$.

Let $\pi$ denote a lower bound to the probability that the 
$2$-adic value of the order
of the restriction of a random element of $S$ (or of $T$ if $G=\Omega(V)$)
to $F$ exceeds the $2$-adic
value of its restriction to $E$, so that the proportion of elements of $G$ that power to an involution as required is greater than
$(\pi/(d-e))(1+\Oh(1/q))$, a bound we often replace with 
$(\pi/d)(1+\Oh(1/q))$.

We now consider the individual cases in Table $\ref{inv-table}$, 
where we use ad-hoc arguments to handle the exceptional cases. 
\begin{itemize}
\item $G=\Sp(d,q)$, $d\equiv2\bmod4$; $e=2$.

Now $v_2(q^{(d-e)/2}-1)\ge v_2(q^2-1)$, which is greater 
than the $2$-adic value of the restriction of $g$ to $E$; so $\pi>1/2$.

\item 
$G=\Omega^+(d,q)$, $d>8$, $q\equiv3\bmod4$, $d\bmod8\ne0$; 
$e=d\bmod8$, so $e<8$; $E$ and $F$ of $+$ type.

Now $v_2(q^{(d-e)/2}-1)-1 \ge v_2(q^4-1)-1$, and this is greater than 
the $2$-adic value of the restriction of $g$ to $E$; so $\pi>1/2$.

\item 
$G=\Omega^-(d,q)$, $q\equiv1\bmod4$, $d\ge6$;  $e=4$; $F$ of $+$ type.

Suppose first that $d>8$.
Since $\Omega^-(4,q)\cong\PSL(2,q^2)$, the proportion
of elements of $\Omega^-(4,q)$ of odd order 
is greater than $1/2$ (see \cite[p.\ 288]{Dornhoff}),  
and so $\pi>1/4$.

If $d\le8$, then our assumption that $d>2e$ fails. 

Suppose that $d=6$.  Consider elements $g$ of $G$ whose characteristic
polynomials factorise as $f(x)=h(x)(x-\alpha)(x-\alpha^{-1})$, where
$h(x)=\tilde{h}(x)$ is irreducible of degree 4, and the $2$-adic value
of the multiplicative order of $\alpha$ is greater than the 
$2$-adic value of the order of $g$ restricted to the kernel of $h(g)$.
This latter order divides $q^2+1$, and hence has $2$-adic value at most
$1$.  Then the involution that is a power of $g$ has $-1$-eigenspace
the sum of the $\alpha$ and $\alpha^{-1}$ eigenspaces of $g$, which is,
as required, of dimension 2 and of $+$ type.  The proportion of
elements of $\Omega^-(6,q)$ of this type, ignoring the restriction
on the order of $\alpha$, but excluding the cases $\alpha=\pm1$,
is $q-3:4(q+1)$, and so, allowing for this restriction, the
proportion of elements of $G$ of the required type is greater than
$1/8 + \Oh(1/q)$.

Now suppose that $d=8$. By the exceptional case of 
Lemma \ref{Omega}, the proportion of elements $g$ 
of $G$ whose characteristic polynomial
has exactly two irreducible factors $h(x)$ and $\tilde{h}(x)$ of 
degree 2 that are the reverse of each other is $1/4+\Oh(1/q)$.
Let $g$ be a random element of $T$.  
Now $g$ lies in $\SO(F)\times\SO(E)$, where $F$ is the null space 
of $h(g)\tilde{h}(g)$, and the probability that $g$ powers to a
suitable involution is greater than $1/2$, since the largest possible value 
for the $2$-adic value of the order of $g$ restricted to $F$ is
greater than the corresponding value  for $E$.  This gives the 
required proportion of elements of $G$ as greater than $1/8+\Oh(1/q)$.

\item 
$G=\Omega^-(d,q)$, $q\equiv3\bmod4$, $d\equiv0\bmod4$; $e_{+}=4$; 
$F$ is of $+$ type.

This can be dealt with exactly as the previous case.

\item 
$G=\Omega^-(d,q)$, $q\equiv3\bmod4$, $d\equiv2\bmod4$; $e=6$; 
$F$ of $+$ type. 

Assume first that $d>2e$.
The proportion of elements of $\Omega^-(6,q)$ of order not a multiple 
of $4$ is easily seen to be at least $1/2+\Oh(1/q)$.
But $v_2(q^{(d-6)/2}-1)-1\ge v_2(q^2-1)-1\ge2$, so $\pi>1/4+\Oh(1/q)$.
Now suppose that $d\le2e$, so $d=10$.  We argue 
as in the case $d>10$, but use Lemma~\ref{Lemma5.9} rather than
Lemma~\ref{Lemma5.8}.  This replaces the factor $1/(d-e)$ by 
$1/(d-e)-1/(d-e)^2=1/4-1/16$.  
Since we simplify our estimates, replacing $1/(d-e)$ by $1/d$, this 
value is within our general bounds.

\item 
$\Omega^-(6,q)$, $q\equiv3\bmod4$; $e=2$; $F$ of $+$ type.

$v_2(q^{(d-e)/2}-1)-1=v_2(q^2-1)-1\ge2$.  The restriction of $g$ to
$E$ has order dividing $(q+1)/2$; so $\pi>1/2$.

\item 
$G=\Omega^0(5,q)$, $q\equiv3\bmod4$, $q>3$; $e_+=1$; $F$ of $+$ type.

$v_2(q^{(d-e)/2}-1)-1=v_2(q^2-1)-1\ge2$, so $\pi>3/4$.
\item 
$G=\Omega^0(d,q)$, $q\equiv1\bmod4$ or $d\equiv3\bmod4$; 
$e=3$; $F$ of $+$ type.

Now $v_2(q^{(d-e)/2}-1)-1$ is at least $v_2(q^2-1)-1$ if $d\equiv3\bmod4$,
and is at least $v_2(q-1)-1$ if $q\equiv1\bmod4$, and hence is at least 1 in
either case.  The proportion of elements of $\Omega(3,q)\cong\PSL(2,q)$ of
odd order is greater than $1/2$; so $\pi>1/4+\Oh(1/q)$.

\item 
$G=\Omega^0(d,q)$, $q\equiv3\bmod4$, $d\equiv1\bmod4$; 
$e=5$; $F$ of $+$ type.

Suppose first that $d>2e$.  Then $v_2(q^{(d-e)/2}-1)-1\ge v_2(q^2-1)-1\ge2$.
Elements of $\SO(E)$ of order not a multiple of $4$ include
those whose characteristic polynomials are of the form $(x-1)f(x)$,
where $f(x)$ is irreducible and $f(x)=\tilde{f}(x)$.  
Such elements correspond to
equivalence classes, under the action of the group generated 
by the Frobenius map, of
elements of $\GF(q^4)$ that do not lie in $\GF(q^2)$, and that 
are of order dividing
$q^2+1$.  Such elements have centralisers in $\SO(E)$ of order $q^2+1$, so
the number of such elements is $|\SO(E)|(q^2-1)/(4(q^2+1))$.
Thus the proportion of elements of $\SO(E)$ of order not a multiple of $4$ is
at least $1/4 + \Oh(1/q^2)$, and the same applies to $\Omega(E)$;
so $\pi>1/8+\Oh(1/q)$.

This leaves the case $d=9$, $e=5$.  Again we proceed as when
$d>2e$, but use Lemma~\ref{Lemma5.9} rather than Lemma~\ref{Lemma5.8}.  
This replaces the factor $1/(d-e)$ by $1/(d-e)-1/(d-e)^2=1/4-1/9$.  Since
this is greater than $1/9$, our stated lower bound holds.

\item 
$G=\SO^-(d,q)$; $e=4$; $F$ of $+$ type.  
 
If $q\equiv1\bmod4$ or $d\equiv0\bmod 4$, then the proportion is  
the same as for $\Omega^-(d,q)$, since $\SO^-(d,q)\cong\Omega^-(d,q)
\times C_2$ in this case.

Now consider $q \equiv 3\bmod 4$.
If $d=6$, then the analysis is similar to that for $\Omega^-(d,q)$ when
$q\equiv 1\bmod 4$ and $e=4$.  
The order of $\alpha$ may now be a multiple of $2$,
but not of $4$. Hence the probability of the order condition being
satisfied is now slightly greater then $1/4$, and the proportion
of elements of the type required is now greater than $1/16 +\Oh(1/q)$.
The case $d=8$ is covered by Lemma~\ref{Lemma5.9}.
If $d>8$ then $v_2(q^{(d-e)/2}-1)\ge v_2(q^2-1)\ge3$, so $\pi>3/8$.

\item 
$G=\SO^0(d,q)$; $e=3$; $F$ is of $+$ type.

Assume $d > 5$.  Since $v_2(q^{(d-e)/2}-1)\ge v_2(q-1)\ge1$, and the
proportion of elements of $\SO(3,q)$ of odd order is greater than
$1/4$, it follows that $\pi>1/8$.

If $d=5$, then we look for elements whose characteristic polynomial 
factorises
as $(x-\alpha)(x-\alpha^{-1})(x-1)h(x)$, where $h(x)=\tilde{h}(x)$ is
irreducible.  The factor $(x-1)h(x)$ is the characteristic polynomial
of an element of $\SO(3,q)$.  We impose the condition $\alpha\ne \pm1$.
The proportion of such elements in $\SO(5,q)$ is 
$((q-3)/2)((q^2-1)/2):(q-1)(q^2+1)=1/4+\Oh(1/q)$.
\end{itemize}
\vspace*{-1cm}
\end{proof}

\subsection{The unitary groups}
We finally turn to the unitary groups. 
If $h(x)\in\GF(q^2)[x]$ is a  monic polynomial 
with non-zero constant term,
then define $\hat{h}(x)$ to be 
the monic polynomial obtained from $\tilde{h}(x)$ by raising each
coefficient to the power $q$.  
We call $\hat{h}(x)$ the \emph{hermitian reverse} of $h(x)$.

\begin{lemma}\label{lhat}
Let $\hat{\ell}_3:=\hat{\ell}_3(q,m)$ denote the number of irreducible 
monic polynomials 
%with non-zero constant term and 
of degree $m$ over $\GF(q^2)$
that are not equal to their hermitian reverse. Then
$${q^{2m}-1\over m}>\hat{\ell}_3 > {q^{2m}(1-q^{-1})\over m}.$$
\end{lemma}
\begin{proof}
If $m$ is even, then no irreducible monic polynomial is equal 
to its hermitian reverse, so
$(q^{2m}-1)/m>\hat{\ell}_3\ge q^{2m}(1-q^{-2})/m$
by Lemma \ref{monic}.
If $m$ is odd, then the number of monic polynomials 
over $\GF(q^2)$ of degree $m$ (reducible or not) that are equal
to their hermitian reverse is $q^{m-1}(q+1)$, 
so $\hat{\ell}_3>q^{2m}(1-q^{-2})/m-q^{m-1}(q+1)
\ge q^{2m}(1-q^{-1})/m$.
\end{proof}

If $m$ is odd, then the irreducible monic polynomials
over $\GF(q^2)$ of degree $m$ that are equal to their
hermitian reverse define elements of $\GF(q^{2m})$
that lie in no proper subfield containing $\GF(q^2)$ and 
have order dividing $q^m + 1$. This could
be used to obtain a precise formula for $\hat{l}_3$.

\begin{lemma}\label{Unitarya} 
Let $G=\SU(d,q)$, and $m>d/4$.
The proportion of elements of
$G$ whose characteristic polynomial has an irreducible factor of degree $m$
that is not equal to its hermitian reverse lies in the interval 
$\big((1-q^{-1})/(2m), 1/(2m)\big)$, and 
is independent of $d$. \ADD
\end{lemma}
\begin{proof}
The proof is almost identical to that of Lemma \ref{Lemma5.8}.
The proportion is $\hat{\ell}_3/(2(q^{2m}-1))$, and the result then follows
from Lemma \ref{lhat}.
\end{proof}

\begin{lemma}\label{Unitaryb} 
Let $G=\SU(d,q)$, and let $m\in(d/6,d/4)$. 
Let $S$ denote the set of elements of $G$ 
whose characteristic polynomial
has exactly two distinct irreducible factors of degree $m$, each the 
hermitian reverse of the other.  Then 
$${\vert S\vert\over\vert G\vert}={1\over2}{\hat{\ell}_3\over q^{2m}-1} - 
{1\over4}\left({\hat{\ell}_3\over q^{2m}-1}\right)^2$$
where $(q^{2m}-1)/m>\hat{\ell}_3\ge q^{2m}(1-q^{-2})/m$.
In particular,
$${\vert S\vert\over\vert G\vert}=
\left({1\over2m}-{1\over4m^2}\right)(1+\Oh(1/q^2)).$$
\end{lemma}

\begin{lemma}\label{Unitaryc}
Let $G=\SU(4m,q)$.
The proportion of elements of $G$ whose characteristic polynomial
has exactly two distinct irreducible factors of degree $m$, each the 
hermitian reverse of the other, lies in the interval 
$$\left({1\over2}{\hat{\ell}_3\over q^{2m}-1} - 
{1\over4}\left({\hat{\ell}_3\over q^{2m}-1}\right)^2(1-q^{-2}),
{1\over2}{\hat{\ell}_3\over q^{2m}-1} - 
{1\over4}\left({\hat{\ell}_3\over q^{2m}-1}\right)^2(1-q^{-2})^{-1}\right).$$
\end{lemma}
\begin{proof}
%The case of the unitary groups of dimension $4m$ is exceptional: 
This case is exceptional:
the constant term of $h(x)\hat{h}(x)$, where $h(x)$
is the irreducible factor in question, need not be 1. 
Consider the excluded case when $g\in G$ has a characteristic 
polynomial with two pairs of hermitian reverse factors
of degree $m$: there is a restriction on the constant terms of 
these polynomials, since $G$ is the special unitary group.
By Lemma \ref{monic-norm}, % ({\it c.f.} Lemma \ref{monic}), 
this restriction multiplies the number of excluded cases by a factor that lies
between $1-q^{-2}$ and $(1-q^{-2})^{-1}$.
\end{proof}

\begin{theorem}\label{Theorem5.3} 
Let $d\ge3$. The proportion of elements of 
$\SU(d,q)$ that power to an involution
whose $-1$-eigenspace has dimension in the range
$(d/3,2d/3]$ is at least $(3/(4d))(1 - q^{-1})$.
\end{theorem}
\begin{proof} 
Using the three previous lemmas, the analysis is similar 
to that for the symplectic and orthogonal groups.
\end{proof}

There remain two unitary cases to consider.
\begin{itemize}
\item $G=\SU(d,q)$, $d\equiv2\bmod4$; $e=2$.

The proportion of elements of $G$ whose characteristic polynomial 
has an irreducible factor of degree $(d-2)/2$ not
equal to its hermitian reverse lies in the interval 
$\big((1-q^{-1})/(d-2), 1/(d-2)\big)$, by Lemma \ref{Unitarya}.
Such an element will power to an involution as required with 
probability greater than $1/2$: observe
$v_2(q^{(d-2)}-1)> v_2(q^2 - 1)$ since $d\equiv2\bmod4$.  

\item 
$G=\SU(d,q)$, $d$ odd; $e=3$.

The proportion of elements $g$ of $G$ whose characteristic
polynomial has two distinct irreducible hermitian reverse factors 
$h(x)$ and $\hat{h}(x)$,
each of degree $(d-3)/2$, and a third irreducible factor $k(x)$ of degree $3$,
is $(1/(3d))(1+\Oh(1/q))$.
Let $E$ denote the null space of $k(g)$, and $F$ the null space 
of $h(g)\hat{h}(g)$.
The order of $g$ restricted to $E$ divides 
$q^3+1$, and hence is an odd multiple of $q+1$.  The order of $g$
restricted to $F$ divides $q^{d-3}-1$, and hence divides $q^2-1$.
Thus the probability that $g$ will power to an involution as required is
greater than $1/2$.  Thus the proportion of elements of $G$ with the
required property is at least $(1/(6d))(1+\Oh(1/q))$.
\end{itemize}
This completes the proof of Theorem \ref{proportions}.

\subsection{Constructing an involution}
We now analyse the cost of determining whether 
a given matrix powers to a suitable involution.

\begin{lemma}\label{Eigenspace}
Given $g\in\GL(d,q)$, one can determine whether or not $g$ is of even order,
and in the positive case determine an integer $n$ such that $g^n$ is an
involution, and find bases for the eigenspaces of this involution,
using a Las Vegas algorithm having complexity
$\Oh(d^3\log d + d^2 \log d\log \log d\log q)$ 
measured in field operations.
\end{lemma}
\begin{proof}
The characteristic polynomial $f(t)$ of $g$ can be computed 
in $\Oh(d^3 \log d)$ field operations. 
It can be factorised as $f(t)=\prod_{i=1}^mf_i(t)^{n_i}$, where the
$f_i(t)$ are distinct monic irreducible polynomials
in $\Oh(d^2 \log d \log\log d\log q)$ field operations. 

Let the $2$-part of the order of $t+(f_i(t))$ in the group 
of units of $\GF(q)[t]/(f_i(t))$ be $2^{x_i}$. 
To compute $x_i$, we first raise $t+(f_i(t))$ to the power $a_i$, where
$a_i$ is the odd part of $q^{\deg (f_i)-1}$; now 
$x_i$ is the number of times that
the resulting field element needs to be squared to give rise to the identity.
Computing $t^k$ in any ring requires at most $2\log k$
ring operations, and $k$ is at most $q^d$, so this can be carried out in 
$\Oh(d^2\log d \log \log d \log q)$ field operations.
%(see \cite[Section 8.3]{vzg}).

If $x := \max_i(x_i) = 0$, then $g$ has odd order.
Otherwise, $n$ is 
$2^{x-1} \cdot \prod_i a_i  \cdot p^{\max_i{\{n_i\}}}$, where 
$\GF(q)$ has characteristic $p$.
Let $I=\{i:x_i=x\}$. 
Clearly the dimension of the $-1$-eigenspace
of $g^n$ is $\sum_{i\in I}n_id_i$, where $d_i$ is the degree of $f_i(t)$.  

To obtain the bases for the eigenspaces, 
we compute $g^n$ using the algorithm of Lemma 
\ref{compute-power}, and evaluate the appropriate nullspaces.
The claim follows.
\end{proof}

Observe that we learn the dimension of the $-1$-eigenspace of 
$g^n$ without evaluating $g^n$, and so can 
decide if $g^n$ is a strong or suitable involution without
computing its eigenspaces.

We now summarise the results of Section \ref{Involution}.
\begin{theorem}\label{Corollary5.1}
There is a Las Vegas algorithm that takes as input a generating set
$X$ for $G$, where $G$ is the first entry in a row in Table $\ref{inv-table}$,
and returns an \SLP\ in $X$ for $g\in G$
that powers to an involution whose eigenspaces satisfy the conditions
imposed in the second entry, together with bases for these eigenspaces,
and an integer $n$ such that $g^n$ is the involution in question. 
This algorithm takes 
$\Oh(d(\xi+d^3 \log d + d^2 \log d\log \log d\log q))$ 
field operations.
\end{theorem}

\subsection{Two related results on orthogonal groups}
We conclude with two results of a flavour similar to 
the other results of Section \ref{Involution}.
These guarantee
that the search in Step 3 of {\tt OneOmegaPlus3}
and {\tt OneOmegaMinus3} terminates 
in $\Oh(n)$ random selections.

\begin{lemma}\label{extra}
If $q\equiv3\bmod4$, then  the proportion of elements 
of $\SO^+(2n,q)$ that are
of twice odd order and have spinor norm $-1$ is $1/8$ if $n=2$, and
is greater than 
$(2n-2)^{-1}(1-q^{-1})/2-q^{-n/2}/4$ if $n>2$.
\end{lemma}
\begin{proof}
Suppose first that $n=2$.  Let $D$ denote the subgroup 
of $\GL(2,q)$ consisting
of elements of determinant $\pm1$.  
Then $\SO^+(4,q)$ is the section of $D\times D$
obtained by taking the subgroup consisting of pairs of 
elements with equal determinants,
and amalgamating the centres;  
$\Omega^+(4,q)$ is obtained from elements of 
$D\times D$ whose entries in each 
factor have determinant 1. Thus an element of 
$\SO^+(4,q)$ of twice odd order and
spinor norm $-1$ arises from an element 
$(g_1,g_2)$ of $D\times D$ where each of $g_1$ and $g_2$
has order a multiple of 4 and dividing $2(q-1)$.  The 
proportion of elements of
$D$ that satisfy this condition is $1/4$, and the result follows.

Now suppose that $n>2$.
By Lemma \ref{Lemma5.8}, the proportion $\pi$ of elements 
$g$ of $\SO^+(2n,q)$ whose
characteristic polynomial has an irreducible factor 
$h(x)\ne\tilde{h}(x)$ of degree $n-1$ 
lies in the interval $\big((2n-2)^{-1}(1-q^{-1})-q^{-n/2}/2,1/(2n-2)\bigl)$.
Since $q^{n-1}+1\equiv2\bmod4$, these elements, restricted to the 
null space of $h(g)\tilde{h}(g)$,
are either of odd order, or of twice odd order; and the restriction of 
$g$ to this null space has spinor norm $-1$ if and only if
the restriction is of even order.
Moreover, $g$ acts on the orthogonal complement of the 
null space as an element of
$\SO^+(2,q)$. %which is cyclic of order $q-1$.  
This is a cyclic group of twice odd order,
the elements with spinor norm $-1$ (in their action on 
this 2-dimensional space)
being those of even order.  Thus the proportion of 
elements of $\SO^+(2n,q)$ whose 
characteristic polynomial satisfies the above condition 
and have spinor norm $-1$ (and
necessarily have twice odd order) is $\pi/2$, and this proves the lemma.
\end{proof}

\begin{lemma}\label{extra-minus}
If $q\equiv3\bmod4$, then the proportion of elements 
of $\SO^-(2n,q)$ that are
of twice odd order and have spinor norm $-1$ is at least $1/(4n)+\Oh(1/q)$,
and is strictly positive.
\end{lemma}
\begin{proof} 
The case $n=1$ being trivial, suppose first that $n=2$.  
Now $\SO^-(4,q)\cong C_2\times\PSL(2,q^2)$.
The proportion of elements of $\PSL(2,q^2)$ of odd order
is greater than $1/2$ (see \cite[p.\ 288]{Dornhoff}).  
Thus the proportion of elements of
$\SO^-(4,q)$ of spinor norm $-1$ and of twice odd order is
greater than $1/4$.

If $n>2$ is odd, then we consider elements 
$g$ of $\SO^-(2n,q)$ whose characteristic
polynomial $f(x)$ is irreducible, and hence satisfy the condition 
$f(x)=\tilde{f}(x)$.  Such elements preserve an irreducible form of
$-$ type, and have order dividing $q^n+1$, which is twice odd.  Since
$-I_{2n}$ has spinor norm $-1$, it follows that $g$ has spinor norm $-1$
if and only if $g$ is of even order.  Hence the proportion of elements of
$\SO^-(2n,q)$ satisfying these conditions is $1/(4n)+\Oh(1/q)$.

If $n>2$ is even, then we consider elements of $\SO^-(2n,q)$ whose
characteristic polynomial has two irreducible factors 
$f(x)$ and $\tilde{f}(x)$, each 
of degree $n-1$.  Again these elements have order not divisible by $4$.
The proportion of such elements having spinor norm $-1$ (and hence of 
twice odd order) is exactly $1/2$.  Hence the proportion of elements 
of this type is $1/(4n-4) +\Oh(1/q)$, as required.
\end{proof}

Lemma \ref{extra} implies that the 
proportion of elements $g$ of $H$
considered in Step 3 of {\tt OneOmegaPlus3}, 
is greater than $(4f-2)^{-1}(1-q^{-1}-q^{-f})/4$.
Lemma \ref{extra-minus} implies that the corresponding
proportion in Step 3 of {\tt OneOmegaMinus3}
is greater than $1/(4n)+\O(1/q)$.

\section{Involutions with eigenspaces of equal dimension}\label{Equal}
Let $G$ be one of the following:
$\SL(4n,q)$, $\Sp(4n,q)$, $\SU(4n,q)$, 
$\Omega^+(4n,q)$ if $q\equiv1\bmod4$,
$\Omega^+(8n,q)$ if $q\equiv3\bmod4$, or $\SO^+(4n,q)$.
We describe an algorithm to construct an 
involution in $G$ with both eigenspaces of the same dimension.
We use this as one component in Algorithm {\tt Two}.

Our algorithm is more general in nature:
it constructs an involution, each of whose eigenspaces
has a specified dimension.
If $G$ is orthogonal, then the eigenspaces
must support forms of $+$ type; 
hence the dimension of the $-1$-eigenspace 
must always be even, and a multiple of $4$
if $G=\Omega^+(d,q)$ and $q\equiv3\bmod4$.

Consider first the case where $G = \SL(d, q)$. 
We outline an algorithm to construct 
an involution with $-1$-eigenspace of dimension 
$e$ where $0 \leq e < d$. 
Its design ensures that recursive calls
involve matrices having dimension at most $2d/3$.
\begin{enumerate}
\item 
Find, by random search, $g \in G$ of even order
that powers to a strong involution $h_1$.

\item Let $r$ and $s$ denote the ranks of the 
$-1$ and $+1$-eigenspaces, $E_-$ and $E_+$ respectively, of $h_1$.

\item If $r = e$ then return the involution $h_1$.
 
\item 
Construct the centraliser in $G$ of $h_1$.
Obtain generators for $\SL(E_-)$ and for $\SL(E_+)$ as subgroups of $G$. 
% $-1$-space, where $S_-$ acts as the identity on the $+1$-eigenspace of $h_1$. 
(See Sections \ref{Pow} and \ref{Bray} for details of the algorithms used.)

\item 
Consider the case where $s \leq e < r$. By recursion, find 
in $\SL(E_-)$ an involution whose $-1$-eigenspace has dimension $e$. 

\item 
Consider the case where $e \leq \min(r, s)$.
If $r < s$, then, by recursion, find 
in $\SL(E_-)$ an involution whose $-1$-eigenspace has dimension $e$. 
Similarly, if $s < r$, then, by recursion, 
find in $\SL(E_+)$ an involution
whose $-1$-eigenspace has dimension $e$. 

\item 
Consider the cases where $s \geq e > r$ or $e \geq \max(r, s)$.
By recursion, find in $\SL(E_+)$ an involution $h_2$ 
whose $-1$-eigenspace has dimension $e-r$. 
Now return $h_1h_2$, an involution of the required type. 
\end{enumerate}
The recursion is founded trivially with the case $d=4$.

\begin{theorem}\label{Corollary5.2}  
In $\Oh(d(\xi + d^3 \log d + d^2 \log d\log\log d \log q))$ field operations, 
this Las Vegas algorithm constructs an involution in $\SL(4n,q)$ 
that has its $-1$-eigenspace of any even dimension in $[0,d]$.
\end{theorem}
\begin{proof}
Theorem \ref{Corollary5.1}
proves that the strong involution $h_1$ in Step 1 can be 
found and constructed using a Las Vegas algorithm in 
$\Oh(d(\xi + d^3 \log d + d^2 \log d\log \log d \log q))$ field operations. 
In Sections \ref{Pow} and \ref{Bray}, we show 
that generators for $\SL(E_-)$ and $\SL(E_+)$ as
subgroups of $G$ can be constructed using the same  
number of operations. 
%in $\Oh(d(\xi + d^3 \log d + d^2\log d \log\log d \log q))$ 
%field operations. 
%Thus the above algorithm requires 
%$\Oh(d(\xi + d^3 \log d + d^2\log d \log\log d \log q))$ field 
%operations, plus 
%We must also consider the number of field operations required by
%the algorithm in its 
%recursive calls. 
Since the dimension of the matrices
in a recursive call is at most $2d/3$, Lemma \ref{d-a-c} implies that 
the total complexity is as stated.
\end{proof}

The other classical groups are dealt with in essentially the same way, 
and the corresponding algorithms have the same complexity.
If $G$ is an orthogonal group preserving a form of $+$ type, 
then the involution constructed in Step 1 has 
both eigenspaces supporting a form of $+$ type, so the involution returned 
has the same property.

\section{Exponentiation}
\label{Exp}

A frequent task in our algorithms is computing 
$g^n$ for some $g\in \GL(d,q)$ and integer $n$
where $n< q^d$. We
could construct $g^n$ with $\Oh(\log n)$ multiplications using the familiar
black-box squaring technique. 
Instead, we describe the following faster Las Vegas
algorithm to perform this task.
\begin{enumerate}
\item 
Construct the Frobenius normal form of $g$ and record
the change-of-basis matrix. 

\item 
 From the Frobenius normal form, 
read off the minimal polynomial
$h(x)$ of $g$, and factorise $h(x)$ 
as a product of irreducible polynomials.

\item 
Following Section \ref{pseudo-order},
compute a multiplicative upper bound, $m$, to the order of $g$. 
%If $\{f_i(x):i\in I\}$ is the set of distinct
%irreducible factors of $h(x)$, and if $d_i$ is the degree of
%$f_i(x)$, then the order of the semi-simple part of $g$ divides
%$\prod_iq^{d_i}-1$, and the order of the unipotent part of $g$ can be
%read off directly. The product of these two factors  gives the
%required upper bound $m$. 

\item If $n>m$, then replace $n$ by $n\bmod m$. 
By repeated squaring, calculate $x^n\bmod h(x)$ 
as a polynomial of degree $k-1$, where $k$ is the degree of $h(x)$.

\item Evaluate this polynomial in $g$ to give $g^n$. 

\item Conjugate $g^n$ by the inverse of the change-of-basis 
matrix to return to the original basis.
\end{enumerate}

We now consider the complexity of this algorithm.
\begin{lemma}\label{compute-power}
Let $g\in\GL(d,q)$ and let $0\le n<q^d$. This 
Las Vegas algorithm computes $g^n$ in 
$\Oh(d^3 \log d + d^2\log d \log\log d \log q)$ field operations.
\end{lemma}
\begin{proof}
Using the Las Vegas algorithm of \cite{Giesbrecht}, 
in $\Oh(d^3 \log d)$ field operations 
we obtain the Frobenius normal form of $g$, the corresponding 
change-of-basis matrix, and thus the minimal polynomial of $g$.
The minimal polynomial can be factored in 
$\Oh(d^2 \log d \log\log d\log q)$ field operations.
% \cite[Theorem 14.14]{vzg}.

Calculating $x^n \bmod h(x)$ requires $\Oh(\log n)$
multiplications in $\GF(q)[x]/(h(x))$,  and hence 
$\Oh(d^2 \log d \log \log d \log q)$ 
field operations. % (see \cite[Section 8.3]{vzg}). 
Evaluating the resultant polynomial in $g$ requires
$\Oh(d)$ matrix multiplications;  but multiplying by 
$g$ only costs $\Oh(d^2)$ field operations, since $g$ is sparse 
in Frobenius normal form. Conjugating $g$ by the inverse of 
the change-of-basis matrix costs $\Oh(d^3)$ field operations.
\end{proof}

\section{Constructing direct factors}\label{Pow}
We consider the following problem.
\begin{problem}
Let $G = \langle X \rangle$ be a subgroup of
the centraliser of an involution $g$ in
$\GX(d,q)$, so $G \leq \GX(E)\times\GX(F)$, where
$E$ and $F$ are the eigenspaces of $g$.  If $G$ contains
$\Omega(E)\times\Omega(F)$, find $($as \SLPs\ in $X)$
generating sets for $\Omega X(E)$ and $\Omega X(F)$.
\end{problem}
We prove the following result.
\begin{theorem}  There is a Las Vegas algorithm, with complexity 
$\Oh(\frac{d\log\log d}{\log d}
(\xi +d^3 \log d + d^2 \log d \log\log d\log q))$
measured in
field operations, that takes as input a
subset $X$ of $\GX(E)\times\GX(F)$, where $E$ and $F$ are the
eigenspaces of an involution in $\GX(d,q)$, such that $X$ generates a group
containing $\Omega X(E)\times\Omega X(F)$, and returns
generating sets for $\Omega X(E)$ and $\Omega X(F)$ as \SLPs\ in $X$.
\end{theorem}

Our proof of this theorem relies heavily on the 
one-sided Monte Carlo recognition algorithm 
of Niemeyer \& Praeger \cite{NP, JAMS}.
We outline this algorithm briefly.  
The input is $\langle X \rangle = G \leq \GX(d,q)$, 
of known type $\X$, where $d>2$. 
It decides whether or not $G$ contains 
$\Omega X(d,q)$, given that
$G$ is an irreducible subgroup of $\GX(d,q)$ that does not
preserve any bilinear or quadratic form not preserved by $\GX(d,q)$.

In order to decide this, a set $\cal S$ of subsets of $\GX(d,q)$ is defined
with the property that any irreducible subgroup of $\GX(d,q)$ that does not
preserve any non-degenerate form not preserved by $\GX(d,q)$, and that
contains a subset $S\in\cal S$, generates a group containing $\Omega X(d,q)$.
In this case $S$ is a {\it witness} to this fact.
For most values of the parameters 
$(\X,d,q)$, the following is the case.  
A set $P$ of pairs of primes or squares of primes, each  
dividing $|\Omega X(d,q)|$ but prime to $q-1$,
is defined.  The elements of $\cal S$ are pairs,
and $S \in \cal S$ if and only if there is a pair  
$(\ell_1,\ell_2)\in P$ such that  $\ell_1$ divides the order of 
one element of $S$, and $\ell_2$ divides the order of the other.

We call parameters $(\X,d,q)$ for which $\cal S$ is defined in 
this way {\it standard}.
(These include all generic cases of \cite{NP}
and some of the non-generic cases of \cite{JAMS}.)
If the parameters are not standard,
then the algorithm requires different types of witness.
To find a witness, a sample of $\Oh(\log\log d)$  random 
elements must be considered; see \cite[Proposition 7.5]{NP}.

Recall that 
a {\it primitive prime divisor} of $q^e-1$ is a prime divisor of
$q^e-1$ that does not divide $q^i-1$ for any positive integer $i<e$.  
If $r$ is a primitive prime divisor of $q^e-1$ then $r\equiv 1\bmod e$, 
and so $r\ge e+1$.  If $(\ell_1,\ell_2)\in P$, then in most cases $\ell_i$ is
a primitive prime divisor of $q^{e_i}-1$ for some $e_i>d/2$ for $i=1,2$, and
$e_1\ne e_2$.  Further conditions may be imposed, and in some cases
$\ell_i$ is the square of a primitive prime divisor of $q^{e_i}-1$.
 We are not concerned here with the precise variations used. 

A sufficient condition for 
$g \in \SX(d,q)$ to have order prime to $\ell_i$ is 
that the characteristic polynomial of $g$ should have no irreducible
factor of degree a multiple of $e_i$. 

Before describing the algorithms to solve our problem, 
we present two related results which assist in our analysis.

\begin{lemma}\label{partition}
Let $\pi$ be a partition of $d > 2$,
and let $\Omega X(d,q)\le G\le\GX(d,q)$. 
Denote by $P(G,\pi)$ the proportion of $g\in G$ such that the
degrees of the irreducible factors of the characteristic polynomial of $g$
partition $d$ as $\pi$.  
\begin{enumerate}
\item[\rm (i)] Let $\X=\mathbf{SL}$ and $\pi=(k, d-k)$:
if $1\le k<d/2$ then $P(G,\pi)>(1-q^{-1})^2/(k(d-k))$.
%; and if $k=1$ or $k=d/2$ then $P(G,\pi)>(1-q^{-1})^2/(2k(d-k)).$

\item[\rm (ii)] 
Let $\X=\mathbf{SL}$ and $\pi=(1, k, d-k-1)$:
if $1<k<d-k-1$ then $P(G,\pi)>(1-q^{-1})^2/(k(d-k-1))$.
%; if $1=k<d-k-1$ or $1<k=d-k-1$ then  $P(G,\pi)>(1-q^{-1})^2/(2k(d-k-1))$;
%and if $k=1$ and $d=3$ then  $P(G,\pi)>(1-q^{-1})^2/(6k(d-k-1))$.

\item[\rm (iii)] 
Let $\X\ne\mathbf{SL}$ and $\pi=(k/2,k/2,(d-k)/2,(d-k)/2)$, where $d$ and $k$ 
are even:
if $1<k<d/2$ then  $P(G,\pi)>(1-q^{-1})^2/(4k(d-k))$
%; and if $k=d/2$ then  $P(G,\pi)>(1-q^{-1})^2/(8k(d-k))$.

\item[\rm (iv)] 
Let $\X=\mathbf{SU}$ or $\mathbf{SO^0}$ and
$\pi=(1,k/2,k/2, (d-k-1)/2,(d-k-1)/2)$, where
$d$ is odd and $k$ is even:
if $2<k<(d-1)/2$ then  $P(G,\pi)>(1-q^{-1})^2/(4k(d-k-1))$
%; if $k=2$ and $k<(d-1)/2$ then $P(G,\pi)>(1-q^{-1})^2/(12k(d-k-1))$;
%if $k=(d-1)/2$
\end{enumerate}
\end{lemma}
The proof is an easy exercise, using the techniques of 
Section \ref{Involution}.

\begin{lemma} 
\label{nofactor}
Let $d > 2$ and $1<\ell\le d$, and 
assume $\Omega X(d,q)\le G\le\GX(d,q)$.  
The proportion of elements of $G$ whose
characteristic polynomial has no irreducible factor of degree a multiple
of $\ell$ is greater than $c\log d/d$ for some positive universal 
constant~$c$.
\end{lemma}
\begin{proof}  
Suppose that $\mathbf{X} = \SL$, and $d>4$.  
An easy modification of Lemma \ref{Lemma5.5a}  shows that 
if $\ell>d/2$, then the proportion of elements of $G$ whose 
characteristic polynomial has no irreducible factor of 
degree $\ell$ is at least $1/2$.  

We now consider smaller values of $\ell$.
We apply Lemma \ref{partition} to obtain 
the proportion of elements of $G$
whose characteristic polynomial has exactly two irreducible factors
of unequal degrees.
Observe that, for any $a>0$,
$\sum_{k=1}^a1/(k(d-k))= (2/d)\sum_1^a1/k$. 
Taking $a=\lfloor(d-1)/2\rfloor$, and letting $k$ denote the smaller degree,
so that $k\le a$,  we see that the proportion in question is
at least $c\log d/d$ for some absolute constant $c>0$.

%, one of degree $k$ and one of degree $d-k$, where $1\le k<d/2$.
%$((\log d+\phi(d))/d)c-4\epsilon/d^2$ where $1\ge c\ge(1-q^{-1})^2$; here
%$\phi(d)$ is $\sum_1^{d-1}1/\ell-\log d$, a monotonically increasing 
%function with limit Euler's $\gamma$, 
%and $\epsilon$ is $1$ if $d$ is even, and $0$ if $d$ is odd;
%,this is to avoid the exceptional case $k=d/2$.

Similarly, if the degree $k$ is required to be congruent to some
fixed value modulo $\ell$, then the proportion in question is at least
$c\log d/(d\ell)$ for some $c>0$.

Now consider the values of $k$ for which $k$ or $d-k$  
is a multiple of $\ell$.  If $\ell>2$ then at least $\ell-2$ of the
$\ell$  residue classes give values 
of $k$ that satisfy the conditions
of the lemma, and complete the proof in this case.

Now assume $\ell = 2$.
If $k$ is odd, and $d$ is even, then $d-k$ is odd; 
so one of the residues classes give values
of $k$ that satisfy the conditions.
If $d\ge9$ is odd, then we consider the proportion of elements 
of $G$ whose
characteristic polynomial has one irreducible factor of degree 1, 
one of degree $k$, and one of degree $d-k-1$, as in case (ii) above. 
The proof now proceeds exactly as before.

The remaining cases occur for bounded $d$ only and there 
clearly exist elements which satisfy the present lemma.

The proof for the other classical groups is essentially the same.
\end{proof}

\subsection{The standard parameter case}
Our task is the following.  Let 
$\Omega X(E)\times\Omega X(F) \leq G = \langle X \rangle \leq
\GX(E)\times\GX(F)$;
find (as \SLPs\ in $X$) a generating set for $\Omega X(E)$. 
Let $e$ and $f$ denote the dimensions of $E$ and $F$ respectively.
We assume that $(\X, e, q)$ is standard; in
particular, this implies that $e > 2$.

Our algorithm, {\tt GenerateFactor}, is the following.
\begin{enumerate}
\item 
Repeatedly construct random 
$(g, h) \in G$, where $g\in\GX(E)$ and $h\in\GX(F)$, until 
we find two elements
$(g_1, h_1)$ and $(g_2, h_2)$ such that $(g_1, g_2)$ acts as a witness 
for $\Omega X(E)$, with corresponding prime powers
$(\ell_1,\ell_2)$, and the pseudo-order $n_i$ of $h_i$ 
is prime to $\ell_i$ for $i=1,2$.  
\item Let $m_i_i(q-1)$.
Compute $g_1^{m_1}=(g_1,h_1)^{m_1}$
and $g_2^{m_2}=(g_2,h_2)^{m_2}$.
\item 
If $\langle g_1^{m_1},g_2^{m_2}\rangle$ is irreducible, 
and it also preserves no non-degenerate bilinear form when $\X=\mathbf{SL}$,
then return $(g_1^{m_1},g_2^{m_2})$; else return to Step 1.
\end{enumerate}

\begin{lemma} \label{standard-parameters}
If the parameters $(\X,e,q)$ for $G$ are standard,
then the Las Vegas algorithm, {\tt GenerateFactor},
constructs a generating pair for $\Omega X(E)$ in
$\Oh(\frac{d \log\log d}{\log d}(\xi+d^3 \log d + d^2 \log d \log\log d\log q))$ field operations.
\end{lemma}
\begin{proof}
The algorithm of \cite{NP} requires $\Oh(\log\log e)$ trials to 
find a pair
of elements $(g_1,h_1)$ and $(g_2,h_2)$ of $G$ such that $(g_1,g_2)$ will act
as a witness for $\Omega X(E)$.
If $(g_1,g_2)$ is a witness because $g_i$ has order
a multiple of $\ell_i$, then Lemma \ref{nofactor} implies that 
the probability that $h_i$ has pseudo-order coprime   
to $\ell_i$ is $\Oh(\log f/f)$.  

We must also consider the probability that $\langle g_1,g_2\rangle$
is reducible, or, if $\X=\mathbf{SL}$, that 
it preserves a non-degenerate form.  Since $g_i$ acts
irreducibly on a subspace of dimension $e_i>d/2$, the probability 
that $\langle g_1,g_2\rangle$ is irreducible is bounded away from 0, 
and tends to 1 as $q$ or $d$ tends to infinity.  
The same is clearly true for the probability 
that $\langle g_1,g_2\rangle$
preserves no non-degenerate form if $\X=\mathbf{SL}$.

Computing and factorising 
the characteristic polynomial of $g \in G$ 
takes $\Oh(d^3 \log d + d^2 \log d \log\log d \log q)$
field operations. 
The powering operation, which need only take place in
$E$, is performed twice, assuming that 
$\langle g_1,g_2\rangle$ is irreducible and does not preserve a form.  
\end{proof}

\subsection{The dimension 2 case} \label{two-factor}
We now consider the case where $(\X, e,q)=(\mathbf{SL}, 2,q)$ and $q>3$.
We first show that, with high probability, $\SL(2, q)$ can be generated
by an irreducible element and a random conjugate.  Let $M(q)$
be the metacyclic group of order $2(q+1)$ defined by the presentation
$\{ a,b\; \vert \; a^{q+1}=1, a^b = a^{-1}, b^2=a^{(q+1)/2}\}$.
\begin{lemma}\label{sl2qgen}
Let $H$ be a maximal irreducible subgroup of
$\SL(2,q)$.
Then $H$ is either conjugate to $\SL(2,r)$,
where $q=r^\ell$ for an odd prime $\ell$;
or to an extension  $\SL(2,r).2$ of $\SL(2,r)$ by
a cyclic group of order $2$, where $q=r^2$;
or is isomorphic to $M(q)$;
or is isomorphic to an extension of a cyclic group of order $2$
by one of $A_4$, $S_4$ or $A_5$.
\end{lemma}
\begin{proof}
This result can be read off from \cite[Hauptsatz II.8.27]{Huppert67}.
\end{proof}
\begin{corollary}\label{psl2q}
Let $q > 3$ and let $g\in\SL(2,q)$ act irreducibly.  The
probability that a random conjugate of $g$, together with $g$, will
generate $\SL(2,q)$ is at least $1 - q^{-2/3}$,
independently of the choice of $g$.
\end{corollary}
\begin{proof}
An irreducible element $g$ of $\SL(2,q)$ lies in a unique cyclic subgroup of
order $q+1$, since distinct cyclic subgroups of $\PSL(2,q)$ of order
$(q+1)/2$ intersect trivially (see \cite[Hauptsatz II.8.5]{Huppert67}).  
Thus the probability that $g$ and a
random conjugate $h$ of $g$ will lie in the same copy of $M(q)$
is $1:k$, where $k= |\SL(2,q) | /2(q+1)=(q^2-q)/2$, unless $g$ has 
order $4$.  If $g$ has order $4$, there remains the possibility that 
$\langle g, h \rangle$ is a quaternion group of order $8$.  

If $g$ has order $4$ and acts irreducibly, then $q\equiv3\bmod 4$.
The elements of $\SL(2,q)$ of order $4$ lie in a single conjugacy class,
of size $q(q - 1)$, so we may assume that 
$$g=\left(\matrix{0&1\cr-1&0\cr}\right).$$
If $\langle g,h\rangle$ is a quaternion group of order $8$,
then a calculation shows that $h$ is of the form
$$\left(\matrix{a&b\cr b&-a}\right),$$ 
where $a^2+b^2=-1$, giving $q + 1$ possibilities.  
Hence the probability that $g$ and $h$ lie in a 
single copy of $M(q)$ is at most $q+3:q(q+1)$. 
(We must consider the $q + 1$ conjugates of $h$, and $g^{\pm 1}$.)

Now let $\SL(2,r)$ be a subgroup of $\SL(2,q)$ containing an element $g$
that acts irreducibly.  This implies that $q$ is an odd power of $r$.  Now $g$
lies in exactly $(q+1)/(r+1)$ conjugates of $\SL(2,r)$, which between them 
contain fewer than
$(q+1)r(r-1)/(r+1)$ of the $q(q-1)$ conjugates of $g$ in $\SL(2,q)$.  
Thus the probability that $h$ lies in one of these subgroups is 
less than $(q+1)r(r-1)/(q(q-1)(r+1)) < q^{-2/3}$.
%This proportion is $\Oh(q^{-2/3})$, the worst case being when $r=q^{1/3}$,
%and is always positive.

The probability that $g$ and $h$ both lie in the
same copy of $2.A_4$ or $2.S_4$ or  $2.A_5$ is $\Oh(1/q^3)$, since $\SL(2,q)$
contains at most two conjugacy classes of any one of these groups
(see \cite[Satz II.8.13-18]{Huppert67}).  
\end{proof}
%Check that A_4 and A_5 can be maximum in PSL(2,q).

We now describe our algorithm, {\tt TwoFactor}, to construct
a generating set for $\SL(2, q)$ in $G$, where
$\Omega X(d-2,q)\times\SL(2,q)\le G = \langle X \rangle 
\le \GX(d-2,q)\times\GL(2,q)$.
The output is a  set of generators for $\SL(2, q)$, 
given as \SLPs\ in $X$.
\begin{enumerate}
\item
If $q+1$ is not a power of $2$, then 
search for $(g, h) \in G$ where:
\begin{itemize}
\item
the characteristic polynomial of $g$ has no irreducible factor of even degree;
\item $h$ has an irreducible characteristic
polynomial, and has pseudo-order divisible by an odd prime $\ell$, where
$\ell$ divides $q+1$.
\end{itemize}

\item
If $q+1$ is a power of $2$, then 
search for $(g, h) \in G$ where, if $g$ and $h$ 
have pseudo-orders $a$ and $b$
respectively, then $v_2(b)>v_2(a)+1$.

\item
Let $k$ be the pseudo-order of $(g,h)$, divided by $\ell$, in Case 1;
and let $k$ be the odd part of the pseudo-order of $(g,h)$ in Case 2.
Evaluate $h^k$ to obtain $(1, x)$. 
%(but see below).  

\item
Now $x$ is an irreducible element of $\SL(2,q)$.
Find, by random search, $y \in G$ such that $x$ and $x^y$
generate $\SL(2,q)$, and return $\{x,x^y\}$.
\end{enumerate}

\begin{lemma} \label{2factor}
The Las Vegas algorithm {\tt TwoFactor} takes 
$\Oh(\frac{d \log\log d}{\log d} 
(\xi + d^3 \log d + d^2 \log d \log\log d \log q)$ field operations.
\end{lemma}
\begin{proof}
The first step is to prove that $(g,h)$ can be found 
with $\Oh(d/\log d)$ trials.
Suppose first that $q+1$ is not a power of $2$. 
Lemma \ref{nofactor} shows that 
the proportion of elements of
$\Omega X(d-2,q)$ with the property that
every irreducible factor of their characteristic polynomials 
has odd degree is $\Oh(\log d /d)$, and the result follows.

If $q+1$ is a power of $2$, then we may require $g$ to have odd order.
Lemma \ref{partition} implies that the proportion of elements of 
$G$ whose characteristic polynomial has at most $5$ irreducible factors, all
of odd degree, is $\Oh(\log d/d)$.  Since $q\equiv3\bmod4$, it follows
that if $g\in G$ has such a characteristic polynomial, then the probability of 
$g$ having odd order is at least $(1/2^5)(1-\Oh(1/q))$. In fact the probability
is at least $(1/2^3)(1-\Oh(1/q))$, since, in the cases where there 
are more than three factors, those of degree greater than $1$ are 
paired as $h(x)$ and $\tilde{h}(x)$
and $k(x)$ and $\tilde{k}(x)$, or as $h(x)$ and $\hat{h}(x)$
and $k(x)$ and $\hat{k}(x)$.

Thus, in either case, we can expect to find a suitable
$(g,h)$ with $\Oh(d/\log d)$ trials.
For each pair $(g,h)$ considered, we compute and factorise
the characteristic polynomial of $g$ in 
$\Oh(d^3\log d + d^2 \log d \log\log d \log q)$ field operations.
In Step 3 we compute the pseudo-order of $(g,h)$ in 
$\Oh(d^3\log d + d^2 \log d \log\log d \log q)$ field operations.
We also need to raise $(g,h)$ to a certain power, but 
only need to power $h$.
(The pseudo-order of $(g,h)$
needs to be computed, rather than the pseudo-order of $h$,
because we need to record $x$ as an \SLP\ in the given generating set.)
Corollary \ref{psl2q} implies that the number of trials 
needed in Step 4 is constant.
\end{proof}

\subsection{Dimension 4 orthogonal cases}
Two further non-standard sets of parameters 
are $(\Omega^\epsilon,4,q)$, for $\epsilon =\pm$.  

Since $\Omega^-(4,q) \cong \PSL(2,q^2)$, this case is
essentially covered by Lemma \ref{2factor}.
We need $(g, h) \in \Omega^-(4,q)\times\Omega^\epsilon(d-4,q)$ 
that powers to an element of $\Omega^-(4,q)$ of order not dividing $q^2-1$.
We thus look for $(g,h)$ where the order of 
$g$ is a multiple of an odd prime dividing $q^2+1$, and the 
order of $h$ is not.
It is sufficient for the characteristic polynomial of $h$ 
to have no irreducible factor of degree a multiple of $4$.

Recall that 
$\Omega^+(4,q)$ is the central product of two copies of $\SL(2,q)$.  
If $q > 3$, then we can find $(g, h)$ where 
$h \in \Omega^+(4,q)$, and its projection to a given copy 
of $\SL(2,q)$ acts irreducibly (in dimension 2), and hence proceed as 
in Section \ref{two-factor}.  

In summary, if $\Omega(E)$ is isomorphic to $\Omega^\epsilon(4,q)$, 
we construct one or (if $\epsilon=+$) two suitable elements of $\Omega(E)$
by powering a suitable element or elements of $G$, found by 
random selection, and 
then construct a generating set for $\Omega(E)$ from conjugates of
this element, or pair of elements.

Thus we arrive at the following lemma.
\begin{lemma} In 
$\Oh(\frac{d \log\log d}{\log d} 
(\xi +d^3\log d + d^2 \log d \log\log d \log q))$ field operations,
this Las Vegas algorithm constructs a 
generating pair for $\Omega(E)\equiv\Omega^\epsilon(4,q)$ 
$($where $q>3$ if $\epsilon=+)$. 
\end{lemma}

\subsection{The other non-standard cases}
%{\tt EOB -- FOLLOWING NEEDS TO BE MADE MORE PRECISE.
%    SHOULD WE SPECIFY THE EXAMPLES EXPLICITLY?}
We are now left with a finite number of possibilities 
for $\Omega X(E)$, 
of which $\SL(2,3)$ and $\Omega^+(4,3)$ are soluble.
Since $\SL(2, 3)$ is the normal closure 
of any one of its 8 elements of order 3, 
these soluble examples pose no problems.

The remaining exceptional cases are listed in \cite{NP}
and are perfect, being simple modulo scalars.  
Since none of these groups consists entirely of diagonal elements,
we can find a 
non-diagonalisable element of the group, and generate
$\Omega X(E)$ with a given degree of confidence by a uniformly bounded 
number of random conjugates of this element.

\subsection{The strong involution case}
Finally we consider the case in which $e\in (d/3,2d/3]$
and obtain a stronger result when $E$ is an eigenspace of 
a strong involution.
We assume that $d$ is sufficiently large to avoid non-standard 
parameters.  Using {\tt GenerateFactor}, we search
for $(g_i,h_i) \in \GX(E)\times\GX(F)$, where $\{g_1, g_2\}$ is 
a witness for $\Omega X(E)$ by virtue of $g_i$ having order a multiple of
some primitive prime divisor of $q^{k_i}-1$ (or of its square), and
the characteristic polynomial of $h_i$ does not have 
an irreducible factor of degree a multiple of $k_i$.
Now $k_i>e/2\ge d/6$, and as $d$ tends to 
infinity the probability that the
characteristic polynomial of $h_i$ will have such a factor clearly 
tends to 0.  Thus the number of
random elements of $G$ that we need to consider, with a given 
probability of success, is  $O(\log \log d)$.

\begin{theorem}  Assume that the parameters $(\X, d, q)$ are standard.
There is a Las Vegas algorithm, with 
complexity $\Oh(\log \log d (\xi +d^3 \log d + d^2\log d \log\log d \log q)$ 
measured in field operations, that takes as input a
subset $X$ of $\GX(E)\times\GX(F)$, where $E$ and $F$ are the 
eigenspaces of a strong involution in $\GX(d,q)$, generates a group
containing $\Omega X(E)\times\Omega X(F)$, where the dimension 
of $E$ is at least $d/3$,
and returns a generating set for $\Omega X(E)$  as an \SLP\ in $X$.
\end{theorem}

\section{Constructing an involution centraliser}
\label{Bray}
In applying our algorithms to groups  in $\C$,
we construct involution centralisers.
In particular, we must solve the following problems.
Let $u$ be an involution in $G=\SX(d, q)$ and let 
$E_+$ and $E_-$ denote the eigenspaces of $u$. 
\begin{enumerate}
\item 
Construct a generating set for a subgroup of $C_G(u)$ 
that contains $\SX(E_+)\times \SX(E_-)$. 
\item 
Suppose that $E_+$ and $E_-$ are isometric. 
Construct the projective centraliser in $G$ of $u$.
\end{enumerate}
If $E_+$ and $E_-$ have the same dimension, then they are isometric, 
except when $G$ is an orthogonal group of $-$ type 
(see Lemma \ref{form-type}).
The second problem arises in Algorithm {\tt Two}
for non-orthogonal groups and for orthogonal groups of $+$ type only.

Elements of the centraliser of an involution in a black-box 
group having an order
oracle can be constructed using an algorithm of Bray \cite{Bray},
which employs the following result.
\begin{theorem}
\label{thm:bray}
If $u$ is an involution in a group $G$, and $g$ is an arbitrary element of $G$,
then $[u,g]$ either has odd order $2k+1$, in which case
$g[u,g]^k$ commutes with $u$, or has even order $2k$, in which case
both $[u,g]^k$ and $[u,g^{-1}]^k$ commute with $u$.
\end{theorem}
That these elements centralise $u$ follows from elementary
properties of dihedral groups. 

Bray \cite{Bray} also proves that if $g$ is uniformly
distributed among the elements of $G$ for which $[u,g]$
has odd order, then $g[u,g]^k$ is uniformly distributed among the
elements of the centraliser of $u$. If $[u,g]$ has even order, 
then the elements returned are involutions; but if just
one of these is selected, then it is independently and uniformly 
distributed within that class of involutions.

Parker \& Wilson \cite{PW05} prove the following.
\begin{theorem}\label{clasthm}
There is a absolute constant $c$ such that if $G$ is a finite
quasisimple classical group, with natural module
of dimension $d$ over a field of odd characteristic,
and $u$ is an involution in $G$, then $[u,g]$ has odd order
for at least a proportion $c/d$ of the elements $g$ of $G$.
\end{theorem}
%This constant is at most 1/12. 

Hence, by a random search of length $\Oh(d)$, we construct
random elements of the centraliser of the involution. 
Liebeck \& Shalev \cite{lish} prove that if $H_0 \leq H \leq {\rm Aut}(H_0)$,
where $H_0$ is a finite simple group, then the probability that
two random elements of $H$ generate a group containing 
$H_0$ tends to 1 as $|H_0|$ tends to infinity. A similar
result clearly holds for a direct product of two simple groups.

In its black-box application, this algorithm 
assumes the existence of an order oracle.
We do not require such an oracle for a linear group.
Recall, from Section \ref{pseudo-order},
that we can deduce if an element of a linear 
group has even order in 
$O(d^3 \log d + d^2 \log d \log\log d\log q)$ field operations. 
Further, the construction of the centraliser
of an involution requires only knowledge of pseudo-orders.

In our context, the analysis of 
\cite{Ryba-paper} implies the following.
\begin{theorem}
The Las Vegas algorithm to construct the centraliser
of an involution in $\SX(d, q)$ has complexity
$\Oh(d(\xi + d^3 \log d + d^2 \log d \log\log d\log q))$ 
measured in field operations.
\end{theorem}

This algorithm can be readily adapted (using projective rather
than linear pseudo-orders) to compute the preimage in 
$\SX(d, q)$ of the centraliser of an involution in the 
projective image of $\SX(d, q)$.

Once we construct a subgroup of  the centraliser containing
its derived group, we can apply 
the algorithms of Section \ref{Pow} 
to obtain generators for the derived
groups of the projections of the
centralisers of the two eigenspaces. 

We summarise the preceding discussion. 
\begin{theorem}
Let $h$ be an involution in $\langle X \rangle = G$, where
$\Omega X(d, q) \leq G \leq \GX(d, q)$. Assume that the 
$-1$-eigenspace of $h$ has dimension $e$ in the range $(d/3, 2d/3]$.
Generating sets for the images in 
$\Omega X(e, q)$ and $\Omega X(d - e, q)$ that centralise
the eigenspaces can be found in 
$\Oh(d(\xi + d^3 \log d + d^2 \log d \log\log d\log q))$ 
field operations.
If the eigenspaces are isometric, so $e = d/2$ and $d \equiv 0 \bmod 4$, then
we can similarly find an element in $\Omega X(d, q) \wr C_2$ that
interchanges the two copies of $\Omega X(d, q)$.
\end{theorem}

\section{The base cases for the non-orthogonal groups}
\label{base}
We now consider the base cases for 
Algorithms {\tt One} and {\tt Two} when $\SX (d, q)$ is
a non-orthogonal group.  If $d = 2n$, then Lemma \ref{wr2} 
shows that $Y_0 := \{s, t, \delta, u, v \}$
generates $\SX(2, q) \wr C_n$  or $\SX(2, q) \wr S_n$
according to the type of $\SX (d, q)$. 
As the first and major task of each algorithm, we construct $Y_0$. 
As a final step, we construct the additional elements $x, y$.

Observe that the elements of $Y_0$ act non-trivially
only on a 4-dimensional space; they can be obtained by 
constructively recognising $\SX(2, q) \wr C_2$, a 
computation practically more efficient than that for $\SX(4, q)$.

Hence we designate the following as 
base cases: $\SX(2, q)$, $\SX(2, q) \wr C_2$, $\SX(3, q)$ and $\SX(4, q)$. 
The last two arise {\it at most once} during
an application of Algorithm {\tt One} or {\tt Two}.

In the remainder of this section, 
we outline the specialised algorithms
for the base cases. 
We first summarise their cost.

\begin{theorem}\label{ryba-alg}
Subject to the availability of a discrete logarithm oracle for $\GF(q)$,
% (or\ $\GF(q^2)$ if the type is $\mathbf{SU}$), 
\SLPs\ for standard generators and other 
elements of $\langle X \rangle = \SX(d, q)$ for $d \leq 4$ can be 
constructed in $\Oh(\xi \log \log q + \log q)$ field operations.
\end{theorem}

\subsection{$\SX(2,q)$}
The base case encountered most frequently is $\SL(2,q)$
in its natural representation.
An algorithm to construct an element of $\SL(2,q)$ as an \SLP\ 
in an arbitrary generating set is described in \cite{Conderetal05}. 
This algorithm requires $\Oh(\log q)$ field operations, and the 
availability of a discrete logarithm oracle for $\GF(q)$.

Observe that $G=\SU(2, q)$ is isomorphic to $\SL(2, q)$.
We can write $G$ over $\GF(q)$  by conjugating 
$G$ by a diagonal matrix $\rm{diag}(\alpha, 1)$
where $\alpha$ is an element of trace 0 in $\GF(q^2)$;
alternatively we could use the algorithm 
of \cite{GLO}; either requires $\Oh(\log q)$ field operations.

\subsection{$\SL(2, q) \wr C_2$}\label{glue-element}
In executing Algorithms {\tt OneEven} or {\tt OneOdd}, or
{\tt TwoTimesFour} or {\tt TwoTwiceOdd}, 
each pair of recursive calls generates
an instance of the following problem.

\begin{problem} \label{glue}
Let $V$ be the natural module of  $G=\SX(4,q)$, and let
$(e_1,f_1,e_2,f_2)$ be a hyperbolic basis for $V$. Given a generating
set for $G$, and the involution $u$, where $u$ maps $e_1$ to $-e_1$
and $f_1$ to $-f_1$, and centralises the other basis
elements, construct the involution $b$ of $G$ that permutes the basis
elements, interchanging $e_1$ with $e_2$, and $f_1$ with $f_2$.
\end{problem}

Consider the procedure {\tt OneEven}.
Observe that in line 14 we construct $\SX(4, q)$.
Now $b$ is the permutation matrix used in line 15 to 
`glue' $v_1$ and $v_2$ together to form $v$, the long cycle. 
We could use the algorithm of Section \ref{ryba-base} to find
$b$ directly in $\SX(4, q)$.
Instead, for reasons of practical efficiency, we 
use the following algorithm to 
find $b$ inside the projective centraliser of $u \in \SX(4, q)$. 

\begin{enumerate}
\item 
Construct the projective
centraliser $H$ of $u$ in $\SX(4,q)$; it contains 
$\SL(2,q)\wr C_2$.

\item 
%Since $\SL(2,q)\wr C_2 \leq H \leq \GL(2, q) \wr C_2$,  
Find $h\in H$ that 
interchanges the spaces $\langle e_1,f_1\rangle$ and $\langle
e_2,f_2\rangle$. 
Observe that $bh$ lies in $\SL(2,q)\times\SL(2,q)$.  

\item 
Using the algorithms described in Section \ref{Pow}, 
construct the two direct factors
and so construct $bh$ and thus $b$ as an \SLP.  
\end{enumerate}
Observe that we can conjugate, using $h$, the solution from one 
copy of $\SL(2, q)$ to the other, thus 
requiring just one constructive recognition of $\SL(2, q)$.
This algorithm has the same complexity as that for $\SL(2, q)$. 
%requires $\Oh(\log q)$ field operations.

\subsection{$\SX(3, q)$ and $\SX(4, q)$}
\label{ryba-base}
For $\SL(3,q)$ we use the algorithm of \cite{sl3q}
to construct standard generators.
It assumes the existence of an oracle
to recognise constructively $\SL(2, q)$
and its complexity is that of the oracle.

We use the {\it involution-centraliser algorithm} of \cite{Ryba-paper}
to construct standard generators for the remaining groups $\SX(3,q)$, 
and the additional elements $x, y \in \SX(4, q)$.  

We briefly summarise this algorithm.  
Assume $G = \langle X \rangle$ is a black-box group
with order oracle. We are given 
$g \in G$ and want to express it as an \SLP\ in $X$.
In our description, if we ``find" an element of $G$, then we 
obtain it as an \SLP\ in $X$.
First find by random search $h\in G$ such that
$gh$ has even order $2\ell$, and $z:=(gh)^\ell$ is a non-central
involution. Now  find, by random search and powering, an involution
$x\in G$ such that $xz$ has even order $2m$, and $y:=(xz)^m$ is a
non-central involution. Note that an \SLP\ is known for $x$, but, at this
stage, not for either of $y$ or $z$. 
Observe that $x$, $y$ and $z$ are non-central involutions. 
We construct their centralisers using the Bray algorithm.
We {\it assume} that we can  solve the explicit membership problem 
in these centralisers; see below for further discussion of this point.
In particular, we find $y$ as an element of the centraliser in $G$ of $x$, 
and $z$ as an
element of the centraliser in $G$ of $y$, and $gh$ as an element of the
centraliser in $G$ of $z$. Now that we know \SLPs\ for both $gh$ and 
$h$, we can construct an \SLP\ for $g$.

In summary, this algorithm reduces the constructive
membership test for $G$ to three constructive membership  tests  in involution
centralisers in $G$.  But  this is an imperfect recursion, since  the 
algorithm may not apply to these centralisers. 
We do not rely on the recursion; instead we construct
explicitly the desired elements of the centralisers, 
since their derived groups are (direct products of) $\SL(2,q)$
and we can use the algorithm of \cite{Conderetal05}.
In this context, the complexity of the involution-centraliser
algorithm is that stated in Theorem \ref{ryba-alg}.

As presented, this is a black-box algorithm  requiring an order oracle.
If $G$ is a linear group, the algorithm does not require 
an order oracle, exploiting instead the multiplicative 
bound for the order of an element which can
be obtained in polynomial time as described in Section \ref{pseudo-order}.

Since the practical performance of
this algorithm is rather slow for large fields,
we organised Algorithms {\tt One} and {\tt Two}
to ensure that they each need {\it at most one application}. 
If the dimension $d$ of the input group is odd, then we invoke 
this algorithm 
once to construct standard generators for $\SX(3, q)$. 
If $d$ is even, then as a final step, we construct the additional generators 
$x$ and $y$ using this algorithm. 
Let $h \in G=\SX(d, q)$ be the involution whose $-1$-eigenspace
is $\langle e_1, f_1, e_2, f_2 \rangle$. 
Observe that $h$ 
% = (s \cdot s^v)^{(p - 1)/2}$, so it 
can be readily constructed from the elements of $Y_0$, 
and that both $x$ and $y$ are elements of $C_G(h)$. 

\section{Base cases for orthogonal groups}\label{base-omega}
\subsection{Groups preserving forms of $+$ type}

Both $\Omega^+(2,q)$ 
and $\SO^+(2,q)$ are cyclic of order dividing $q-1$.
Hence the cost of their constructive recognition 
is the cost of a call to a discrete logarithm oracle
for $\GF(q)$.

The remaining base cases occur in dimension 4.
As we observed in Lemma \ref{omegap4}, 
$\Omega^+(4, q)$ is the central product 
of two copies of $\SL(2,q)$ arising from a tensor 
decomposition of the underlying space.

This tensor decomposition is readily made explicit:
by random selection, we construct an element of $\Omega^+(4, q)$ 
which acts as a scalar on one of the tensor factors and, 
using the algorithm of \cite[\S 4]{tensor},
construct the tensor factors.
Subject to a discrete logarithm oracle for $\GF(q)$,  
we now use the algorithm of \cite{Conderetal05} to recognise 
constructively the copies of $\SL(2,q)$. 

The complexity of this Las Vegas algorithm, 
measured in field operations, 
is constant, given a constant number of calls 
to the discrete logarithm oracle for $\GF(q)$.

Similar comments apply to $\SO^+(4, q) =
C_2.(\PSL(2,q)\times\PSL(2,q)).C_2$. 

\subsection{Groups preserving forms of $-$ type}
As we observed in Lemma \ref{omegam4}, 
$\Omega^-(4, q) \cong \PSL(2,q^2)$.
Subject to a discrete logarithm oracle for $\GF(q^2)$,  
we use the algorithm of \cite{Conderetal05} 
to recognise constructively this group, 
Similar comments apply to $\SO^-(4,q) \cong 
C_2\times\PSL(2,q^2)$.

We must also consider $G=\Omega^-(6,q)$ when $q\equiv 3\bmod 4$.
The centraliser of a non-central involution in $G$ 
contains $\Omega^+(4,q)\times\Omega^-(2,q)$ and so 
{\tt OneOmegaMinus3} does not apply. 
Instead, we outline a new algorithm to obtain standard generators 
for $\Omega^-(6,q)$, assuming that $q > 3$.
Recall that $V$ denotes the underlying 6-dimensional space.

\begin{enumerate}
\item Find, by random search, an element of $G$ that powers up to 
an involution $i$, with an eigenspace
$E$ of dimension 4 supporting a form of $+$ type and 
an eigenspace $F$ of dimension 2 supporting a form of $-$ type.

\item 
Construct a generating set for $\Omega(F)$ in $C_G(i)$. 

\item 
Now find, by random search, $h \in G$ such that 
$T=E\cap E^h$ is of dimension 2, and supports a form of $+$ type.

\item 
The centraliser of $T$ in $G$ contains $\Omega(F)$ and $\Omega(F^h)$.
With high probability, the union of these two cyclic groups 
generates the centraliser $H := \Omega^-(4,q)$  of $T$ in $G$.
Decide this using the `naming' algorithm of \cite{NP}.
If not, repeat Steps 3 and 4 until it is true.
%this centraliser is a copy $H$ of $\Omega^-(4,q)$.  
%In fact $\Omega(F)$ and $\Omega(F^h)$ will together be likely to generate $H$.

\item 
Construct a hyperbolic basis $(e_2,f_2,x,y)$  for the orthogonal 
complement $T^\perp$ of $T$.  

\item 
Now construct standard generators for $H$.  
One of the standard generators for $H$ is $\delta$, 
and $\delta^{(q^2-1)/4}$ is 
the involution whose $+1$ and $-1$-eigenspaces,
restricted to $T^\perp$, are
$\langle e_2,f_2\rangle $ and $\langle x,y \rangle$.  

\item 
Allowing this involution to act on the whole of $V$, the $-1$-eigenspace
is unchanged; and in the centraliser of this involution we find a copy 
of $K=\Omega^+(4,q)$.  

\item 
Construct a hyperbolic basis $(e_1,f_1)$ for $T$, 
so that $(e_1,f_1,e_2,f_2,x,y)$ is a hyperbolic basis for $V$.  
Rewrite the standard generators of $H$ with respect to this basis.
All but one of the standard generators of $G$ now appear among the 
standard generators of $H$.

\item 
The remaining standard generator for $G$ 
is $(e_1,e_2)^-(f_1,f_2)^-$ and is an element of $K$.  
We now construct this generator
as an \SLP\ in the generators of $K = \Omega^+(4, q)$. 
\end{enumerate}

If $q=3$ then $\Omega(F)$ is of order 2, and this method fails.
Instead we use permutation group techniques to 
construct standard generators for $\Omega^-(6, 3)$.

\begin{lemma}
The complexity of this Las Vegas algorithm, measured in field operations, 
is constant, given a constant number of calls to the discrete logarithm 
oracle for $\GF(q^2)$.
\end{lemma}
\begin{proof}
For Step 1, see Theorem \ref{proportions}.
To compute the probability 
that $E\cap E^h$ is of dimension 2, and supports a form of
$+$ type, we count the number of pairs of subspaces of dimension 4 that 
support a form of $+$ type, and count the number of
pairs that in addition intersect in a space of $+$ type.  
This gives a probability that converges rapidly to $1/2$.  
To estimate the probability
that the union of $\Omega(F)$ and $\Omega(F^h)$ generates 
$H$, we compute the probability that these 
subgroups lie in a maximal subgroup.  For
example, the probability that they both lie in a copy of 
$\PSL(2,q)$ is $\Oh(1/q)$, and one sees easily that the 
probability of failure is $\Oh(1/q)$.  The use of the naming algorithm in
Step 4 is not necessary; we can simply start again if Step 6 fails.
The discrete logarithm oracle for $\GF(q^2)$ is used in Step 6.
The other steps clearly require a bounded number of field operations.
\end{proof}

\subsection{Groups preserving forms of $0$ type}
As we observed in Lemma \ref{omega03}, 
$\Omega(3, q) \cong \PSL(2,q)$. 
Subject to a discrete logarithm oracle for $\GF(q)$,  
we use the algorithm of \cite{Conderetal05} 
to recognise constructively this group. 
Similar comments apply to $\SO(3,q)$. 

We must also consider $G=\Omega(5,q)$ when $q\equiv 3\bmod 4$.
The centraliser of a non-central involution in $G$ 
contains $\Omega^-(2,q)$ and so 
{\tt OneOmegaCircle3} does not apply. 
Instead, we outline a new algorithm to obtain standard generators 
for $\Omega(5,q)$, assuming that $q > 3$.

\begin{enumerate}
\item  Find, by random search, an element of $G$ that powers up to an 
involution $i$ whose $-1$-eigenspace $E$ has dimension 4
and supports a form of $+$ type.

\item  Find, by random search, an element $h$ of $G$ such that 
$T=E\cap E^h$ has dimension 3 
and supports a non-degenerate form, and $T^\perp$ supports a form of $+$ type.

\item  Construct standard generators in $C_G(i)$ for the 
centraliser $\Omega(E)$ of $E^\perp$, 
and hence for the centraliser $\Omega(E^h)$ of $(E^\perp)^h$.

\item  Construct a hyperbolic basis $(e_2,f_2,x)$ of $T$. Find 
the standard generators for the centraliser $\Omega(T)$ of $T^\perp$ 
with respect to this basis as \SLPs\ in the given generators of 
$G$ by using explicit membership testing in $\Omega(E)$.

\item Observe that the centraliser $K$ in $\Omega(E)$ of $x$
acts as $\Omega(3,q)$ on the orthogonal complement 
of $\langle x\rangle$ in $E$.
Since we have found standard generators for $\Omega(E)$, we can now 
construct standard generators for $K$ as \SLPs\ in these standard generators. 

\item  In the same way we construct generators for the centraliser 
$L$ of $x$ in $\Omega(E^h)$.

\item Construct a hyperbolic basis $(e_1,f_1)$ for the orthogonal 
complement of $T$ in $V$.

\item  The union of $K$ and $L$ generates the 
centraliser $M$ of $x$ in $G$, which acts as $\Omega^+(4,q)$ on 
the orthogonal complement of $x$.  

\item Construct standard generators for $M = \Omega^+(4, q)$, and so
obtain $v=(e_1,e_2)^-(f_1,f_2)^-$ as an \SLP\ in the generators of $M$.

\item  The standard generators of $G$ with respect to the basis 
$(e_1,f_1,e_2,f_2,x)$ are the standard generators 
for $\Omega(T)$, together with $v$.
\end{enumerate}

\begin{lemma}
The complexity of this Las Vegas algorithm, 
measured in field operations, is constant,
given a constant number of calls to 
the discrete logarithm oracle for $\GF(q)$.
\end{lemma}
\begin{proof}
For Step 1, see Theorem \ref{proportions}.
The discrete logarithm oracle for $\GF(q)$ is used in Step 9.
\end{proof}

We can easily find standard generators for $\Omega(5, 3)$, 
for example, by considering
it as a permutation group acting on the set of isotropic vectors.

\section{Complexity of the algorithms} 
\label{Analysis}
We now analyse the principal
algorithms, and in the next section estimate the length of the \SLPs\ 
that express the canonical generators as words in  the given
generators. The time analysis is based on counting the number of
field operations, the number of random elements selected, and 
the number of calls to the discrete logarithm oracle. Use of discrete
logarithms in a given field requires first the setting up of certain
tables, and these tables are consulted for each application. The
time spent in the discrete logarithm oracle, and the space that it
requires, are not proportional to the number of applications in a
given field. 

A hyperbolic basis for a vector space with a given
non-degenerate bilinear form can be constructed 
in $\Oh(d^3)$ field operations (see \cite{Brooksbank03} 
for an algorithm to perform this task).

If a matrix group acts absolutely irreducible  on its underlying 
vector space, then we can determine the classical forms 
it preserves in $\Oh(d^3)$ field operations 
(see \cite[Section 7.5.4]{HoltEickOBrien05}).

Babai \cite{Babai91} presented a Monte Carlo algorithm to
construct in polynomial time independent nearly uniformly distributed 
random elements of a finite group.  An alternative is the 
{\it product replacement algorithm} of Celler
{\it et al.\ }\cite{Celleretal95}.
That this is also polynomial time was
established by Pak \cite{Pak00}.
For a discussion of both algorithms, see \cite[pp.\ 26-30]{Seress03}.

We now complete our analysis of the main algorithms.
\begin{theorem}\label{Theorem1}  
The number of field operations carried out in the 
Las Vegas algorithm {\tt OneEven} is 
$\Oh(d (\xi + d^3 \log d + d^2\log d \log\log d \log q))$. 
\end{theorem}
\begin{proof} 
The proportion of elements of $G$ with the required property in line 4
is at least $k/d$ for some absolute constant $k$, as proved in Section
\ref{Involution}. Theorem \ref{Corollary5.1} shows that the 
involution can be constructed in 
$\Oh(d(\xi + d^3 \log d + d^2 \log d \log\log d \log q))$ field operations.

Lines 8 and 14 require 
$\Oh(d(\xi + d^3 \log d + d^2 \log d \log\log d \log q))$ field operations
as proved in Section \ref{Bray}.

The recursive calls in  lines 9 and 10 involve matrices of dimension at
most $2d/3$; Lemma \ref{d-a-c} implies that they increase 
the number of field operations by only a constant factor. 

The result follows.
\end{proof}

We estimate the number of calls to the $\SL(2, q)$ constructive
recognition algorithm and the associated discrete logarithm oracle.
\begin{theorem}\label{Theorem2} 
If $d > 2$, then Algorithms {\tt OneEven} and {\tt TwoEven}
generate at most $2d-3$ and $6 \log d$ calls  to 
the discrete logarithm oracle for $\GF(q)$ respectively.
\end{theorem}
\begin{proof}
Each call to the constructive recognition oracle for $\SL(2,q)$ generates 
three calls to the discrete logarithm oracle for $\GF(q)$ 
(see \cite{Conderetal05}).
Each solution to Problem \ref{glue} requires three calls to 
the discrete logarithm oracle. 

Let $f(d)$ be the number of calls to the discrete logarithm oracle 
generated by applying {\tt OneEven} to $\SX(d, q)$. 
Then $f(2) = f(4) = 3$ and 
$f(d) = f(e) + f(d - e) + 3$ for $d > 4$ and some $e \in (d/3, 2d/3]$.
It follows that $f(d) \leq 2d - 3$ for $d > 2$.

Let $g(d)$ be the number of calls generated by 
applying {\tt TwoEven} to $\SX(d, q)$, where $d$ is even.
Again $g(2) = g(4) = 3$ and $g(2n) \leq g(n) + 6$ for $n > 2$.
Hence $g(d) \leq 6 \log d$.
\end{proof}

Similar results hold for the other algorithms.
If we use the involution-centraliser algorithm \cite{Ryba-paper} 
to construct either standard generators for $\SX(3, q)$, 
or the additional generators $x, y \in \SX(4, q)$, 
then the number of calls to the oracle in each case is 9.

\section{Straight-line programs}\label{SLP}
We now consider the length of the \SLPs\ for the 
standard generators for $\SX(d, q)$ constructed by our algorithms.

In its simplest form, 
an \SLP\ on a subset $X$ of a group $G$ is a
string, each of whose entries is either a pointer to an element of $X$, 
or a pointer to a previous
entry of the string, or an ordered pair of pointers to (not necessarily 
distinct) previous entries.
Every entry of the string defines an element of $G$.  An 
entry that points to an element of
$X$ defines that element.  An entry that points to a previous entry defines 
the inverse of the element defined by that entry.  An entry that points to 
two previous entries defines the
product, in that order, of the elements defined by those entries.

Such a simple \SLP\ defines an element 
of $G$, namely the element defined by the last entry, and it 
can be obtained by computing in turn
the elements for successive entries.
The \SLP\ is primarily used by replacing the elements 
$X$ of $G$ by the elements $Y$ of some group
$H$, where $X$ and $Y$ are in one-to-one correspondence, and then evaluating 
the element of $H$ that the \SLP\ then defines.

Before we estimate their lengths, we 
identify other critical properties of \SLPs. 
\begin{enumerate}
\item 
We replace the second type of node, which defines the 
inverse of a previously defined element, by a node type with two fields, 
one pointing to a previous entry, and one containing 
a possibly negative integer.  The element defined is then the 
element defined by the entry to which
the former field points, raised to the power defined by 
the latter field.  This reflects the fact that we
raise group elements to very large powers, and have 
an efficient algorithm described in Section \ref{Exp} for performing this.

\item 
An \SLP\ may define a number of elements of $G$,
and not just one element, so a sequence of nodes may be specified as 
giving rise to elements
of $G$.  Thus we wish to return a single \SLP\ that defines 
all of the standard generators of $\SX(d,q)$, rather than an \SLP\ 
for each generator.
This avoids duplication when two or more of the standard generators 
rely on common calculations.

\item 
A critical concern is how the number of trials in a random search for 
a group element affects the length of an \SLP\ that defines that element.  
Any discussion of this requires consideration of the algorithm
used to generate random elements.  We make two reasonable assumptions:
\begin{enumerate}
\item[(a)] the associated random process is a stochastic process taking place 
in a graph whose vertices are defined by a {\it seed};
\item[(b)] a random number generator now determines which 
edge adjoining the current vertex in the graph will be followed in 
the stochastic process.  
\end{enumerate}
By default, the length of the \SLP\ will then increase by a constant amount 
for every trial, {\it successful or unsuccessful}.  
Should its length reflect only those trials that are successful? 
One additional assumption which allows us to explore this question is 
the following: 
\begin{quotation}
\noindent 
{\it 
When embarking on a search that is expected to require $d$ trials, 
we record the value of the seed, and repeatedly carry
out a random search, using our random process, but returning, after 
every $\ell (d)$ steps, for some function $\ell$ of $d$, to the 
stored value of the seed, until we succeed.
}
\end{quotation}
We hypothesise that values for $\ell(d)$ range from $\log d$ to $d$ 
and now analyse the lengths of the \SLPs\ for the boundary values. 
\end{enumerate}

\begin{theorem} 
If the \SLPs\ constructed satisfy properties $1-3$ above,
then their lengths are the following.
%\begin{table}[h]
%%\caption{Length of SLP}
%\label{SLP-len}
\begin{center}
\begin{tabular}
{|c|r|r|} \hline
$\ell (d)$  & {\tt OneMain}  & {\tt TwoMain} \rule{0cm}{3.0ex}\\
\hline
%1  &  $d + \log d$ & $\log^2 d$ \rule{0cm}{3.0ex}\\ \hline
$\log d$  &  $\Oh(d \log d)$ & $\Oh(\log^3 d)$ \rule{0cm}{3.0ex}\\ \hline
$d$  &  $\Oh(d \log d)$ & $\Oh(d \log d)$ \rule{0cm}{3.0ex}\\ \hline
\end{tabular}
\end{center}
%\end{table}

\end{theorem}
\begin{proof}
For each hypothesised value of $\ell(d)$, 
we wish to find functions $f(d)$ and $g(d)$ such that the lengths of
the \SLPs\ returned by Algorithms {\tt One} and {\tt Two} are bounded
above by these functions respectively.

Let $e\in(d/3,2d/3]$. 
Our analysis of Algorithm {\tt One} implies that 
$f(d) \leq f(e) + f(d - e) + c \cdot \ell (d)$ for 
some constant $c > 0$.

%It suffices for $f$ to satisfy
%$f(d)\ge f(e)+f(d-e) + c\ell(d)$ whenever $d\ge5$ for some constant
%$c$, if $f(d)$ is large enough for small values of $d$. Since of
%necessity $f(d)>f(e)+f(d-e)$ it follows that $f(d)$ is at least linear in $d$.

Consider, for example, the case where $\ell(d)=d$.
We wish to prove that 
$f (d) \leq k \cdot d \log d$ for some positive constant  $k$.
Let $k>3c/(3\log(3) - 2)$, taking all logarithms to base 2. Assume by induction
that $f(n) < kn\log(n)$ for  all $n<d$ for some $d>4$.  Then 
$$f(d)\le f(e)+f(d-e) +cd  <  
ke\log(e) + k(d-e)\log(d-e) + cd  <  kd\log(d),$$
as required, since $e\log(e)+(d-e)\log(d-e)$ takes its maximum value,
for $e$ in the given range, when $e=2d/3$.
The results are similar if $\ell(d)=\log d$.

Algorithm {\tt Two} recurses either from the case $d=4n$ to the case
$d=2n$ in one step, or from the case $d=4n+2$ to the case $d=4n$ and
then to the case $d=2n$. It is easy to see that the effect on the
length of the \SLP\ in the latter situation is dominated by the second
step. If $d$ is initially odd, then the contribution of the reduction to
the even case, which is carried out once, may also be ignored here.
The main contribution to the length of the \SLP\ in passing from $d=4n$
to $d=2n$ arises from constructing an involution whose eigenspaces
have dimension $2n$. This involution is constructed recursively,
where the length of the recursion is $\Oh(\log d)$. Thus the
contribution to the length of the \SLP\ in constructing this involution
is $\Oh(\log d\ell(d))$. 
Hence, $g(4n)\le g(2n)+c\log(n)\ell(n)$ and $g(4n+2)\le g(2n)+c\log(n)\ell(n)$
for some $c > 0$.

If $\ell(d)=\Oh(\log d)$, then  the inequality 
$g(n)\le g(\lceil n/2\rceil)+c\log^2(n)$
is satisfied by $g(n)=k\log^3(n)$ for sufficiently large $k$. 
Similar calculations can be carried for the other case,
yielding the stated results. 
\end{proof}
\section{An implementation}\label{implementation}
Our implementation of these algorithms is publicly available in {\sc Magma}.
It uses:
\begin{itemize}
\item 
the product replacement algorithm \cite{Celleretal95}
to generate random elements;  
\item a new implementation of this algorithm by
B\"a\"arnhielm \& Leedham-Green \cite{prospector}
which realises the properties 
identified in Section \ref{SLP};
\item our implementations of Bray's algorithm \cite{Bray}
and the involution-centraliser algorithm \cite{Ryba-paper}.
\item our implementations of the algorithms of \cite{Conderetal05}
and \cite{sl3q}.
\end{itemize}

The computations reported in Table \ref{table1} were carried out
using {\sc Magma} V2.13 on a Pentium IV 2.8 GHz processor.
We list the CPU time in seconds taken to construct 
the standard generators for $\SX(d, q)$ for the 
non-orthogonal groups, 
and for $\Omega^\epsilon(d, q)$ for a range of values of $d$ and $q$.
We use Algorithm {\tt Two} for the non-orthogonal 
groups,  Algorithm {\tt One} for the orthogonal groups.
The time is averaged over three runs.

\begin{table}[htp]
\label{table1}
\begin{center}
\begin{tabular}
{|c|r|r|r|r|r|r|r|} \hline
$d$ & $q$ & $\SL$ & $\Sp$ & $\SU$ & $\Omega^+$ & $\Omega^-$ & $\Omega^0$  
\rule{0cm}{2.5ex}\\ \hline
5 & 5  & $0.1$ & -- & 1.4 & -- & -- & 2.8 \rule{0cm}{2.5ex}\\ \hline
6 & 5  & $0.4$ & 2.7 & 1.4 & 3.3 & 2.2 & -- \rule{0cm}{2.5ex}\\ \hline
10  & 5  & 0.5 & 4.5 & 1.6 & 5.4 & 4.8 & -- \rule{0cm}{2.5ex}\\ \hline
20  & 5  & 0.9 & 6.1 & 2.3 & 14.0 & 12.2 & -- \rule{0cm}{2.5ex}\\ \hline
25 & 5  & 1.5 & -- & 4.8 & -- & -- & 17.0  \rule{0cm}{2.5ex}\\ \hline
40  & 5  & 1.9 & 31.0 & 6.2 & 31.1 & 32.8 & -- \rule{0cm}{2.5ex}\\ \hline
45 & 5  & $5.4$ & -- & 12.6 &    -- & -- & 41.7 \rule{0cm}{2.5ex}\\ \hline
60  & 5  & 6.2 & 13.0 & 26.8 & 51.1 & 64.2 & -- \rule{0cm}{2.5ex}\\ \hline
80  & 5  & 13.0 & 16.5 & 39.3 & 40.3 & 114.2 & -- \rule{0cm}{2.5ex}\\ \hline
100  & 5  & 34.7 & 24.3 & 83.8 & 120.0 & 203.9 & -- \rule{0cm}{2.5ex}\\ \hline
5 & $5^4$  & $0.7$ & -- & 5.1 & -- & -- & 5.2 \rule{0cm}{2.5ex}\\ \hline
6 & $5^4$ & $1.1$ & 7.1 & 8.8 & 7.0 & 5.6 & -- \rule{0cm}{2.5ex}\\ \hline
10  & $5^4$  & 2.1 & 18.8 & 13.1  & 12.8 & 12.3 & -- \rule{0cm}{2.5ex}\\ \hline
20  & $5^4$  & 3.7 & 25.6 & 19.1 & 32.7 & 32.3 & -- \rule{0cm}{2.5ex}\\ \hline
25 & $5^4$  & $7.2$ & -- & 37.3 & -- & -- & 56.4 \rule{0cm}{2.5ex}\\ \hline
40  & $5^4$  & 18.6 & 39.8 & 41.6 & 103.0 & 128.7 & -- \rule{0cm}{2.5ex}\\ \hline
45 & $5^4$  & $21.2$ & -- & 98.6 & -- & -- & 297.9 \rule{0cm}{2.5ex}\\ \hline
60  & $5^4$  & 82.4 & 74.5 & 151.9 & 241.5 & 418.6 & -- \rule{0cm}{2.5ex}\\ \hline
80  & $5^4$  & 167.7 & 110.5 & 202.5 & 530.2 & 729.7 & -- \rule{0cm}{2.5ex}\\ \hline
100  & $5^4$  & 501.9 & 244.4 & 404.9 & 996.0 & 1571.6 & -- \rule{0cm}{2.5ex}\\ \hline
\end{tabular}
\end{center}
\caption{Performance of implementation for a sample of groups}
\end{table}

\begin{thebibliography}{10}

\bibitem{Ambrose}
Sophie Ambrose. Matrix Groups: Theory, Algorithms and Applications.
PhD thesis, University of Western Australia, 2006.

\bibitem{Babai91}
L\'aszl\'o Babai.  
Local expansion of vertex-transitive graphs and
  random generation in finite groups.  {\it Theory of Computing}, (Los
  Angeles, 1991), pp.\ 164--174. Association for Computing Machinery, 
New York, 1991.

\bibitem{BabaiSzemeredi84}
L\'aszl\'o Babai and Endre Szemer\'edi.
\newblock On the complexity of matrix group problems, {I}.
\newblock In {\em Proc.\ $25$th IEEE Sympos.\ Foundations Comp.\ Sci.}, pages
  229--240, 1984.

\bibitem{HB}
Henrik B\"a\"arnhielm. 
Recognising the Suzuki groups in their natural representations.  
{\it J. Algebra} {\bf 300} (2006), 171--198.
 
\bibitem{prospector}
Henrik B\"a\"arnhielm and C.R. Leedham-Green.
% Lawrence Pettit. 
Extending the product replacement algorithm.
Preprint 2007. 

\bibitem{BLNPS1} Robert Beals, Charles R. Leedham-Green, Alice C. Niemeyer,
Cheryl E. Praeger and \'Akos Seress.
\newblock A black-box group algorithm for recognizing finite symmetric
and alternating groups I,
\newblock {\em Trans. Amer. Math. Soc.}  {\bf 355} (2003), 2097--2113.

\bibitem{Magma}
Wieb Bosma, John Cannon, and Catherine Playoust.
\newblock The {\sc Magma} algebra system I: The user language,
\newblock {\em J.\ Symbolic Comput.}, {\bf 24}, 235--265, 1997.

\bibitem{Bray} J.N. Bray. An improved method of finding
the centralizer of an involution. {\it Arch. Math. (Basel)}
{\bf 74} (2000), 241--245.

\bibitem{Brooksbank03}
Peter A. Brooksbank.
\newblock Constructive recognition of classical groups
in their natural representation.
\newblock {\em J. Symbolic Comput.} {\bf 35} (2003), 195--239.

\bibitem{Brooksbank03a}
Peter~A. Brooksbank.
\newblock Fast constructive recognition of black-box unitary groups.
\newblock {\em LMS J. Comput. Math.}, 6:162--197 (electronic), 2003.

\bibitem{BK}
Peter A. Brooksbank and William M. Kantor.
\newblock On constructive recognition of a black box {${\rm PSL}(d,q)$}.
\newblock In {\em Groups and Computation, III (Columbus, OH, 1999)}, volume~8
  of {\em Ohio State Univ. Math. Res. Inst. Publ.}, de Gruyter, Berlin,
  95--111,  2001.

\bibitem{BK06}
Peter~A. Brooksbank and William M.\ Kantor.
\newblock Fast constructive recognition of black box orthogonal groups.
\newblock {\em J. Algebra}, {\bf 300}, 2006, 256-288. 

\bibitem{Carter}
Roger Carter.
\newblock Simple groups of Lie Type.
\newblock Wiley-Interscience, 1989.

\bibitem{Celleretal95}
Frank Celler, Charles R.\ Leedham-Green, Scott H.\ Murray, Alice C.\
  Niemeyer and E.A.\ O'Brien. Generating random elements of a 
finite group. {\it Comm.\ Algebra}, {\bf 23} (1995), 4931--4948.

\bibitem{CLG97}
Frank Celler and C.R.\ Leedham-Green.
\newblock Calculating the order of an invertible matrix.
\newblock In {\em {Groups and Computation {II}}}, volume~28 of {\em Amer.\
  Math.\ Soc.\ DIMACS Series}, pages 55--60. (DIMACS, 1995), 1997.

\bibitem{CellerLeedhamGreen98}
F.~Celler and C.R.~Leedham-Green.
\newblock A constructive recognition algorithm for the special linear group.
\newblock In {\em The atlas of finite groups: ten years on (Birmingham, 1995)},
  volume 249 of {\em London Math. Soc. Lecture Note Ser.}, pages 11--26,
  Cambridge, 1998. Cambridge Univ. Press.

\bibitem{ConderLeedhamGreen01}
Marston Conder and Charles~R. Leedham-Green.
\newblock Fast recognition of classical groups over large fields.
\newblock In {\em Groups and Computation, III (Columbus, OH, 1999)}, volume~8
  of {\em Ohio State Univ. Math. Res. Inst. Publ.}, pages 113--121, Berlin,
  2001. de Gruyter.

\bibitem{Conderetal05}
M.D.E. Conder, C.R. Leedham-Green, and E.A. O'Brien.
\newblock Constructive recognition of PSL$(2, q)$.
\newblock {\em Trans.\ Amer.\ Math.\ Soc.} {\bf 358}, 1203--1221, 2006.

\bibitem{Dornhoff} L. Dornhoff, {\it Group Representation Theory, Part A},
Marcel Dekker, 1971.

\bibitem{Giesbrecht} Mark Giesbrecht.
Nearly optimal algorithms for canonical matrix forms. 
PhD thesis, University of Toronto, 1993.
%SIAM J. Comput. 24 (1995), no. 5, 948--969. 

\bibitem{GLO} S.P.\ Glasby, C.R.\ Leedham-Green, and E.A. O'Brien.
Writing projective representations over subfields.
{\it J. Algebra}, 295, 51-61, 2006.

\bibitem{GLS3}
Daniel Gorenstein, Richard Lyons, and  Ronald Solomon.
The classification of the finite simple groups. Number 3. Part I,
American Mathematical Society, Providence, RI, 1998. 

\bibitem{Ryba-paper} P.E. Holmes, S.A. Linton, E.A. O'Brien, A.J.E. Ryba and
R.A. Wilson. Constructive membership in black-box groups. Preprint 2007.

\bibitem{HoltEickOBrien05}
Derek~F.\ Holt, Bettina Eick, and Eamonn~A.\ O'Brien.
\newblock {\em Handbook of computational group theory}.
\newblock Chapman and Hall/CRC, London, 2005.

\bibitem{Huppert67}
B.\ Huppert.
\newblock {\em {E}ndliche {G}ruppen {I}}, 
volume 134 of {\em Grundlehren Math.\ Wiss.}
\newblock Springer-Verlag, Berlin, Heidelberg, New York, 1967.

\bibitem{KantorSeress01}
William~M. Kantor and {\'A}kos Seress.
\newblock Black box classical groups.
\newblock {\em Mem. Amer. Math. Soc.}, {\bf 149}, 2001.

\bibitem{KG}
W.\ Keller-Gehrig. Fast algorithms for the characteristic
polynomial. {\it Theoret. Comput. Sci.} {\bf 36}, 309--317, 1985. 
    
\bibitem{lish} M.W. Liebeck and A. Shalev. The probability of generating
a finite simple group, {\it Geom. Ded.} {\bf 56} (1995), 103--113.

\bibitem{sl3q}
F.\ L{\"u}beck, K.\ Magaard, and E.A. O'Brien. 
Constructive recognition of $\SL_3(q)$.
{\it J. Algebra}, {\bf 316}, 2007, 619--633.

\bibitem{tensor}
C.R.\ Leedham-Green and E.A.\ O'Brien. 
Tensor Products are Projective Geometries.
{\it J. Algebra}, {\bf 189}, 514--528, 1997.

%\bibitem{NeumannPraeger92}
%Peter~M.\ Neumann and Cheryl~E.\ Praeger.
%\newblock A recognition algorithm for special linear groups.
%\newblock {\em Proc.\ London Math.\ Soc.\ $(3)$}, 65:555--603, 1992.

\bibitem{NP} Alice C. Niemeyer and Cheryl E. Praeger.
A recognition algorithm for classical groups over finite fields,
{\it Proc. London Math. Soc.} {\bf 77} (1998), 117--169.

\bibitem{JAMS}
Alice C. Niemeyer and Cheryl E.\ Praeger.
A recognition algorithm for non-generic classical groups over finite fields. 
J. Austral. Math. Soc. Ser. A {\bf 67} (1999), no. 2, 223--253.

%\bibitem{NHZ}
%Ivan Niven, Herbert S. Zuckerman and Hugh L. Montgomery,
%An introduction to the theory of numbers. 5th edition, John Wiley,
%New York, 1991.

\bibitem{OBrien05}
E.A. O'Brien. Towards effective algorithms for linear groups.
{\it Finite Geometries, Groups and Computation},
(Colorado), pp. 163-190. De Gruyter, Berlin, 2006.

\bibitem{Pak00}
Igor Pak. The product replacement algorithm is polynomial.
In {\it 41st Annual Symposium on Foundations of Computer Science
(Redondo Beach, CA, 2000)}, 476--485,
IEEE Comput. Soc. Press, Los Alamitos, CA, 2000.

\bibitem{PW05}
C.W.\ Parker and R.A.\ Wilson.
Recognising simplicity in black-box groups. 
Preprint 2007.

\bibitem{Seress03}
{\'A}kos Seress.
\newblock {\em Permutation group algorithms}, volume 152 of {\em Cambridge
  Tracts in Mathematics}.
\newblock Cambridge University Press, Cambridge, 2003.

%\bibitem{Storjohann98}
%Arne Storjohann.
%An $\Oh(n\sp 3)$ algorithm for the Frobenius normal form. In
%{\em Proceedings of the 1998 International Symposium on Symbolic
%and Algebraic Computation} (Rostock), 101--104, ACM, New York, 1998.

\bibitem{Taylor} 
Donald E. Taylor. 
The geometry of the classical groups.
Sigma Series in Pure Mathematics, {\bf 9}. Heldermann Verlag, Berlin, 1992. 

\bibitem{vzg}
Joachim von zur Gathen and J\"urgen Gerhard,
{\it Modern Computer Algebra}, Cambridge University Press, 2003.

\bibitem{Zassenhaus}©
Hans Zassenhaus. On the spinor norm.  Arch. Math.  13  1962 434--451.
\end{thebibliography}

\begin{tabbing}
\=\hspace{70mm}\=\kill
\>School of Mathematical Sciences \>Department of Mathematics \\
\>Queen Mary, University of London \>Private Bag 92019, Auckland \\
\>London E1 4NS,   \>University of Auckland  \\
\>United Kingdom  \> New Zealand  \\
\> C.R.Leedham-Green@qmul.ac.uk \> obrien@math.auckland.ac.nz
\end{tabbing}

\vspace*{2mm}
\noindent 
Last revised \today

\end{document}
